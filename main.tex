\documentclass[a4paper, 11pt, twoside]{book}

\usepackage{comment}

% ---------------- Encodage ------------- %
\usepackage[utf8]{inputenc}
\usepackage[T1]{fontenc}
\usepackage[french]{babel}
\usepackage[autolanguage]{numprint}
\usepackage{csquotes}
% ---------------- Encodage ------------- %

%------- Bibliographie ----------
\usepackage[backend=biber, maxbibnames=99, style=science]{biblatex}
\addbibresource{refs.bib}
%------- Bibliographie ----------

% -------------- Mise en page ---------- %
\usepackage[top=1.5cm, bottom=1.5cm, left = 1.5cm, right=3cm]{geometry}

\usepackage{fancyhdr}
\pagestyle{fancy}

\usepackage{lastpage}
%\fancyfoot[C]{\thepage \ / \pageref{LastPage}}
\fancyhead[L]{L'essentiel des mathématiques de première année de Licence}
\fancyhead[R]{Année scolaire 2022-2023}
\usepackage{indentfirst}
\usepackage{setspace}
\usepackage[usenames, dvipsnames]{xcolor}
\usepackage{booktabs}
\usepackage{multicol}
\usepackage{caption}
\usepackage{subcaption}
\usepackage{diagbox}
\usepackage{xspace}

\setlength{\tabcolsep}{10pt}
\usepackage{needspace}
\usepackage[Lenny]{fncychap}
\usepackage{lmodern}
% -------------- Mise en page ---------- %

% ------------ Boxes -------------- %
\usepackage{framed}
\usepackage[most]{tcolorbox}
% ------------- Boxes -------------- %

% ---------------- Maths -------------- %
\usepackage{mathtools, amssymb, amsthm}
\usepackage[makeroom]{cancel}
\usepackage{centernot}
\usepackage{systeme}
\usepackage{stmaryrd}
\usepackage{dsfont}
% ---------------- Maths -------------- %

% ------------- Théorèmes ------------ %
\newtcolorbox{theorem}[1][]{enhanced, colback=white, colframe=green!60!black, title=\textbf{Théorème} : {#1}, breakable, attach boxed title to top text left={yshift*=-3mm}, boxed title style={size=small, colback=green!60!black, sharpish corners}, sharpish corners}

\newtcolorbox{lemma}[1][]{enhanced, colback=white, colframe=red!60!black, title=\textbf{Lemme} : {#1}, breakable, attach boxed title to top text left={yshift*=-3mm}, boxed title style={size=small, colback=red!60!black, sharpish corners}, sharpish corners}

\newtcolorbox{corollary}[1][]{enhanced, colback=white, colframe=green!60!black, title=\textbf{Corollaire} : {#1}, breakable, attach boxed title to top text left={yshift*=-3mm}, boxed title style={size=small, colback=green!60!black, sharpish corners}, sharpish corners}

\newtcolorbox{proposition}[1][]{enhanced, colback=white, colframe=red!60!black, title=\textbf{Proposition} : {#1}, breakable, attach boxed title to top text left={yshift*=-3mm}, boxed title style={size=small, colback=red!60!black, sharpish corners}, sharpish corners}

\newtcolorbox{definition}[1][]{enhanced, colback=white, colframe=blue!60!black, title=\textbf{Définition} : {#1}, breakable, attach boxed title to top text left={yshift*=-3mm}, boxed title style={size=small, colback=blue!60!black, sharpish corners}, sharpish corners}

\newtcolorbox{axiom}[1][]{enhanced, colback=white, colframe=blue!60!black, title=\textbf{Axiome} : {#1}, breakable, attach boxed title to top text left={yshift*=-3mm}, boxed title style={size=small, colback=blue!60!black, sharpish corners}, sharpish corners}

\theoremstyle{definition}
\newtheorem{example}{Exemple}[chapter]
\newtheorem{exercise}{Exercice}[chapter]

\theoremstyle{remark}
\newtheorem*{remark}{Remarque}
\newtheorem*{notation}{Notation}
\newtheorem*{nomenclature}{Nomenclature}
% ------------- Théorèmes ------------ %

% -------- Liens -------
\usepackage{hyperref}

%\hypersetup{
%    colorlinks=true,
%    allcolors=black,
%    pdftitle={Essentiel mathématiques L1},
%}
% -------- Liens -------

% ------------- Images --------------- %
\usepackage{graphicx}
% ------------- Images --------------- %

% ------------ TIKZ -------------- %
\usepackage{tikz}
\usepackage{pgfplots}
\pgfplotsset{compat=1.15}
\usepackage{mathrsfs}
\usetikzlibrary{arrows, angles, quotes, calc}

% -------- TIKZ ------------------ %

% ------------------ Physique ---------------- %
\usepackage{physics}
\usepackage{siunitx}
\AtBeginDocument{\RenewCommandCopy\qty\SI}
% ------------------ Physique ---------------- %

% ------------ Ensembles de nombres ---------- %
\newcommand{\N}{\mathbb{N}}
\newcommand{\Z}{\mathbb{Z}}
\newcommand{\D}{\mathbb{D}}
\newcommand{\Q}{\mathbb{Q}}
\newcommand{\R}{\mathbb{R}}
\newcommand{\C}{\mathbb{C}}
\newcommand{\K}{\mathbb{K}}
% ------------ Ensembles de nombres ---------- %

% ------------ Macros ---------------- %
\DeclareMathOperator{\pgcd}{pgcd}
\DeclareMathOperator{\ppcm}{ppcm}
\DeclareMathOperator{\rg}{rg}
\DeclareMathOperator{\com}{com}
\DeclareMathOperator{\End}{End}
\DeclareMathOperator{\Aut}{Aut}

\newcommand{\deriv}[2]{\frac{\diffd{#1}}{\diffd{#2}}}
\newcommand{\link}[2]{\href{#1}{\textcolor{blue}{\underline{#2}}}}

\newcommand{\appli}[5]{
	\begin{array}{rrcl}
		#1 \colon & #2 & \longrightarrow & #3 \\
		          & #4 & \longmapsto     & #5
	\end{array}
}
% ------------ Macros ---------- %

%-----------Bloc de code------------%
\usepackage{listings}
\usepackage{color}

\definecolor{dkgreen}{rgb}{0,0.6,0}
\definecolor{gray}{rgb}{0.5,0.5,0.5}
\definecolor{mauve}{rgb}{0.58,0,0.82}

\definecolor{xdxdff}{rgb}{0.49019607843137253,0.49019607843137253,1}
\definecolor{qqwuqq}{rgb}{0,0.39215686274509803,0}
\definecolor{qqqqff}{rgb}{0,0,1}

\lstset{frame=single,
	language=C++,
	aboveskip=3mm,
	belowskip=3mm,
	showstringspaces=false,
	columns=flexible,
	basicstyle={\small\ttfamily},
	numbers=none,
	numberstyle=\tiny\color{gray},
	keywordstyle=\color{blue},
	commentstyle=\color{dkgreen},
	stringstyle=\color{mauve},
	breaklines=true,
	breakatwhitespace=true,
	tabsize=4
}

\newcommand*{\inlineCode}{\fontfamily{pnc}\selectfont\ttfamily}

\usepackage{comment}
%-----------Bloc de code----------------%

\renewcommand{\leq}{\leqslant}
\renewcommand{\geq}{\geqslant}

%\includeonly{chapters/algebre/complexes}

\begin{document}

\newgeometry{margin=2.5cm}
\begin{titlepage}
	\begin{center}
		\vspace*{\fill}
			\vspace*{1cm}
			\hrule
			\vspace{0.5cm}
			\Large \textbf{L'essentiel des mathématiques de première année de Licence}
			\\
			Raphaël Heng
			\\
            Alyce Théobald
			\vspace{0.5cm}
			\hrule
			\vspace{0.5cm}
			\begin{figure}[!h]
				\centering
				\includegraphics[scale=0.6]{img/logo_lyon1.jpg}
			\end{figure}
			\vspace{0.5cm}
			Université Claude Bernard Lyon 1 
			\\
			Licence 1 - Portail Mathématiques-Informatique 
			\\
			Année universitaire 2022-2023 
			\\
			\vspace*{1cm}
		\vspace*{\fill}
	\end{center}
\end{titlepage}
\restoregeometry

\tableofcontents

\part{Introduction}
\def\arraystretch{1}

\par Avant toute chose, nous tenons à préciser que cette mise en page est destinée à une impression sous la forme d'un livre. Ainsi pour profiter d'une bonne lecture sur la version numérique, nous vous recommandons de la lire en mode \og double pages \fg avec les plus grandes marges orientées vers le centre. 
\\
\par \noindent Ce document repose principalement sur les enseignements de nos professeurs Guillaume AUBRUN, Kenji IOHARA et Thomas STROBL
mais nous avons utilisé des ressources complémentaires telles que \textbf{Bibmath} \cite{bibmath}, \textbf{Wikipédia} \cite{wikipedia}, \textbf{Exo7} \cite{exo7} ou encore les cours disponibles sur le site de la \textbf{licence Mathématiques} \cite{licence_maths}.

\par \noindent Il regroupe l'essentiel des compétences mathématiques à maîtriser à la fin de la première année de Licence. Vous y trouverez les définitions et les théorèmes à connaître accompagnés d'exercices à savoir refaire. Nous essaierons de démontrer le plus de théorèmes possibles, cependant les preuves ne sont pas toute à retenir (cela dépend également de votre orientation : mathématiques ou informatique).
Il se peut également que les outils mathématiques ne soient pas présentées scrupuleusement comme aux cours magistraux, nous avons éventuellement paraphrasé certains passages. Par exemple, il se peut que les notations utilisées ne soient pas les mêmes que celles vues en cours, nous avons préféré utiliser des notations qui nous semblent plus claires.
\\
\par \noindent Nous pensons qu'il est intéressant de définir certains mots de vocabulaires définis ci-dessous :
\begin{itemize}
    \item \textbf{Assertion} : Une assertion est une affirmation mathématique qui est soit vraie soit fausse. 
    \item \textbf{Axiome} : Un axiome est une assertion que l'on considère vraie sans démonstration.
    \item \textbf{Définition} : Une définition énonce comment un objet mathématique est construit.
    \item \textbf{Théorème} : Un théorème est une assertion d'importance particulière ayant été démontrée.
    \item \textbf{Corollaire} : Un corollaire est un résultat découlant d'un théorème.
    \item \textbf{Lemme} : Un lemme est un résultat intermédiaire sur lequel on s'appuie pour démontrer un théorème.
    \item \textbf{Proposition} : Une proposition est un résultat simple qui n'est pas associé à un théorème.
    \item \textbf{Conjecture} : Une conjecture est une proposition dont on ignore la véracité.
\end{itemize}
\par \noindent Rappelons également les ensembles de nombres étudiés au lycée :
\[ \text{L'ensemble des entiers naturels : } \N = \{0, 1, \ldots \} \]
\[ \text{L'ensemble des entiers relatifs : } \Z = \{ \ldots, -1, 0, 1, \ldots \} \]
\[ \text{L'ensemble des nombres décimaux : } \D = \left\{ \frac{a}{10^n} : a \in \Z, n \in \N \right\} \]
\[ \text{L'ensemble des nombres rationnels : } \Q = \left\{ \frac{a}{b}  : a \in \Z, b \in \Z^* \right\} \]
\[ \text{L'ensemble des nombres réels : } \R = ]-\infty, +\infty[ \]
\par \noindent Nous définirons l'ensemble des nombres réels plus rigoureusement dans le \autoref{chap:nb_reels}.
\\
\par \noindent Pour désigner un ensemble privé de 0, nous pouvons lui ajouter \og * \fg en exposant. 
\\ 
Par exemple $\N^* = \{1, 2, \ldots\}$.
\\
\par \noindent Définissons également certaines notations :
\begin{itemize}
    \item $\forall$ : \og Pour tout \fg ou \og Quelque soit \fg.
    \item $\exists$ : \og Il existe \fg.
    \item $\exists!$ : \og Il existe un unique \fg.
    \item $\in$ : \og Appartient à \fg.
    \item $\subseteq$ : \og Inclus dans  ou égal à \fg.
    \item $\subset$ : \og Strictement inclus dans \fg.
    \item $P \implies Q$ : \og Si $P$ alors $Q$ \fg.
    \item $P \iff Q$ : \og $P$ équivaut à $Q$ \fg. Autrement dit : \og $P$ si et seulement si $Q$ \fg.
    \item $x \coloneqq y$ : \og $x$ est défini par $y$ \fg. Pour les lecteurs informaticiens, \og $\coloneqq$ \fg se comporte comme le \og = \fg en programmation.
    \item $\equiv$ : Selon le contexte, il peut désigner plusieurs choses \cite{symbole_congru_wikipedia}:
    \begin{itemize}
        \item En arithmétique, il désigne une congruence sur des entiers.
        \item En logique, il désigne une équivalence.
        \item Sinon il désigne une \og identité \fg. C'est-à-dire une égalité qui est vraie quelque soient les valeurs des variables employées.
    \end{itemize}
    \item $\square$ : Quand il est utilisé à la fin d'une démonstration, il signifie : \og Ce qu'il fallait démontrer \fg.
    \item $\llbracket a, b \rrbracket$ : Désigne l'intervalle d'entiers entre $a$ et $b$ inclus.
    \item $[a, b]$ : Désigne l'intervalle de réels entre $a$ et $b$ inclus.
\end{itemize}


\part{Analyse}

\chapter{Nombres réels}\label{chap:nb_reels}

\begin{definition}[Nombre réel \cite{wikipedia_nb_reel}]
    Un nombre réel est un nombre qui peut être représenté par une partie entière et une liste finie ou infinie de décimales.
\end{definition}

\begin{proposition}[Addition et multiplication sur $\R$]
    On peut définir sur $\R$ une addition (notée \og + \fg) et une multiplication (notée \og $\times$ \fg ou \og $\cdot$ \fg) qui prolonge l'addition et la multiplication de $\N$ et vérifie les règles suivantes pour $(a, b, c) \in \R^3$ :
    \begin{enumerate}
        \item Commutativité : 
        \[ a + b = b + a \text{ et } a \cdot b = b \cdot a \]
        \item Associativité : 
        \[ a + (b + c) = (a + b) + c \]
        \[ a \cdot (b \cdot c) = (a \cdot b) \cdot c \]
        \item Distributivité : 
        \[ a \cdot (b + c) = a \cdot b + a \cdot b \]
        \item \'Elements neutres ou absorbants :
        \[ a + 0 = a \]
        \[ a \cdot 1 = a \]
        \[a \cdot 0 = 0 \]
    \end{enumerate}
\end{proposition}

\begin{proposition}[Relation d'ordre sur $\R$]
    On peut définir une relation d'ordre sur $\R$, notée \og $\leq$ \fg, qui prolonge l'ordre de $\N$ et vérifie les règles suivantes pour $(a, b, c) \in \R^3$ :
    \begin{enumerate}
        \item Réflexivité : 
        \[ a \leq a \]
        \item Antisymétrie : 
        \[ a \leq b \land b \leq a \implies a = b \]
        \item Transitivité : 
        \[ a \leq b \land b \leq c \implies a \leq c \]
        \item Ordre total : 
        \[ a \leq b \lor b \leq a \]
        \item Compatibilité avec l'addition :
        \[ a + c \leq b + c \]
        \item Compatibilité avec la multiplication par un réel positif :
        \[ a \leq b \land c \geq 0 \implies a \cdot c \leq b \cdot c \]
    \end{enumerate}
\end{proposition}

\begin{definition}[Valeur absolue]
    Soit $x \in \R$, on définit la valeur absolue ainsi :
    \begin{align*}
        \abs{x} =
        \begin{cases}
            x &\text{ si } x \geq 0 \\
            -x &\text{ si } x < 0
        \end{cases}
    \end{align*}
\end{definition}

\begin{proposition}
    $\forall (a, b) \in \R^2$.
    \begin{enumerate}
            \item $\abs{a + b} \leq \abs{a} + \abs{b}$
            \item $\abs{a \cdot b} = \abs{a} \cdot \abs{b}$
            \item $\abs{a - b} \geq \abs{a} - \abs{b}$
            \item $\abs{a} = \sqrt{a^2}$
        \end{enumerate}
\end{proposition}

\begin{definition}[Intervalle]
    Un intervalle $I$ est une partie de $\R$ tel que :
    \begin{align*}
        \forall (x, y) \in I^2 \implies \forall z \in I \text{ et } z \in \R,\ x \leq z \leq y
    \end{align*}
\end{definition}

\begin{definition}
    Soient $A$ une partie de $\R$ et $m \in \R$.
    \begin{enumerate}
        \item On dit que $m$ est un \textbf{majorant} de $A$ si et seulement si : $A \iff \forall x \in A,\ x \leq m$.
        \item On dit que $m$ est un \textbf{minorant} de $A$ si et seulement si : $\forall x \in A,\ x \geq m$.
    \end{enumerate}
    On dit que $A$ est \textbf{majorée} si elle admet un \textbf{majorant}, \textbf{minorée} si elle admet un \textbf{minorant} et \textbf{bornée} si elle est \textbf{majorée} et \textbf{minorée}.
\end{definition}

\begin{theorem}
    Soit $A$ une partie non-vide de $\R$. \\
    Si $A$ est \textbf{majorée}, elle admet un \textbf{plus petit majorant} appelé la \textbf{borne supérieure} de $A$, notée : $\sup(A)$.
    \\
    Si $A$ est \textbf{minorée}, elle admet un \textbf{plus grand minorant} appelé la \textbf{borne inférieure} de $A$, notée : $\inf(A)$.
\end{theorem}

\begin{proposition}
    Soient $A$ une partie de $\R$ non-vide, $M$ un majorant de $A$ et $m$ un minorant de $A$.  
    \begin{enumerate}
        \item $M = \sup(A) \iff \forall \varepsilon > 0,\ ]M - \varepsilon,\ M] \cap A \neq \varnothing$.
        \item $m = \inf(A) \iff \forall \varepsilon > 0,\ [m, m + \varepsilon[ \cap A \neq \varnothing $
    \end{enumerate}
\end{proposition}
\chapter{Fonctions réelles}
\def\arraystretch{1}

\section{Définitions}
\begin{definition}[Fonction]
	Une fonction $f$ est la donnée de :
	\begin{enumerate}
		\item Un ensemble de départ $E$.
		\item Un ensemble d'arrivée $F$.
		\item Une flèche : $ f : E \to F $ à tout élément $x \in E$ associe un élément $f(x) \in F$.
	\end{enumerate}		
	On appelle \textbf{images} les éléments de $F$ et \textbf{antécédents} les éléments de $E$.
\end{definition}

\begin{definition}[Graphe d'une fonction]
	Soient $E$ et $F$ deux ensembles et $f : E \to F$ une fonction.
	\\
	Le graphe de $f$ est défini comme :
	\[ \operatorname{Gr}(f) = \operatorname{Gph}(f) = \{ (x, f(x)),\ x \in E \} \subseteq E \times F \] 
\end{definition}

\begin{definition}
	Soit $f : E \to F$ une fonction. On dit que :
    \begin{enumerate}
        \item $f$ est \textbf{injective} si et seulement si : 
        \[ \forall x_1, x_2 \in E : f(x_1) = f(x_2) \implies x_1 = x_2 \]
        \item $f$ est \textbf{surjective} si et seulement si : 
        \[ \forall y \in F,\ \exists x \in E : f(x) = y \]
        \item $f$ est \textbf{bijective} si et seulement si :
        \[ \forall y \in E,\ \exists ! x \in E : f(x) = y \]
    \end{enumerate}
\end{definition}

\begin{definition}[Bijection réciproque]
	Lorsque $f : E \to F$ est une bijection. On peut définir $f^{-1} : F \to E$ la bijection réciproque de $f$ qui associe à tout élément de $F$ son unique antécédent dans $E$.
\end{definition}

\begin{proposition}
	Soient $x \in E,\ y \in F$.
    \begin{multicols}{2}
        \begin{enumerate}
            \item $f^{-1} (f(x)) = x$
            \item $f(f^{-1}(y)) = y$
        \end{enumerate}
    \end{multicols}
\end{proposition}

\begin{definition}
	Soient $I \subseteq \R$, $f : I \to \R$ et $T \in \R_+$. On dit que :
    \begin{enumerate}
        \item $f$ est \textbf{paire} si et seulement si : 
        \[ \forall x \in I : f(x) = f(-x) \]
        \item $f$ est \textbf{impaire} si et seulement si  :
        \[ \forall x \in I : -f(x) = f(-x) \]
        \item $f$ est \textbf{$T$-périodique} si et seulement si :
        \[ \forall x \in \R,\ z \in \Z : f(x + nT) = f(x) \]
    \end{enumerate}
\end{definition}

\begin{definition}
	Soient $I \subseteq \R,\ f : I \to \R$. On dit que :
	\begin{enumerate}
	    \item $f$ est \textbf{majorée} ou qu'elle admet un \textbf{majorant} si et seulement si : 
	    \[ \exists M \in \R,\ \forall x \in I : f(x) \leq M \]
        \item $f$ est \textbf{minorée} ou qu'elle admet un \textbf{minorant} si et seulement si :
        \[ \exists m \in \R,\ \forall x \in I : f(x) \geq m \]
        \item $f$ est \textbf{bornée} si et seulement si elle est \textbf{majorée} et \textbf{minorée}.
	\end{enumerate}
\end{definition}

\begin{definition}
	Soient $I \subseteq \R$, $a \in \R$ et $f : I \to \R$. On dit que :
    \begin{enumerate}
        \item $f$ est \textbf{croissante} si et seulement si :
        \[ \forall x, y \in I : x \leq y \implies f(x) \leq f(y) \]
        \item $f$ est \textbf{décroissante} si et seulement si : 
        \[ \forall x, y \in I : x \leq y \implies f(x) \geq f(y) \]
        \item $f$ est \textbf{monotone} si et seulement si elle est \textbf{croissante} ou \textbf{décroissante}.
        \item $f$ est \textbf{strictement croissante} si et seulement si : 
        \[ \forall x, y \in I : x < y \implies f(x) < f(y) \]
        \item $f$ est \textbf{strictement décroissante} si et seulement si :
        \[ \forall x, y \in I : x < y \implies f(x) > f(y) \]
        \item $f$ est \textbf{strictement monotone} si et seulement si elle est \textbf{strictement croissante} ou \textbf{strictement décroissante}.
        \item $f$ est \textbf{constante} si et seulement si :
        \[ \forall x \in I : f(x) = a \]
    \end{enumerate}
\end{definition}

\section{Opérations sur les fonctions}
\begin{definition}[Opérations sur les fonctions]
	Soient $f, g$ deux fonctions et $A \subseteq \R$.
	\begin{center}
		$
		\begin{array}{cc}
			\appli{f+g}{A}{\R}{x}{f(x)+g(x)}
			&
			\appli{fg}{A}{\R}{x}{f(x)g(x)}
		\end{array}
		$
	\end{center}
	Si $g$ ne s'annule pas :
	\begin{center}
		$
		\appli{\frac{f}{g}}{A}{\R}{x}{\frac{f(x)}{g(x)}}
		$
	\end{center}
\end{definition}

\begin{definition}[Composition de fonctions]
	Soient $E, F, G, H \subseteq \R,\ F \subseteq G,\ f : E \to F,\ g : G \to H$.
	\begin{center}
		$
		\appli{g \circ f}{E}{H}{x}{g(f(x))}
		$
	\end{center}
\end{definition}

\begin{definition}[Fonction identité]
    Soient $E, F \subseteq \R$.
	\begin{center}
		$
		\appli{id_E}{E}{F}{x}{x}
		$
	\end{center}
\end{definition}

\begin{proposition}
	Soit $f : E \to F$ une bijection.
    \begin{multicols}{2}
        \begin{enumerate}
            \item $f^{-1} \circ f = id_E$.
            \item $f \circ f^{-1} = id_F$.
        \end{enumerate}
    \end{multicols}
\end{proposition}

\begin{definition}[Image directe]
    Soient $f : E \to F,\ A \subseteq E$.
	\[ f(A) = \{ f(x),\ x \in A \} : f(A) \subseteq F \]
\end{definition}

\begin{definition}[Image réciproque]
	Soient $f : E \to F,\ B \subseteq F$.
	\[ f^{-1}(B) = \{ \forall x \in E,\ f(x) \in B \} : f^{-1} \subseteq E \]
\end{definition}

\begin{proposition}
    Soient $f : E \to F,\ A_1, A_2 \subseteq E,\ B_1, B_2 \subseteq F$.
    \begin{enumerate}
        \item $f(A_1 \cup A_2) = f(A_1) \cup f(A_2)$.
        \item $f^{-1} (B_1 \cup B_2) = f^{-1} (B_1) \cup f^{-1} (B_2)$.
        \item $f^{-1} (B_1 \cap B_2) = f^{-1} (B_1) \cap f^{-1} (B_2)$.
    \end{enumerate}
\end{proposition} 


\chapter{Fonctions usuelles}


\begin{definition}[Fonction polynomiale]
	Soient $n \in \N,\ a_0, \ldots, a_n \in \R$.
	\begin{align*}
		f : \R &\to \R \\
            x &\mapsto \sum_{i = 0}^{n} a_i x^i
	\end{align*}
\end{definition}

\begin{definition}[Fonction partie entière]
	\[ \forall x \in \R,\ \exists ! E(x) \in \Z,\ E(x) \leq x < E(x + 1) \]
	\begin{align*}
		E : \R &\to \Z \\
        x &\mapsto E(x)
	\end{align*}
\end{definition}

\begin{definition}[Fonction puissance]
	Soit $a \in \R$.
	\begin{align*}
		f : \R_+^* &\to \R \\
        x &\mapsto x^a
	\end{align*}
\end{definition}

\begin{proposition}
	$\forall (a, b, x) \in \R^3$.
    \begin{multicols}{4}
        \begin{enumerate}
            \item $1^a = 1$.
            \item $x^a \cdot x^b = x^{a + b}$.
            \item $(xy)^a = x^a y^a$.
            \item $(x^a)^b = x^{ab}$.
        \end{enumerate}
    \end{multicols}
\end{proposition}

\section{Fonctions trigonométriques}
\noindent $\forall x \in \R$.
\begin{table}[!h]
	\centering
	\begin{tabular}{cccc}
		\toprule 
		$f$ & $x \mapsto \cos(x)$ & $x \mapsto \sin(x)$ & $x \mapsto \tan(x) = \frac{\sin(x)}{\cos(x)}$ \\
		\midrule
		Domaine de définition & $[0, \pi] \to [-1, 1]$ & $[-\frac{\pi}{2}, \frac{\pi}{2}] \to [-1, 1]$ & $]-\frac{\pi}{2}, \frac{\pi}{2}[ \to \R$ \\
		Parité & Paire & Impaire & Impaire \\
		Périodicité & $2\pi$ & $2\pi$ & $\pi$ \\
		\bottomrule
	\end{tabular}
	\caption{Fonctions trigonométriques}
\end{table}

\begin{proposition}\leavevmode
    \begin{enumerate}
        \item $\forall x \in ]-\frac{\pi}{2}, \frac{\pi}{2}[$ :
        \begin{multicols}{2}
            \begin{enumerate}
                \item $\arcsin(\sin(x)) = x$.
                \item $\arctan(\tan(x)) = x$.
            \end{enumerate}
        \end{multicols}
        \item $\forall x \in [0, \pi],\ \arccos(\cos(x)) = x$.
        \item $\forall x \in [-1, 1]$ :
        \begin{multicols}{2}
            \begin{enumerate}
                \item $\sin(\arcsin(x)) = x$.
                \item $\cos(\arccos(x)) = x$.
            \end{enumerate}
        \end{multicols}
        \item $\forall x \in \R,\ \tan(\arctan(x)) = x$.
    \end{enumerate}
\end{proposition}

\begin{proposition}
	$\forall (a, b) \in \R^2$.
    \begin{multicols}{2}
        \begin{enumerate}
            \item $\sin(a + b) = \sin(a) \cos(b) + \sin(b) \cos(a)$.
            \item $\cos(a + b) = \cos(a) \cos(b) - \sin(a) \sin(b)$.
        \end{enumerate}
    \end{multicols}
\end{proposition}

\begin{proof}
    Nous pouvons procéder avec des produits scalaires, mais nous allons utiliser les nombres complexes ici.
    \\
    D'une part :
    \[e^{i (a + b)} = \cos(a + b) + i\sin(a + b)\]
    D'autre part :
    \begin{align*}
        e^{i (a + b)} &= e^{ia} \cdot e^{ib} \\
        &= [\cos(a) + i \sin(a)] \cdot [\cos(b) + i \sin(b)] \\
        &= \cos(a) \cos(b) + i\sin(b)\cos(a) + i\sin(a)\cos(b) - \sin(a)\sin(b) \\
        &= \cos(a)\cos(b) - \sin(a)\sin(b) + i [\sin(b)\cos(a) + \sin(a) \cos(b)].
    \end{align*}
    Par identification de la partie réelle et de la partie imaginaire :
    \[ \sin(a + b) = \sin(a) \cos(b) + \sin(b) \cos(a) \]
	\[ \cos(a + b) = \cos(a) \cos(b) - \sin(a) \sin(b) \]
\end{proof}

\begin{proposition}
    $\forall x \in \R,\ \cos^2(x) + \sin^2(x) = 1$.
\end{proposition}

\begin{proof}
    C'est une application du théorème de Pythagore sachant que le rayon du cercle trigonométrique est égal à 1.
\end{proof}

\begin{proposition}
    \begin{enumerate}
        \item $\lim_{x \to +\infty} \arctan(x) = \frac{\pi}{2}$.
        \item $\lim_{x \to -\infty} \arctan(x) = -\frac{\pi}{2}$.
    \end{enumerate}
\end{proposition}

\section{Exponentielle et logarithme}

\begin{definition}[Fonction exponentielle]
	\begin{align*}
		\exp : \R &\to \R_+^* \\
        x &\mapsto \exp(x) \equiv e^x
	\end{align*}
\end{definition}

\begin{proposition}
	La fonction exponentielle est \emph{bijective} et \emph{strictement croissante} et $\exp(0) = 1$.
    \begin{multicols}{2}
        \begin{enumerate}
            \item $\lim_{x \to -\infty} e^x = 0$.
            \item $\lim_{x \to +\infty} e^x = +\infty$.
        \end{enumerate}
    \end{multicols}
    \noindent $\forall x, y \in \R$.
    \begin{multicols}{2}
        \begin{enumerate}
            \item $\exp(x + y) = \exp(x) \cdot \exp(y)$.
            \item $\exp(-x) = \frac{1}{\exp(x)}$.
            \item $\exp(x - y) = \frac{\exp(x)}{\exp(y)}$.
        \end{enumerate}
    \end{multicols}
\end{proposition}

\begin{definition}[Logarithme néperien]
	\begin{align*}
		\ln : \R_+^* &\to \R \\
        x &\mapsto \ln(x)
	\end{align*}
\end{definition}

\begin{proposition}
	$\forall x \in \R_+^*,\ \forall y \in \R$.
    \begin{multicols}{2}
        \begin{enumerate}
            \item $\exp(\ln(x)) = x$.
            \item $\ln(\exp(y)) = y$.
        \end{enumerate}
    \end{multicols}
\end{proposition}

\begin{proposition}
	La fonction logarithme néperien est \emph{bijective} et \emph{strictement croissante} et $\ln(1) = 0$.
    \begin{multicols}{2}
        \begin{enumerate}
            \item $\lim_{x \to 0} \ln(x) = -\infty$.
            \item $\lim_{x \to +\infty} \ln(x) = +\infty$.
        \end{enumerate}
    \end{multicols}
    \noindent $\forall x, y \in \R$.
    \begin{multicols}{2}
        \begin{enumerate}
            \item $\ln(xy) = \ln(x) + \ln(y)$.
            \item $\ln(\frac{1}{x}) = -\ln(x)$.
            \item $\ln(\frac{x}{y}) = \ln(x) - \ln(y)$.
            \item $\ln(x^n) = n\ln(x)$.
        \end{enumerate}
    \end{multicols}
\end{proposition}

\section{Fonctions hyperboliques}
\noindent $\forall x \in \R$
\begin{table}[!ht]
	\centering
	\begin{tabular}{cccc}
		\toprule
		Fonction & $x \mapsto \cosh(x)$ & $x \mapsto \sinh(x)$ & $x \mapsto \tanh(x)$ \\
		\midrule
		Domaine de définition & $\R$ & $\R$ & $\R$ \\
		Parité & Paire & Impaire & Impaire \\
		Définition & $\frac{e^x + e^{-x}}{2}$ & $\frac{e^x - e^{-x}}{2}$ & $\frac{\sinh(x)}{\cosh(x)} = \frac{e^x - e^{-x}}{e^x + e^{-x}}$ \\
		\bottomrule
	\end{tabular}
	\caption{Fonctions hyperboliques}
\end{table}

\begin{proposition}
	$\forall (x, y) \in \R^2$.
    \begin{enumerate}
        \item $\cosh(x + y) = \cosh(x) \cosh(y) + \sinh(x) \sinh(y)$.
        \item $\sinh(x + y) = \cosh(x) \sinh(y) + \sinh(x) \cosh(y)$.
    \end{enumerate}
\end{proposition}

\begin{proof}
    Calcul direct avec les définitions de $\cosh$ et de $\sinh$.
\end{proof}

\chapter{Suites réelles}
\begin{definition}[Suite réelle]
    On appelle \textbf{suite réelle} une fonction de $\N \to \R$. 
    On note $(u_n)_{n \in \N}$ la fonction $x \mapsto u_n$.
\end{definition}

\begin{remark}
    Ainsi les propriétés des fonctions réelles s'appliquent également aux suites.
\end{remark}

\section{Suites usuelles}
\begin{definition}[Suite arithmétique]
    $\forall (r, u_0) \in \R^2$. 
    \\
    On définit une \textbf{suite arithmétique} $(u_n)_{n \in \N}$ de la manière suivante :
    \begin{align*}
        (u_n)_{n \in \N} \coloneqq
        \begin{cases}
            u_{n+1} = u_n + r \\ 
            u_n = u_0 + nr
        \end{cases}
    \end{align*}
\end{definition}

\begin{proposition}
    Soit $(u_n)_{n \in \N}$ une suite arithmétique.
    \begin{align*}
        \sum_{k = 0}^{n} u_k = (n+1)u_0 + r \frac{n(n+1)}{2}
    \end{align*}
\end{proposition}

\begin{definition}[Suite géométrique]
    $\forall (q, u_0) \in \R^* \times \R$. 
    \\
    On définit une \textbf{suite géométrique} $(u_n)_{n \in \N}$ de la manière suivante : 
    \begin{align*}
        (u_n)_{n \in \N} \coloneqq 
        \begin{cases}
            u_{n+1} = qu_n \\
            u_n = u_0 q^n
        \end{cases}
    \end{align*}
\end{definition}

\begin{proposition}
    Soit $(u_n)_{n \in \N}$ une suite géométrique.
    \begin{align*}
        \sum_{k = 0}^{n} u_k = 
        \begin{cases}
            u_0 \frac{1 - q^{n+1}}{1 - q} \text{ si } q \neq 1\\
            u_0(n+1) \text{ si } q = 1
        \end{cases}
    \end{align*}
\end{proposition}

\begin{definition}[Suite arithmético-géométrique]
    $\forall (r, u_0, q) \in \R^2 \times \R^*$. 
    \\
    On définit une \textbf{suite arithmético-géométrique} $(u_n)_{n \in \N}$ de la manière suivante :
    \begin{align*}
        (u_n)_{n \in \N} \coloneqq 
        \begin{cases}
            u_{n+1} = qu_n + r \\ 
            u_n = a + (u_0 - a) q^n,\ a = \frac{r}{1 - q}
        \end{cases}
    \end{align*}
\end{definition}

En pratique, pour l'étude des suites arithmético-géométrique, on commence par résoudre $a = qa + r \iff a = \frac{r}{1 - q}$, puis on pose $v_n = u_n - a$ et $v_{n+1} = u_{n+1} - a$ qui est une suite géométrique, ainsi $v_n = v_0 q^n$ et finalement $u_n = v_n + a \iff u_n = v_0 q^n + a \iff u_n = (u_0 - a) q^n + a$.

\section{Convergence d'une suite}
\begin{definition}
    Soient $(u_n)_{n \in \N}$ une suite et $\ell \in \R$. 
    \\
    On dit que $(u_n)_{n \in \N}$ \textbf{tend} vers $\ell$ si et seulement si :
    \[ \forall \varepsilon > 0,\ \exists N \in \N,\ \forall n \geq N,\ \abs{u_n - \ell} \leq \varepsilon \]
    On note alors :
    \begin{align*}
        u_n \xrightarrow[n \to +\infty]{} \ell \text{ ou } \lim_{n \to +\infty} u_n = \ell
    \end{align*}
\end{definition}

\begin{definition}
    Soit $(u_n)_{n \in \N}$ une suite réelle. On dit que :
    \begin{enumerate}
        \item $(u_n)_{n \in \N}$ \textbf{converge} si et seulement si : 
        \[ \exists \ell \in \R,\ \varepsilon > 0,\ \exists N \in \N,\ \abs{u_n - \ell} \leq \varepsilon \]
        \item $(u_n)_{n \in \N}$ \textbf{diverge} si et seulement si : 
        \[ \forall \ell \in \R,\ \varepsilon > 0,\ \forall N \in \N,\ \exists n \geq N,\ \abs{u_n - \ell} > \varepsilon \]
        \item $(u_n)_{n \in \N}$ \textbf{ne converge pas vers} $\ell \in \R$ si et seulement si : 
        \[ \exists \varepsilon > 0,\ \forall N \in \N,\ \exists n \geq N,\ \abs{u_n - \ell} > \varepsilon \]
    \end{enumerate}
\end{definition}

\begin{theorem}
    La limite d'une suite convergente $(u_n)_{n \in \N}$ est unique.
\end{theorem}

\begin{proof}
    Procédons à un raisonnement par l'absurde.
    \\
    On suppose que $\ell_1 \neq \ell_2$. Posons $\varepsilon \coloneqq \frac{1}{3} \abs{\ell_1 - \ell_2} > 0$.
    \\
    Par définition de la limite :
    \begin{align*}
        &\exists N_1 \in \N,\ \forall n \geq N_1,\ \abs{u_n - \ell_1} \leq \varepsilon 
        &
        &\exists N_2 \in \N,\ \forall n \geq N_2,\ \abs{u_n - \ell_2} \leq \varepsilon
    \end{align*}
    Posons $N \coloneqq \max(N_1, N_2)$, si $n \geq N$ alors :
    \begin{align*}
        \abs{u_n - \ell_1} \leq \varepsilon \text{ et } \abs{u_n - \ell_2} \leq \varepsilon
    \end{align*}
    \[ \abs{\ell_1 - \ell_2} = \abs{\ell_1 - u_n + u_n - \ell_2} \leq \abs{u_n - \ell_1} + \abs{u_n - \ell_2} \]
    Alors : 
    \begin{align*}
        \abs{u_n - \ell_1} + \abs{u_n - \ell_2} &\leq \varepsilon + \varepsilon = \frac{2}{3} \abs{\ell_1 - \ell_2} \\
        \abs{\ell_1 - \ell_2} &\leq \frac{2}{3} \abs{\ell_1 - \ell_2} \\
        \frac{1}{3} \abs{\ell_1 - \ell_2} \leq 0 \\
        \varepsilon \leq 0
    \end{align*}
    ce qui est absurde. Ainsi on a montré que $\ell_1 = \ell_2$.
\end{proof}

\begin{theorem}
    Toute suite convergente est bornée.
\end{theorem}

\begin{proof}
    Supposons qu'une suite $(u_n)_{n \in \N}$ converge vers $\ell \in \R$.
    Posons $\varepsilon \coloneqq 1$. 
    \\
    Par définition de la limite :
    \[ \exists N \in \N,\ \forall n \geq N,\ \abs{u_n - \ell} \leq 1 \iff \ell - 1 \leq u_n \leq \ell + 1 \]
    Posons $M \coloneqq \max(u_0, \ldots, u_{N-1}, \ell + 1)$ et $m = \min(u_0, \ldots, u_{N - 1}, \ell - 1)$.
    \begin{align*}
        \forall n \in \N,\ 
        \begin{cases}
            m \leq u_n \leq M, &\text{ si } n < N \\
            \ell - 1 \leq u_n \leq \ell + 1, &\text{ si } n > N
        \end{cases}
    \end{align*}
    Sachant que $m \leq \ell - 1$ et $M \geq \ell + 1$.
    On a :
    \[ \forall n \in \N,\ m \leq u_n \leq M \]
    ce qui signifie que $(u_n)_{n \in \N}$ est bornée.
\end{proof}

\begin{theorem}
    Soient $(\ell_1, \ell_2) \in \R^2$, $(u_n)_{n \in \N}$ et $(v_n)_{n \in \N}$ deux suites convergentes telles que :
    \[ \lim_{n \to +\infty} u_n = \ell_1 \text{ et } \lim_{n \to +\infty} v_n = \ell_2 \]
    Alors :
    \[ \lim_{n \to +\infty} (u_n + v_n) = \ell_1 + \ell_2 \]
\end{theorem}

\begin{proof}
    Soit $\varepsilon > 0$.
    \\
    \\
    Par définition de la limite :
    \begin{align*}
        &\exists N_1 \in \N,\ \forall n \geq N_1,\ \abs{u_n - \ell_1} \leq \varepsilon &
        &\exists N_2 \in \N,\ \forall n \geq N_2,\ \abs{v_n - \ell_2} \leq \varepsilon
    \end{align*}
    Posons $N \coloneqq \max(N_1, N_2)$. Si $n \geq N$, alors :
    \begin{align*}
        \abs{u_n - \ell_1} \leq \varepsilon \text{ et } \abs{v_n - \ell_2} \leq \varepsilon
    \end{align*}
    Puis : 
    \begin{align*}
        \abs{u_n + v_n - (\ell_1 + \ell_2)} = \abs{u_n - \ell_1 + v_n - \ell_2} &\leq \abs{u_n - \ell_1} + \abs{v_n - \ell_2} \\
        \abs{u_n - \ell_1} + \abs{v_n - \ell_2} &\leq \varepsilon + \varepsilon = 2 \varepsilon
    \end{align*}
    Posons $\varepsilon' \coloneqq 2\varepsilon$.
    Ainsi : 
    \[ \abs{u_n - \ell_1} + \abs{v_n - \ell_2} \leq \varepsilon' \]
    C'est-à-dire que $\lim_{n \to +\infty} (u_n + v_n) = \ell_1 + \ell_2$.
\end{proof}

\begin{theorem}
    Soient $\ell_1, \ell_2 \in \R$, $(u_n)_{n \in \N}$ et $(v_n)_{n \in \N}$ deux suites convergentes telles que :
    \[ \lim_{n \to + \infty} u_n = \ell_1 \text{ et } \lim_{n \to +\infty} v_n = \ell_2 \]
    Alors :
    \[ \lim_{n \to +\infty} (u_n \cdot v_n) = \ell_1 \cdot \ell_2 \]
\end{theorem}

\begin{proof}
    Soit $\varepsilon > 0$.
    \\
    Comme $(u_n)_{n \in \N}$ converge, elle est bornée :
    \[ \exists M \in \R, \forall n \in \N,\ \abs{u_n} \leq M \]
    Par définition de la limite :
    \begin{align*}
        &\exists N_1,\ \forall n \geq N_1,\ \abs{u_n - \ell_1} \leq \varepsilon &
        &\exists N_2,\ \forall n \geq N_2,\ \abs{v_n - \ell_2} \leq \varepsilon
    \end{align*}
    Posons $N \coloneqq \max(N_1, N_2)$. Si $n \geq N$ alors :
    \[ \abs{u_n - \ell_1} \leq \varepsilon \text{ et } \abs{v_n - \ell_2} \leq \varepsilon \]
    Puis : 
    \begin{align*}
        \abs{u_n \cdot v_n - \ell_1 \cdot \ell_2} &= \abs{u_n \cdot v_n - u_n \cdot \ell_2 + u_n \cdot \ell_2 - \ell_1 \cdot \ell_2} \\
        &= \abs{u_n(v_n - \ell_2) + \ell_2 (u_n - \ell_1)} \\ 
        &\leq \abs{u_n} \abs{v_n - \ell_2} + \abs{\ell_2} \abs{u_n - \ell_1} \\ 
        &\leq M \varepsilon + \abs{\ell_2} \varepsilon = (M + \abs{\ell_2}) \varepsilon
    \end{align*}
    Posons $\varepsilon' \coloneqq \varepsilon$. Ainsi :
    \[ \abs{u_n \cdot v_n - \ell_1 \cdot \ell_2} = \varepsilon' \]
    C'est-à-dire que $\lim_{n \to +\infty} (u_n \cdot v_n) = \ell_1 \cdot \ell_2$.
\end{proof}

\begin{theorem}
    Soient $(u_n)_{n \in \N}$ et $(v_n)_{n \in \N}$ deux suites convergentes telles que $\forall n \in \N, u_n \leq v_n$. 
    \[ \lim_{n \to +\infty} u_n \leq \lim_{n \to +\infty} v_n \]
\end{theorem}

\begin{proof}
    Soient $\ell_1, \ell_2 \in \R$.
    Posons :
    \begin{align*}
        \ell_1 &\coloneqq \lim_{n \to +\infty} u_n & \ell_2 &\coloneqq \lim_{n \to +\infty} v_n
    \end{align*}
    On raisonne par l'absurde en supposant que $\ell_1 > \ell_2$.
    \\
    Posons : 
    \[ \varepsilon \coloneqq \frac{\ell_1 - \ell_2}{3} > 0 \]
    \begin{align*}
        &\exists N_1,\ \forall n \geq N_1,\ \abs{u_n - \ell_1} \leq \varepsilon & 
        &\exists N_2,\ \forall n \geq N_2,\ \abs{v_n - \ell_2} \leq \varepsilon
    \end{align*}
    Autrement dit :
    \begin{align*}
        &\forall n \geq N_1,\ u_n \geq \ell_1 - \varepsilon &
        &\forall n \geq N_2,\ v_n \leq \ell_2 + \varepsilon
    \end{align*}
    Posons $N \coloneqq \max(N_1, N_2)$. Si $n \geq N$ alors :
    \begin{align*}
        v_n \leq \ell_2 + \varepsilon < \ell_1 - \varepsilon \leq u_n \implies v_n < u_n
    \end{align*}
    ce qui est absurde. Ainsi $\ell_1 \leq \ell_2$.
\end{proof}

\begin{corollary}
    Soient $(u_n)_{n \in \N}$ et $\ell \in \R$ tels que $\lim_{n \to +\infty} = \ell$.
    \begin{multicols}{2}
        \begin{enumerate}
            \item Si $\forall n \in \N,\ u_n \leq M$ alors $\ell \leq M$.
            \item Si $\forall n \in \N,\ u_n \geq m$ alors $\ell \geq m$.
        \end{enumerate}
    \end{multicols}
\end{corollary}

\begin{theorem}[Théorème des gendarmes]
    Soient $\ell \in \R$, $(u_n)_{n \in \N}, (v_n)_{n \in \N}, (w_n)_{n \in \N}$ des suites telles que :
    \begin{multicols}{2}
        \begin{enumerate}
            \item $\forall n \in \N,\ u_n \leq v_n \leq w_n$.
            \item $\lim_{n \to +\infty} u_n = \ell$ et $\lim_{n \to +\infty} w_n = \ell$.
        \end{enumerate}
    \end{multicols} 
    \[ \lim_{n \to + \infty} v_n = \ell \]
\end{theorem}

\begin{proof}
    \begin{align*}
        &\exists N_1 \in \R,\ \forall n \geq N_1,\ \abs{u_n - \ell} \geq \varepsilon &
        &\exists N_2 \in \R,\ \forall n \geq N_2,\ \abs{w_n - \ell} \geq \varepsilon
    \end{align*}
    Posons $N \coloneqq \max(N_1, N_2)$. Si $n \geq N$ alors :
    \begin{align*}
        \abs{u_n - \ell} \leq \varepsilon \text{ et } \abs{w_n - \ell} \leq \varepsilon
    \end{align*}
    ce qui revient à dire : 
    \begin{align*}
        &\ell - \varepsilon \leq u_n - \ell \leq \ell + \varepsilon & &\ell - \varepsilon \leq w_n - \ell \leq \ell + \varepsilon
    \end{align*}
    Sachant que :
    \[ u_n \leq v_n \leq w_n \]
    on a :
    \begin{align*}
        \ell - \varepsilon \leq u_n - \ell \leq v_n - \ell \leq w_n - \ell \leq \ell + \varepsilon
    \end{align*}
    et donc finalement :
    \[ \ell - \varepsilon \leq v_n - \ell \leq \ell + \varepsilon \iff \abs{v_n - \ell} \leq \varepsilon \]
    C'est-à-dire $\lim_{n \to +\infty} v_n = \ell$.
\end{proof}

\begin{theorem}
    \begin{enumerate}
        \item Toute suite croissante majorée converge.
        \item Toute suite décroissante minorée converge.
    \end{enumerate}
\end{theorem}

\begin{theorem}[Théorème des suites adjacentes]
    Soient $(u_n)_{n \in \N}$ et $(v_n)_{n \in \N}$ deux suites telles que :
    \begin{multicols}{3}
        \begin{enumerate}
            \item $(u_n)_{n \in \N}$ est croissante.
            \item $(v_n)_{n \in \N}$ est décroissante.
            \item $\lim_{n \to +\infty} (v_n - u_n) = 0$
        \end{enumerate}
    \end{multicols}
    \noindent Alors $(u_n)_{n \in \N}$ et $(v_n)_{n \in \N}$ convergent vers la même limite.
\end{theorem}

\begin{proof}
    Posons $w_n \coloneqq v_n - u_n$.
    \\ 
    On sait que $(v_n)_{n \in \N}$ est décroissante et que $(u_n)_{n \in \N}$ est croissante. 
    \\ 
    Ainsi $v_{n+1} - v_n \leq 0$ et $u_{n+1} - u_n \geq 0$.
    Etudions la variation de $(w_n)_{n \in \N}$.
    \begin{align*}
        w_{n+1} - w_n &= v_{n+1} - v_n - (u_{n+1} - u_n) < 0
    \end{align*}
    Ainsi $(w_n)_{n \in \N}$ est décroissante et sa limite est $0$.
    On a alors  :
    \[ w_n \geq 0 \iff  v_n - u_n \geq 0 \iff v_n \geq u_n \]
    D'après les monotonies de $(v_n)_{n \in \N}$ et $(u_n)_{n \in \N}$, on a l'encadrement suivant : 
    \[ u_0 \leq u_n \leq v_n \leq v_0 \]
    $(u_n)_{n \in \N}$ est majorée par $v_0$ et est croissante, donc elle converge vers une limite $\ell_1$.
    \\ 
    $(v_n)_{n \in \N}$ est minorée par $u_0$ et est croissante, donc elle converge vers une limite $\ell_2$.
    D'une part :
    \[ \lim_{n \to +\infty} w_n = 0 \]
    D'autre part :
    \[ \lim_{n \to +\infty} w_n = \lim_{n \to +\infty} (v_n - u_n) = \ell_2 - \ell_1 \]
    Donc :
    \[ \ell_2 - \ell_1 = 0 \iff \ell_2 = \ell_1 \]
\end{proof}

\section{Suites extraites}
\begin{definition}[Extraction]
    Une extraction est une fonction $\varphi : \N \to \N$ qui est strictement croissante.
\end{definition}

\begin{definition}[Suite extraite]
    Une suite extraite ou une sous-suite d'une suite $(u_n)_{n \in \N}$ est une suite de la forme $(u_{\varphi(n)})_{n \in \N}$ où $\varphi$ est une extraction.
\end{definition}

\begin{proposition}
    Soit $(u_n)_{n \in \N}$ une suite et $(u_{\varphi(n)})_{n \in \N}$ une de ses sous-suites.
    \begin{itemize}
        \item Si $(u_n)_{n \in \N}$ est croissante, alors $(u_{\varphi(n)})_{n \in \N}$ aussi.
        \item Si $(u_n)_{n \in \N}$ est décroissante, alors $(u_{\varphi(n)})_{n \in \N}$ aussi.
        \item Si $(u_n)_{n \in \N}$ est majorée, alors $(u_{\varphi(n)})_{n \in \N}$ aussi.
        \item Si $(u_n)_{n \in \N}$ est minorée, alors $(u_{\varphi(n)})_{n \in \N}$ aussi.
        \item Si $(u_n)_{n \in \N}$ est converge, alors $(u_{\varphi(n)})_{n \in \N}$ aussi.
    \end{itemize}
\end{proposition}

\begin{proposition}
    Soit $(u_n)_{n \in \N}$ une suite, alors :
    \[ (u_n)_{n \in \N} \text{ converge} \iff (u_{2n})_{n \in \N} \text{ et } (u_{2n + 1})_{n \in \N} \text{ convergent vers la même limite} \]
\end{proposition}

\begin{theorem}[Théorème de Ramsey]
    Toute suite admet une sous-suite monotone.
\end{theorem}

\begin{proof}
    Soit $(u_n)_{n \in \N}$ une suite. Soit $E = \{ n \in \N,\ \forall m \geq n,\ u_m \leq u_n \}$.
    \\ 
    \textbf{Cas 1 :} $E$ est fini, donc majoré par un entier $N,\ \forall n \leq N,\ n \notin E$ donc $\exists m > n,\ u_m > u_n$. On définit alors par récurrence une extraction $\varphi : \N \to \N$ en posant $\varphi(0) = N + 1$, puis, étant donnés $\varphi(0) < \varphi(1) < \cdots < \varphi(K)$, on choisit $\varphi(K + 1)$ tel que $u_{\varphi(K+1)} > u_{\varphi(K)}$ et la suite extraite $(u_{\varphi(n)})_{n \in \N}$ est croissante.
    \\ 
    \textbf{Cas 2 :} $E$ est infini. On pose $E = \{ \varphi(n) : n \in \N \}$ avec $\varphi : \N \to \N$.
    \[ \forall k \in \N,\ \varphi(k) \in E, \text{ comme } \varphi(K + 1) > \varphi(K),\ u_{\varphi(K+1)} \leq u_{\varphi(K)} \]
    et la sous-suite $(u_{\varphi(n)})_{n \in \N}$ est décroissante.
\end{proof}

\begin{theorem}[Théorème de Bolzano-Weierstrass]
    Toute suite bornée admet une sous-suite convergente.
\end{theorem}

\begin{proof}
    Soit $(u_n)_{n \in \N}$ une suite bornée. D'après le théorème de Ramsey, il existe une sous-suite monotone $(u_{\varphi(n)})_{n \in \N}$. Comme $(u_{\varphi(n)})_{n \in \N}$ est monotone et bornée, alors elle converge.
\end{proof}

\section{Limites infinies}

\begin{definition}[Limites infinies]
    Soit $(u_n)_{n \in \N}$ une suite.
    \begin{enumerate}
        \item $\lim_{n \to +\infty} u_n = +\infty \iff \forall A \in \R,\ \exists N \in \N,\ \forall n \geq N,\ u_n \geq A$.
        \item $\lim_{n \to -\infty} u_n = -\infty \iff \forall A \in \R,\ \exists N \in \N,\ \forall n \geq N,\ u_n \leq A$.
    \end{enumerate}
\end{definition}

\begin{theorem}
    Soit $(u_n)_{n \in \N}$ une suite. \\
    \begin{enumerate}
        \item Si elle est \textbf{croissante} alors :
        \begin{multicols}{2}
            \begin{itemize}
                \item ou bien elle converge.
                \item ou bien elle tend vers $+\infty$.
            \end{itemize}
        \end{multicols}
        \item Si elle est \textbf{décroissante} alors :
        \begin{multicols}{2}
            \begin{itemize}
                \item ou bien elle diverge.
                \item ou bien elle tend vers $-\infty$.
            \end{itemize}
        \end{multicols}
    \end{enumerate}
\end{theorem}

\begin{proof}
    Démontrons les propriétés si $(u_n)_{n \in \N}$ est croissante. \\ 
    On distingue deux cas :
    \begin{enumerate}
        \item Si $(u_n)$ est majorée, elle converge, d'après le théorème de convergence monotone car elle est croissante et majorée.
        \item Si $(u_n)$ n'est pas majorée, montrons qu'elle tend vers $+\infty$.
        Soit $A$ un réel. Comme $(u_n)_{n \in \N}$ n'est pas majorée :
        \[ \exists N \in \N,\ u_N \geq A \]
        \[ \forall n \geq N,\ u_n \geq u_N \geq A \]
    \end{enumerate}
    On utilise un raisonnement analogue si $(u_n)_{n \in \N}$ est décroissante.
\end{proof}

\begin{theorem}[Limites par comparaison]
   Soient $(u_n)_{n \in \N}$ et $(v_n)_{n \in \N}$ deux suites telles que $u_n \leq v_n$.
   \[ \lim_{n \to +\infty} u_n = +\infty \implies \lim_{n \to +\infty} v_n = +\infty \]
   \[ \lim_{n \to +\infty} v_n = -\infty \implies \lim_{n \to +\infty} u_n = -\infty \]
\end{theorem}

\begin{table}[!h]
    \centering
    \begin{tabular}{cc}
         \toprule
         Hypothèses & Conclusion \\ 
         \midrule
         \og $+\infty +\infty$ \fg & $+\infty$ \\ 
         \og $-\infty -\infty$ \fg & $-\infty$ \\
         \og $\pm \infty + \ell$ \fg & $\pm \infty$ \\
         \og $-\infty \cdot \ell > 0$ \fg & $-\infty$ \\ 
         \og $-\infty \cdot \ell < 0$ \fg & $+\infty$ \\ 
         \og $+\infty \cdot \ell > 0$ \fg & $+\infty$ \\ 
         \og $+\infty \cdot \ell < 0$ \fg & $-\infty$ \\ 
         \og $+\infty \cdot +\infty$ \fg & $+\infty$ \\ 
         \og $-\infty \cdot -\infty$ \fg & $+\infty$ \\ 
         \og $-\infty \cdot +\infty$ \fg & $-\infty$ \\  
         \og $\infty - \infty$ \fg & FI \\
         \og $0 \cdot \infty$ \fg  & FI \\
         \og $\frac{0}{0}$ \fg & FI \\ 
         \og $\frac{\infty}{\infty}$ \fg & FI \\
         \bottomrule
    \end{tabular}
    \caption{Limites infinies ($\ell \in \R$) et formes indéterminées}
    \label{tab:limites_infinies_et_fi}
\end{table}

\chapter{Continuité et limites de fonctions}
\def\arraystretch{1}

\begin{definition}[Limite d'une fonction]
	
    \begin{enumerate}
        \item En un point $a \in \R$, $\ell \in \R$ :
        \begin{enumerate}
            \item $ \lim_{x \to a} f(x) = \ell \iff 
	\forall \varepsilon > 0,\ \exists \delta > 0,\ \abs{x - a} \leq \delta \implies \abs{f(x) - \ell} \leq \varepsilon $.
            \item $ \lim_{x \to a} f(x) = +\infty \iff 
	\forall A \in \R,\ \exists \delta > 0,\ \abs{x - a} \leq \delta \implies f(x) \geq A $.
            \item $ \lim_{x \to a} f(x) = -\infty \iff 
	\forall A \in \R,\ \exists \delta > 0,\ \abs{x - a} \leq \delta \implies f(x) \leq A $.
        \end{enumerate}
        \item En l'infini, $\ell \in \R$ : 
        \begin{enumerate}
            \item $ \lim_{x \to +\infty} f(x) = \ell \iff 
	\forall \varepsilon > 0,\ \exists A \in \R,\ x \geq A \implies \abs{f(x) - \ell} < \varepsilon $.
            \item $ \lim_{x \to -\infty} f(x) = \ell \iff 
	\forall \varepsilon > 0,\ \exists A \in \R,\ x \leq A \implies \abs{f(x) - \ell} \leq \varepsilon $.
            \item $\lim_{x \to +\infty} f(x) = +\infty \iff 
	\forall A \in \R,\ \exists B \in \R,\ x \geq B \implies f(x) \geq A $.
            \item $ \lim_{x \to +\infty} f(x) = -\infty \iff 
	\forall A \in \R,\ \exists B \in \R,\ x \geq B \implies f(x) \leq A $.
            \item $ \lim_{x \to -\infty} f(x) = +\infty \iff 
	\forall A \in \R,\ \exists B \in \R,\ x \leq B \implies f(x) \geq A $.
            \item $\lim_{x \to -\infty} f(x) = -\infty \iff 
	\forall A \in \R,\ \exists B \in \R,\ x \leq B \implies f(x) \leq A$.
        \end{enumerate}
    \end{enumerate}
\end{definition}

\begin{theorem}
	Soit $I \subseteq \R,\ f : I \to \R,\ a \in I \cup \{-\infty, +\infty\},\ \ell \in \R \cup \{-\infty, +\infty\}$
	\[ \lim_{x \to a} f(x) = \ell \iff \forall (u_n)_{n \in \N} : \lim_{n \to +\infty} u_n = a \implies \lim_{n \to +\infty} f(u_n) = \ell \]
\end{theorem}

\begin{definition}[Limite à gauche et à droite]
    Soient $I \subseteq \R,\ a \in I,\ f : I \to \R $.
    \begin{enumerate}
        \item $ \lim_{\substack{x \to a \\ x < a}} f(x) \equiv \lim_{x \to a^-} f(x) = \ell \iff \forall \varepsilon > 0,\ \exists \delta > 0,\ a - \delta < x < a \implies \abs{f(x) - \ell} \leq \varepsilon $.
        \item $\lim_{\substack{x \to a \\ x > a}} f(x) \equiv \lim_{x \to a^+} f(x) = \ell \iff \forall \varepsilon > 0,\ \exists \delta > 0,\ a < x < a + \delta \implies \abs{f(x) - \ell} \leq \varepsilon$.
    \end{enumerate}
\end{definition}

\begin{definition}[Continuité]
	Soient $I$ un intervalle, $a \in I,\ f : I \to \R$.
        \\ 
        On dit que $f$ est \textbf{continue} si et seulement si :
        \[\lim_{x \to a} f(x) = f(a).\] 
	On dit que $f$ est continue sur $I$ si elle est continue en tout point de $I$.\\
	On peut également définir la continuité à gauche et à droite.
\end{definition}

\begin{remark}
	Les théorèmes d'opérations avec les limites, de comparaison et des gendarmes sont analogues à ceux vus  dans le chapitre sur les suites réelles.
\end{remark}

\begin{theorem}[Composition de limites]
	Soient $I, J \subseteq \R$ et $f : I \to J,\ g : J \to \R,\ a \in I$ tels que :
	\begin{enumerate}
		\item $\lim_{x \to a} f(x) = y \in I$.
		\item $\lim_{y \to z} g(z) = \ell$ existe.
	\end{enumerate}
	\[ \lim_{x \to a} g(f(x)) = \ell \]
\end{theorem}

\begin{theorem}[Théorème des valeurs intermédiaires]
	$\forall a, b \in \R,\ a < b,\ f : [a, b] \to \R$ une fonction continue.
	\[ \forall y \in [f(a), f(b)],\ \exists c \in [a, b] : f(c) = y \]
\end{theorem}

\begin{proof}
	On utilise la borne supérieure.
	\\
	Soit $E = \{ x \in I \mid f(x) \leq y \}$. $a \in E$ donc $E \neq \varnothing$. On sait que $E \subseteq I$ donc $E$ est majoré.
	\\
	Posons $c \coloneqq \sup(E)$.
	\\
	Puisque $c = \sup(E)$, il existe une suite $(c_n)_{n \in \N}$ d'éléments de $E$ telle que $\lim_{n \to +\infty} c_n = c$. 
	\\
	Comme $f$ est continue, on a :
	\[ \lim_{n \to +\infty} f(c_n) = f(c) \] 
	Puisque $c_n \in E,\ f(c_n) \leq y$. En passant à la limite, on a :
	\[ f(c) \leq y \]
	Montrons maintenant que $f(c) \geq y$.
	\begin{itemize}
		\item Si $c = b$, on a bien :
		\[ f(c) = f(b) = \geq y \]
		\item Si $c < b$, pour $n$ assez grand :
		\[ c < c + \frac{1}{n} \leq b \]
		Sachant que $c = \sup(E)$, $c + \frac{1}{n} \notin E$, on a donc :
		\[ f \left( c + \frac{1}{n} \right) > y \]
		On a $\lim_{n \to +\infty} c + \frac{1}{n} = c$ et $f$ étant continue :
		\[ \lim_{n \to +\infty} f \left( c + \frac{1}{n} \right) = f(c) \]
		Sachant que $f \left( c + \frac{1}{n} \right) > y$, en passant à la limite :
		\[ f(c) \geq y \]
	\end{itemize}
\end{proof}

\begin{theorem}
	$\forall a, b \in \R,\ a < b,\ f :\ ]a, b[ \to \R$. 
	Si $f$ est croissante.
	\begin{enumerate}
		\item $f$ admet une limite en $b$, qui est finie si et seulement si $f$ est \textbf{majorée}.
		\item $f$ admet une limite en $a$, qui est finie si et seulement si $f$ est \textbf{minorée}.
	\end{enumerate}
	Si $f$ est décroissante.
	\begin{enumerate}
		\item $f$ admet une limite en $b$, qui est finie si et seulement si $f$ est \textbf{minorée}.
		\item $f$ admet une limite en $a$, qui est finie si et seulement si $f$ est \textbf{majorée}.
	\end{enumerate}
	$\forall x_0 \in \ ]a, b[$, $f$ a une limite à gauche et à droite en $x_0$ et :
	\[ \lim_{\substack{x \to a \\ x < x_0}} f(x) \leq f(x_0) \leq \lim_{\substack{x \to a \\ x > x_0}} f(x) \]
\end{theorem}

\begin{theorem}
	$\forall a, b \in \R,\ a < b,\ f : [a, b] \to \R$ une fonction continue.
	\begin{enumerate}
	    \item $ f \text{ strictement croissante} \implies f : [a, b] \to [f(a), f(b)] \text{ est une bijection} $.
            \item $ f \text{ strictement décroissante} \implies f : [a, b] \to [f(b), f(a)] \text{ est une bijection} $.
	\end{enumerate}
\end{theorem}

\begin{theorem}
	Soit $I \subseteq \R$ et $f : I \to \R$ une injection continue.
	\\
	Alors $f$ est strictement monotone, donc bijective. Si on pose $J = f(I)$, $f^{-1} : J \to I$ est continue.
\end{theorem}

\begin{definition}[Segment]
	Un segment est un intervalle fermé borné.
\end{definition}

\begin{theorem}
	Soient $a, b \in \R$ tels que $a < b$ et $f : [a, b] \to \R$ une fonction continue. Alors $f$ est bornée sur $[a, b]$ et elle atteint ses bornes.
	\[ \exists m, M \in \R,\ \forall x \in [a, b] : m \leq f(x) \leq M \text{ et } \exists x_0, x_1 \in [a, b] : f(x_0) = m \text{ et } f(x_1) = M \]
\end{theorem}

\begin{definition}[Prolongement par continuité]
	Soient $I \subseteq \R$, $x_0 \in I,\ f : I\backslash\{x_0\} \to \R$.
	On suppose que pour $\ell \in \R$, $\lim_{x \to x_0} f(x) = \ell$ existe. Alors la fonction :
	\begin{center}
		$
		\appli{\overset{\sim}{f}}{I}{\R}{x}{
		\begin{cases}
			f(x) \text{ si } x \neq x_0 \\
			\ell \text{ sinon}
		\end{cases}		
		}
		$
	\end{center}
\end{definition}
\chapter{Dérivabilité et accroissements finis}
\def\arraystretch{1}

\section{Dérivabilité, théorèmes de Rolle et des accroissements finis}
\begin{definition}
	Soient $I \subseteq \R,\ f : I \longrightarrow \R$, $f$ est dérivable en $a \in I$ si et seulement s'il existe $\ell \in \R$ tel que :
	\begin{align*}
		  \lim_{x \to a} \frac{f(x) - f(a)}{x - a} &= \ell
	\end{align*}
    Une autre manière de définir la dérivabilité :
    \begin{align*}
        \lim_{h \to 0} \frac{f(a + h) - f(a)}{h} &= \ell
    \end{align*}
	On note $f'(a) = \ell$ la dérivée de $f$ en $a$.
	Ainsi une fonction dérivable est une fonction dérivable en tout point de $I$.
	On peut également vérifier la limite à gauche et à droite de $a$.
\end{definition}

\begin{proposition}
	Soient $I \subseteq \R,\ f : I \longrightarrow \R$ et $a \in I$, si $f$ est \textbf{dérivable} en $a$ alors elle est \textbf{continue} en $a$.
\end{proposition}

\begin{proof}
	Quand $f$ est dérivable en $a$, le développement limité suivant existe :
	\[ f(x) \underset{x \to a}{=} f(a) + f'(a)(x - a) + o(x - a) \]
	En passant à la limite :
	\begin{align*}
		\lim_{x \to a} f(x) &= \lim_{x \to a} \left( f(a) + f'(a)(x - a) + o(x - a) \right) \\
		&= \lim_{x \to a} f(a) + \underbrace{\lim_{x \to a} \left( f'(a)(x - a) + o(x - a) \right)}_{= 0} \\
		\lim_{x \to a} f(x) &= f(a)
	\end{align*}
	ce qui veut dire que $f$ est continue en $a$.
\end{proof}

\begin{theorem}
	Soient $I \subseteq \R$, $a \in I$, $\lambda \in \R$ et $f, g$ deux fonctions dérivables en $a$.
	\begin{enumerate}
		\item $f + \lambda g$ est dérivable en $a$ et $(f + \lambda g)'(a) = f'(a) + \lambda g'(a)$.
		\item $f g$ est dérivable en $a$ et $(fg)'(a) = f'(a) g(a) + f(a) g'(a)$.
	\end{enumerate}
\end{theorem}

\begin{proof}
	\leavevmode
	\begin{enumerate}
		\item On calcule $\lim_{x \to a} \frac{f(x) + \lambda g(x) - (f(a) + \lambda g(a))}{x - a}$.
		\begin{align*}
			\lim_{x \to a} \frac{f(x) + \lambda g(x) - (f(a) + \lambda g(a))}{x - a} &= \lim_{x \to a} \frac{f(x) + \lambda g(x) - f(a) - \lambda g(a)}{x - a} \\
			&= \lim_{x \to a} \frac{f(x) - f(a)}{x - a} + \lambda \lim_{x \to a} \frac{g(x) - g(a)}{x - a} \\
			&= f'(a) + \lambda g'(a)
		\end{align*}
		\item On calcule $\lim_{x \to a} \frac{(fg)(x) - (fg)(a)}{x - a}$. On utilise le fait qu'une fonction dérivable en $a$ est continue en $a$ et la définition de la continuité en $a$.
		\begin{align*}
			\lim_{x \to a} \frac{(fg)(x) - (fg)(a)}{x - a} &= \lim_{x \to a} \frac{f(x) g(x) - f(a) g(a)}{x - a} \\
			&= \lim_{x \to a} \left[ \frac{f(x) g(x) - f(a) g(a)}{x - a} + g(x) \frac{f(x) - f(a)}{x - a} \right] \\
			&= \lim_{x \to a} \frac{f(x)g(x) - f(a)g(a)}{x - a} + \lim_{x \to a} g(x) \frac{f(x) - f(a)}{x - a} \\
			&= f(a) \lim_{x \to a} \frac{g(x) - g(a)}{x - a} + g(a) \lim_{x \to a} \frac{f(x) - f(a)}{x - a} \\
			&= f(a) g'(a) + f'(a) g(a)
		\end{align*}
	\end{enumerate}
\end{proof}

\begin{theorem}[\cite{derivation_wikiversite}]
	Soient $I, J \subseteq \R$, $f : I \longrightarrow \R$ et $g : J \longrightarrow \R$ telles que $f(I) \subseteq J$ et $a$ un point de $I$. \\
	Si $f$ est dérivable au point $a$ et $g$ est dérivable au point $f(a)$ alors la composée $g \circ f$ est dérivable au point $a$ et :
	\[ (g \circ f)'(a) = g'(f(a)) f'(a) \]
\end{theorem}

\begin{proof}
	\cite{derivation_wikiversite}. Notons $b = f(a)$. Puisque $g$ est dérivable en $b$, il existe une fonction $u : J \to \R$ telle que :
	\[ u(b) = \lim_{y \to b} u(y) = g'(b) \]
	et $\forall y \in J :$ 
	\[ g(y) - g(b) = u(y) (y - b) \]
	En particulier, $f$ est continue au point $a$ car elle y est dérivable: 
	\[ \lim_{x \to a} u(f(x)) = g'(b) \]
	et $\forall x \in I :$
	\[ g(f(x)) - g(f(a)) = u(f(x)) (f(x) - f(a)) \]
	Le taux de variation au point $a$ de la fonction $g \circ f$ s'exprime alors sous la forme :
	\[ \frac{g(f(x)) - g(f(a))}{x - a} = u(f(x)) \cdot \frac{f(x) - f(a)}{x - a} \]
	et quand $x$ tend vers $a$, cette expression tend vers $g'(b) \cdot f'(a) = g'(f(a)) f'(a)$.
\end{proof}

\begin{definition}[Maximum, minimum]
    Soient $I \subseteq \R,\ f : I \longrightarrow \R$ et $a \in I$.
    \begin{enumerate}
        \item On dit que $a$ est un \textbf{maximum} si et seulement si : $\forall x \in I,\ f(a) \geq f(x)$.
        \item On dit que $a$ est un \textbf{minimum} si et seulement si : $\forall x \in I,\ f(a) \leq f(x)$.
    \end{enumerate}
    On appelle extremum un point qui est soit un maximum soit un minimum.
    \begin{enumerate}
        \item On dit que $a$ est un \textbf{maximum local} si et seulement si : $\exists \varepsilon > 0,\ a \text{ est un maximum de } f_{|]a - \varepsilon, a + \varepsilon[}$.
        \item On dit que $a$ est un \textbf{minimum local} si et seulement si : $\exists \varepsilon > 0,\ a \text{ est un minimum de } f_{|]a - \varepsilon, a + \varepsilon[}$.
    \end{enumerate}
\end{definition}

\begin{theorem}
	Soient $I \subseteq \R,\ f : I \longrightarrow \R$ une fonction dérivable et $a \in \overset{\circ}{I}$, $\overset{\circ}{I}$ désigne l'intérieur de $I$. 
	\[ a \text{ est un extremum local} \implies f'(a) = 0 \]
\end{theorem}

\begin{proof}
	Soient $a$ un extremum local et $f_{|]a - \varepsilon, a + \varepsilon[}$.
	\\
	\begin{enumerate}
		\item Quand $h > 0$ : 
		\[ \frac{f(a + h) - f(a)}{h} \leq 0,\ \lim_{h \to 0} f'(a) \leq 0 \]
		\item Quand $h < 0$ :
		\[ \frac{f(a + h) - f(a)}{h} \geq 0,\ \lim_{h \to 0} f'(a) \geq 0 \]
	\end{enumerate}
	alors $f'(a) = 0$.
\end{proof}

\begin{theorem}[Théorème de Rolle]
    Soient $a, b \in \R$ tels que $a < b$ et $f : [a, b] \longrightarrow \R$ une fonction telle que :
    \begin{enumerate}
            \item $f$ est continue sur $[a, b]$.
            \item $f$ est dérivable sur $]a, b[$.
            \item $f(a) = f(b)$.
        \end{enumerate}
    \par \noindent Il existe un $c \in \ ]a, b[$ tel que : \[ f'(c) = 0 \]
\end{theorem}

\begin{proof}\cite{exo7_analyse1}
	\leavevmode
	\begin{enumerate}
		\item Si $f$ est constante, alors n'importe quel $c \in \ ]a, b[$ convient.
		\item Sinon il existe $x_0 \in [a, b]$ tel que $f(x_0) \neq f(a)$. 
		\\
		Supposons par exemple $f(x_0) > f(a)$. Alors $f$ est continue sur $[a, b]$, donc elle admet un maximum en un point $c \in [a, b]$. Mais $f(c) \geq f(x_0) > f(a)$ donc $c \neq a$. De même comme $f(a) = f(b)$ alors $c \neq b$. Ainsi $c \in \ ]a, b[$. En $c$, $f$ est dérivable et admet un maximum local donc $f'(c) = 0$.
	\end{enumerate}
\end{proof}

\begin{theorem}[Théorème des accroissements finis]
    Soient $a, b \in \R$ tels que $a < b$ et $f : [a, b] \longrightarrow \R$, telle que : 
    \begin{enumerate}
            \item $f$ est continue sur $[a, b]$.
            \item $f$ est dérivable sur $]a, b[$.
        \end{enumerate}
    \par \noindent Il existe un $c \in ]a, b[$ tel que :
	\[ f'(c) = \frac{f(b) - f(a)}{b - a} \]
\end{theorem}

\begin{proof}\cite{exo7_analyse1}
	Posons $\ell = \frac{f(b) - f(a)}{b - a}$ et $g(x) = f(x) - \ell (x - a)$.
	 \\
	 Alors $g(a) = f(a),\ g(b) = f(b) - \frac{f(b) - f(a)}{b - a} \cdot (b - a) = f(a)$.
	 \\
	  Par le théorème de Rolle, il existe un $c \in \ ]a, b[$ tel que $g'(c) = 0$. Or $g'(x) = f'(x) - \ell$. Ce qui donne $f'(c) = \frac{f(b) - f(a)}{b - a}$.
\end{proof}

\begin{corollary}[Inégalité des accroissements finis]
	Soient $a,b \in \R$ tels que $a < b$, $f : \ ]a, b[ \ \longrightarrow \R$ une fonction dérivable sur $]a, b[$ et $M$ une constante telle que pour tout $x \in \ ]a, b[,\ \abs{f'(x)} \leq M$.
	\[ \abs{\frac{f(b) - f(a)}{b - a}} \leq M \]
\end{corollary}

\begin{proof}
	D'après le théorème des accroissements finis, il existe un $c \in \ ]a, b[$ tel que $\frac{f(b) - f(a)}{b - a}$. Or pour tout $x \in \ ]a, b[$, $\abs{f'(x)} \leq M$ donc $\abs{f'(c)} \leq M$ et donc :
	\[ \abs{\frac{f(b) - f(a)}{b - a}} \leq M \]
\end{proof}

\begin{proposition}[\cite{exo7_analyse1}]
    Soient $f : [a, b] \longrightarrow \R$ une fonction continue sur $[a, b]$ et dérivable sur $]a, b[$.
    \begin{enumerate}
            \item $\forall x \in \ ]a, b[,\ f'(x) \geq 0 \iff f \text{ croissante}$.
            \item $\forall x \in \ ]a, b[,\ f'(x) \leq 0 \iff f \text{ décroissante}$.
            \item $\forall x \in \ ]a, b[,\ f'(x) = 0 \iff f \text{ constante}$.
            \item $\forall x \in \ ]a, b[,\ f'(x) > 0 \implies f \text{ strictement croissante}$.
            \item $\forall x \in \ ]a, b[,\ f'(x) < 0 \implies f \text{ strictement décroissante}$ .
        \end{enumerate}
\end{proposition}

\begin{proof}
	Montrons le 1. 
	\\
	\boxed{\implies} : Supposons que pour $x \in \ ]a, b[,\ f'(x) \geq 0$. \\
	Soient $x, y \in \ ]a, b[$ tels que $x \leq y$. D'après le théorème des accroissements finis, il existe un $c \in \ ]x, y[$ tel que $\frac{f(x) - f(y)}{x - y} = f'(c)$.
	\[ f(x) - f(y) = f'(c)(x - y) \]
	Or $f'(x) \geq 0$ pour $x \in \ ]a, b[$ donc $f'(c) \geq 0$ et $x \leq y$ donc $x - y \leq 0$ et $f(x) - f(y) \leq 0$ donc  $f(x) \leq f(y)$.
	\\
	\boxed{\impliedby} : Supposons que pour $x, y \in \ ]a, b[$ tels que $x \leq y$ et $f(x) \leq f(y)$. \\
	On a donc :
	\begin{align*}
		f(y) - f(x) &\geq 0 \\
		\frac{f(y) - f(x)}{y - x} &\geq 0 
	\end{align*}
	On sait que :
	\[ \lim_{y \to x} \frac{f(y) - f(x)}{y - x} = f'(x) \]
	donc :
	\[ f'(x) \geq 0 \]
\end{proof}

\begin{definition}
	Soient $I \subseteq \R,\ f : I \longrightarrow \R$ et $n \in \N$.
	\begin{enumerate}
		\item $f \in \mathcal{D}^n(I, \R)$ signifie que $f$ est $n$ fois dérivable sur $I$.
		\item $f \in \mathcal{C}^n(I, \R)$ signifie que $f$ est $n$ fois dérivable et que sa dérivée $n$-ième est continue.
		\item $f \in \mathcal{C}^{\infty}(I, \R) = \mathcal{D}^{\infty}(I, \R)$ signifie que $f \in \mathcal{C}^n(I, \R),\ \forall n \in \N$. On dit que les fonctions $\mathcal{C}^{\infty}$ sont des \textbf{fonctions lisses}.
	\end{enumerate}
\end{definition}

\begin{proposition}
	$\forall f, g \in \mathcal{C}^n(I, \R) \implies f + g, f \cdot g, f \circ g \in \mathcal{C}^n(I, \R)$
\end{proposition}

\section{Convexité}
\begin{definition}
    Soient $I \subseteq \R,\ f : I \longrightarrow \R$.
    \begin{enumerate}
        \item On dit que $f$ est \textbf{convexe} si et seulement si : 
        \[ \forall x, y \in I,\ \lambda \in [0, 1] : f(\lambda x + (1 - \lambda)y) \leq \lambda f(x) + (1 - \lambda) f(y) \]
        \item On dit que $f$ est \textbf{concave} si et seulement si : 
        \[ \forall x, y \in I,\ \lambda \in [0, 1] : f(\lambda x + (1 - \lambda)y) \geq \lambda f(x) + (1 - \lambda) f(y) \]
    \end{enumerate}
\end{definition}

\par Géométriquement, $f$ est convexe signifie que son graphe passe sous les cordes de $f$ et que les tangentes passent sous le graphe. $f$ est concave signifie que son graphe au-dessus des cordes de $f$ et que les tangentes passent par-dessus le graphe.

\begin{theorem}
	Soient $I \subseteq \R,\ f \in \mathcal{D}^2(I, \R)$.
        \begin{enumerate}
                \item $f \text{ convexe} \iff f'' \geq 0$.
                \item $f \text{ concave} \iff f'' \leq 0$.
            \end{enumerate}
\end{theorem}

\begin{theorem}[Inégalité de Jensen]
	Soient $I \subseteq \R$, $f : I \longrightarrow \R$ une fonction convexe, $x_i \in I$ et $\lambda_i \in [0, 1]$ tels que $\sum_{i=1}^n \lambda_i = 1$.
	\[ f \left( \sum_{i=1}^{n} \lambda_i x_i \right) \leq \sum_{i=1}^{n} \lambda_i f(x_i) \]
	Si $f$ est concave, 
	\[ f \left( \sum_{i=1}^{n} \lambda_i x_i \right) \geq \sum_{i=1}^{n} \lambda_i f(x_i) \]
\end{theorem}

\begin{proof}\cite{inegalite_jensen_bibmath}
	On procède par récurrence pour montrer $P(n) : f \left( \sum_{i=1}^{n} \lambda_i x_i \right) \leq \sum_{i=1}^{n} \lambda_i f(x_i)$. Le principe de la preuve est similaire pour l'autre inégalité.
	\begin{enumerate}
		\item \textbf{Initialisation :} Pour $n = 1$ et $n = 2$ il s'agit de la définition d'une fonction convexe.
		\item \textbf{Hérédité :} Supposons que $P(n)$ vraie pour un $n > 2$. Soient $x_1, \ldots, x_{n+1} \in I$ et $\lambda_1, \ldots, \lambda_{n+1} \in [0,1]$ tels que $\lambda_1 + \cdots + \lambda_n + \lambda_{n+1} = 1$. On veut estimer :
		\[ f(\lambda_1 x_1 + \cdots + \lambda_n x_n + \lambda_{n+1} x_{n+1} ) \]
		On pose :
		\[ 
		\begin{cases}
			\lambda_n' = \lambda_n + \lambda_{n+1} \\
			\lambda_n' x_n' = \lambda_n x_n + \lambda_{n+1} x_{n+1}
		\end{cases}
		\]
		Alors $\lambda_n' \in [0, 1]$. En effet :
		\[ \lambda_n \geq 0,\ \lambda_{n+1} \geq 0 \implies \lambda_n' \geq 0 \]
		et 
		\[ \lambda_n' = 1 - (\lambda_1+ \cdots + \lambda_{n-1}) \leq 1 \]
		On a aussi $x_n' \in I$. En effet, si $x_n \leq x_{n+1}$, alors :
		\[ x_n' = \frac{\lambda_n}{\lambda_n'} x_n + \frac{\lambda_{n+1}}{\lambda_n'} x_{n+1} \leq \frac{\lambda_n}{\lambda_n + \lambda_{n+1}} x_{n+1} + \frac{\lambda_{n+1}}{\lambda_n + \lambda_{n+1}} x_{n+1} \leq x_{n+1} \]
		De même, 
		\[ x_n' \geq x_n \]
		On a 
		\begin{align*}
			f(\lambda_1 x_1 + \cdots + \lambda_n x_n + \lambda_{n+1} x_{n+1}) &= f(\lambda_1 x_1 + \cdots + \lambda_{n-1} x_{n-1} + \lambda_n' x_n') \\
			&\leq \lambda_1 f(x_1) + \cdots  + \lambda_{n-1} f(x_{n-1}) + \lambda_n' f(x_n') 
		\end{align*}
		Puisque que $f$ est convexe,
		\begin{align*}
			f(x_n') &= f \left(\frac{\lambda_n}{\lambda_n'} x_n + \frac{\lambda_{n+1}}{\lambda_n'} x_{n+1}\right) \\
			&\leq \frac{\lambda_n}{\lambda_n'} f(x_n) + \frac{\lambda_{n+1}}{\lambda_n'} f(x_{n+1})
		\end{align*}
		On conclut que :
		\[ f(\lambda_1 x_1 + \cdots + \lambda_n x_n + \lambda_{n+1} x_{n+1}) \leq \lambda_1f(x_1) + \cdots  +\lambda_{n+1} f(x_{n+1}) \]
		Donc $P(n+1)$ est vraie.
	\end{enumerate}
\end{proof}

\begin{proposition}
	Soient $I \subseteq \R,\ f : I \longrightarrow \R$ et $a \in I$, la tangente de $f$ en $a$ est :
	\[ \mathcal{T}_a(x) = f(a) + f'(a)(x - a) \]
\end{proposition}

\begin{proposition}
	Soient $a, b \in \R$ tels que $a < b,\ f : [a, b] \longrightarrow \R$.
	La corde $c$ reliant les points $(a, f(a))$ et $(b, f(b))$ est définie par l'équation suivante :
	\[ c = \frac{f(b) - f(a)}{b - a} (x - a) + f(a) \]
\end{proposition}

\begin{proposition}
	Soient $I \subseteq \R,\ f \in \mathcal{D}^2(I, \R),\ a \in I$.
	\begin{align*}
		\begin{cases}
			f'(a) = 0 \\
			f''(a) < 0
		\end{cases}
		\implies a \text{ est un maximum local, si } f''(a) > 0,\ a \text{  est un minimum local}
	\end{align*}
\end{proposition}

\begin{definition}[Suite récurrente]
	Soient $I \subseteq \R,\ f : I \longrightarrow \R$ et $u_0 \in I$.
	Si $\forall n \in \N,\ u_n \in I$ alors on peut définir :
	\[ u_{n + 1} = f(u_n) \]
\end{definition}

\begin{lemma}
	Soit $I \subseteq \R$. S'il existe un $\ell \in I$ tel que $\lim_{n \to +\infty} u_n = \ell$ et si $f$ est continue en $\ell$ alors 
	\[ f(\ell) = \ell \]
\end{lemma}

\begin{definition}
	Soient $I \subseteq \R,\ f : I \longrightarrow \R$. On dit que $f$ est stable sur $I$ si et seulement si : 
	\[ f(I) \subseteq I \]
\end{definition}

\begin{proposition}
	Soit $I \subseteq \R$.
	\begin{enumerate}
		\item Si $f$ est \textbf{croissante} sur $I$ alors 
		$
		(u_n)_{n \in \N} = 
		\begin{cases}
			u_0 \in I \\
			u_{n + 1} = f(u_n)
		\end{cases}
		$
		est monotone.
		\[ u_1 \geq u_0 \iff (u_n) \text{ croissante} \]
		\[ u_1 \leq u_0 \iff (u_n) \text{ décroissante} \]
		\item Si $f$ est \textbf{décroissante} sur $I$, alors les suites extraites $(v_n)_{n \in \N} = u_{2n}$ et $(w_n)_{n \in \N} = u_{2n + 1}$ sont monotones, l'une est \textbf{croissante}, l'autre est \textbf{décroissante}.
	\end{enumerate}
\end{proposition}

\begin{definition}[Point fixe]
	Soient $f : \mathcal{D}_f \longrightarrow \R$ et $x \in \mathcal{D}_f,\ f(x) = x$. On dit que $x$ est un \textbf{point fixe} de $f$.
\end{definition}

\begin{definition}[Coefficient de convergence]
    Soient $(u_n)_{n \in \N}$ tel que $\lim_{n \to +\infty} u_n = \ell \in \R$, $\varepsilon_n = \abs{u_n - \ell}$ et supposons que $\lim_{n \to +\infty} \frac{\varepsilon_{n + 1}}{\varepsilon_n} = K \in \R_+$. 
    On appelle $K$ coefficient de la convergence.
    \begin{itemize}
        \item Si $K = 1$ la convergence est \textbf{lente}.
        \item Si $K = 0$ la convergence est \textbf{rapide}.
        \item Si $0 < K < 1$ la convergence est \textbf{géométrique}.
    \end{itemize}
\end{definition}

\begin{definition}[Fonction contractante]
	Soient $I \subseteq \R,\ f : I \longrightarrow \R$.
    On dit que $f$ est \textbf{contractante} si et seulement si :
    \[ \exists k \in \ ]0, 1[,\ \forall x, y \in I : \abs{f(x) - f(y)} \leq k \abs{x - y} \]
\end{definition}

\begin{theorem}[Théorème du point fixe]
	Soient $I$ un intervalle fermé et $f : I \longrightarrow I$ une fonction contractante et continue et $(u_n)_{n \in \N}$ sa suite récurrente associée.
	\begin{enumerate}
        \item Il existe un unique point fixe $\ell \in I$.
        \item $\lim_{n \to +\infty} u_n = \ell$.
        \item La convergence de $(u_n)_{n \in \N}$ est géométrique.
    \end{enumerate}
\end{theorem}

\begin{proof}
	Soient $a, b$ des réels tels que $a < b$.
	\begin{enumerate}
		\item \textbf{Existence} : Il existe un point fixe d'après le théorème des valeurs intermédiaires pour $g(x) = f(x) - x$.
		\begin{enumerate}
			\item $g(a) = f(a) - a \geq 0$.
			\item $g(b) = f(b) - b \leq 0$.
		\end{enumerate}
		\[ \exists c \in [a, b] : g(c) = f(c) - c = 0 \iff f(c) = c \]
		\item \textbf{Unicité} : Soient $\ell_1, \ell_2$ des points fixes.
		Supposons que $\ell_1 \neq \ell_2$ .
		\begin{align*}
			\abs{\ell_1 - \ell_2} &= \abs{f(\ell_1) - f(\ell_2)} \\
								  &\leq k \abs{\ell_1 - \ell_2} \\
								  &< \abs{\ell_1 - \ell_2}
		\end{align*}
		ce qui est contradictoire donc $\ell_1 = \ell_2$.
	\end{enumerate}
\end{proof}

\begin{theorem}[Suites récurrentes linéaires d'ordre 2]
	Soit $u_{n+2} = a u_{n+1} + b u_n$ pour $a, b \in \R$.
	On pose l'équation suivante pour $r \in \R$ :
	\[ r^2 - ar - b = 0 \qquad \Delta = a^2 + 4b \]
	Pour $\lambda, \mu, \alpha \in \R$ :
	\begin{itemize}
        \item $\Delta > 0 \implies u_n = \lambda r_1^n + \mu r_2^n$ avec $r_1, r_2$ tels que :
        \begin{align*}
            r_1 &= \frac{a - \sqrt{\Delta}}{2} & r_2 &= \frac{a + \sqrt{\Delta}}{2}
        \end{align*}
        \item $\Delta = 0 \implies u_n = (\lambda + \mu n) r_0^n$ avec $r_0$ tel que :
        \begin{align*}
            r_0 = \frac{a}{2}
        \end{align*}
    \end{itemize}    
    On trouve $\lambda, \mu$ grâce aux conditions sur les deux premiers termes de la suite.
\end{theorem}



\chapter{Intégration}

\begin{theorem}[Théorème fondamental de l'analyse]
	Soit $f \in \mathcal{C}^0([a, b])$ et 
	\begin{align*}
		\forall x \in [a,b],\ F(x) \coloneqq \int_a^x f(t) \ \diffd t
	\end{align*}
	alors $F \in \mathcal{C}^1(]a, b[)$ et $\forall x \in ]a, b[,\ F'(x) = f(x)$.    
\end{theorem}

\begin{corollary}
    Si $F \in \mathcal{C}^1(]a, b[)$ tel que $\forall x \in ]a, b[,\ F'(x) = f(x)$ alors
	\begin{align*}
		\int_a^b f(x) \ \diffd x = [F(x)]_a^b \equiv F(b) - F(a) 
	\end{align*}
\end{corollary}

\begin{proposition}
	Soient $(\lambda, a, b) \in \R^3$ et $(f, g) \in (\mathcal{C}^0([a, b]))^2$.
	\begin{align*}
		\int_a^b f(x) + \lambda g(x) \ \diffd x = \int_a^b f(x) \ \diffd x + \lambda \int_a^b g(x) \ \diffd x 
	\end{align*}
\end{proposition}

\begin{proof}
    \begin{align*}
        \int_{a}^{b} f(x) + \lambda g(x) \ \diffd x &= [F(x) + \lambda G(x)]_a^b \\
                                                    &= (F(b) + \lambda G(b)) - (F(a) + \lambda G(a)) \\
                                                    &= F(b) - F(a) + \lambda (G(b) - G(a)) \\
                                                    &= \int_{a}^{b} f(x) \ \diffd x + \lambda \int_{a}^{b} g(x) \ \diffd x
    \end{align*}
\end{proof}

\begin{proposition}
	Soient $(a, b) \in \R^2$ et $c \in ]a, b[$.
	\begin{align*}
		\int_a^b f(x) \ \diffd x = \int_a^c f(x) \ \diffd x + \int_c^b f(x) \ \diffd x
	\end{align*}
\end{proposition}

\begin{theorem}[Théorème de la moyenne]
    Soit $f \in \mathcal{C}^0([a, b])$. $\exists c \in ]a, b[$  tel que :
    \[ \frac{\int_{a}^{b} f(x) \ \diffd x}{b - a} = f(c) \]
\end{theorem}

\begin{theorem}[Intégration par parties]
	Soient $(u, v) \in (\mathcal{C}^1([a, b]))^2$ alors
	\begin{align*}
		\int_a^b u'(x) v(x) \ \diffd x = [u(x)v(x)]_a^b - \int_a^b u(x) v'(x) \ \diffd x
	\end{align*}
  \end{theorem}

\begin{proof}
    \begin{align*}
        (uv)'(x) &= u'(x) v(x) + u(x)v'(x) \\
        \iff u'(x)v(x) &= (uv)'(x) - u(x)v'(x)
    \end{align*}
    Par croissance et linéarité de l'intégrale :
    \begin{align*}
        \int_a^b u'(x)v(x) \diffd x &= \int_a^b (uv)'(x) \diffd x - \int_a^b u(x)v'(x) \ \diffd x \\
        \int_a^b u'(x)v(x) \diffd x &=  [u(x)v(x)]_a^b - \int_a^b u(x)v'(x) \ \diffd x 
    \end{align*}
\end{proof}

\begin{theorem}[Intégration par changement de variable]
	Soit $f \in \mathcal{C}^0(I)$ et $\varphi \in \mathcal{C}^1([a, b])$ tel que $\varphi ([a, b]) \subset I$ alors 
	\begin{align*}
		\int_a^b f(\varphi(x)) \cdot \varphi'(x) \ \diffd x = \int_{\varphi(a)}^{\varphi(b)} f(x) \ \diffd x
	\end{align*}
\end{theorem}

\begin{proof}
    \begin{align*}
        (f \circ \varphi)'(x) &= f' \circ \varphi(x) \cdot \varphi'(x) \\
        \int_a^b (f \circ \varphi)'(x) \diffd x &= \int_a^b f' \circ \varphi(x)  \cdot \varphi'(x) \diffd x \\
         &= [F(\varphi(x))]_a^b \\
         &= F(\varphi(b)) - F(\varphi(a)) \\
         &= [F(x)]_{\varphi{a}}^{\varphi(b)} \\
         &= \int_{\varphi(a)}^{\varphi(b)} f(x) \ \diffd x 
    \end{align*}
\end{proof}

Lorsque nous sommes confrontés à une intégrale de fonctions trigonométriques, on peut se ramener à une intégrale de fraction rationnelle en posant le changement de variable suivant :
\[ u = \tan(\frac{x}{2}) \]
\begin{multicols}{3}
    \begin{enumerate}
        \item $\cos(x) = \frac{1 - u^2}{1 + u^2}$
        \item $\sin(x) = \frac{2u}{1 + u^2}$
        \item $\diffd x = \frac{2}{1 + u^2} \diffd u$
    \end{enumerate}
\end{multicols}
\begin{proof}
    \begin{enumerate}
        \item \[ \cos(x) = \cos^2\left( \frac{x}{2} \right) - \sin^2\left( \frac{x}{2} \right) \]
            \begin{align*}
                \cos^2\left( \frac{x}{2} \right) - \sin^2\left( \frac{x}{2} \right) &= \cos^2\left( \frac{x}{2} \right) - \sin^2\left( \frac{x}{2} \right) \cdot \frac{1 + \tan^2 \left( \frac{x}{2} \right)}{1 + \tan^2 \left( \frac{x}{2} \right)}
            \end{align*}
            Posons $A(x) = \left( \cos^2\left( \frac{x}{2} \right) - \sin^2\left( \frac{x}{2} \right) \right) \left( 1 + \tan^2\left( \frac{x}{2} \right) \right)$.
            \begin{align*}
                A(x) &= \cos^2\left( \frac{x}{2} \right) - \sin^2\left( \frac{x}{2} \right) + \left[ \cos^2\left( \frac{x}{2} \right) - \sin^2\left( \frac{x}{2} \right) \right] \frac{\sin^2 \left( \frac{x}{2} \right)}{\cos^2 \left( \frac{x}{2} \right)} \\
                &= \cos^2 \left( \frac{x}{2} \right) - \frac{\sin^4 \left( \frac{x}{2} \right)}{\cos^2 \left( \frac{x}{2} \right)} \\
                &= \frac{\cos^4 \left( \frac{x}{2} \right) - \sin^4 \left( \frac{x}{2} \right)}{\cos^2 \left( \frac{x}{2} \right)}\\
                &= \frac{\left[ \cos^2\left( \frac{x}{2} \right) - \sin^2 \left( \frac{x}{2} \right) \right] \left[ \cos^2\left( \frac{x}{2} \right) + \sin^2 \left( \frac{x}{2} \right) \right]}{\cos^2 \left( \frac{x}{2} \right)}
            \end{align*}
             On sait que $\forall x \in \R, \cos^2(x) + \sin^2(x) = 1$.
             Ainsi : 
             \begin{align*}
                 A(x) = 1 - \frac{\sin^2\left( \frac{x}{2} \right)}{\cos^2\left( \frac{x}{2} \right)} = 1 - \tan^2 \left( \frac{x}{2} \right)
             \end{align*}
             Ainsi en posant $u = \tan(\frac{x}{2})$, on retrouve bien :
             \begin{align*}
                 \cos^2\left( \frac{x}{2} \right) - \sin^2\left( \frac{x}{2} \right) = \cos(x) = \frac{1 - u^2}{1 + u^2}
             \end{align*}
     \item \[ \sin(x) =  2 \sin(\frac{x}{2}) \cos(\frac{x}{2}) \]
        \begin{align*}
            2\sin(\frac{x}{2})\cos(\frac{x}{2}) &= 2\sin(\frac{x}{2})\cos(\frac{x}{2}) \cdot \frac{1 + \tan^2 \left( \frac{x}{2} \right)}{1 + \tan^2 \left( \frac{x}{2} \right)} \\
            &= \frac{2\sin(\frac{x}{2})\cos(\frac{x}{2}) \cdot \left( 1 + \tan^2 \left( \frac{x}{2} \right) \right)}{1 + \tan^2 \left( \frac{x}{2} \right)}
        \end{align*}
        Posons $A(x) = 2\sin(\frac{x}{2})\cos(\frac{x}{2}) \cdot \left( 1 + \tan^2 \left( \frac{x}{2} \right) \right)$.
        \begin{align*}
            A(x) &= 2\sin(\frac{x}{2})\cos(\frac{x}{2}) + 2\sin(\frac{x}{2})\cos(\frac{x}{2}) \tan^2 \left( \frac{x}{2} \right) \\
            &= 2\sin(\frac{x}{2})\cos(\frac{x}{2}) + 2\sin(\frac{x}{2})\cos(\frac{x}{2}) \frac{\sin^2 \left( \frac{x}{2} \right)}{\cos^2 \left( \frac{x}{2} \right)} \\
            &= 2\sin(\frac{x}{2})\cos(\frac{x}{2}) + \frac{2 \sin^3 \left( \frac{x}{2} \right)}{\cos(\frac{x}{2})} \\ 
            &= 2 \left( \sin(\frac{x}{2}) \left[ \cos(\frac{x}{2}) + \frac{\sin^2 \left( \frac{x}{2} \right)}{\cos(\frac{x}{2})} \right] \right) \\
            &= 2 \left( \sin(\frac{x}{2}) \left[ \frac{\cos^2 \left(\frac{x}{2}\right) + \sin^2 \left( \frac{x}{2} \right)}{\cos(\frac{x}{2})} \right] \right)
        \end{align*}
        On sait que $\forall x \in \R, \cos^2(x) + \sin^2(x) = 1$. Ainsi :
        \begin{align*}
            A(x) &= 2 \frac{\sin(\frac{x}{2})}{\cos(\frac{x}{2})} \\
            &= 2 \tan(\frac{x}{2})
        \end{align*}
        On a donc :
        \begin{align*}
            2\sin(\frac{x}{2})\cos(\frac{x}{2}) = \sin(x) = \frac{2 \tan(\frac{x}{2})}{1 + \tan^2 \left( \frac{x}{2} \right)}
        \end{align*}
        En posant $u = \tan(\frac{x}{2})$ on retrouve :
        \[ \sin(x) = \frac{2u}{1 + u^2} \]
        \item 
        \begin{align*}
            u = \tan(\frac{x}{2}) &\iff \arctan(u) = \frac{x}{2} \\
            &\iff x = 2 \arctan(u) \\
            &\iff \diffd x = \frac{2}{1 + u^2} \diffd u
        \end{align*}
    \end{enumerate}
\end{proof}


\chapter{Equations différentielles linéaires}
\def\arraystretch{1}

\noindent Pour résoudre une équation différentielle, nous allons suivre ces 3 étapes :
\begin{enumerate}
	\item Déterminer les solutions de l'équation homogène associée.
	\item Déterminer une solution particulière de l'équation différentielle.
	\item Combiner les solutions précédentes pour obtenir la solution générale.
\end{enumerate}

\section{\'Equations différentielles d'ordre 1}

\begin{definition}[\'Equation différentielle d'ordre 1]
	Soient $y \in \mathcal{C}^1(\R, \R)$ et $a, b \in \mathcal{C}^0(\R, \R)$. \\
	Une équation différentielle d'ordre 1 est une équation de la forme
	\[ y' + a(x)y = b(x) \]
\end{definition}

\begin{theorem}
	Soient $y \in \mathcal{C}^1(\R, \R)$ et $a \in \mathcal{C}^0(\R, \R)$, les solutions de :
	\[ y' + a(x)y = 0 \]
	sont de la forme $C \in \R,\ A'(x) = a(x)$:
	\[ y = C \cdot \exp(-A(x)) \]
\end{theorem}

\begin{proof}
    Il faut montrer l'inclusion dans les deux sens. \\
    \textbf{Inclusion réciproque :}
	Définissons
	\[ y = C e^{-A(x)} \]
	on aurait donc
	\[ y' = -C a(x)e^{-A(x)} \]
	puis
	\[ y' + a(x)y(x) = -C a(x) e^{-A(x)} + a(x)Ce^{-A(x)} = 0 \]
    \textbf{Inclusion directe} \cite{bibmath_resolution_eq_diff} :
    Supposons $y$ solution de $y' + a(x) y = 0$.
    Alors il existerait un $C \in \R$ tel que $y(x) = Ce^{-A(x)}$.
    Posons $f(x) = y(x)e^{A(x)}$.
    \begin{align*}
        f'(x) &= y'(x)e^{A(x)} + y(x)e^{A(x)}A'(x) \\
        &= -a(x)y(x)e^{A(x)} + a(x)y(x)e^{A(x)} \\
        &= 0
    \end{align*}
    Cela implique que $f(x) = C,\ C \in \R$. Donc :
    \begin{align*}
        C = y(x)e^{A(x)} \iff y(x) = Ce^{-A(x)}
    \end{align*}
\end{proof}

\begin{example}
    $(E_1) : y' - 2xy = 0$.
    En appliquant le théorème précédent on obtient les solutions :
    \begin{align*}
        y_0 = Ce^{x^2}
    \end{align*}
\end{example}

Pour trouver une solution particulière d'une équation différentielle d'ordre 1. Nous utilisons $y_{h}(x)$, sauf qu'ici, $C$ n'est plus une constante mais une fonction. Cette méthode est appelée \textbf{variation de la constante}.
    \begin{align*}
        \begin{cases}
            y_{p} = C(x)e^{-A(x)} \\
            y_{p}' = C'(x)e^{-A(x)} - C(x)a(x)e^{-A(x)}
        \end{cases}
    \end{align*}
    On obtient alors en remplaçant dans l'équation générale :
    \begin{align*}
        &y_{p}' + a(x) y_{p} = b(x) \\
        \iff &C'(x)e^{-A(x)} - C(x)a(x) e^{-A(x)} + a(x)C(x)e^{-A(x)} = b(x) \\
        \iff &C'(x) e^{-A(x)} = b(x) \\
        \iff &C(x) = \int b(x) e^{A(x)}
    \end{align*}

\begin{example}
    $(E_1) : y' - 2xy = \exp(x^2 - x)$.
    On utilise donc la solution homogène trouvée précédemment avec la variation de la constante.
    \begin{align*}
        \begin{cases}
            y_p = C(x) e^{x^2} \\ 
            y_p' = C'(x) e^{x^2} + C(x)2x e^{x^2}
        \end{cases}
    \end{align*}
    On remplace dans l'équation $(E_1)$.
    \begin{align*}
        y_p' - 2x y_p &= \exp(x^2 - x) \\
        C'(x) e^{x^2} + C(x)2x e^{x^2} - 2x C(x)e^{x^2} &= \exp(x^2 - x) \\
        C'(x)e^{x^2} &= \exp(x^2 - x) \\ 
        C'(x) &= \frac{\exp(x^2 - x)}{e^{x^2}} \\
        C'(x) &= \exp(x^2 - x - x^2) \\
        C'(x) &= \exp(-x) \\
        C(x) &= -\exp(-x) + k, k \in \R
    \end{align*}
    Ainsi on a l'ensemble des solutions de $(E_1)$ :
    \begin{align*}
        y &= (-e^{-x} + k) e^{x^2}, k \in \R \\
        y &= -\exp(x^2 - x) + ke^{x^2}, k \in \R
    \end{align*}
\end{example}

\section{\'Equations différentielles d'ordre 2}

\begin{definition}[\'Equations différentielle d'ordre 2]
	Soient $y \in \mathcal{C}^2(\R, \R)$, $b \in \mathcal{C}^0(\R, \R)$ et $p, q \in \R$. \\
	Une équation différentielle d'ordre 2 est une équation de la forme
	\[ y'' + py' + q = b(x) \]
\end{definition}

\begin{theorem}
  Soient $y \in \mathcal{C}^2(\R, \R),\ p, q \in \R$ et $(E)$ l'équation suivante :
  \[ (E) : y'' + py' + qy = 0, \]
  On s'intéresse d'abord à cette équation associée :
  \[ \ \lambda^{2} + p\lambda + q = 0 \]
  Pour $\Delta = p^2 - 4q$. 
  \begin{enumerate}
	\item Cas 1 :
		  $\Delta > 0$
		  \[ y = C_{1} e^{\lambda_{1} x} + C_{2} e^{\lambda_{2} x},\ \lambda_{i} C_{i} \in \R \]
        \[ \lambda_{1/2} = \frac{-p \pm \sqrt{\Delta}}{2} \]
	\item Cas 2:
		  $\Delta = 0$
		  \[ y = (C_{1} + C_{2}x) e^{\lambda x},\ C_{i}, \lambda \in \R \]
        \[ \lambda = \frac{-p}{2} \]
	\item Cas 3 :
		  $\Delta < 0$
		  $,\ \lambda_{1} = a + ib,\ a, b \in \R,\ \lambda_{2} = a - ib \equiv \overline{\lambda_{1}}$
		  \[ y = e^{ax}(C_1 \cos(\abs{b}x) + C_2 \sin(\abs{b}x)), C_i \in \R \]
        \[ \lambda_{1/2} = \frac{-p \pm i \sqrt{-\Delta}}{2} \]
  \end{enumerate}
\end{theorem}

\par La variation de la constante est difficilement applicable sur les équations différentielles de second ordre, nous devons trouver d'autres méthodes.
Toutes les équations de second ordre qu'on étudiera dans ce chapitre auront pour second membre une composé de fonctions polynomiales, exponentielles et trigonométriques. Ainsi, nous pouvons utiliser ces propriétés.
\\
\par \noindent Voici une méthode pour trouver une solution particulière d'une équation de type :
    \[ y'' + py' + qy = b(x) \]
    pour $p, q \in \R,\ b \in \mathcal{C}^0(\R, \R),\ y \in \mathcal{C}^2(\R, \R)$. \\
    Soient $\alpha, \beta \in \R$, $m \in \N$, $P, P_1, P_2, Q_1, Q_2 \in \R[X]$. 
    \begin{itemize}
        \item \cite{exo7_analyse1} Si $b(x) = e^{\alpha x}(P_1(x) \cos(\beta x) + P_2(x) \sin(\beta x) )$
        \[y_{p} = x^{m} e^{\alpha x}(Q_1(x) \cos(\beta x) + Q_2(x) \sin(\beta x)) ,\ \deg(Q_1), \deg(Q_2) \leq \max\left\{ \deg(P_1), \deg(P_2) \right\} \]
        \[ m = \begin{cases}
            0 \text{ si } \alpha + i\beta \text{ n'est pas racine de l'équation caractéristique} \\
            1 \text{ sinon}
        \end{cases} \]
        \item Si $b(x) = P(x) e^{\alpha x}$ 
        \[ y_p = x^m Q(x)e^{\alpha x},\ \deg(Q) \leq \deg(P) \]
        avec $m$ l'ordre de multiplicité (voir \autoref{def:ordre_mult}) de la racine $\alpha$ par rapport à l'équation caractéristique associée.
\end{itemize}

\begin{remark}
    Il existe des propriétés analogues pour les équations différentielles du premier ordre, parfois cela est plus rapide qu'avec la variation de la constante.
\end{remark}

\begin{example}
    $(E) : y'' - 2y' + 3y = 9x^2 e^{2x} + 4e^x$
    Tout d'abord, résolvons l'équation homogène associée :
    \[ (E_0) : y'' - 2y' + 3y = 0 \]
    \[ \lambda^2 - 2\lambda + 3 = 0 \]
    \[ \Delta = (-2)^2 - 4 \cdot 3 = -8 < 0 \]
    \begin{align*}
        \lambda_1 &= \frac{2 - i\sqrt{8}}{2} & \lambda_2 &= \frac{2 + i\sqrt{8}}{2} \\
        &= 1 - i\sqrt{2} & &= 1 + i\sqrt{2}
    \end{align*}
    Ainsi les solutions de $(E_0)$ sont :
    \[ y_0 = e^x \left(C_1 \cos(\sqrt{2}x) + C_2 i \sin(\sqrt{2}x) \right) \]
    Trouvons une solution particulière de l'équation :
    \[ (E_1) : y'' - 2y' + 3y = 9x^2 e^{2x} \]
    On remarque que le second membre (le \og $b(x)$ \fg) est de la forme $P(x) e^{\alpha x}$ avec $\deg(P) = 2$ et $\alpha \notin \{1 \pm i\sqrt{2}\}$, ainsi la solution particulière est de la forme $y_1 = (ax^2 + bx + c)e^{2x},\ a, b, c \in \R$.
    On a donc :
    \begin{align*}
        \begin{cases}
            y_1 = (ax^2 + bx + c)e^{2x} \\
            y_1' = (2ax + b)e^{2x} + 2(ax^2 + bx + c)e^{2x} = (2ax + b)e^{2x} + 2y_1 \\
            y_1'' = 2a e^{2x} + 2(2ax + b)e^{2x} + 2y_1'
        \end{cases}
    \end{align*}
    En remplaçant dans $(E_1)$ :
    \begin{align*}
        2ae^{2x} + 2(2ax + b)e^{2x} + 2y_1' - 2[(2ax + b)e^{2x} + 2y_1] + 3y_1 &= 9x^2 e^{2x} \\
        2ae^{2x} + (4ax + 2b)e^{2x} + 2y_1' - (4ax + 2b)e^{2x} - 4y_1 + 3y_1 &= 9x^2 e^{2x} \\
        2ae^{2x}+ 2y_1' - y_1 &= 9x^2e^{2x} \\
        2ae^{2x} + 2[(2ax + b)e^{2x} + 2y_1] - y_1 &= 9x^2e^{2x} \\
        2ae^{2x} + (4ax + 2b)e^{2x} + 4 y_1 - y_1 &= 9x^2 e^{2x} \\
        2ae^{2x} + (4ax + 2b)e^{2x} + 3y_1 &= 9x^2e^{2x} \\
        2ae^{2x} + (4ax + 2b)e^{2x} + 3[(ax^2 + bx + c)e^{2x}] &= 9x^2 e^{2x} \\ 
        2ae^{2x} + (4ax + 2b)e^{2x} + (3ax^2 + 3bx + 3c)e^{2x} &= 9x^2 e^{2x} \\
        (2a + 4ax + 2b + 3ax^2 + 3bx + 3c)e^{2x} &= 9x^2 e^{2x} \\
        [3ax^2 + (4a + 3b)x + (2a + 2b + 3c)]e^{2x} &= 9x^2 e^{2x}
    \end{align*}
    On procède par identification :
    \begin{align*}
        \systeme{
            3a = 9,
            4a + 3b = 0,
            2a + 2b + 3c = 0
        }
        \iff 
        \systeme{
            a = 3,
            4a + 3b = 0,
            2a + 2b + 3c = 0
        }
        \iff 
        \systeme{
            a = 3,
            3b = -12,
            2a + 2b + 3c = 0
        }
        \\ 
        \iff 
        \systeme{
            a = 3,
            b = -4,
            2a + 2b + 3c = 0
        }
        \iff 
        \systeme{
            a = 3,
            b = -4,
            3c = 2
        }
        \iff 
        \systeme{
            a = 3,
            b = -4,
            c = \frac{2}{3}
        }
    \end{align*}
    \[ y_1 = \left(3x^2 - 4x + \frac{2}{3} \right)e^{2x} \]
    Maintenant trouvons une solution particulière de l'équation :
    \[ (E_2) : y'' - 2y' + 3y = 4e^x \]
    Ici on a encore une forme $P(x)e^{\alpha x}$ avec $P(x) = 1, \deg(P) = 0$ et $\alpha \notin \{ 1 \pm i \sqrt{2} \}$ ainsi $y_2 = k e^{x}, k \in \R$
    \begin{align*}
        \begin{cases}
            y_2 = k e^{x} \\
            y_2' = k e^{x} \\
            y_2'' = k e^{x}
        \end{cases}
    \end{align*}
    \begin{align*}
        k e^{x} -2k e^{x} + 3k e^{x} &= 4e^x \\ 
        2k e^x &= 4e^x \\
        k e^x = 2e^x
    \end{align*}
    \[ y_2 = 2e^x \]
    Solution générale :
    \begin{align*}
        y &= y_0 + y_1 + y_2 \\
        y &= e^x \left(C_1 \cos(\sqrt{2}x) + C_2 i \sin(\sqrt{2}x) \right) + \left(3x^2 - 4x + \frac{2}{3} \right)e^{2x} + 2e^x \\
        y &= \left(C_1 \cos(\sqrt{2}x) + C_2 i \sin(\sqrt{2}x) + 2 \right)e^x + \left(3x^2 - 4x + \frac{2}{3} \right)e^{2x}
    \end{align*} 
\end{example}
\chapter{Développements limités et formules de Taylor}

Dans cette partie, en l'absence de précisions supplémentaires, $I$ désigne un intervalle de $\R$.

\begin{notation}
    Dans cette section, on définit la notation suivante : 
    \[ \overline{I} \coloneqq I \cup \{\pm \infty\} \]
\end{notation}

\par Il arrive parfois des situations où l'on se retrouve avec des formes indéterminées de type \og $\frac{0}{0}$ \fg ou \og $\frac{\infty}{\infty}$ \fg lorsque nous essayons de calculer les limites, le théorème suivant permet de lever l'indétermination assez simplement.
Plus tard dans ce chapitre, nous pourrons également utiliser les développements limités pour lever les indéterminations.
\section{Règle de l'Hôpital}
\begin{theorem}[Règle de l'Hôpital]\cite{regle_hopital_bibmath}
    Soient $(a, b) \in \R^2,\ \ell \in \overline{\R}$ tels que $a < b$ et $f, g : ]a, b[ \to \R$ deux fonctions dérivables telles que $g'$ ne s'annule pas. 
    \begin{enumerate}
        \item Si $\lim_{x \to a} f(x) = \lim_{x \to a} g(x) = 0$ et si $\lim_{x \to a} \frac{f'(x)}{g'(x)} = \ell$, alors $\lim_{x \to a} \frac{f(x)}{g(x)} = \ell$.
        \item Si $\lim_{x \to a} g(x) = \pm \infty$ et si $\lim_{x \to a} \frac{f'(x)}{g'(x)} = \ell$, alors $\lim_{x \to a} \frac{f(x)}{g(x)} = \ell$.
    \end{enumerate}
\end{theorem}


\begin{example}
    $\lim_{x \to 0} \frac{\sin(2x)}{x}$, on a ici une forme indéterminée \og $\frac{0}{0}$ \fg.
    On remarque que $\sin(2x)$ et $x$ sont dérivables en 0 et $x' = 1 \neq 0$, on utilise donc la règle de l'Hôpital pour lever l'indétermination.
    \begin{align*}
        \lim_{x \to 0} \frac{\sin(2x)}{x} &= \lim_{x \to 0} \frac{\sin'(2x)}{x'} \\
        &= \lim_{x \to 0} \frac{2 \cos(x)}{1} \\
        &= 2 \lim_{x \to 0} \cos(x) \\
        &= 2
    \end{align*}
\end{example}

\begin{definition} 
    Soit $a \in \R$.
    On dit que $x$ est au voisinage d'un point $a$ si et seulement si :
    \[ \exists \varepsilon > 0, \text{ tel que } x \in ]a - \varepsilon, a + \varepsilon[ \]
\end{definition}

\section{Relations de négligeabilité, domination, d'équivalence}
\begin{definition}
    Soient $I \subset \R$, $f, g : I \to \R$, $a \in \overline{I}$ et $\varepsilon$ telle que $\lim_{x \to a} \varepsilon(x) = 0$.
    \begin{enumerate}
        \item On dit que $f$ est \textbf{dominée} par $g$ au voisinage de $a$ s'il existe un $B \in \R_+$ tel que $\abs{f(x)} \leq B \abs{g(x)}$ au voisinage de $a$. 
        On écrit alors $f \underset{a}{=} O(g)$ ou $f = O_a(g)$.
        \item On dit que $f$ est \textbf{négligeable} devant $g$ au voisinage de $a$ si $f(x) = g(x) \cdot \varepsilon(x)$.
        On écrit alors $f \underset{a}{=} o(g)$ ou $f = o_a(g)$.
        \item On dit que $f$ est \textbf{équivalente} à $g$ au voisinage de $a$ si $f(x) = g(x) (1 + \varepsilon(x))$.
        On écrit alors $f \underset{a}{\sim} g$.
    \end{enumerate}
\end{definition}

\begin{proposition}\leavevmode
    \begin{enumerate}
        \item Si $\lim_{x \to a} \frac{f(x)}{g(x)} = 0$ alors $f$ est négligeable devant $g$.
        \item Si $\lim_{x \to a} \frac{f(x)}{g(x)} = 1$ alors $f$ est équivalente à $g$.
        \item Si $\lim_{x \to a} \frac{f(x)}{g(x)}$ est bornée, alors $f$ est dominée par $g$.
    \end{enumerate}   
\end{proposition}

\begin{proposition}
\begin{multicols}{2}
    \begin{enumerate}
        \item $o(1) + o(1) = o(1)$
        \item $\forall \lambda \in \R,\ \lambda \cdot o(1) = o(1)$
        \item $\forall n \in \N^*,\ (o(1))^n = o(1)$
        \item $\forall \alpha > 0,\ (o(1))^{\alpha} = o(1)$
        \item $\forall \alpha \in \R,\ (1 + o(1))^{\alpha} = 1 + o(1)$
        \item $O(1) + O(1) = O(1)$
        \item $\forall \lambda \in \R,\ \lambda \cdot O(1) = O(1)$
        \item $\forall n \in \N^*,\ (O(1))^n = O(1)$
        \item $o(1) \cdot O(1) = o(1)$
    \end{enumerate}
\end{multicols}
\end{proposition}

\begin{proposition}
\begin{multicols}{2}
    \begin{enumerate}
        \item $\ln(x) \underset{+\infty}{\sim} o(x)$
        \item $\forall \alpha, \beta \in \R_+^*,\ (\ln(x))^{\beta} = o(x^{\alpha})$
        \item $\forall \alpha, \beta \in \R_+^*,\ x^{\beta} = o(e^{\alpha x})$
    \end{enumerate}
\end{multicols}
\end{proposition}

\begin{proposition}
    Soit $f$ une fonction polynomiale. 
    \begin{enumerate}
        \item Un équivalent de $f$ en l'infini est un son monôme de plus haut degré.
        \item Un équivalent de $f$ en 0 est son monôme de plus bas degré.
    \end{enumerate}
\end{proposition}

\section{Développements limités}

\begin{definition}[Polynôme de Taylor]
    Soit $f \in \mathcal{D}^n(I, \R)$ alors son polynôme de Taylor en $x_0$ est :
    \[ T_{n, x_0}^f(x) = \sum_{k = 0}^{n} \frac{f^{(k)}(x_0)}{k!} (x - x_0)^k \]
\end{definition}

\begin{theorem}[Formule de Taylor-Young]
    Soient $f \in \mathcal{C}^{n}(I, \R),\ x_0 \in I$.
    \begin{align*}
        f(x) = \sum_{k = 0}^{n} \frac{f^{(k)}(x_0)}{k!} (x - x_0)^k + o((x - x_0)^n) 
    \end{align*}
\end{theorem}

\begin{theorem}[Formue de Taylor-Lagrange]
    Soient $f \in \mathcal{C}^{n+1}(I, \R),\ x_0 \in I$. 
    \begin{align*}
        \exists c \in 
        \begin{cases}
            ]x_0, x[ \text{ si } x > x_0 \\
            ]x, x_0[ \text{ si } x < x_0
        \end{cases},\
        f(x) = \sum_{k = 0}^{n} \frac{f^{(k)}(x_0)}{k!} (x - x_0)^k + 
        \frac{f^{(n+1)}(c)}{(n+1)!} (x - x_0)^{n+1}
    \end{align*}
\end{theorem}

\begin{theorem}[Formule de Taylor avec reste intégral]
    Soient $f \in \mathcal{C}^{n+1}(I, \R),\ x_0 \in I$.
    \begin{align*}
        f(x) = \sum_{k = 0}^{n} \frac{f^{(k)}(x_0)}{k!} (x - x_0)^k + \int_{x_0}^{x} \frac{f^{(n+1)}(t)}{n!} (x - t)^n \diffd t
    \end{align*}
\end{theorem}

\begin{corollary}[Inégalité de Taylor-Lagrange]
    Si $\forall x \in I,\ \abs{f^{(n+1)}(x)} \leq M$ alors
    \[ \abs{\int_{x_0}^{x} \frac{f^{(n+1)}(t)}{n!} (x - t)^n \ \diffd t} \leq M \frac{\abs{x - x_0}^{n+1}}{(n+1)!} \]
\end{corollary}

\begin{definition}[Développement limité]
    Un polynôme $P_n(x)$ de degré $n$ satisfaisant :
    \[ f(x) = P_n(x) + o((x - x_0)^n) \]
    est un développement limité d'ordre $n$ de la fonction $f$.
\end{definition}

\begin{remark}
    Il est courant d'abréger développement limité par DL s'il n'y a pas d'ambiguïté.
\end{remark}

\begin{proposition}
    Si une fonction admet un développement limité, alors il est unique.
\end{proposition}

\subsection{Opérations sur les développements limités}
\begin{proposition}
    Soient $(c_0, \ldots, c_n) \in \R^{n+1}, (d_0, \ldots, d_n) \in \R^{n+1}$ et $f, g$ deux fonctions admettant des développements limités en 0 telles que :
    \[ f(x) = c_0 + c_1x + \cdots + c_n x^n + o(x^n) \]
    \[ g(x) = d_0 + d_1x + \cdots + d_n x^n + o(x^n) \]
    \begin{enumerate}
        \item Addition : $f + g$ admet un développement limité en 0 à l'ordre $n$.
        \[ f(x) + g(x) = (c_0 + d_0) + (c_1 + d_1)x + \cdots + (c_n + d_n) x^n + o(x^n) \]
        \item Multiplication : $f \cdot g$ admet un développement limité en 0 à l'ordre $n$.
        \[ (c_0 + c_1 x + \cdots + c_n x^n) \cdot (d_0 + d_1 x + \cdots + d_n x^n) \]
        où l'on conserve les monômes de degré inférieur ou égal à $n$.
        \item Composition : Si $g(0) = 0$ alors la fonction $f \circ g$ admet un développement limité en 0 à l'ordre $n$.\\
        Posons $C(x) \coloneqq c_0 + c_1x + \cdots + c_n x^n$ et $D(x) \coloneqq d_0 + d_1x + \cdots + d_n x^n$.\\
        Sa partie polynomiale est le polynôme tronqué à l'ordre $n$ (on conserve les monômes de degré inférieur ou égal à $n$) de la composition $C(D(x))$. 
    \end{enumerate}
\end{proposition}


\part{Algèbre}
\chapter{Calcul Algèbrique}
\par \noindent Dans cette partie, $\K$ désigne soit $\N$, soit $\Z$, soit $\Q$, soit $\R$.

\begin{axiom}[Loi de composition \og $+$ \fg]
    $\forall a, b, c \in \K,\ \K \backslash \{\N\}$ 
    \begin{align*}
        a + (b + c) &= (a + b) + c & a + b &= b + a \\
        a + 0 &= a & \exists a' \in \K,\ a + a' &= 0 \implies a' = -a
    \end{align*}
\end{axiom}

\begin{axiom}[Loi de composition \og $\cdot$ \fg]
    $\forall a, b, c \in \K$ 
    \begin{align*}
        a \cdot (b \cdot c) &= (a \cdot b) \cdot c & a \cdot b &= b \cdot a \\
        a \cdot 1 &= a & a \cdot 0 &= 0 \\
        a \cdot (b + c) &= a \cdot b + a \cdot c & (a + b) \cdot c &= a \cdot c + b \cdot c 
    \end{align*}
\end{axiom}

\begin{proposition}[Opérations sur les fractions]
    $\forall a, c \in \Z,\ \forall b, d \in \Z^*$
    \begin{align*}
        \frac{a}{b} + \frac{c}{d} &= \frac{ad + bc}{bd} & \frac{a}{b} \cdot \frac{c}{d} &= \frac{a \cdot c}{b \cdot d}
    \end{align*}
\end{proposition}

\begin{definition}[Somme]
    $\forall m, n \in \N,\ m \leq n,\ \forall a_k \in \R,\ m \leq k \leq n$ 
    \[ \sum_{k = m}^n a_k = a_m + a_{m + 1} + \cdots + a_n \]
\end{definition}

\begin{notation}
    La somme est parfois notée de cette manière :
    \[ \sum_{k = m}^n \equiv \sum_{m \leq k \leq n} \]
\end{notation}

\begin{proposition}[Linéarité de la somme]
    $\forall m, n \in \N,\ m \leq n,\ \forall a_k, \lambda \in \R,\ m \leq k \leq n$
    \[ \sum_{k = m}^{n} (a_k + \lambda b_k) = \sum_{k = m}^n a_k + \lambda \sum_{k = m}^n b_k \]
\end{proposition}

\begin{proposition}[Somme téléscopique]
    $\forall m, n \in \N,\ m \leq n,\ \forall a_k \in \R,\ m \leq k \leq n$
    \[ \sum_{k = m}^n (a_k - a_{k - 1}) = a_n - a_{m - 1} \]
\end{proposition}

\begin{proposition}
	$\forall n, p \in \N,\ n \geq p$,
	\[ \binom{n}{p} = \frac{n!}{p!(n - p)!} \]
	$\forall n \in \N$,
	\begin{align*}
		\sum_{k = 0}^{n} k &= \frac{n(n+1)}{2} & 
		\sum_{k = 0}^{n} k^2 &= \frac{n(n+1)(2n+1)}{6} &
		\sum_{k = 0}^{n} k^3 &= \frac{n^2(n+1)^2}{4}
	\end{align*}
	$\forall a, b \in \K,\ n \in \N$,
	\begin{align*}
		(a + b)^n = \sum_{k = 0}^{n} \binom{n}{k} a^kb^{n-k}
	\end{align*}
\end{proposition}
\begin{proof}
    Pour les démonstrations, on procède par interprétation combinatoire et par récurrence.
\end{proof}

\begin{definition}[Produit]
    $\forall m, n \in \N,\ m \leq n, a_k \in \R, m \leq k \leq n$
    \[ \prod_{k = m}^{n} \equiv \prod_{m \leq k \leq n} = a_1 \cdot a_2 \cdot \ldots \cdot a_n \]
\end{definition}
\chapter{Ensembles}
\def\arraystretch{1}

Nous allons tout d'abord donner une définition intuitive d'un ensemble : Un ensemble $E$ est une collection d'objets appelés éléments. Si $E$ contient un élément $x$, on dit que $x$ appartient à $E$, noté $x \in E$.

\begin{definition}[Ensemble vide]
  L'ensemble vide noté $\varnothing$ est l'ensemble ne contenant aucun élément.
\end{definition}

\begin{definition}[Inclusion]
  Soient $E$ et $F$ deux ensembles.
  $$F \subseteq E \iff \forall x \in F,\ x \in E$$
  On dit que $F$ est inclu dans $E$.
\end{definition}

\begin{definition}[\'Egalité d'ensembles]
  Soient $E$ et $F$ deux ensembles.
  $$E = F \iff E \subset F \text{ et } F \subset E$$
\end{definition}

\begin{definition}[Singleton]
  Un singleton est un ensemble ne contenant qu'un seul élément.
\end{definition}

\begin{definition}[Réunion d'ensembles]
  Soient $E$ et $F$ deux ensembles.
  \[ E \cup F = \{ x \in E \cup F \ | \ x \in E \text{ ou } x \in F \} \]
  On lit \og $E$ union $F$ \fg.
\end{definition}

\begin{definition}[Intersection d'ensembles]
  Soient $E$ et $F$ deux ensembles.
  \[ E \cap F = \{ x \in E \cap F \ | \ x \in E \text{ et } x \in F \} \]
  On lit \og $E$ inter $F$ \fg.
\end{definition}

\begin{definition}[Complémentaire d'un ensemble]
  Soient $E$ et $F$ deux ensembles.
  \[ E \backslash F = \{ x \in E \backslash F \ | \ x \in E,\ x \notin F \} \]
\end{definition}

\begin{proposition}
	Soient $A, B, C$ et $E$ des ensembles.
	\begin{enumerate}
		\item La réunion et l'intersection sont commutatives et assocatives.
		\item \'Elément neutre :
		\begin{enumerate}
			\item $A \cup \varnothing = A$.
			\item $A \cap A = A$.
		\end{enumerate}
		\item $A \subseteq E \iff A \cap E = E \cap A = A$.
		\item Distributivité :
		\begin{enumerate}
			\item $A \cup (B \cap C) = (A \cup B) \cap (A \cup C)$.
			\item $A \cap (B \cup C) = (A \cap B) \cup (A \cap C)$.
		\end{enumerate}
	\end{enumerate}
\end{proposition}

\begin{proposition}[Lois de Morgan]
  Soient $A$ et $B$ deux ensembles.
  \begin{align*}
    (A \cup B)^{C} &= A^{C} \cap B^{C} & (A \cap B)^{C} &= A^{C} \cup B^{C}
  \end{align*}
\end{proposition}

\begin{definition}[Produit cartésien]
    Soient $E$ et $F$ des ensembles. On définit le produit cartésien :
    \[ E \times F \coloneqq \{ (x, y),\ x \in E,\ y \in F \} \]
    Par convention : $\underbrace{E \times \cdots \times E}_{n \text{ fois}} = E^n$.
\end{definition}

\chapter{Logique}
\noindent Notations logiques :
\begin{enumerate}
	\item $\neg$ : \og non \fg.
	\item $\land$ : \og et \fg.
	\item $\lor$ : \og ou \fg.
	\item $\veebar$ : \og ou exclusif \fg.
\end{enumerate}

\begin{definition}[Assertion]
  Une assertion est une affirmation mathématique soit vraie soit fausse.
\end{definition}

\begin{definition}[Prédicat]
  Un prédicat est un énoncé mathématique dont la véracité dépend d'une ou plusieurs variables.
\end{definition}

Soient $P$ et $Q$ deux prédicats.
\begin{table}[!h]
	\centering
	\begin{tabular}{cccccc}
		\toprule
		$P$ & $Q$ & $P \land Q$ & $P \veebar Q$ & $\neg P$ & $P \implies Q$ \\
		\midrule
		$V$ & $V$ & $V$ & $V$ & $F$ & $V$ \\
		$V$ & $F$ & $F$ & $V$ & $F$ & $F$ \\
		$F$ & $V$ & $F$ & $V$ & $V$ & $V$ \\
		$F$ & $F$ & $F$ & $F$ & $V$ & $V$ \\
		\bottomrule
	\end{tabular}
	\caption{Table de vérité}
\end{table}

\begin{proposition}
  Soient $P$ et $Q$ deux prédicats.  
	\begin{enumerate} 
		\item $(P \implies Q) \land (Q \implies P) \implies (P \iff Q)$.
		\item $P \implies Q \iff \neg P \lor Q$.
		\item $\neg (P \lor Q) \iff \neg P \land \neg Q$.
		\item $\neg (P \land Q) \iff \neg P \lor \neg Q$.
		\item $P \implies Q \iff \neg Q \implies \neg P$.
	\end{enumerate}
\end{proposition}

\begin{remark}
    Les 2. et 3. sont les lois de Morgan, la 4. est la contraposée.
\end{remark}

\par \noindent Voici quelques négations usuelles :
\begin{itemize}
    \item Le contraire de \og $\forall x \in E, P(x)$ \fg est \og $\exists x \in E, \neg P(x)$ \fg.
    \item Le contraire de \og $x < y$ \fg est \og $x \geq y$ \fg.
\end{itemize}
\chapter{Nombres complexes}

On définit l'ensemble des nombres complexes, noté $\C$, comme une extension de l'ensemble des nombres réels.
Cette extension introduit un nouvel élément, noté $i$, appelé \emph{nombre imaginaire} et défini comme $i^2=-1$.

\subsection{Vision algébrique des nombres complexes}
\begin{definition}[Forme algébrique d'un nombre complexe]
    Soient $a, b \in \R$ et $z \in \C$, on appelle \emph{forme algébrique} de $z$ l'expression $z = a + ib$.
    \\
    $a$ est appelé \og \emph{partie réelle} \fg, notée $\Re(z)$ et $b$ est appelé \og \emph{partie imaginaire} \fg, notée $\Im(z)$.
\end{definition}

\begin{definition}[Module d'un nombre complexe]
    Soit $z = a + ib$ avec $a, b \in \R$. 
    \\
    On définit $\abs{z}$ tel que :
    \[ \abs{z} = \sqrt{a^2 + b^2} \]
    qu'on appelle \emph{module} de $z$.
\end{definition}

\begin{definition}[Conjugué d'un nombre complexe]
    Soit $z = a + ib$ avec $a, b \in \R$. On appelle conjugué de $z$ qu'on note $\overline{z}$ tel que :
    \[ \overline{z} = a - ib \]
\end{definition}

\begin{proposition}
    Soient $z_1, z_2 \in \C$.
    \begin{multicols}{3}
        \begin{enumerate}
            \item $\abs{z_1 + z_2} \leq \abs{z_1} + \abs{z_2}$.
            \item $\abs{z_1 - z_2} \geq \abs{z_1} - \abs{z_2}$.
            \item $\abs{z_1 \cdot z_2} = \abs{z_1} \cdot \abs{z_2}$.
        \end{enumerate}
    \end{multicols}
\end{proposition}


\chapter{Arithmétique}
\begin{definition}
    Soient $a \in \Z, b \in \Z^*$.
    \[ a \text{ est un multiple de } b \iff b \text{ est un diviseur de } a \iff b \mid a \iff \exists q \in \Z, a = bq \]
\end{definition}

\section{Divisibilité}

\begin{theorem}[Division euclidienne]
    Soient $a \in \Z, b \in \Z^*$.
    \[ \exists ! (q, r) \in \Z^2, a = bq + r, (0 \leq r < \abs{b}) \]
\end{theorem}

\begin{proof}\cite{livre_prepa}
	\leavevmode
	\begin{enumerate}
		\item \emph{Existence :} Supposons $a \in \N$ et considérons M = $\{n \in \N : nb \leq a\}$ l'ensemble des multiples de $b$ inférieurs à $a$. $M$ est une partie de $\N$. Nous avons deux propriétés : 
		\begin{enumerate}
			\item $M$ est non vide car 0 est un multiple de $b$ inférieur à $a$.
			\item $M$ est majoré par $a$ d'après sa définition.
		\end{enumerate}
		Ainsi $M$ admet un plus grand élément que l'on note $q$, vérifiant :
		\begin{enumerate}
			\item $qb \leq a$ car $q \in M$ 
			\item $(q + 1)b > a$ car $q + 1 > q$ sachant que $q$ est le plus grand élément de $M$, $q + 1 \notin M$.
		\end{enumerate}
		Posons : $r \coloneqq a - bq$. Sachant que $a \geq bq,\ r \geq 0$. On a $r < b$ car $b = (q + 1)b - qb > a - bq = r$. Supposons que $a \in \Z$. 
		\begin{enumerate}
			\item Si $a$ est positif, on se ramène au cas précédent.
			\item Dans le cas où $a < 0,\ -a \geq 0$, ainsi il existe $(q', r') \in \Z^2$ tel que :
		\[ -a = bq' + r',\ 0 \leq r' < \abs{b} \]
		\[a = b(-q') - r' \]
			\begin{enumerate}
				\item Si $r' = 0$, on pose $q = -q'$ et $r = 0$ et on obtient le couple recherché.
				\item Si $r' \neq 0$, $r' \in \llbracket 1, b-1 \rrbracket$ et $a = b(-q' -1) + (b - r')$, on pose $q = -q' - 1$ et $r = b - r'$ et on obtient le couple recherché. 
			\end{enumerate}
		\end{enumerate}
		\item \emph{Unicité :} Soit $(q, q', r, r') \in \Z^4$. \\
	On a d'une part : $a = bq + r$ et d'autre part : $a = bq' + r'$.
	On sait que $0 \leq r < b$ et $0 \leq r' < b$ donc :
	\[ b \abs{q' - q} = \abs{r' - r} < b \]
	ce qui n'est possible que si $\abs{q' - q} = 0$ ce qui impliquerait $q = q'$. Ceci entraîne donc $r = r'$.
	\end{enumerate}
\end{proof}

\begin{nomenclature}
    Pour $a, b, c, d$ définis comme dans le théorème précédent.
    \begin{multicols}{2}
        \begin{itemize}
        \item $a$ est appelé le \emph{dividende}
        \item $b$ est appelé le \emph{diviseur}
        \item $q$ est appelé le \emph{quotient}
        \item $r$ est appelé le \emph{reste}
    \end{itemize}
    \end{multicols}
\end{nomenclature}

\section{PGCD et PPCM}

\begin{definition}
	Soit $(a, b) \in (\Z^*)^2$. 
	\begin{enumerate}
		\item L'ensemble des diviseurs de $\N^*$ commun à $a$ et $b$ admet un plus grand élément. C'est le \emph{plus grand commun diviseur} des entiers $a$ et $b$. On le note $\pgcd(a,b)$.
		\item L'nesemble des diviseurs de $\N^*$ commun à $a$ et $b$ admet un plus petit élément. C'est le \emph{plus petit commun multiple} des entiers $a$ et $b$. On le note $\ppcm(a, b)$.
	\end{enumerate}
\end{definition}

\begin{theorem}
	Soit $(a, b, d) \in (\Z^*)^2 \times \Z$.
	\begin{enumerate}
		\item $a \mid d \text{ et } b \mid d \implies \ppcm(a, b) \mid d$
		\item $d \mid a \text{ et } d \mid b \implies d \mid \pgcd(a, b)$
	\end{enumerate}
\end{theorem}

\begin{proof}
	\leavevmode
	\begin{enumerate}
		\item Posons $\ell \coloneqq \ppcm(a,b)$. 
		\[ \exists ! (q,r) \in \Z^2,\ d \coloneqq q \ell + R,\ 0 \leq r < \ell \]
		\[ r \coloneqq d - q \ell,\ d \text{ et } \ell \text{ sont multiples de } a \text{ et } r \text{ est aussi un multiple de } a \text{ et } b \]
		Par la minimalité de $\ell,\ r = 0 \implies m = q \ell$.
		\item Posons $m = \pgcd(a, b)$. Montrons que :
		\[ \pgcd(m, d) = m \]
		Soit $\ell \coloneqq \ppcm(m,d)$, $\ell \geq m$, $a$ et $b$ sont multiples de $m$ et $d$. D'après 1. :
		\[ \ell \mid a \text{ et } \ell \mid b,\ \ell \leq m \]
		Sachant que $\ell \geq m$ et $\ell \leq m$, $\ell = m$.
	\end{enumerate}
\end{proof}

\begin{definition}
	Soit $(a,b) \in (\Z^*)^2$. On dit que $a$ et $b$ sont \emph{premiers entre eux} si et seulement si $\pgcd(a, b) = 1$.
\end{definition}

\section{Algorithme d'Euclide}
\begin{proposition}[Algorithme d'Euclide]
	Soit $(a, b) \in (\Z^*)^2$ tel que $\abs{a} > \abs{b}$.
	\[ \exists ! (q,r) \in \Z^2,\ a = bq + r,\ 0 \leq r < \abs{b} \]
	\[ \pgcd(a, b) = \pgcd(b, a) = \pgcd(b, a - qb) = \pgcd(b, r) \]
	\begin{enumerate}
		\item Si $r = 0$ alors $a = qb$ et donc $\pgcd(a, b) = b$.
		\item Si $r \neq 0$ alors :
		\[ \exists ! (q_1, r_1) \in \Z^2,\ b = q_1 r + r_1,\ 0 \leq r_1 < r \]
		Ensuite : 
		\begin{enumerate}
			\item Si $r_1 = 0$ alors $b = q_1 r$ et donc $\pgcd(a, b) = r$.
			\item Si $r_1 \neq 0$ alors :
			\[ \exists ! (q_2, r_2) \in \Z^2,\ r = q_2 r_1 + r_2,\ 0 \leq r_2 < r_1 \]
		\end{enumerate}
	\end{enumerate}
	
	On procède de cette manière jusqu'à obtenir un reste nul, le $\pgcd$ de $a$ et $b$ est le dernier reste non nul.
\end{proposition}

\begin{theorem}[Identité de Bézout]
	Soit $(a, b) \in \Z^2$.
	\[ \exists (u, v) \in \Z^2,\ au + bv = \pgcd(a, b) \]
\end{theorem}

\begin{corollary}
	Soient $(a, b) \in (\Z^*)^2$ et $d \in \Z$.
	\[ \exists (u, v) \in \Z^2,\ au + bv = d \iff \pgcd(a,b) \mid d \]
\end{corollary}

\begin{lemma}[Lemme de Gauss]
	Soit $(a, b) \in (\Z^*)^2$ tel que $\pgcd(a, b) = 1$.
	\[ \forall c \in \Z,\ a \mid bc \implies a \mid c \]
\end{lemma}

\begin{proof}
	\begin{align*}
		\pgcd(a, b) = 1 &\implies \exists(u, v) \in \Z^2,\ au + bv = 1 \\
		&\implies a(cu) + b(cv) = c \\ 
		&\implies \pgcd(a, bc) \mid c
	\end{align*}
\end{proof}

\section{Nombres premiers}
\begin{definition}[Nombre premier]
	Soit $p \in \N^*$. On dit que $p$ est \emph{premier} si et seulement s'il admet exactement deux diviseurs : $1$ et $p$.
\end{definition}

\begin{theorem}[Théorème d'Euclide]
	Il existe une infinité de nombres premiers.
\end{theorem}

\begin{proof}
	Supposons qu'il existe $k$ nombres premiers $p_1, \ldots, p_k$.
	\[ N \coloneqq p_1 \cdots p_k + 1 \implies p_i \nmid N \]
\end{proof}

\begin{lemma}
	Soit $n \in \N$ tel que $n \geq 2$.
	\\
	Si $p$ est le plus petit diviseur de $n$ tel que $p > 2$ alors $p$ est premier.
\end{lemma}

\begin{theorem}[Décomposition en facteurs premiers]
	Soit $n \in \N^*$ tel que $n \geq 2$. 
	\\
	Il existe une unique écriture de $n$ sous la forme de :
	\[ p_1^{\alpha_1} \cdots p_k^{\alpha_k} \]
	\begin{enumerate}
		\item Pour $1 \leq i \leq k$, les $p_i$ sont premiers.
		\item $\alpha_i \in \N^*$.
		\item $p_1 < p_2 < \cdots < p_k$.
	\end{enumerate}
\end{theorem}

\begin{proposition}
	Soient $(a, b) \in \Z^2$ et $(i, k) \in \N^2$. Pour déterminer $\pgcd(a, b)$ on peut utiliser leurs décompositions en facteurs premiers.
	\[ a = p_1^{\alpha_1} \cdots p_k^{\alpha_k} \]
	\[ b = n_1^{\beta_1} \cdots n_i^{\beta_i} \]
	$\pgcd(a, b)$ correspond au produit des facteurs premiers communs.
\end{proposition}

\section{Congruences}
\begin{definition}
	Soient $(a,b) \in \Z^2$ et $n \in \N$ tel que $n \geq 2$.
	\\
	On dit que $a$ et $b$ sont \emph{congrus modulo} $n$ s'il existe un $k$ tel que :
	\[ a - b = k n \]
	On écrit généralement $a \equiv b \mod n$ ou $a \equiv b \ [n]$.
\end{definition}

\begin{proposition}
	Soient $(a, b, c, d) \in \Z^4$ et $(k, n) \in \N^2$ tel que $n \geq 2$.
	\begin{enumerate}
		\item $a \equiv a \mod n$ 
		\item $a \equiv b \mod n \iff b \equiv a \mod n$ 
		\item $a \equiv b \mod n \text{ et } b \equiv c \mod n \implies a \equiv c \mod n$
		\item $a \equiv b \mod n \text{ et } c \equiv d \mod n \implies a + c \equiv b + d \mod n \text{ et } ac \equiv bd \mod n$
		\item $a \equiv b \mod n \implies a^k \equiv b^k \mod n$
	\end{enumerate}
\end{proposition}

\begin{proof}
	Immédiate en utilisant la définition de la congruence.
\end{proof}

\begin{theorem}
	Soient $(m_1, m_2) \in \N^2$ tel que :
	\begin{enumerate}
		\item $m_1 > 1$
		\item $\pgcd(m_1, m_2) = 1$
	\end{enumerate}
	Soient $(a_1, a_2) \in \Z^2$ tel que pour $x \in \Z$ :
	\begin{enumerate}
		\item $x \equiv a_1 \mod m_1$
		\item $x \equiv a_2 \mod m_2$
	\end{enumerate}
	Notons $\mathcal{S}$ l'ensemble des solutions :
	$\exists (k_0, k) \in \Z^2$ tel que :
	\begin{enumerate}
		\item $\mathcal{S} = k_0 + k m_1 m_2$ 
		\item $0 \leq k_0 < m_1 m_2$
	\end{enumerate}
\end{theorem}

\begin{remark}
	Ce théorème est un cas particulier du théorème des restes chinois.
\end{remark}
\chapter{Polynômes}

\section{Définitions}
\begin{definition}[Polynôme]
    Un \emph{polynôme} est un élément de l'ensemble 
    \begin{align*}
        \K[X] = \{ a_0 1 + a_1 X + a_2 X^2 + \cdots a_n X^n \mid a_i \in \K,\ n \in \N \}
    \end{align*}
    Soit $P \in \K[X]$, si $a_n \neq 0$, $n$ est le degré du polynôme, on le note 
    $ \deg(P) = n. $
\end{definition}

\begin{proposition}
	$\forall \lambda \in \K^*,\ (P, Q) \in (\K[X])^2$. 
	\begin{multicols}{2}
        \begin{enumerate}
    		\item $\operatorname{deg}(\lambda) = 0$.
    		\item $\operatorname{deg}(P \cdot Q) = \operatorname{deg}(P) + \operatorname{deg}(Q)$.
    		\item $\operatorname{deg}(P + Q) = \mathrm{max}\left(\operatorname{deg}(P),\ \operatorname{deg}(Q)\right)$.
            \item $\operatorname{deg}(0) = -\infty$.
    	\end{enumerate}
    \end{multicols}
\end{proposition}

\section{Arithmétique des polynômes}

\begin{theorem}[Division euclidienne de polynômes]
	Soient $P_1, P_2$ deux polynômes non nuls.
	\begin{align*}
		\exists ! (Q, R) \in (\K[X])^2 \text{ tel que } P_1 = P_2 Q + R 
	\end{align*}
	Vérifiant :
	$\operatorname{deg}(R) = -\infty$ ou $0 \leq \operatorname{deg}(R) < \operatorname{deg}(Q)$.
\end{theorem}

\begin{definition}[Polynôme irréductible]
	Un polynôme $P \in \K[X]$ non constant est dit irréductible, s'il n'existe pas $(P_1, P_2) \in (\K[X])^2$ tel que $P = P_1 P_2$ et $\operatorname{deg}(P_1) < \operatorname{deg}(P_2)$.
\end{definition}

\begin{definition}\label{def:ordre_mult}
	Soient $P$ un polynôme non constant, $\alpha \in \K$ et $m \in \N^*$ $\alpha$ est une racine d'ordre de multiplicité $m$ de $P$ si et seulement si
	\begin{align*}
		(X - \alpha)^m \mid P \text{ et } (X - \alpha)^{m+1} \nmid P.
	\end{align*}
\end{definition}

\begin{theorem}
	Soient $P$ un polynôme non constant, $\alpha \in \K$ et $m \in \N^*$
	\begin{align*}
		(X - \alpha)^m \mid P \iff P(\alpha) = P'(\alpha) = \cdots = P^{m-1}(\alpha) = 0.
	\end{align*}
\end{theorem}

\begin{theorem}[Théorème fondamental de l'algèbre]
	Soit $P(X)$ un polynôme à coefficients complexes de degré $n$. $P(X)$ admet $n$ racines complexes. 
\end{theorem}

\section{Fractions rationnelles}
\begin{definition}[Fraction rationnelle]
	$F(X)$ est appelée fraction rationnelle s'il existe $P(X), Q(X) \in \K[X]$ tels que :
	\begin{align*}
		F(X) = \frac{P(X)}{Q(X)},\ Q(X) \neq 0
	\end{align*}
	avec $\deg(F) = \deg(P) - \deg(Q)$. \\
	Si $\deg(P) > \deg(Q)$, alors il existe $E(X)$ appelée la \emph{partie entière} et des polynômes $R(X), S(X)$ tels que $\deg(R) < \deg(S)$ et : 
	\begin{align*}
		F(X) = E(X) + \frac{R(X)}{S(X)}.
	\end{align*}
\end{definition}

\begin{theorem}[Décomposition en éléments simples]
	Soient $F(X) = \frac{P(X)}{Q(X)}$ et $m, n, \alpha, \beta \in \N^*$.
	\\
	Sur $\R$ :
	\begin{align*}
		Q(X) &= (X - a_1)^{\alpha_1} \ldots (X - a_n)^{\alpha_n} (X^2 + b_1 X + c_1)^{\beta_1} \ldots (X^2 + b_m X + c_m)^{\beta_m} \\ 
            &= \prod_{i = 0}^{n} (X - a_i)^{\alpha_i} \cdot \prod_{k = 0}^{m} (X^2 + b_k X + c_k)^{\beta_k}
	\end{align*}
	Alors $\forall A, B, C \in \R$, $F$ s'écrit de manière unique sous la forme :
	\begin{align*}
		F(X) &= \left[ \left( \frac{A_{11}}{(X - a_1)^1} + \cdots + \frac{A_{1n}}{(X - a_1)^{\alpha_1}} \right) + \cdots + \left( \frac{A_{n1}}{(X - a_n)^1} + \cdots + \frac{A_{nn}}{(X - a_n)^{\alpha_n}}  \right) \right] \\
		+ &\left[ \left( \frac{B_{11}X + C_{11}}{(X^2 + b_1 X + c_1)^1} + \cdots + \frac{B_{1m}X + C_{1m}}{(X^2 + b_1 X + c_1)^{\beta_1}} \right) + \cdots + \right. \\
		&\left. \left( \frac{B_{m1}X + C_{m1}}{(X^2 + b_m X + c_m)^1} + \cdots + \frac{B_{mm}X + C_{mm}}{(X^2 + b_m X + c_m)^{\beta_m}}  \right) \right] \\
        F(X) &=  \sum_{i = 1}^{n} \sum_{j = 1}^{\alpha_i} \frac{A_{ij}}{\left(X - a_i\right)^{j} } + \sum_{k = 1}^{m} \sum_{l = 1}^{\beta_k} \frac{B_{kl}X + C_{kl}}{(X^2 + b_kX + c_k)^{l}}
	\end{align*}
	Sur $\C$ :
	\begin{align*}
		Q(X) &= (X - a_1)^{\alpha_1} \ldots (X - a_n)^{\alpha_n} \\
            &= \prod_{i = 0}^{n} (X - a_i)^{n}
	\end{align*}
	Alors $\forall A, B, C \in \C, \ \forall \alpha \in \N^*$  $F$ s'écrit de manière unique sous la forme :
	\begin{align*}
		F(X) &= \left[ \left( \frac{A_{11}}{(X - a_1)^1} + \cdots + \frac{A_{1n}}{(X - a_1)^{\alpha_1}} \right) + \cdots + \left( \frac{A_{n1}}{(X - a_n)^1} + \cdots + \frac{A_{nn}}{(X - a_n)^{\alpha_n}}  \right) \right] \\
        F(X) &=  \sum_{i = 1}^{n} \sum_{j = 1}^{\alpha_i} \frac{A_{ij}}{\left(X - a_i\right)^{j} }
	\end{align*}    
\end{theorem}

\begin{proof}\cite{math-sup.fr_decomp}
    Sur $\C$ : il est recommandé de consulter la section sur les espaces vectoriels avant (voir \autoref{chap:espaces_vectoriels}).
    Rappelons tout d'abord :
    \[ \C[X] \coloneqq \left\{ \sum_{k = 0}^{n} a_k X^k : n \in \N,\ a_k \in \C \right\} \]
    Notons l'espace des fractions rationnelles :
    \[ \C(X) \coloneqq \left\{ \frac{P(X)}{Q(X)} : P, Q \in \C[X],\ Q \neq 0_{\C[X]} \right\} \]
    La stratégie de la démonstration consiste à montrer une égalité de deux ensembles :
    \[ E = \left\{ \frac{P(X)}{Q(X)} : P \in \C[X],\ \deg(P) < \deg(Q) \right\} \]
    \[ F = \left\{ \sum_{i = 1}^{n} \sum_{j = 1}^{m_i} \frac{a_{ij}}{(X - \alpha_i)^{j}} : a_{ij} \in \C \right\} \]
    \emph{\'Etude de $E$ :} Posons $d \coloneqq \deg(Q)$
    \begin{align*}
        E &= \left\{ \frac{P(X)}{Q(X)} : P \in \C[X],\ \deg(P) < d \right\} \\
          &= \left\{ \frac{a_0 + a_1X + \cdots + a_{d - 1} X^{d - 1}}{Q(X)} : a_0, \ldots, a_{d - 1} \in \C \right\} \\
          &= \left\{ a_0 \frac{1}{Q(X)} + a_1 \frac{X}{Q(X)} + \cdots + a_{d - 1} \frac{X^{d-1}}{Q(X)} : a_0, \ldots, a_{d - 1} \in \C \right\} \\
          &= \operatorname{Vect}\left\{ \frac{1}{Q(X)}, \frac{X}{Q(X)}, \ldots, \frac{X^{d-1}}{Q(X)} \right\}
    \end{align*}
    Notons : $\mathcal{F} \coloneqq \operatorname{Vect}\left\{ \frac{1}{Q(X)}, \frac{X}{Q(X)}, \ldots, \frac{X^{d-1}}{Q(X)} \right\}$
    Nous avons réussi à exprimer $E$ sous la forme d'une famille de vecteurs, nous en déduisons que $E$ est un espace vectoriel admettant $\mathcal{F}$ comme famille génératrice, montrons ensuite que $\mathcal{F}$ est libre :
    Soient $ \lambda_0,\ldots,\lambda_{d-1} \in \C$
    \begin{align*}
        \lambda_0 \frac{1}{Q(X)} + \cdots + \lambda_{d - 1} \frac{X^{d-1}}{Q(X)} &= \frac{\lambda_0 + \lambda_{1} X + \cdots + \lambda_{d - 1} X^{d - 1}}{Q(X)}
    \end{align*}
    \[ \frac{\lambda_0 + \lambda_{1} X + \cdots + \lambda_{d - 1} X^{d - 1}}{Q(X)} = 0_{\C(X)} \iff \lambda_0 + \lambda_{1} X + \cdots + \lambda_{d - 1} X^{d - 1} = 0_{\C[X]} \]
    Un polynôme est nul si tous ses coefficients sont nuls, ainsi :
    \[ \lambda_0 = \cdots = \lambda_{d-1} = 0 \]
    Ainsi $\mathcal{F}$ est libre. $\mathcal{F}$ est libre et génératrice, elle forme donc une base de $E$ et ainsi : 
    \[ \dim(E) = d \]
    \emph{\'Etude de F :}
    \[ F = \left\{ \sum_{i = 1}^{n} \sum_{j = 1}^{m_i} \frac{a_{ij}}{(X - \alpha_i)^{j}} : a_{ij} \in \C \right\} \]
    On peut ainsi écrire $F$ ainsi : 
    \begin{align*}
        F &= \left\{ \frac{a_{11}}{(X - \alpha_1)} + \cdots + \frac{a_{1m_1}}{(X - a_1)^{m_1}} + \cdots + \frac{a_{n1}}{(X - \alpha_n)} + \cdots + \frac{a_{nm_n}}{(X - \alpha_n)^{m_n}} : a_{ij} \in \C \right\} \\
        &= \operatorname{Vect} \left\{ \frac{1}{(X - \alpha_1)}, \ldots, \frac{1}{(X - \alpha_1)^{m_1}}, \ldots, \frac{1}{(X - \alpha_n)}, \ldots, \frac{1}{(X - \alpha_n)^{m_n}}  \right\}
    \end{align*}
    Notons $\mathcal{F}_2 \coloneqq \operatorname{Vect} \left\{ \frac{1}{(X - \alpha_1)}, \ldots, \frac{1}{(X - \alpha_1)^{m_1}}, \ldots, \frac{1}{(X - \alpha_n)}, \ldots, \frac{1}{(X - \alpha_n)^{m_n}}  \right\}$. Nous en déduisons que $F$ est un espace vectoriel admettant $\mathcal{F}_2$ comme famille génératrice, montrons que $\mathcal{F}_2$ est libre : Soient $a_{ij} \in \C$ tels que :
    \begin{align*}
        \begin{split}
            a_{11} \frac{1}{(X - \alpha_1)} + a_{12} \frac{1}{(X - \alpha_1)^2} + \cdots + a_{1(m_1 -1)} \frac{1}{(X - \alpha_1)^{m_1 - 1}} + \cdots \\
            + a_{n1} \frac{1}{(X - \alpha_n)} + a_{n2} \frac{1}{(X - \alpha_n)^2} + \cdots + a_{nm_n} \frac{1}{(X - \alpha_n)^{m_n}} = 0_{\C(X)}
        \end{split}
    \end{align*}
    En multipliant l'équation par $(X - \alpha_1)^{m_1}$ :
    \begin{align*}
        \begin{split}
            a_{11} (X - \alpha_1)^{m_1 - 1} + a_{12} (X - \alpha_1)^{m_1 - 2} + \cdots + a_{1m_1} + \\
            (X - \alpha_1)^{m_1}
            \left( 
            a_{21} \frac{1}{(X - \alpha_2)} + \cdots + a_{n1} \frac{1}{(X - \alpha_n)} + \cdots + a_{nm_n} \frac{1}{(X - \alpha_n)^{m_n}}
            \right)
            = 0_{\C(X)}
         \end{split}
    \end{align*}
    En posant $X = \alpha_1$, on trouve que $a_{1m_1} = 0$.
    En remplaçant $a_{1m_1}$ par sa valeur dans l'équation initiale, celle-ci devient : 
    \begin{align*}
        \begin{split}
            a_{11} \frac{1}{(X - \alpha_1)} + a_{12} \frac{1}{(X - \alpha_1)^2} + \cdots + a_{1(m_1 - 1)} \frac{1}{(X - \alpha_1)^{m_1 - 1}} + \cdots \\
            + a_{n1} \frac{1}{(X - \alpha_n)} + a_{n2} \frac{1}{(X - \alpha_n)^2} + \cdots + a_{nm_n} \frac{1}{(X - \alpha_n)^{m_n}} = 0_{\C(X)}
        \end{split}
    \end{align*}
    En multipliant l'équation par $(X - \alpha_1)^{m_1 - 1}$ et en posant $X = \alpha_1$, on trouve que $a_{1(m_1 - 1)} = 0$. On procède de la même manière jusqu'à prouver $a_{11} = \cdots = a_{1m_1} = 0$.
    On continue ainsi de suite pour montrer que tous les coefficients sont nuls et donc que $\mathcal{F}_2$ est libre. $\mathcal{F}_2$ est libre et génératrice, elle forme donc une base de $F$.
    \[ \dim(F) = m_1 + \cdots + m_n \]
    Mais aussi : 
    \[ \deg(Q) = \deg \left( \prod_{i=1}^{n} (X - \alpha_i)^{m_i} \right) = m_1 + \cdots + m_n \]
    Ainsi : \[ \dim(F) = \deg(Q) = d \]
    \emph{Inclusion de $F$ dans $E$ :}
    \\
    Un élément de $F$ est de la forme suivante : 
    \[ \frac{a_{11}}{(X - \alpha_1)} + \cdots + \frac{a_{1m_1}}{(X - \alpha_1)^{m_1}} + \cdots + \frac{a_{n1}}{(X - \alpha_n)} + \cdots + \frac{a_{nm_n}}{(X - \alpha_n)^{m_n}} \]
    En mettant toutes les fractions sur le même dénominateur, on obtient :
    \[ \frac{a_{11}(X - \alpha_1)^{m_1 - 1} \prod_{i = 2}^{n}(X - \alpha_i)^{m_i} + \cdots + a_{1m_1} \prod_{i = 2}^{n}(X - \alpha_i)^{m_i} + \cdots + a_{nm_n} \prod_{i = 1}^{n - 1}(X - \alpha_i)^{m_i}}{Q(X)} \]
    Le degré du numérateur est strictement inférieur à celui de $Q$. Donc c'est un élément de $E$. Ainsi :
    \[ F \subset E \]
    Nous avons montré que $\dim(E) = \dim(F)$ et que $F \subset E$, ainsi $E = F$.
    Ce qui revient à dire que toute fraction rationnelle 
    \begin{align*}
        \frac{P(X)}{Q(X)},\ \deg(P) < \deg(Q)
    \end{align*}
    s'exprime sous la forme 
    \begin{align*}
        \sum_{i=1}^n \sum_{j = 1}^{m_i} \frac{a_{ij}}{(X - \alpha_i)^j}
    \end{align*}
    L'unicité de la décomposition découle du fait que tout élément d'un espace vectoriel s'exprime par une unique combinaison linéaire des vecteurs d'une de ses bases.
\end{proof}
\chapter{Systèmes linéaires et matrices}
\def\arraystretch{1}

\section{Définitions et opérations élémentaires}
\begin{definition}[Matrice]
	Soient $(m,n) \in (\N^*)^2$ et $(a_{11}, \ldots, a_{mn}) \in \K^{mn}$. \\
    Une \textbf{matrice} est un tableau de données appartenant à $\mathcal{M}_{m,n} (\K)$ :
    \begin{align*}
        \begin{pmatrix}
            a_{11} & a_{12} & \cdots & a_{1n} \\
            a_{21} & a_{22} & \cdots & a_{2n} \\
            \vdots & \vdots & \ddots & \vdots \\
            a_{m1} & a_{m2} & \cdots & a_{mn}
        \end{pmatrix}
    \end{align*}
\end{definition}

\begin{definition}[Système linéaire]
    Soient $(m,n) \in (\N^*)^2$, $(x_1, \ldots, x_n) \in \K^n$, $(a_{11}, \ldots, a_{mn}) \in \K^{mn}$ et $(b_1, \ldots, b_m) \in \K^m$.
    Un \textbf{système linéaire} est décrit par :
    \begin{align*}
        \begin{cases}
            a_{11} \cdot x_1 + a_{12} \cdot x_2 + \cdots + a_{1n} \cdot x_n = b_1 \\
            a_{21} \cdot x_1 + a_{22} \cdot x_2 + \cdots + a_{2n} \cdot x_n = b_2 \\
            \vdots \\
            a_{m1} \cdot x_1 + a_{m2} \cdot x_2 + \cdots + a_{mn} \cdot x_n = b_m
        \end{cases}
    \end{align*}        
    Sa matrice associée est : 
    \begin{align*}
        \left(
        \begin{matrix}    
            a_{11} & a_{12} & \cdots & a_{1n} \\
            a_{21} & a_{22} & \cdots & a_{2n} \\
            \vdots & \vdots & \ddots & \vdots \\
            a_{m1} & a_{m2} & \cdots & a_{mn}
        \end{matrix}
        \
        \middle|
        \
        \begin{matrix}
            b_1 \\
            b_2 \\
            \vdots \\
            b_m
        \end{matrix}
        \right)
    \end{align*}
\end{definition}

\begin{definition}[Opérations élémentaires]
	Soient $i, j$ tels que $i \neq j$ des numéros de ligne et $\lambda \in \K^*$.
	\begin{multicols}{2}
        \begin{enumerate}
		\item $L_i \leftarrow L_i + \lambda L_j$ 
		\item $L_i \leftrightarrow L_j$
		\item $L_i \leftarrow \lambda \cdot L_i$
	\end{enumerate}
    \end{multicols}
	Opérations analogues sur les colonnes.
\end{definition}

\begin{definition}
	Une matrice est \textbf{échelonnée} en lignes si et seulement si le nombre de zéros commençant une ligne croît strictement ligne par ligne jusqu'à ce qu'il ne reste plus que des zéros.
\end{definition}

\begin{proposition}[Pivot de Gauss]
    L'algorithme sera décrit ici pour les lignes, l'énoncé pour les colonnes est analogue sauf qu'on applique les opérations sur les colonnes.
   	On applique les opérations élémentaires afin d'échelonner le système ou la matrice.
    \\
    Ensuite le but est d'appliquer les opérations élémentaires pour \og remonter \fg dans l'algorithme et d'obtenir une matrice ou un système de cette forme :
    \begin{align*}
        \begin{pmatrix}
            a_{11} & 0 & \cdots & 0 \\
            0 & a_{22} & \cdots & 0 \\ 
            \vdots & \vdots & \ddots & \vdots \\ 
            0 & 0 & \cdots & a_{mn}
        \end{pmatrix}
    \end{align*}
\end{proposition}

\begin{definition}[Rang d'une matrice]
	Le rang d'une matrice $A$ est son nombre de lignes non nulles après échelonnage. Il est noté $\rg(A)$.
\end{definition}

\begin{theorem}
    Soient $\mathcal{S}$ un système linéaire de $m$ lignes et $n$ inconnues, $A \in \mathcal{M}_{m,n}(\K)$ et $B \in \mathcal{M}_{m,1}(\K)$ telles que $A|B$ forme la matrice associée à $\mathcal{S}$. $\mathcal{S}$ est solvable si et seulement si :
    \[ \rg(A) = \rg(A|B) \]
\end{theorem}

\section{Opérations sur les matrices}
\begin{definition}[Opérations sur les matrices]
	Soient $(m,n,p,q) \in (\N^*)^4$, $A \in \mathcal{M}_{m, n}(\K)$, $B \in \mathcal{M}_{p, q}(\K)$, $\lambda \in \K$, $(a_{11}, \ldots, a_{mn}) \in \K^{mn}$ et $(b_{11}, \ldots, b_{pq}) \in \K^{pq}$ tels que :
	\begin{align*}
		A &=
		\begin{pmatrix}
			a_{11} & \cdots & a_{1n} \\
			\vdots & \ddots & \vdots \\
			a_{m1} & \cdots & a_{mn}
		\end{pmatrix}
		&
		B &= 
		\begin{pmatrix}
			b_{11} & \cdots & b_{1q} \\
			\vdots & \ddots & \vdots \\
			b_{p1} & \cdots & b_{pq}
		\end{pmatrix}
	\end{align*}
	\begin{enumerate}
		\item Si $m = p$ et $n = q$ alors on peut définir l'addition entre $A$ et $ B$ et la multiplication par $\lambda$.
		\begin{align*}
			A + \lambda B &= 
			\begin{pmatrix}
				(a_{11} + \lambda b_{11}) & \cdots & (a_{1n} + \lambda b_{1q}) \\
				\vdots & \ddots & \vdots \\
				(a_{m1} + \lambda b_{p1}) & \cdots & (a_{mn} + \lambda b_{pq})
			\end{pmatrix}
		\end{align*}
		\item Si $n = p$ alors on peut définir la multiplication entre $A$ et $B$.\\
		Soit $C = AB$. Chaque coefficient $C_{ij}$ est défini par :
		\begin{align*}
			C_{ij} = \sum_{k=1}^{n} a_{ik} b_{kj}
		\end{align*}
        Autrement dit, le coefficient $c_{ij}$ pour $C = AB$ est donné par :
        \begin{align*}
            &\begin{pmatrix}
                b_{11} & \cdots & \color{red}b_{1j} & \cdots & b_{1q} \\
                b_{21} & \cdots & \color{green}b_{2j} & \cdots & b_{2q} \\
                \vdots & \cdots & \vdots & \cdots & \vdots \\
                b_{p1} & \cdots & \color{blue}b_{pj} & \cdots & b_{pq}
            \end{pmatrix}
            \\
            \begin{pmatrix}
                a_{11} & a_{12} & \cdots & a_{1n} \\ 
                \vdots & \vdots & \vdots & \vdots \\
                \color{red}a_{i1} & \color{green}a_{i2} & \cdots & \color{blue}a_{in} \\ 
                \vdots & \vdots & \vdots & \vdots \\
                a_{m1} & a_{m2} & \cdots & a_{mn}
            \end{pmatrix}
            &\begin{pmatrix}
                c_{11} & c_{12} & \cdots & \cdots & c_{1q} \\
                \vdots & \vdots & \vdots & \vdots & \vdots \\
                \cdots & \cdots & \boxed{c_{ij}} & \cdots & c_{iq} \\
                \vdots & \vdots & \vdots & \vdots & \vdots \\
                c_{m1} & \cdots & \cdots & \cdots & c_{mq}
            \end{pmatrix}
        \end{align*}
        \[ \boxed{c_{ij}} = \color{red}a_{i1} \cdot b_{1j} \color{black}+ \color{green} a_{i2} \cdot b_{2j} \color{black}+ \color{blue}\cdots + a_{in} \cdot b_{pj} \]
		Attention, la multiplication n'est pas commutative $(AB \neq BA)$.
	\end{enumerate}
\end{definition}

\begin{proposition}
	Soit $(m, n, k, l) \in (\N^*)^4$.
	\begin{enumerate}
		\item Soit $(A, B, C) \in (\mathcal{M}_{m,n} (\K))^3$.
		\[A + B = B + A\]
		\[A + (B + C) = (A + B) + C\]
		\item Soient $A \in \mathcal{M}_{m, k}(\K),\ B \in \mathcal{M}_{k, l}(\K),\ C \in \mathcal{M}_{l, n}(\K)$.
		\[A \cdot (B \cdot C) = (A \cdot B) \cdot C\]
		\item Soient $A \in \mathcal{M}_{m, k}(\K),\ (B, C) \in (\mathcal{M}_{k, n}(\K))^2,\ \lambda \in \K$.
		\[A \cdot (B + \lambda C) = A \cdot B + \lambda \cdot A \cdot C\]
		\[(A + \lambda B) \cdot C = A \cdot C + \lambda \cdot B \cdot C\]
	\end{enumerate}
\end{proposition}

\begin{definition}[Matrice nulle]
	La matrice nulle est la matrice dont tous les coefficients sont $0$. On la note $0_{m,n}$, $m$ étant le nombre de lignes, $n$ le nombre de colonnes. 
\end{definition}

\begin{definition}[Matrice identité]
	La matrice identité est la matrice dont tous les coefficients sont $0$ à l'exception de ceux de la diagonale principale à $1$. On la note $I_n$, $n$ étant le nombre de lignes.
	\begin{align*}
		I_n = 
		\begin{pmatrix}
			1 & 0 & \cdots & 0 \\
			0 & 1 & \cdots & 0 \\
			\vdots & \ddots & \ddots & \vdots \\
			0 & \cdots & 0 & 1
		\end{pmatrix}
	\end{align*}
\end{definition}

\begin{remark}
    La matrice identité est parfois notée : $\mathds{1}_n$.
\end{remark}

\begin{lemma}
	Soit $(m,n) \in (\N^*)^2$.
	\begin{multicols}{2}
	    \begin{enumerate}
    		\item $\forall A \in \mathcal{M}_{m, n}(\K),\ A + 0_{m,n} = A$
    		\item $\forall A \in \mathcal{M}_{n}(\K),\ A \cdot I_n = I_n \cdot A = A$
    	\end{enumerate}
	\end{multicols}
\end{lemma}

\begin{definition}[Transposée]
	Soient $(m,n) \in (\N^*)^2$, $A \in \mathcal{M}_{m, n}(\K)$.
	\\
	On définit $A^T \in \mathcal{M}_{n, m}(\K)$ la transposée de $A$ par :
	\[(A^T)_{ij} = A_{ji}\]
    Autrement dit, les colonnes de $A$ deviennent ses lignes et réciproquement.
\end{definition}

\begin{definition}
	Soient $n \in \N^*$ et $M \in \mathcal{M}_n(\K)$.
	\begin{multicols}{2}
	    \begin{enumerate}
    		\item $M$ est symétrique $\iff M^T = M$
    		\item $M$ est anti-symétrique $\iff M^T = -M$
    	\end{enumerate}
	\end{multicols}
\end{definition}

\begin{proposition}[Calculer le déterminant d'une matrice carrée]
    Pour calculer le déterminant d'une matrice carrée, il existe plusieurs méthodes.
    Soit $A$ une matrice carrée de taille $n \times n$. 
    \begin{itemize}
        \item $n = 2$. 
        \[ 
        A = 
        \begin{pmatrix}
            a & b \\
            c & d
        \end{pmatrix}
        \quad 
        \det(A) = ad - bc 
        \]
        \item $n = 3$
        \[
        A = 
        \begin{pmatrix}
            a & b & c \\
            d & e & f \\
            g & h & i
        \end{pmatrix}
        \quad 
        \det(A) = (aei + bfg + cdh) - (ceg + bdi + afh)
        \]
        \item $n \geq 3$.
        (Meilleure rédaction prévue)
        \begin{align*}
            \det(A) = \sum_{j = 1}^{n} A_{ij} = (-1)^{i+j} \cdot \det(A^{ij})
        \end{align*}
    \end{itemize}
\end{proposition}

\begin{proposition}
	Soient $n \in \N^*$, $(A, B) \in (\mathcal{M}_n(\K))^2,\ \lambda \in \K$.
	\begin{multicols}{2}
	    \begin{enumerate}
		\item $\det(A \cdot B) = \det(A) \cdot \det(B)$
		\item $\det(\lambda A) = \lambda^n \det(A)$
		\item $\det(A^T) = \det(A)$
		\item $\det(A^{-1}) = \frac{1}{\det(A)}$ si $\det(A) \neq 0$
	\end{enumerate}
	\end{multicols}
\end{proposition}

\begin{definition}[Matrice inversible]
	Soient $n \in \N^*$ et $A \in \mathcal{M}_n(\K)$.
	\\
	$A$ est inversible si et seulement si existe une unique matrice $B \equiv A^{-1}$ telle que 
	\[AB = BA = I_n.\]
\end{definition}

\begin{proposition}
	Une matrice est inversible si et seulement si son déterminant est non nul.
\end{proposition}

\begin{definition}[Comatrice]
	La comatrice $\com(A)$ d'une matrice $A \in \mathcal{M}_n(\K)$.
	$$
	[\com(A)]_{ij} = (-1)^{i + j} \det(A^{(ij)}) 
	$$
\end{definition}

\begin{proposition}[Calculer l'inverse d'une matrice]
    Lorsque nous devons calculer l'inverse d'une matrice, plusieurs cas sont possibles.
    Si on a une matrice
    $A =
    \begin{pmatrix}
        a & b \\
        c & d
    \end{pmatrix}
    $.
    \[
    A^{-1} = \frac{1}{\det(A)} 
    \begin{pmatrix}
        d & -b \\
        -c & a
    \end{pmatrix}
    \]
    Une méthode générale pour calculer l'inverse d'une matrice $A$ est d'utiliser l'algorithme de Gauss-Jordan.
    On part de la matrice 
    \[ A|I_n \]
    et on applique les opérations élémentaires pour avoir une matrice de la forme
    \[I_n|B\]
    $B \equiv A^{-1}$ est l'inverse de $A$.
    \\
    Une autre méthode générale est d'utiliser la comatrice de $A$.
    \[ A^{-1} = \frac{1}{\det(A)} \left( \com(A) \right)^T \]
\end{proposition}

\begin{definition}[Trace d'une matrice]
	Soit $A \in \mathcal{M}_n(\K)$, la trace de $A$, notée $\tr(A)$ est définie par l'application suivante.
	\begin{align*}
		\tr : \mathcal{M}_n(\K) &\to \K \\
		A &\mapsto \sum_{i = 1}^{n} A_{ii}
	\end{align*}
    Autrement dit, c'est la somme des coefficients de la diagonale principale (celle allant du premier au dernier coefficient).
\end{definition}

\begin{lemma}
	\begin{multicols}{2}
	    \begin{enumerate}
    		\item $\tr(A^T) = \tr(A)$
    		\item $\tr(A \cdot B) = \tr(B \cdot A)$
    	\end{enumerate}
	\end{multicols}
\end{lemma}

\begin{proof}\leavevmode
    \begin{enumerate}
        \item Par application directe de la définition de la trace et de la transposée.
        \item On utilise la définition du produit matriciel.
        \begin{align*}
            \tr(AB) = \tr(BA) \iff \tr(AB) - \tr(BA) = 0
        \end{align*}
        \begin{align*}
            \tr(AB) - \tr(BA) &=\sum_{i=1}^n (AB)_{ii} - \sum_{i=1}^n (BA)_{ii} \\ 
            &= \sum_{i=1}^n [(AB)_{ii} - (BA)_{ii}] \\ 
            &= \sum_{k=1}^n \left[ \sum_{i=1}^n \left( A_{ik} B_{ki} - B_{ik}A_{ki} \right) \right] \\ 
            = 
            \begin{split}
                &[(\cancel{a_{11} b_{11}} - \cancel{b_{11} a_{11}}) +  (\cancel{a_{12} b_{21}} - \cancel{b_{12} a_{21}}) + \cdots + (\cancel{a_{1n} b_{n1}} - \cancel{b_{1n} a_{n1}})] \\
                + &[(\cancel{a_{21} b_{12}} - \cancel{b_{21}a_{12}}) + (\cancel{a_{22} b_{22}} - \cancel{b_{22} a_{22}}) + \cdots + (\cancel{a_{2n}b_{n2}} - \cancel{b_{2n}a_{n2}})] \\
                + &\cdots + \\ 
                + &[(\cancel{a_{n1} b_{1n}} - \cancel{b_{n1} a_{1n}}) + (\cancel{a_{n2} b_{2n}} - \cancel{b_{n2} a_{2n}}) + \cdots + (\cancel{a_{nn} b_{nn}} - \cancel{b_{nn} a_{nn}})]
            \end{split}
            \\
            &= 0
        \end{align*}
    \end{enumerate}
\end{proof}

\begin{proposition}
    Pour passer d'une forme cartésienne à une forme paramétrique, on applique le pivot de Gauss sur les lignes.
    Pour passer d'une forme paramétrique à une forme cartésienne, on utilise le déterminant.
\end{proposition}

\chapter{Espaces vectoriels}\label{chap:espaces_vectoriels}
\def\arraystretch{1}

\section{Définitions}
\begin{definition}[Loi de composition]
	Soient $E$, $F$ deux ensembles et $f$ une application.
	\begin{enumerate}
		\item On dit que $f$ est une loi de composition \textbf{interne} si et seulement si : $f : E \times E \to E$.
		\item On dit que $f$ est une loi de composition \textbf{externe} si et seulement si : $f : E \times F \to E$.
	\end{enumerate}
\end{definition}

\begin{definition}[Magma]
	On appelle \textbf{magma} un ensemble $M$ muni d'une loi de composition \textbf{interne} \og $*$ \fg. 
	\\
	On le note $(M, *)$.
\end{definition}

\begin{definition}[Monoïde \cite{monoide_bibmath}]
	On appelle \textbf{monoïde} un ensemble $M$ muni d'une loi de composition \textbf{interne} \og $*$ \fg \textbf{associative}. C'est-à-dire que pour tous $x, y, z \in M$ :
	\[ x * (y * z) = (x * y) * z \]
\end{definition}

\begin{definition}[Groupe]
	Soit $(G, *)$ un \textbf{monoïde}.
	\\ 
	On dit que $(G, *)$ est un \textbf{groupe} si et seulement si pour tous $g_1, g_2 \in G$ : 
    \begin{enumerate}
    		\item $\exists 0_G \in G,\ g_1 * 0_G = 0_G * g_1 = g_1$
    		\item $\exists g^{-1} \in G,\ g_1 * g^{-1} = g^{-1} * g_1 = 0_G$
    	\end{enumerate}
    \noindent De plus, si et seulement :
    \[ g_1 * g_2 = g_2 * g_1 \]
    on dit que $(G, *)$ est un groupe \textbf{commutatif} ou \textbf{abélien}.
\end{definition}

\begin{definition}[Anneau \cite{bibmath_resume_groupes}]
	Soit $A$ un ensemble muni de deux lois de compositions \textbf{internes} \og $+$ \fg et \og $\cdot$ \fg sur $A$ telles que pour tous $a, b, c \in A$ : 
	\begin{enumerate}
			\item $(A, +)$ est un groupe \textbf{commutatif}.
			\item $a \cdot (b \cdot c) = (a \cdot b) \cdot c$.
			\item $a \cdot (b + c) = a \cdot b + a \cdot c$.
			\item $(b + c) \cdot a = b \cdot a + c \cdot a$.
			\item \og $\cdot$ \fg possède un \textbf{élément neutre}.
		\end{enumerate}
	\noindent On dit que $A$ est \textbf{intègre} si :
	\begin{enumerate}
			\item $A$ est commutatif : $a \cdot b = b \cdot a$.
			\item $a \cdot b = 0 \implies a = 0 \text{ ou } b = 0$.
		\end{enumerate}
\end{definition}

\begin{definition}[Corps \cite{bibmath_resume_groupes}]
    Un corps est un anneau \textbf{commutatif} dans lequel tout élément non nul est inversible.
\end{definition}

\begin{definition}[$\K$-espace vectoriel]
	Soit $\K$ un corps. 
	\\
	Un $\K$-espace vectoriel est un ensemble $E$ composé d'une loi de composition \textbf{interne} \og $+$ \fg et d'une loi de composition \textbf{externe} \og $\cdot$ \fg telles que :
	\begin{center}
		$
		\begin{array}{cc}
			\appli{+}{E \times E}{E}{(x,y)}{x+y}
			&
			\appli{\cdot}{\K \times E}{E}{(\lambda, u)}{\lambda \cdot u}
		\end{array}
		$	
	\end{center}		
	
	\begin{enumerate}
		\item $(E, +)$ est un \textbf{groupe commutatif}.
		\item $\forall \lambda_1, \lambda_2 \in \K,\ u, v \in E :$
		\begin{enumerate}
			\item $\lambda_1 \cdot (u + v) = \lambda_1 \cdot u + \lambda_1 \cdot v$
			\item $(\lambda_1 + \lambda_2) \cdot u = \lambda_1 \cdot u + \lambda_2 \cdot u$
			\item $(\lambda_1 \cdot \lambda_2) \cdot u = \lambda_1 \cdot (\lambda_2 \cdot u)$
			\item $1 \cdot u = u$
		\end{enumerate}
	\end{enumerate}
\end{definition}

\begin{definition}[Sous-espace vectoriel]
	Soit $E$ un $\K$-espace-vectoriel, $F$ est un sous-espace vectoriel de $E$ si :
	\begin{enumerate}
    		\item $F \subseteq E$
    		\item $F \neq \varnothing$
    		\item $\forall u, v \in F,\ \lambda \in \K : u + \lambda v \in F$
    	\end{enumerate}
\end{definition}

\begin{definition}[Somme directe]
	Soient $E$ un $\K$-espace vectoriel et $F_1, F_2 \subseteq E$. On dit que $F_1$ et $F_2$ sont en \textbf{somme directe} ou qu'ils sont \textbf{supplémentaires} dans $E$ si et seulement si :
	\begin{multicols}{2}
	    \begin{enumerate}
		\item $F_1 \cap F_2 = \{ 0_E \}$
		\item $F_1 + F_2 = E$
	\end{enumerate}
	\end{multicols}
	On note alors :
	\[ F_1 \oplus F_2 = E \]
\end{definition}

\begin{proposition}
	Soient $E$ un $\K$-espace vectoriel et $F_1, F_2 \subseteq E$, on a $F_1 + F_2 \subseteq E$. 
\end{proposition}

\begin{proof}
	Tout d'abord $0_E \in F_1$ et $0_E \in F_2$, on a donc $0_E \in F_1 + F_2$.
	\\
	Ensuite, soient $x, y \in F_1 + F_2,\ \lambda \in \K$, posons pour $v_1, w_1 \in F_1$ et $v_2, w_2 \in F_2$ :
	\begin{align*}
		\begin{cases}
			x = v_1 + v_2 \\
			y = w_1 + w_2
		\end{cases}
	\end{align*}
	On a :
	\begin{align*}
		x + \lambda y &= v_1 + v_2 + \lambda (w_1 + w_2) \\
		&= v_1 + \lambda w_1 + v_2 + \lambda w_2
	\end{align*}
	ce qui implique que $x + \lambda y \in F_1 + F_2$.
\end{proof}

\begin{proposition}
	$\Vect(u_1, \ldots, u_k) \subseteq \R^n$ est un sous-espace vectoriel.
\end{proposition}

\begin{proof}
	$0_{\R^n} \in \Vect(u_1, \ldots, u_k)$ pour $\lambda_1 = \cdots = \lambda_k = 0$.
	\\
	Soient $v = \sum_{i=1}^{k} \alpha_i u_i$, $w = \sum_{i=1}^{k} \beta_i u_i$ et $\lambda \in \R$ alors on a :
	\begin{align*}
		v + \lambda w &= \sum_{i=1}^{k} (\alpha_i u_i) + \lambda \sum_{i=1}^{k} (\beta_i u_i) \\
		&= \sum_{i=1}^{k} (\alpha_i + \lambda \beta_i) u_i \in \Vect(u_1, \ldots, u_k)
	\end{align*}
\end{proof}

\begin{proposition}
	Soient $E$ un $\K$-espace vectoriel, $F_1, F_2 \subseteq E$ alors $F_1 \cap F_2 \subseteq E$.
\end{proposition}

\begin{proof}
	Tout d'abord, $0_E \in F_1$, $0_E \in F_2$ alors $0_E \in F_1 \cap F_2$.
	\\
	Ensuite pour tout $u, v \in F_1 \cap F_2,\ \lambda \in \K$ on a :
	\begin{enumerate}
		\item $u + \lambda v \in F_1$ car $F_1 \subseteq E$. 
		\item $u + \lambda v \in F_2$ car $F_2 \subseteq E$.
	\end{enumerate}
	ainsi $u + \lambda v \in F_1 \cap F_2$.
\end{proof}

\begin{definition}
	Soient $n \in \N^*$, $E$ un espace vectoriel et $\mathcal{F} = (u_1, \ldots, u_n)$ une famille de vecteurs de $E$.
	\begin{enumerate}
		\item On dit que $\mathcal{F}$ est \textbf{libre} si et seulement si :
		\begin{align*}
			\forall \lambda_i \in \K : \sum_{i = 1}^n \lambda_i u_i = 0_E \implies \lambda_1 = \cdots = \lambda_n = 0
		\end{align*}
		\item On dit que $\mathcal{F}$ est \textbf{génératrice} si et seulement si :
		\begin{align*}
			\forall x \in E,\ \exists \lambda_i \in \K : x = \sum_{i = 1}^n \lambda_i u_i 
		\end{align*}
	\end{enumerate}
\end{definition}

\begin{remark}
	Si une famille n'est pas libre, on dit qu'elle est liée.
\end{remark}

\section{Base et dimension}
\begin{definition}[Base]
	Une famille de vecteurs est une base si elle est \textbf{libre} et \textbf{génératrice}.
\end{definition}

\begin{proposition}
	Soient un espace vectoriel $E$ et $\mathcal{F} = (u_1, \ldots, u_n)$ une famille de vecteurs de $E$.
	\[ 
	\mathcal{F} \text{ est une base} \iff 
	\forall x \in E,\ \exists ! \lambda_i \in \K^n : x = \sum_{i = 1}^n \lambda_i u_i
	\]
\end{proposition}

\begin{proof}
	L'existence est évidente car $\mathcal{F}$ est génératrice. 
	\\
	Montrons l'unicité : Soient $\lambda_1, \ldots, \lambda_n, \mu_1, \ldots, \mu_n \in \K$.
	On a d'une part :
	\[ \forall x \in E,\ x = \sum_{i=1}^{n} \lambda_i u_i \]
	puis d'autre part :
	\[ \forall x \in E,\ x = \sum_{i=1}^{n} \mu_i u_i \]
	Donc on a :
	\begin{align*}
		&\sum_{i=1}^{n} \lambda_i u_i = \sum_{i=1}^{n} \mu_i u_i \\
		\iff &\sum_{i=1}^{n} (\lambda_i - \mu_i) u_i = 0 
	\end{align*}
	Or $\mathcal{F}$ est libre, donc pour tout $i \in \llbracket 1, n \rrbracket,\ \lambda_i - \mu_i = 0 \implies \lambda_i = \mu_i$.
\end{proof}

\begin{proposition}
	Soit $\mathcal{F} = (u_1, \ldots, u_n)$ une famille de vecteurs de $\R^n$. $\mathcal{F}$ est une base de $\R^n$ si et seulement si :
	\[ \det(\mathcal{F}) \neq 0 \]
\end{proposition}

\begin{proof}
	$\mathcal{F} = (u_1, \ldots, u_n)$ est une base $\iff \forall x \in \R^n,\ \exists ! (\lambda_1, \ldots, \lambda_n) \in \R^n : x = \sum_{i=1}^{n} \lambda_i u_i$.
	\begin{align*}
		x = \sum_{i=1}^{n} \lambda_i u_i &= 
		\begin{pmatrix}
			\lambda_1 u_{1,1} + \cdots + \lambda_n u_{1,n} \\
			\vdots \\
			\lambda_1 u_{n,1} + \cdots + \lambda_n u_{n,n}
		\end{pmatrix}
		\\
		&= 
		\begin{pmatrix}
			u_{1,1} & \cdots & u_{1,n} \\
			\vdots & \ddots & \vdots \\
			u_{n,1} & \cdots & u_{n,n}
		\end{pmatrix}
		\begin{pmatrix}
			\lambda_1 \\
			\vdots \\
			\lambda_n
		\end{pmatrix}
	\end{align*}
	Posons 
	$
	A =
	\begin{pmatrix}
		u_{1,1} & \cdots & u_{1,n} \\
		\vdots & \ddots & \vdots \\
		u_{n,1} & \cdots & u_{n,n}
	\end{pmatrix}
	$
	et 
	$\lambda =
	\begin{pmatrix}
		\lambda_1 \\
		\vdots \\
		\lambda_n
	\end{pmatrix}
	$\\
	On a donc :
	\[ x = A \cdot \lambda \]
	Il existe une solution unique si et seulement s'il existe l'inverse de $A$, c'est-à-dire que $\det(A) \neq 0$.
\end{proof}

\begin{definition}[Dimension d'un espace vectoriel]
	Soit $E$ un espace vectoriel, on appelle dimension de $E$, notée $\dim(E)$, le nombre d'éléments d'une base de $E$. 
\end{definition}

\begin{proposition}
	Soient $E$ un espace vectoriel de dimension $n$ et $\mathcal{F} = (u_1, \ldots, u_n)$ une famille de vecteurs de $E$. Alors on a :
	\begin{enumerate}
		\item $\mathcal{F}$ est une base.
		\item $\mathcal{F}$ est libre.
		\item $\mathcal{F}$ est génératrice.
	\end{enumerate}
	\noindent Ainsi il suffit de montrer que $\mathcal{F}$ est libre pour montrer les deux autres propriétés.
\end{proposition}

\begin{theorem}[Théorème de la base incomplète]
	Toute famille libre peut être complétée en une base.
\end{theorem}

\begin{theorem}[Théorème de la base extraite]
    De toute famille génératrice, on peut extraire une base.
\end{theorem}

\begin{theorem}
    Chaque espace vectoriel admet une base.
\end{theorem}

\begin{corollary}
	Soient $n, N \in \N^*,\ \mathcal{F} = (u_1, \ldots, u_N)$, $E$ un espace vectoriel tel que $\dim(E) = n$.
	\begin{enumerate}
		\item Si $N > n$ alors $\mathcal{F}$ n'est pas libre.
		\item Si $N < n$ alors $\mathcal{F}$ n'est pas génératrice.
	\end{enumerate}
\end{corollary}

\begin{proposition}
	Soient $E$ un espace vectoriel et $F, G\subseteq E$.
	\begin{multicols}{2}
	    \begin{enumerate}
    		\item $\dim(F \oplus G) = \dim(F) + \dim(G)$ .
    		\item $\dim(F + G) \leq \dim(E)$.
    	\end{enumerate}
	\end{multicols}
\end{proposition}

\begin{theorem}[Théorème de Grassman]
	Soient $F$ et $G$ deux espaces vectoriels.
	\[ \dim(F+G) = \dim(F) + \dim(G) - \dim(F\cap G) \]
\end{theorem}


\chapter{Applications linéaires}
\def\arraystretch{1}

\section{Définitions}
\begin{definition}[Application linéaire]
    Soient $E$ et $F$ deux $\K$-espaces vectoriels et $f : E \to F$.
    On dit que $f$ est une \textbf{application linéaire} si et seulement si : 
    \[ \forall x_1, x_2 \in E,\ \lambda \in \K : f(x_1 + \lambda x_2) = f(x_1) + \lambda f(x_2) \]
\end{definition}

\begin{definition}
    Soient $E$ et $F$ deux $\K$-espaces vectoriels et $f : E \to F$ une application linéaire.
    \\
    On définit le \textbf{noyau} de $f$, noté $\ker(f)$ tel que :
    \[ \ker(f) = \{ x \in E : f(x) = 0_F \} \] 
    On définit l'\textbf{image} de $f$, notée $\Im(f)$ telle que :
    \[ \Im(f) = \{ y \in F,\ \exists x \in E : y = f(x) \} \]
\end{definition}

\begin{theorem}[\cite{applications_lineaires_bibmath}]
	Soient $E$ et $F$ deux $\K$-espaces vectoriels. \\
	$f \in \mathcal{L}(E, F)$ est injective si et seulement si $\ker(f) = \{ 0_E \}$.
\end{theorem}

\begin{proof}
	\leavevmode
	\begin{enumerate}
		\item \boxed{\implies} : Supposons $f$ injective.
		\\
		Si $x \in \ker(f)$ alors $f(x) = 0_F$. On sait que $f(0_E) = 0_F$, or $f$ est injective donc $0_E = x$ et :
		\[ \ker(f) = \{ 0_E \} \]
		\item \boxed{\impliedby} : Supposons que $\ker(f) = \{ 0_E \}$.
		\\
		Soient $x, y \in E$ tels que $f(x) = f(y)$.
		On a ensuite :
		\[ 0_F = f(x) - f(y) \]
		puis car $f$ est linéaire :
		\[ 0_F = f(x - y) \]
		ce qui veut dire que $x - y \in \ker(f)$ or $\ker(f) = \{ 0_E \}$ donc $0_E = x - y$, c'est-à-dire $x = y$.
		\\
		On a bien montré que $f$ est injective.
	\end{enumerate}
\end{proof}

\begin{theorem}
	Soient $E, F$ des espaces vectoriels et $f : E \to F$ une application linéaire.
	\begin{enumerate}
		\item $\ker(f) \subseteq E$.
		\item $\Im(f) \subseteq F$.
	\end{enumerate}
\end{theorem}

\begin{proof}
	\leavevmode 
	\begin{enumerate}
		\item $\forall x_1, x_2 \in \ker(f),\ \lambda \in \K$.
		\begin{align*}
			f(x_1 + \lambda x_2) &= f(x_1) + \lambda f(x_2) \\ 
								 &= 0_F + \lambda 0_F \\
								 &= 0_F
		\end{align*}
		\item $\forall y_1, y_2 \in \Im(f),\ \lambda \in \K$.
		\[ y_1 \in \Im(f) \iff \exists x_1 : f(x_1) = y_1 \]
		\[ y_2 \in \Im(f) \iff \exists x_2 : f(x_2) = y_2 \]
		\begin{align*}
			y_1 + \lambda y_2 &= f(x_1) + \lambda f(x_2) \\
			                  &= f(x_1 + \lambda x_2)
		\end{align*}
		Ainsi $y_1 + \lambda y_2 \in \Im(f)$.
	\end{enumerate}
\end{proof}

\begin{definition}
    \par \noindent Soient $E$ et $F$ deux $\K$-espaces vectoriels et $f$ une application linéaire de $E$ vers $F$.
    \begin{enumerate}
        \item On dit que $f$ est un \textbf{morphisme} de $E$ vers $F$, on note $f \in \mathcal{L}(E, F)$.
        \item Si $E = F$, on dit que $f$ est un \textbf{endomorphisme} de $E$, on note $f \in \mathcal{L}(E)$ ou $\End(E)$.
        \item Si $f \in \mathcal{L}(E, F)$ est une \textbf{bijection}, alors $f$ est un \textbf{isomorphisme}. On le note $f : E \overset{\sim}{\to} F$.
        \item Si $f \in \mathcal{L}(E)$ est un \textbf{isomorphisme}, on dit que $f$ est un \textbf{automorphisme} de $E$ et on le note $\Aut(E)$.
        \\
        $E$ et $F$ sont appelés \textbf{isomorphes} s'il existe un \textbf{isomorphisme} de l'un vers l'autre, on écrit parfois $E \cong F$.
    \end{enumerate}
\end{definition}

Soit $\mathcal{E} = \{ f \in \mathcal{C}^n(I) : f^{(n)} + a_{n - 1} f^{(n - 1)} + \cdots + a_1 f' + a_0 f = 0 \},\ n \in \N,\ a_i \in \mathcal{C}^0(I),\ I$ un intervalle ouvert.

\begin{proposition}
    Soit $n \in \N$.
    \begin{multicols}{2}
        \begin{enumerate}
            \item $\mathcal{E} \subseteq \mathcal{C}^n(I)$.
            \item $\dim(\varepsilon) = n$.
        \end{enumerate}
    \end{multicols}
\end{proposition}

\begin{proof}
	Montrons 1. \\
	On définit :
	\begin{center}
		$\appli{\Psi}{\mathcal{C}^n(I)}{\mathcal{C}^0(I)}{f}{f^{(n)} + a_{n-1}f^{(n-1)} + \cdots + a_1 f' + a_0 f}$
	\end{center}
	Alors on voit que $\mathcal{E} = \ker(\Psi)$.
\end{proof}

Soient $E = \{ (u_n)_{n \in \N},\ u_n \in \R,\ \forall n \in \N \}$, $N \in \N$ et $a = (a_0, \ldots, a_{N-1}) \in \R^N$.
\\
On définit :
\[ F = \{ (u_n)_{n \in \N} : u_{n + N} + a_{N-1} u_{n + N - 1} + \cdots + a_n u_{n+1} + a_0 u_n = 0,\ \forall n \in \N \} \]

\begin{theorem}
	\begin{multicols}{2}
		\begin{enumerate}
			\item $F \subseteq E$.
			\item $\dim(F) = N$.
		\end{enumerate}
	\end{multicols}
\end{theorem}

\begin{definition}[Rang d'un morphisme]
    Soit $f \in \mathcal{L}(E, F)$. On appelle $\rg(f) = \dim(\Im(f))$ le rang de $f$.
\end{definition}

\begin{theorem}[Théorème du rang]
    Soient $E, F$ des espaces vectoriels de dimensions finies ou infinies et $f \in \mathcal{L}(E, F)$, alors :
    \[ \dim(E) = \dim(\ker(f)) + \rg(f) \]
\end{theorem}

\begin{corollary}
    Soit $f \in \mathcal{L}(E)$.
    \begin{align*}
        \begin{cases}
            \ker(f) + \Im(f) = \ker(f) \oplus \Im(f) \\
            \ker(f) \oplus \Im(f) = E
        \end{cases}
        \iff 
        \ker(f) \cap \Im(f) = \{0_E\}
    \end{align*}
\end{corollary}

\begin{proposition}
    Soit $f \in \mathcal{L}(E, F)$ et $(b_1, \ldots, b_n)$ une base de $E$. Si l'on connait $f(b_i) \in F$, $1 \leq i \leq n$, on connait toute l'application $f$.
\end{proposition}

\begin{proof}
	$\forall x \in E,\ \exists ! (\lambda_1, \ldots, \lambda_n) \in \R^n$ tel que $x = \sum_{i=1}^{n} \lambda_i b_i$.
	\[ f(x) = f \left( \sum_{i=1}^{n} \lambda_i b_i \right) = \sum_{i=1}^{n} \lambda_i f(b_i) \]
\end{proof}

\begin{corollary}
    Soit $(b_1, \ldots, b_n)$ une base de $E$. Alors $\varphi \in \mathcal{L}(E, \R^n)$ est un isomorphisme donné par :
    \[ \varphi(b_i) = e_i,\ 1 \leq i \leq n \]
    \[ \varphi \left( \sum_{i = 1}^{n} x_i b_i \right) = \sum_{i = 1}^{n} x_i e_i = 
    \begin{pmatrix}
        x_1 \\
        \vdots \\
        x_n
    \end{pmatrix}
    \in \R^n
     \]
\end{corollary}

\begin{lemma} Soient $\K$ un corps et $E$ un $\K$-espace-vectoriel de dimension $n \in \N$.
    \[ \dim(E) = n \implies E \cong \K^n \]
\end{lemma}

\begin{lemma}
    Soient $\mathcal{B} = (b_1, \ldots, b_n)$ une base de $E$, $\lambda_1, \ldots, \lambda_n \in \K$ et $\varphi \in \mathcal{L}(E, \K^n)$  alors :
    \begin{center}
    	$
    	\appli{\varphi}{E}{\K^n}{\sum_{i = 1}^{n} \lambda_i b_i}{
    	\begin{pmatrix}
    		\lambda_1 \\
    		\vdots \\
    		\lambda_n
    	\end{pmatrix}
    	}
    	$
    \end{center}
    est une bijection, alors c'est un isomorphisme.
\end{lemma}

\begin{proposition}
    Soient $E, F$ des espaces vectoriels $f \in \mathcal{L}(E, F)$ un morphisme et $A$ sa matrice associée. 
    \[ y = f(x) \iff y = A \cdot x \]
\end{proposition}

\section{Projecteurs et symétries}
\begin{definition}
    Soit $f \in \mathcal{L}(E)$. Notons $f^2 = f \circ f$.
    \begin{enumerate}
        \item On dit que $f$ \textbf{est idempotente/une projection} si et seulement si : $f^2 = f$.
        \item On dit que $f$ \textbf{est involutive/une symétrie linéaire} si et seulement si : $f^2 = id_E$.
    \end{enumerate}
\end{definition}

\begin{proposition} 
    Soit $p \in \mathcal{L}(E)$.
    \begin{enumerate}
        \item $id_E - p \text{ est une projection} \iff p \text{ est une projection}$
        \item $2p - id_E \text{ est une symétrie} \iff p \text{ est une projection}$
    \end{enumerate}
\end{proposition}

\begin{proof}
	\leavevmode
    \begin{enumerate}
        \item Posons $f(x) = id_E(x) - p(x)$. Montrons que $f(f(x)) = id_E(x) - p(x)$ si et seulement si $p(p(x))$.
        \begin{align*}
            f(f(x)) &= id_E(id_E(x) - p(x)) - p(id_E(x) - p(x)) \\ 
            &= id_E(id_E(x)) - id_E(p(x)) - p(id_E(x)) + p(p(x)) \\ 
            &= id_E(x) - p(x) - p(x) + p(p(x)) \\ 
            &= id_E(x) - 2p(x) + p(p(x)) \\ 
            &= id_E(x) - [2p(x) - p(p(x))]
        \end{align*}
        \begin{align*}
            id_E(x) - [2p(x) - p(p(x))] = id_E(x) - p(x) &\iff p(p(x)) = p(x) \\ 
            &\iff 2p(x) - p(p(x)) = p(x) \\ 
            &\iff 2p(x) = p(x) + p(p(x)) \\
            &\iff p(x) = p(p(x))
        \end{align*}
        \item Posons $s(x) = 2p(x) - id_E(x)$. 
        \\
        Montrons que $s(s(x)) = id_E(x)$ si et seulement si $p(p(x)) = p(x)$.
        \begin{align*}
            s(s(x)) &= s(2p(x) - id_E(x)) \\
            &= 2p(2p(x) - id_E(x)) - id_E(2p(x) - id_E(x)) \\
            &= 4p(p(x)) - 2p(id_E(x)) - 2id_E(p(x)) + id_E(id_E(x))
        \end{align*}
        \begin{align*}
            4p(p(x)) - 4p(x) + id_E(x) = id_E(x) &\iff 4p(p(x)) - 4p(x) = 0 \\
            &\iff 4p(p(x)) = 4p(x) \\
            &\iff p(p(x)) = p(x)
        \end{align*}
    \end{enumerate}
\end{proof}

\begin{definition}
    Soient $E = F \oplus G$, $u \in F$, $v \in G$, alors l'application 
    \begin{center}
    	$
    	\appli{p_F}{E}{E}{u+v}{u}
    	$
    \end{center}
    est appelée un \textbf{projecteur} sur $F$ \textbf{parallélement} à $G$.
\end{definition}

\begin{proposition}
    Soit $p_F$ définie comme dans la définition précédente.
    \begin{enumerate}
        \item $p_F$ est une projection.
        \item Soit $p$ une projection, $p$ est un projecteur sur $\Im(p)$ parallélement à son noyau $\ker(p)$.
    \end{enumerate}
\end{proposition}

\section{Rotations.}
\begin{definition}
	Soit $n \in \N^*$.
    \[ \operatorname{GL}(\R, n) = \{ A \in \mathcal{M}_n(\R) : \det(A) \neq 0 \}. \]
\end{definition}

\begin{definition}
	Soit $n \in \N^*$.
    \[ \operatorname{SL}(\R, n) = \{ A \in \mathcal{M}_n(\R) : \det(A) = 1 \}. \]
\end{definition}

\begin{definition}
	Soit $n \in \N^*$.
    \[ \operatorname{O}(n) = \{ R \in \mathcal{M}_n(\R) : \forall x, y \in \R^n : \langle Rx|Ry \rangle = \langle x|y \rangle \}. \]
\end{definition}

\begin{definition}
    Soit $n \in \N^*$.
    \[ \operatorname{SO}(n) = \{ R \in \operatorname{O}(n) : \det(R) = 1 \}. \]
\end{definition}

\begin{proposition}
	$\forall R \in \mathcal{M}_n(\K)$ :
    \begin{enumerate}
        \item $R \in \operatorname{O}(n) \iff R^T \cdot R = I_n$.
        \item $R \in \operatorname{O}(n) \implies \det(R) \in \{ \pm 1 \}$.
    \end{enumerate}
\end{proposition}

\begin{corollary}
	$\forall n \in \N^* : \operatorname{O}(n) = \{ R \in \mathcal{M}_n(\K) : R^T \cdot R = I_n \}$.
\end{corollary}

\begin{lemma}
	 $\forall x, y \in \R^n,\ A \in \mathcal{M}_n(\R) : \langle y \mid A \cdot x \rangle = \langle A^T \cdot y \mid x \rangle$.
\end{lemma}

\begin{proof}
	Soient $a_{1,1}, \ldots, a_{n,n} \in \R$ les coefficients de $A$.
	\begin{align*}
		\langle y \mid A \cdot x \rangle &= \sum_{i = 1}^{n} y_i (A \cdot x)_i \\
		&= \sum_{i = 1}^{n} y_i \left( \sum_{j=1}^{n} a_{i,j} x_{j,i} \right) \\
		&= \sum_{j = 1}^{n} \left( \sum_{i = 1}^{n} a_{j,i} y_i \right) \cdot x_j \\
		&= \langle A^T \cdot y \mid x \rangle
	\end{align*}
\end{proof}

\section{Changements de bases et matrices associées aux applications linéaires.}

\begin{definition}[Matrice d'une application linéaire]
    Soient $n, p \in \N^*$, $E$ et $F$ deux $\K$-espaces vectoriels, $\mathcal{B} = (b_1, \ldots, b_p)$ une base de $E$, $\mathcal{B}' = (b_1', \ldots, b_n')$ une base de $F$ et $f \in \mathcal{L}(E, F)$.
    \\
    On peut définir une matrice $\Mat_{\mathcal{B},\mathcal{B}'} (f) \in \mathcal{M}_{n,p}(\K)$ telle que :
    \begin{align*}
        \hspace{0.4cm}
        \begin{matrix}
            f(b_1) & \cdots & f(b_p) 
        \end{matrix}
    \end{align*}
    \begin{align*}
        \begin{matrix}
            b_1' \\
            \vdots \\ 
            b_n'
        \end{matrix}
        \begin{pmatrix}
            a_{1,1} & \cdots & a_{1,p} \\
            \vdots & \ddots & \vdots \\ 
            a_{n,1} & \cdots & a_{n,p}
        \end{pmatrix}
    \end{align*}
    avec les coefficients $a_{1,1}, \ldots, a_{n,p} \in \K$ tels que : 
    \[ f(b_j) = \sum_{i = 1}^n a_{i,j} b_i' \]
\end{definition}

\begin{example}
	Soit $\Psi$ une application linéaire telle que :
	\begin{center}
		$\appli{\Psi}{\R_2[X]}{\R_3[X]}{P}{P + (X + X^2)P' + (-2 + 3X - X^3)P''}$
	\end{center}
	On prend $\mathcal{B} = (1, X, X^2)$ comme base de $\R_2[X]$ et $\mathcal{B}' = (1, X, X^2, X^3)$ comme base de $\R_3[X]$.
	\\
	On calcule :
	\[ \Psi(1) = 1 \cdot 1 + (X + X^2) \cdot 0 + (-2 + 3X - X^3) \cdot 0 = 1 \cdot 1 + 0 \cdot X + 0 \cdot X^2 + 0 \cdot X^3 \]
	\[ \Psi(X) = X + (X + X^2) \cdot 1 + (-2 + 3X - X^3) \cdot 0 = 0 \cdot 1 + 2 \cdot X + 1 \cdot X^2 + 0 \cdot X^3 \]
	\[ \Psi(X^2) = X^2 + (X + X^2) \cdot 2X + (-2 + 3X - X^3) \cdot 2 = -4 \cdot 1 + 6 \cdot X + 2 \cdot X^2 + 1 \cdot X^3 \]
	\begin{align*}
		&\begin{matrix}
			\Psi(1) & \Psi(X) & \Psi(X^2)
		\end{matrix}
		\\
		\begin{matrix}
			1 \\
			X \\
			X^2 \\
			X^3
		\end{matrix}
		&
		\begin{pmatrix}
			1 & 0 & -4 \\
			0 & 2 & 6 \\
			0 & 1 & 2 \\
			0 & 0 & 1
		\end{pmatrix}
		= \Mat_{\mathcal{B}, \mathcal{B'}}(\Psi)
	\end{align*}
\end{example}

\begin{definition}[Matrice de passage]
    Soient $n \in \N^*$, $E$ un $\K$-espace vectoriel de dimension $n$, $\mathcal{B} = (b_1, \ldots, b_n)$ et $\mathcal{B}' = (b_1', \ldots, b_n')$ deux bases de $E$. On appelle \textbf{matrice de passage} de $\mathcal{B}$ à $\mathcal{B}'$ la matrice carrée de taille $n$ dont la $j-$ième colonne est formée des coordonnées de $b_j'$ dans la base $\mathcal{B}$. Nous la noterons $P_{\mathcal{B}, \mathcal{B}'}$.
    \begin{align*}
    	\hspace{0.45cm}
    	\begin{matrix}
    		b_1' & \cdots & b_n' 
    	\end{matrix}
    \end{align*}
    \begin{align*}
    	\begin{matrix}
    		b_1 \\
    		\vdots \\ 
    		b_n
    	\end{matrix}
    	\begin{pmatrix}
    		a_{1,1} & \cdots & a_{1,n} \\
    		\vdots & \ddots & \vdots \\ 
    		a_{n,1} & \cdots & a_{n,n}
    	\end{pmatrix}
    \end{align*}
    Avec les coefficients $a_{1,1}, \ldots, a_{n,n} \in \K$ tels que :
    \[ b_j' = \sum_{i=1}^{n} a_{i,j} b_i  \]
\end{definition}

\begin{example}
	Soient $\mathcal{E} = ((1, 0), (0, 1))$ et $\mathcal{B} = ((1, 2), (3, -1))$ deux bases de $\R^2$.
	On a :
	\[
	\begin{pmatrix}
		1 \\
		2
	\end{pmatrix}
	= 
	1 \cdot
	\begin{pmatrix}
		1 \\
		0
	\end{pmatrix}
	+ 2 \cdot
	\begin{pmatrix}
		0 \\
		1
	\end{pmatrix}
	\]
	
	\[
	\begin{pmatrix}
		3 \\
		-1
	\end{pmatrix}
	=
	3 \cdot
	\begin{pmatrix}
		1 \\
		0
	\end{pmatrix}
	- 1 \cdot 
	\begin{pmatrix}
		0 \\
		1
	\end{pmatrix}
	\]
	Ainsi on a :
	\begin{align*}
		&\begin{matrix}
			(1, 2) & (3, -1)
		\end{matrix}
		\\
		\begin{matrix}
			(1, 0) \\
			(0, 1)
		\end{matrix}
		&\begin{pmatrix}
			1 & 3 \\
			2 & -1
		\end{pmatrix} = P_{\mathcal{E},\mathcal{B}}
	\end{align*}
\end{example}

\begin{definition}
    Soient $A, B \in \mathcal{M}_{m,n}(\K)$.
    \begin{enumerate}
        \item On dit que $A$ et $B$ sont \textbf{équivalentes} si et seulement si : 
        \[ \exists P \in \operatorname{GL}(n),\ Q \in \operatorname{GL}(m),\ B = Q^{-1} A P \]
        On note $A \sim B$.
        \item On dit que $A$ et $B$ sont \textbf{semblables} si et seulement si :
        \[ \exists P \in \operatorname{GL}(n),\ A = P B P^{-1} \]
        On note $A \simeq B$.
    \end{enumerate}
\end{definition}

\begin{lemma}
	$\forall A, B \in \mathcal{M}_n(\K),\ A \simeq B : \tr(A) = \tr(B)$.
\end{lemma}

\begin{proof}
	$B = P^{-1} A P,\ \tr(B) = \tr(P^{-1} A P) = \tr(A P P^{-1}) = \tr(A)$.
\end{proof}

\part{Annexes}
\def\arraystretch{1.5}
\begin{table}[!h]
    \centering
    \begin{tabular}{cc}
        \toprule
        Soient $f, g$ des fonctions \\
        \midrule
        $C \in \R,\ C f(x)$ & $C f'(x)$  \\
        $(f + g)'(x)$ & $f'(x) + g'(x)$ \\
        $(f \cdot g)'(x)$ & $f'(x) \cdot g(x) + f(x) \cdot g'(x)$ \\
        $\left( \frac{f}{g} \right)'(x)$ & $\frac{f'(x) \cdot g(x) - f(x) \cdot g'(x)}{(g(x))^2}$ \\
        $(f \circ g)'(x)$ & $f'(g(x)) \cdot g'(x) $ \\
        \bottomrule
    \end{tabular}
    \caption{Formules de dérivation.}
    \label{tab:formules_derivation}
\end{table}

\begin{table}[!h]
    \centering
    \begin{tabular}{ccc}
        \toprule
        $f(x)$ & $f'(x)$ & $\mathcal{D}_f$ \\ 
        \midrule
        $C \in \R$ & $0$ & $\R$ \\
        $x^a,\ a \in \R$ & $a x^{a - 1}$ & $\R^* \text{ si } a \in \Z^-, \R_+^* \text{ sinon }$ \\
        $\sqrt{x}$ & $\frac{1}{2\sqrt{x}}$ & $\R_+^*$ \\
        $e^x$ & $e^x$ & $\R$ \\
        $\ln{\abs{x}}$ & $\frac{1}{x}$ & $\R_+^*$ \\
        $\cos{x}$ & $-\sin{x}$ & $\R$ \\
        $\sin{x}$ & $\cos{x}$ & $\R$ \\
        $\tan{x}$ & $\frac{1}{\cos^2{x}} = 1 + \tan^2{x}$ & $\forall k \in \Z, \R \backslash \left\{ \frac{\pi}{2} + k \pi \right\}$ \\
        $\cosh{x}$ & $\sinh{x}$ & $\R$ \\
        $\sinh{x}$ & $\cosh{x}$ & $\R$ \\
        $\tanh{x}$ & $1 - \tanh^2{x} = \frac{1}{\cosh^2{x}}$ & $\R$ \\
        $\arctan{x}$ & $\frac{1}{1 + x^2}$ & $\R$ \\
        $\arcsin{x}$ & $\frac{1}{\sqrt{1 - x^2}}$ & $]-1, 1[$ \\
        $\arccos{x}$ & $-\frac{1}{\sqrt{1 - x^2}}$ & $]-1, 1[$ \\
        \bottomrule
    \end{tabular}
    \caption{Dérivées usuelles.}
    \label{tab:derivees_usuelles}
\end{table}

\begin{table}[!h]
    \centering
    \begin{tabular}{ccc}
         \toprule
         $f(x)$ & $F(x)$ & $I$ \\
         \midrule
         $x^a,\ a \in \R \backslash \{1\}$ & $\frac{1}{a + 1} x^{a+1} + C$ & $\R_+^* \text{ ou } \R_-^* \text{ si } a \in \Z^-,\ \R_+^* \text{ sinon}$ \\
         $\frac{1}{x}$ & $\ln(\abs{x}) + C$ & $\R_+^* \text{ ou } \R_-^*$ \\
         $e^{ax},\ a \in \R$ & $\frac{1}{a} e^{ax} + C$ & $\R$ \\
         $\cos{x}$ & $\sin{x} + C$ & $\R$ \\
         $\sin{x}$ & $-\cos{x} + C$ & $\R$ \\
         $1 + \cos^2{x} = 1 + \tan^2{x}$ & $\tan{x} + C$ & $\forall k \in \Z,\ \left] -\frac{\pi}{2} + k \pi, \frac{\pi}{2} + k \pi \right[$ \\
         $\cosh{x}$ & $\sinh{x} + C$ & $\R$ \\
         $\sinh{x}$ & $\cosh{x} + C$ & $\R$ \\
         $\frac{1}{\cosh^2{x}} \text{ ou } 1 + \tanh^2{x}$ & $\tanh{x} + C$ & $\R$ \\
         $\frac{1}{1 + x^2}$ & $\arctan{x} + C$ & $\R$ \\
         $\frac{1}{\sqrt{1 - x^2}}$ & $-\arccos{x} + C \text{ ou } \arcsin{x} + C$ & $]-1, 1[$ \\
         \bottomrule
    \end{tabular}
    \caption{Primitives usuelles, $C \in \R$.}
    \label{tab:primitives_usuelles}
\end{table}

\begin{table}[!h]
    \centering
    \begin{tabular}{cc}
         \toprule
         $f(x)$ & DL \\ 
         \midrule
         $e^x$ & $\sum_{k=0}^n \frac{x^k}{k!} + o(x^n)$ \\
         $\cosh(x)$ & $ \sum_{k=0}^n \frac{x^{2k}}{(2k)!} + o(x^n)$ \\
         $\sinh(x)$ & $ \sum_{k=0}^n \frac{x^{2k + 1}}{(2k + 1)!} + o(x^{2n + 1})$ \\
         $\cos(x)$ & $ \sum_{k=0}^n (-1)^k \frac{x^{2k}}{(2k)!} + o(x^{2n})$ \\
         $\sin(x)$ & $ \sum_{k=0}^n (-1)^k \frac{x^{2k+1}}{(2k+1)!} + o(x^{2n + 1})$ \\
         $(1 + x)^{a}$ & $ \sum_{k=0}^n \binom{a}{k} x^k + o(x^n)$ \\ 
         $\frac{1}{1 - x}$ & $ \sum_{k=0}^n x^k + o(x^n)$ \\ 
         $\frac{1}{1 + x}$ & $ \sum_{k=0}^n (-1)^k x^k + o(x^n)$ \\ 
         $\ln(1 + x)$ & $ \sum_{k=1}^n (-1)^{k - 1} \frac{x^k}{k} + o(x^n)$ \\ 
         $\ln(1 - x)$ & $ \sum_{k=1}^n \frac{x^k}{k} + o(x^n)$ \\ 
         $\arctan(x)$ & $ \sum_{k=0}^n (-1)^k \frac{x^{2k+1}}{2k+1} + o(x^{2n+1})$ \\
         \bottomrule
    \end{tabular}
    \caption{Développements limités usuels en 0.}
    \label{tab:dl_usuels}
\end{table}

\begin{table}[!h]
    \centering
    \begin{tabular}{cc}
         \toprule 
         $e^x - 1 \underset{x \to 0}{\sim} x$ \\
         $\ln(1+x) \underset{x \to 0}{\sim} x$ \\
         $\ln(x) \underset{x \to 1}{\sim} x - 1$ \\ 
         $\sin(x) \underset{x \to 0}{\sim} \sinh(x) \underset{x \to 0}{\sim} \tan(x) \underset{x \to 0}{\sim} \tanh(x) \underset{x \to 0}{\sim} \arcsin(x) \underset{x \to 0}{\sim} \arctan(x) \underset{x \to 0}{\sim} x$ \\ 
         $1 - \cos(x) \underset{x \to 0}{\sim} \frac{x^2}{2}$ \\ 
         $\cosh(x) - 1 \underset{x \to 0}{\sim} \frac{x^2}{2}$ \\ 
         $\arccos(x) \underset{\substack{x \to 1 \\ x < 1}}{\sim} \sqrt{2}\sqrt{1-x}$ \\ 
         $(1 + x)^a - 1 \underset{x \to 0}{\sim} ax,\ a \in \R^*$ \\ 
         $\frac{1}{1-x} - 1 \underset{x \to 0}{\sim} x$ \\ 
         $\cosh(x) \underset{x \to +\infty}{\sim} \sinh(x) \underset{x \to +\infty}{\sim} \frac{e^x}{2}$ \\ 
         \bottomrule
    \end{tabular}
    \caption{\'Equivalents usuels}
    \label{tab:equiv_usuels}
\end{table}

\printbibliography

\end{document}
