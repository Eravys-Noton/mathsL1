\chapter{Logique et raisonnements}

\section{Logique}
\noindent Notations logiques :
\begin{enumerate}
	\item $\neg$ : \og non \fg.
	\item $\land$ : \og et \fg.
	\item $\lor$ : \og ou \fg.
	\item $\veebar$ : \og ou exclusif \fg.
\end{enumerate}

\begin{definition}[Assertion]
  Une assertion est une affirmation mathématique soit vraie soit fausse.
\end{definition}

\begin{definition}[Prédicat]
  Un prédicat est un énoncé mathématique dont la véracité dépend d'une ou plusieurs variables.
\end{definition}

Soient $P$ et $Q$ deux prédicats.
\begin{table}[!h]
	\centering
	\begin{tabular}{cccccc}
		\toprule
		$P$ & $Q$ & $P \land Q$ & $P \veebar Q$ & $\neg P$ & $P \implies Q$ \\
		\midrule
		$V$ & $V$ & $V$ & $V$ & $F$ & $V$ \\
		$V$ & $F$ & $F$ & $V$ & $F$ & $F$ \\
		$F$ & $V$ & $F$ & $V$ & $V$ & $V$ \\
		$F$ & $F$ & $F$ & $F$ & $V$ & $V$ \\
		\bottomrule
	\end{tabular}
	\caption{Table de vérité}
\end{table}

\begin{proposition}
  Soient $P$ et $Q$ deux prédicats.  
	\begin{enumerate} 
		\item $(P \implies Q) \land (Q \implies P) \implies (P \iff Q)$.
		\item $P \implies Q \iff \neg P \lor Q$.
		\item $\neg (P \lor Q) \iff \neg P \land \neg Q$.
		\item $\neg (P \land Q) \iff \neg P \lor \neg Q$.
		\item $P \implies Q \iff \neg Q \implies \neg P$.
	\end{enumerate}
\end{proposition}

\begin{remark}
    Les 2. et 3. sont les lois de Morgan, la 4. est la contraposée.
\end{remark}

\par \noindent Voici quelques négations usuelles :
\begin{itemize}
    \item Le contraire de \og $\forall x \in E, P(x)$ \fg est \og $\exists x \in E, \neg P(x)$ \fg.
    \item Le contraire de \og $x < y$ \fg est \og $x \geq y$ \fg.
\end{itemize}

\section{Raisonnements}

\begin{definition}[Raisonnement par récurrence]
    Il existe plusieurs variantes du raisonnement par récurrence, définissons d'abord la récurrence simple. L'objectif est de montrer qu'une propriété $P_n$ est vraie pour tout entier naturel $n$. 
    \begin{enumerate}
        \item \textbf{Initialisation} : On montre que $P_0$ est vraie.
        \item \textbf{Hérédité} : On suppose que pour un $k$ tel que $0 < k < n,\ P_k$ est vraie et on montre que $P_{k+1}$ est vraie.
    \end{enumerate}
\end{definition}

\begin{definition}[Raisonnement par l'absurde]
    Soit $P$ une assertion. Le raisonnement par l'absurde consiste à montrer que la assertion contraire de $P$, que l'on note $\overline{P}$ dans cette définition, est fausse impliquant que $P$ est vraie.
    Pour ce faire, on suppose que $\overline{P}$ est vraie et on commence à raisonner, s'il l'on arrive à une absurdité ou une contradiction, on a montré que $\overline{P}$ est fausse, impliquant que $P$ est vraie.
\end{definition}

\begin{definition}[Raisonnement par analyse-synthèse \cite{analyse_synthese_bibmath}]
	Raisonnement utilisé pour démontrer l'\textbf{existence} et l'\textbf{unicité} d'un objet.
	\begin{enumerate}
		\item \textbf{Analyse} : On suppose que l'objet existe et on cherche les conditions nécessaires que doit vérifier l'objet. Cette partie démontre l'\textbf{unicité}.
		\item \textbf{Synthèse} : On considère l'objet identifié dans la partie analyse et on vérifie qu'il a les propriétés souhaitées. Cette partie démontre l'\textbf{existence}.
	\end{enumerate}
\end{definition}