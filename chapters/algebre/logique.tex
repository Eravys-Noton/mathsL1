\chapter{Logique}
\begin{definition}[Assertion]
  Une assertion est une affirmation mathématique soit vraie soit fausse.
\end{definition}

\begin{definition}[Prédicat]
  Un prédicat est un énoncé mathématique dont la véracité dépend d'une ou plusieurs variables.
\end{definition}

\begin{proposition}[Opérations logiques de base]
  Soient $P, Q$ deux prédicats.
  \begin{table}[!h]
    \centering
    \begin{tabular}{cccccc}
      \toprule
      $P$ & $Q$ & $P$ et $Q$ & $P$ ou $Q$ & non($P$) & $P \implies Q$ \\
      \midrule
      $V$ & $V$ & $V$ & $V$ & $F$ & $V$ \\
      $V$ & $F$ & $F$ & $V$ & $F$ & $F$ \\
      $F$ & $V$ & $F$ & $V$ & $V$ & $V$ \\
      $F$ & $F$ & $F$ & $F$ & $V$ & $V$ \\
      \bottomrule
    \end{tabular}
  \end{table}
  
  \noindent Si $P \implies Q$ et $Q \implies P$ alors $P \iff Q$.
  \begin{multicols}{2}
    \begin{enumerate} 
    \item $P \implies Q \iff \operatorname{non}(P) \text{ ou } Q$
    \item $\operatorname{non}(P \text{ ou } Q) \iff \operatorname{non}(P) \text{ et } \operatorname{non}(Q)$
    \item $\operatorname{non}(P \text{ et } Q) \iff \operatorname{non}(P) \text{ ou } \operatorname{non}(Q)$
    \item $P \implies Q \iff \operatorname{non}(Q) \implies \operatorname{non}(P)$
  \end{enumerate}
  \end{multicols}
\end{proposition}

\begin{remark}
    Les 2. et 3. sont les lois de Morgan, la 4. est la contraposée.
\end{remark}

\begin{notation}
    En logique le \og ET \fg peut se noter $\land$ et le \og OU \fg $\lor$.
\end{notation}

\par \noindent Voici quelques négations usuelles :
\begin{itemize}
    \item Le contraire de \og $\forall x \in E, P(x)$ \fg est \og $\exists x \in E, \operatorname{non}(P(x))$ \fg.
    \item Le contraire de \og $x < y$ \fg est \og $x \geq y$ \fg.
\end{itemize}