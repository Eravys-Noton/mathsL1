\chapter{Logique}
\noindent Notations logiques :
\begin{enumerate}
	\item $\neg$ : \og non \fg.
	\item $\land$ : \og et \fg.
	\item $\lor$ : \og ou \fg.
	\item $\veebar$ : \og ou exclusif \fg.
\end{enumerate}

\begin{definition}[Assertion]
  Une assertion est une affirmation mathématique soit vraie soit fausse.
\end{definition}

\begin{definition}[Prédicat]
  Un prédicat est un énoncé mathématique dont la véracité dépend d'une ou plusieurs variables.
\end{definition}

\begin{proposition}[Opérations logiques de base]
  Soient $P$ et $Q$ deux prédicats.
  \begin{table}[!h]
    \centering
    \begin{tabular}{cccccc}
      \toprule
      $P$ & $Q$ & $P \land Q$ & $P \veebar Q$ & $\neg P$ & $P \implies Q$ \\
      \midrule
      $V$ & $V$ & $V$ & $V$ & $F$ & $V$ \\
      $V$ & $F$ & $F$ & $V$ & $F$ & $F$ \\
      $F$ & $V$ & $F$ & $V$ & $V$ & $V$ \\
      $F$ & $F$ & $F$ & $F$ & $V$ & $V$ \\
      \bottomrule
    \end{tabular}
  \end{table}
  
  \noindent $(P \implies Q) \land (Q \implies P) \implies (P \iff Q)$.
  \begin{multicols}{2}
	\begin{enumerate} 
		\item $P \implies Q \iff \neg P \lor Q$.
		\item $\neg (P \lor Q) \iff \neg P \land \neg Q$.
		\item $\neg (P \land Q) \iff \neg P \lor \neg Q$.
		\item $P \implies Q \iff \neg Q \implies \neg P$.
	\end{enumerate}
  \end{multicols}
\end{proposition}

\begin{remark}
    Les 2. et 3. sont les lois de Morgan, la 4. est la contraposée.
\end{remark}

\par \noindent Voici quelques négations usuelles :
\begin{itemize}
    \item Le contraire de \og $\forall x \in E, P(x)$ \fg est \og $\exists x \in E, \neg P(x)$ \fg.
    \item Le contraire de \og $x < y$ \fg est \og $x \geq y$ \fg.
\end{itemize}