\chapter{Ensembles}
\def\arraystretch{1}

Nous allons tout d'abord donner une définition intuitive d'un ensemble : Un ensemble $E$ est une collection d'objets appelés éléments. Si $E$ contient un élément $x$, on dit que $x$ appartient à $E$, noté $x \in E$.

\begin{definition}[Ensemble vide]
  L'ensemble vide noté $\varnothing$ est l'ensemble ne contenant aucun élément.
\end{definition}

\begin{definition}[Inclusion]
  Soient $E$ et $F$ deux ensembles.
  $$F \subseteq E \iff \forall x \in F : x \in E$$
  On dit que $F$ est inclu dans $E$.
\end{definition}

\begin{example}
	Soient $E = \{1, 2, 3, 4\}$ et $F = \{1, 2\}$.
	\[F \subseteq E\]
\end{example}

\begin{definition}[\'Egalité d'ensembles]
  Soient $E$ et $F$ deux ensembles.
  $$E = F \iff E \subseteq F \text{ et } F \subseteq E$$
\end{definition}

\begin{definition}[Singleton]
  Un singleton est un ensemble ne contenant qu'un seul élément.
\end{definition}

\begin{definition}[Réunion d'ensembles]
  Soient $E$ et $F$ deux ensembles.
  \[ E \cup F = \{ x : x \in E \text{ ou } x \in F \} \]
  On lit \og $E$ union $F$ \fg.
\end{definition}

\begin{example}
	Soient $E = \{ 1, 2, 3 \}$ et $F = \{ 4, 5, 6 \}$.
	\[ E \cup F = \{ 1, 2, 3, 4, 5, 6 \} \]
\end{example}

\begin{definition}[Intersection d'ensembles]
  Soient $E$ et $F$ deux ensembles.
  \[ E \cap F = \{ x : x \in E \text{ et } x \in F \} \]
  On lit \og $E$ inter $F$ \fg.
\end{definition}

\begin{example}
	Soient $E = \{ 1, 2, 3, 4, 5 \}$ et $F = \{ 3, 4, 5, 6, 7 \}$.
	\[ E \cap F = \{ 3, 4, 5 \} \]
\end{example}

\begin{definition}[Complémentaire d'un ensemble]
  Soient $E$ et $F$ deux ensembles. 
  \\
  Le complémentaire de $F$ dans $E$ noté $E \backslash F$ ou $\complement_E F$ est défini par :
  \[ E \backslash F = \{ x : x \in E,\ x \notin F \} \]
  Si on parle du complémentaire sans préciser d'ensemble, on peut se permettre de noter $E^C$ ou $\complement E$.
\end{definition}

\begin{example}
	Soient $E = \{ 1, 2, 3, 4, 5 \}$ et $F = \{ 3, 4, 5, 6, 7 \}$.
	\[ E \backslash F = \{ 1, 2 \} \]
\end{example}

\begin{proposition}
	Soient $A, B, C$ et $E$ des ensembles.
	\begin{enumerate}
		\item La réunion et l'intersection sont commutatives et assocatives.
		\item \'Elément neutre :
		\begin{enumerate}
			\item $A \cup \varnothing = A$.
			\item $A \cap A = A$.
		\end{enumerate}
		\item $A \subseteq E \iff A \cap E = E \cap A = A$.
		\item Distributivité :
		\begin{enumerate}
			\item $A \cup (B \cap C) = (A \cup B) \cap (A \cup C)$.
			\item $A \cap (B \cup C) = (A \cap B) \cup (A \cap C)$.
		\end{enumerate}
	\end{enumerate}
\end{proposition}

\begin{proposition}[Lois de Morgan]
  Soient $A$ et $B$ deux ensembles.
  \begin{align*}
    (A \cup B)^{C} &= A^{C} \cap B^{C} & (A \cap B)^{C} &= A^{C} \cup B^{C}
  \end{align*}
\end{proposition}

\begin{proof}
	Soient $A$ et $B$ deux ensembles et $x$ un élément quelconque. 
	\begin{enumerate}
		\item \boxed{\subseteq} : Par définition du complémentaire :
		\[ x \in (A \cup B)^C \iff x \notin (A \cup B) \]
		$x \notin A$ car $A \subseteq (A \cup B)$ et $x \notin B$ car $B \subseteq (A \cup B)$, ainsi $ \in A^C$ et $x \in B^C$. 
		\\
		Par définition de l'intersection :
		\[ x \in (A^C \cup B^C) \]
		et donc :
		\[ (A \cup B)^C \subseteq (A^C \cap B^C) \]
		\boxed{\supseteq} : Par définition de l'intersection :
		\begin{align*}
			x \in (A^C \cap B^C) &\iff x \in A^C \text{ et } x \in B^C \\
			&\iff x \notin A \text{ et } x \notin B \\
			&\iff x \notin (A \cup B)^C
		\end{align*}
		d'où :
		\[ (A^C \cap B^C) \subseteq (A \cup B)^C \]
		\item \boxed{\subseteq} : Par définition du complémentaire :
		\begin{align*}
			x \in (A \cap B)^C &\iff x \notin (A \cap B) \\
			&\iff x \notin A \text{ et } x \notin B \\
			&\iff x \in A^C \text{ et } x \in B^C \\
			&\iff x \in (A^C \cap B^C)
		\end{align*}
		Sachant que :
		\[ (A^C \cap B^C) \subseteq (A^C \cup B^C) \]
		On a :
		\[ x \in (A^C \cap B^C) \implies x \in (A^C \cup B^C) \]
		d'où :
		\[ (A \cap B)^C \subseteq (A^C \cup B^C) \]
		\boxed{\supseteq} : Par définition de la réunion :
		\begin{align*}
			x \in (A^C \cup B^C) &\iff x \in A^C \text{ ou } x \in B^C \\
			&\iff x \notin A \text{ ou } x \notin B \\
			&\iff x \notin (A \cap B) \\
			&\iff x \in (A \cap B)^C
		\end{align*}
		Ainsi : 
		\[ (A^C \cap B^C) \subseteq (A \cup B)^C \]
	\end{enumerate}
\end{proof}

\begin{definition}[Produit cartésien]
    Soient $E$ et $F$ des ensembles. On définit le produit cartésien :
    \[ E \times F = \{ (x, y),\ x \in E,\ y \in F \} \]
    Par convention : $\underbrace{E \times \cdots \times E}_{n \text{ fois}} = E^n$.
\end{definition}
