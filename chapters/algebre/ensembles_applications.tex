\chapter{Ensembles}
Nous allons tout d'abord donner une définition intuitive d'un ensemble : Un ensemble $E$ est une collection d'objets appelés éléments. Si $E$ contient un élément $x$, on dit que $x$ appartient à $E$, noté $x \in E$.

\begin{definition}[Ensemble vide]
  L'ensemble vide noté $\varnothing$ est l'ensemble ne contenant aucun élément.
\end{definition}

\begin{definition}[Inclusion]
  Soient $E, F$ deux ensembles.
  $$F \subset E \equiv F \subseteq E \iff \forall x \in F,\ x \in E$$
  On dit que $F$ est inclu dans $E$.
\end{definition}

\begin{definition}[\'Egalité d'ensembles]
  Soient $E, F$ deux ensembles.
  $$E = F \iff E \subset F \land F \subset E$$
\end{definition}

\begin{definition}[Singleton]
  Un singleton est un ensemble ne contenant qu'un seul élément.
\end{definition}

\begin{definition}[Réunion d'ensembles]
  Soient $E, F$ deux ensembles.
  \[ E \cup F = \{ x \in E \cup F \ | \ x \in E \lor x \in F \} \]
\end{definition}

\begin{definition}[Intersection d'ensembles]
  Soient $E, F$ deux ensembles.
  \[ E \cap F = \{ x \in E \cap F \ | \ x \in E \land x \in F \} \]
\end{definition}

\begin{definition}[Complémentaire d'un ensemble]
  Soient $E, F$ deux ensembles.
  \[ E \backslash F = \{ x \in E \backslash F \ | \ x \in E,\ x \notin F \} \]
\end{definition}

\begin{proposition}[Lois de Morgan]
  Soient $A, B$ deux ensembles.
  \begin{align*}
    (A \cup B)^{C} &= A^{C} \cap B^{C} & (A \cap B)^{C} &= A^{C} \cup B^{C}
  \end{align*}
\end{proposition}

\begin{definition}[Produit cartésien]
    Soient $E$ et $F$ des ensembles. On définit le produit cartésien :
    \[ E \times F \coloneqq \{ (x, y),\ x \in E,\ y \in F \} \]
    Par convention : $\underbrace{E \times \cdots \times E}_{n \text{ fois}} = E^n$.
\end{definition}
