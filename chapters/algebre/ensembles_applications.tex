\chapter{Ensembles}
\def\arraystretch{1}

Nous allons tout d'abord donner une définition intuitive d'un ensemble : Un ensemble $E$ est une collection d'objets appelés éléments. Si $E$ contient un élément $x$, on dit que $x$ appartient à $E$, noté $x \in E$.

\begin{definition}[Ensemble vide]
  L'ensemble vide noté $\varnothing$ est l'ensemble ne contenant aucun élément.
\end{definition}

\begin{definition}[Inclusion]
  Soient $E$ et $F$ deux ensembles.
  $$F \subseteq E \iff \forall x \in F,\ x \in E$$
  On dit que $F$ est inclu dans $E$.
\end{definition}

\begin{definition}[\'Egalité d'ensembles]
  Soient $E$ et $F$ deux ensembles.
  $$E = F \iff E \subset F \text{ et } F \subset E$$
\end{definition}

\begin{definition}[Singleton]
  Un singleton est un ensemble ne contenant qu'un seul élément.
\end{definition}

\begin{definition}[Réunion d'ensembles]
  Soient $E$ et $F$ deux ensembles.
  \[ E \cup F = \{ x \in E \cup F \ | \ x \in E \text{ ou } x \in F \} \]
  On lit \og $E$ union $F$ \fg.
\end{definition}

\begin{definition}[Intersection d'ensembles]
  Soient $E$ et $F$ deux ensembles.
  \[ E \cap F = \{ x \in E \cap F \ | \ x \in E \text{ et } x \in F \} \]
  On lit \og $E$ inter $F$ \fg.
\end{definition}

\begin{definition}[Complémentaire d'un ensemble]
  Soient $E$ et $F$ deux ensembles.
  \[ E \backslash F = \{ x \in E \backslash F \ | \ x \in E,\ x \notin F \} \]
\end{definition}

\begin{proposition}
	Soient $A, B, C$ et $E$ des ensembles.
	\begin{enumerate}
		\item La réunion et l'intersection sont commutatives et assocatives.
		\item \'Elément neutre :
		\begin{enumerate}
			\item $A \cup \varnothing = A$.
			\item $A \cap A = A$.
		\end{enumerate}
		\item $A \subseteq E \iff A \cap E = E \cap A = A$.
		\item Distributivité :
		\begin{enumerate}
			\item $A \cup (B \cap C) = (A \cup B) \cap (A \cup C)$.
			\item $A \cap (B \cup C) = (A \cap B) \cup (A \cap C)$.
		\end{enumerate}
	\end{enumerate}
\end{proposition}

\begin{proposition}[Lois de Morgan]
  Soient $A$ et $B$ deux ensembles.
  \begin{align*}
    (A \cup B)^{C} &= A^{C} \cap B^{C} & (A \cap B)^{C} &= A^{C} \cup B^{C}
  \end{align*}
\end{proposition}

\begin{definition}[Produit cartésien]
    Soient $E$ et $F$ des ensembles. On définit le produit cartésien :
    \[ E \times F \coloneqq \{ (x, y),\ x \in E,\ y \in F \} \]
    Par convention : $\underbrace{E \times \cdots \times E}_{n \text{ fois}} = E^n$.
\end{definition}
