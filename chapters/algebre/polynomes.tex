\chapter{Polynômes}
\def\arraystretch{1}

\section{Définitions}
\begin{definition}[Polynôme]
    Un \textbf{polynôme} est un élément de l'ensemble 
    \begin{align*}
        \K[X] = \left\{ \sum_{i=0}^n a_i X^i : \ a_i \in \K,\ n \in \N \right\}
    \end{align*}
    Si $a_n \neq 0$, $n$ est le degré du polynôme, on le note $ \deg(P) = n$.
\end{definition}

\begin{proposition}
	Soient $P, Q, R \in \K[X]$.
	\begin{enumerate}
		\item Associativité :
		\begin{enumerate}
			\item $(P + Q) + R = P + (Q + R)$.
			\item $(P \cdot Q) \cdot R = P \cdot (Q \cdot R)$.
		\end{enumerate}
		\item Commutativité :
		\begin{enumerate}
			\item $P + Q = Q + P$.
			\item $P \cdot Q = Q \cdot P$.
		\end{enumerate}
		\item \'Elément neutre :
		\begin{enumerate}
			\item $P + 0 = P$.
			\item $P \cdot 1 = P$.
		\end{enumerate}
		\item \'Elément absorbant : $P \cdot 0 = 0$.
		\item Distributivité :
		\begin{enumerate}
			\item $(P + Q) \cdot R = P \cdot R + Q \cdot R$.
			\item $P \cdot (Q + R) = P \cdot Q + P \cdot R$.
		\end{enumerate}
	\end{enumerate}
\end{proposition}

\begin{proposition}
	$\forall \lambda \in \K^*,\ P, Q \in \K[X]$. 
	\begin{enumerate}
    		\item $\deg(\lambda) = 0$.
    		\item $\deg(P \cdot Q) = \deg(P) + \deg(Q)$.
    		\item $\deg(P + Q) = \mathrm{max}\left(\deg(P),\ \deg(Q)\right)$.
    		\item $\deg(P \circ Q) = \deg(P) \cdot \deg(Q)$.
            \item $\deg(0) = -\infty$.
    	\end{enumerate}
\end{proposition}

\section{Arithmétique des polynômes}

\begin{theorem}[Division euclidienne de polynômes]
	Soient $P_1, P_2$ deux polynômes non nuls.
	\begin{align*}
		\exists ! (Q, R) \in (\K[X])^2 \text{ tel que } P_1 = P_2 Q + R 
	\end{align*}
	Vérifiant :
	$\deg(R) = -\infty$ ou $0 \leq \deg(R) < \deg(Q)$.
\end{theorem}

\begin{definition}[Polynôme irréductible]
	Un polynôme $P \in \K[X]$ non constant est dit irréductible, s'il n'existe pas $(P_1, P_2) \in (\K[X])^2$ tel que $P = P_1 P_2$ et $\deg(P_1) < \deg(P_2)$.
\end{definition}

\begin{proposition}
	Soit $P \in \K[X]$ non constant et irréductible.
	\begin{enumerate}
		\item $\K = \C \iff \deg(P) = 1$.
		\item $\K = \R \iff \deg(P) = 1 \text{ ou } \deg(P) = 2$. 
	\end{enumerate}
\end{proposition}

\begin{proposition}
	Soient $P \in K[X]$ non constant et $\alpha \in \K$.
	\\
	On dit que $\alpha$ est une racine si et seulement si :
	\[ P(\alpha) = 0 \iff X - \alpha \mid P(X) \]
\end{proposition}

\begin{definition}\label{def:ordre_mult}
	Soient $P$ un polynôme non constant, $\alpha \in \K$ et $m \in \N^*$ $\alpha$ est une racine d'ordre de multiplicité $m$ de $P$ si et seulement si :
	\begin{align*}
		(X - \alpha)^m \mid P \text{ et } (X - \alpha)^{m+1} \nmid P.
	\end{align*}
\end{definition}

\begin{theorem}
	Soient $P$ un polynôme non constant, $\alpha \in \K$ et $m \in \N^*$
	\begin{align*}
		(X - \alpha)^m \mid P \iff P(\alpha) = P'(\alpha) = \cdots = P^{m-1}(\alpha) = 0.
	\end{align*}
\end{theorem}

\begin{theorem}[Théorème fondamental de l'algèbre]
	Soit $P(X)$ un polynôme à coefficients complexes de degré $n$. $P(X)$ admet $n$ racines complexes. 
\end{theorem}

\begin{theorem}
	Soient $P$ un polynôme de degré $n$ et $a_0, \ldots, a_n \in \Z$ ses coefficients tels que :
	\[ P(X) = a_n X^n + \cdots + a_1 X + a_0 \]
	\[ \exists (p,\ q) \in \Z \times \Z^*,\ \pgcd(p,\ q) = 1,\ P\left( \frac{p}{q} \right) \implies p \mid a_0 \land q \mid a_n \]
\end{theorem}

\begin{proof}
	\[ P \left(\frac{p}{q}\right) = 0 \]
	\begin{align}\label{proof_racines_rationnelles_1}
		a_n \frac{p^n}{q^n} + \cdots + a_1 \frac{p}{q} + a_0 &= 0 
	\end{align}
	\begin{align}\label{proof_racines_rationnelles_2}
		(\ref{proof_racines_rationnelles_1}) \cdot q^n &\iff a_n p^n + \cdots + a_1 p q^{n-1} + a_0 q^n = 0 \\
	&\iff a_0 q^n = -a_n p^n - \cdots - a_1 p q^{n-1} \nonumber \\
	&\iff a_0 = -\frac{a_n}{q^n} p^n - \cdots - \frac{a_1}{q} p \nonumber \\
	&\iff a_0 = p \left( - \frac{a_1}{q} - \cdots - \frac{a_n}{q^n} p^{n-1} \right) \nonumber \\
	&\iff p \mid a_0 \nonumber
	\end{align}
	\begin{align*}
		(\ref{proof_racines_rationnelles_2}) &\iff a_n p^n = - a_{n-1} p^{n-1} q - \cdots - a_1 p q^{n-1} - a_0 q^n \\
		&\iff a_n = - \frac{a_{n-1}}{p} q - \cdots - \frac{a_1}{p^{n-1}} q^{n-2} - \frac{a_0}{p^n} q^n \\
		&\iff a_n = q \left( -\frac{a_{n-1}}{p} - \cdots - \frac{a_0}{p^n} q^{n-1} \right) \\
		&\iff q \mid a_n
	\end{align*}
\end{proof}

\section{Fractions rationnelles}
\begin{definition}[Fraction rationnelle]
	$F(X)$ est appelée fraction rationnelle s'il existe $P(X), Q(X) \in \K[X]$ tels que :
	\begin{align*}
		F(X) = \frac{P(X)}{Q(X)},\ Q(X) \neq 0
	\end{align*}
	avec $\deg(F) = \deg(P) - \deg(Q)$. \\
	Si $\deg(P) > \deg(Q)$, alors il existe $E(X)$ appelée la \textbf{partie entière} et des polynômes $R(X), S(X)$ tels que $\deg(R) < \deg(S)$ et : 
	\begin{align*}
		F(X) = E(X) + \frac{R(X)}{S(X)}.
	\end{align*}
\end{definition}

\begin{theorem}[Décomposition en éléments simples]
	Soient $P, Q \in \K[X],\ F(X) = \frac{P(X)}{Q(X)},\ m, n, \alpha_i, \beta_i \in \N^*$.
	\\
	Sur $\R$ : Soient $a_i, b_i, c_i \in \R$.
	\[ Q(X) = \prod_{i = 1}^{n} (X - a_i)^{\alpha_i} \cdot \prod_{i = 1}^{m} (X^2 + b_i X + c_i)^{\beta_i} \]
	Soient $A_{i, j},\ B_{i, j},\ C_{i, j} \in \R$. $F$ s'écrit de manière unique (à l'ordre des termes près) sous la forme :
	\[ F(X) = \sum_{i = 1}^{n} \sum_{j = 1}^{\alpha_i} \frac{A_{i, j}}{\left(X - a_i\right)^{j}} + \sum_{i = 1}^{m} \sum_{j = 1}^{\beta_i} \frac{B_{i, j}X + C_{i, j}}{(X^2 + b_iX + c_i)^{j}} \]
	Sur $\C$ : Soient $a_i \in \C$.
	\[ Q(X) = \prod_{i = 1}^{n} (X - a_i)^{n} \]
	Soient $A_{i, j} \in \C$. $F$ s'écrit de manière unique (à l'ordre des termes près) sous la forme :
	\[ F(X) = \sum_{i = 1}^{n} \sum_{j = 1}^{\alpha_i} \frac{A_{i, j}}{\left(X - a_i\right)^{j}} \]
\end{theorem}

\begin{proof}\cite{math-sup.fr_decomp}
    Sur $\C$ : il est recommandé de consulter la section sur les espaces vectoriels avant (voir \autoref{chap:espaces_vectoriels}).
    Rappelons tout d'abord :
    \[ \C[X] = \left\{ \sum_{k = 0}^{n} a_k X^k : n \in \N,\ a_k \in \C \right\} \]
    Notons l'espace des fractions rationnelles :
    \[ \C(X) = \left\{ \frac{P(X)}{Q(X)} : P, Q \in \C[X],\ Q \neq 0_{\C[X]} \right\} \]
    La stratégie de la démonstration consiste à montrer une égalité de deux espaces vectoriels :
    \[ E = \left\{ \frac{P(X)}{Q(X)} : P \in \C[X],\ \deg(P) < \deg(Q) \right\} \]
    \[ F = \left\{ \sum_{i = 1}^{n} \sum_{j = 1}^{m_i} \frac{a_{i, j}}{(X - \alpha_i)^{j}} : a_{ij} \in \C \right\} \]
    \textbf{\'Etude de $E$ :} Posons $d = \deg(Q)$
    \begin{align*}
        E &= \left\{ \frac{P(X)}{Q(X)} : P \in \C[X],\ \deg(P) < d \right\} \\
          &= \left\{ \frac{a_0 + a_1X + \cdots + a_{d - 1} X^{d - 1}}{Q(X)} : a_0, \ldots, a_{d - 1} \in \C \right\} \\
          &= \left\{ a_0 \frac{1}{Q(X)} + a_1 \frac{X}{Q(X)} + \cdots + a_{d - 1} \frac{X^{d-1}}{Q(X)} : a_0, \ldots, a_{d - 1} \in \C \right\} \\
          &= \operatorname{Vect}\left\{ \frac{1}{Q(X)}, \frac{X}{Q(X)}, \ldots, \frac{X^{d-1}}{Q(X)} \right\}
    \end{align*}
    Notons : $\mathcal{F} = \operatorname{Vect}\left\{ \frac{1}{Q(X)}, \frac{X}{Q(X)}, \ldots, \frac{X^{d-1}}{Q(X)} \right\}$
    Nous avons réussi à exprimer $E$ sous la forme d'une famille de vecteurs, nous en déduisons que $E$ est un espace vectoriel admettant $\mathcal{F}$ comme famille génératrice, montrons ensuite que $\mathcal{F}$ est libre :
    Soient $ \lambda_0,\ldots,\lambda_{d-1} \in \C$
    \begin{align*}
        \lambda_0 \frac{1}{Q(X)} + \cdots + \lambda_{d - 1} \frac{X^{d-1}}{Q(X)} &= \frac{\lambda_0 + \lambda_{1} X + \cdots + \lambda_{d - 1} X^{d - 1}}{Q(X)}
    \end{align*}
    \[ \frac{\lambda_0 + \lambda_{1} X + \cdots + \lambda_{d - 1} X^{d - 1}}{Q(X)} = 0_{\C(X)} \iff \lambda_0 + \lambda_{1} X + \cdots + \lambda_{d - 1} X^{d - 1} = 0_{\C[X]} \]
    Un polynôme est nul si tous ses coefficients sont nuls, ainsi :
    \[ \lambda_0 = \cdots = \lambda_{d-1} = 0 \]
    Ainsi $\mathcal{F}$ est libre. $\mathcal{F}$ est libre et génératrice, elle forme donc une base de $E$ et ainsi : 
    \[ \dim(E) = d \]
    \textbf{\'Etude de} $F$ :
    \[ F = \left\{ \sum_{i = 1}^{n} \sum_{j = 1}^{m_i} \frac{a_{i, j}}{(X - \alpha_i)^{j}} : a_{ij} \in \C \right\} \]
    On peut ainsi écrire $F$ ainsi : 
    \begin{align*}
        F &= \left\{ \frac{a_{1, 1}}{(X - \alpha_1)} + \cdots + \frac{a_{1, m_1}}{(X - a_1)^{m_1}} + \cdots + \frac{a_{n, 1}}{(X - \alpha_n)} + \cdots + \frac{a_{n, m_n}}{(X - \alpha_n)^{m_n}} : a_{ij} \in \C \right\} \\
        &= \operatorname{Vect} \left\{ \frac{1}{(X - \alpha_1)}, \ldots, \frac{1}{(X - \alpha_1)^{m_1}}, \ldots, \frac{1}{(X - \alpha_n)}, \ldots, \frac{1}{(X - \alpha_n)^{m_n}}  \right\}
    \end{align*}
    Notons $\mathcal{F}_2 = \operatorname{Vect} \left\{ \frac{1}{(X - \alpha_1)}, \ldots, \frac{1}{(X - \alpha_1)^{m_1}}, \ldots, \frac{1}{(X - \alpha_n)}, \ldots, \frac{1}{(X - \alpha_n)^{m_n}}  \right\}$. Nous en déduisons que $F$ est un espace vectoriel admettant $\mathcal{F}_2$ comme famille génératrice, montrons que $\mathcal{F}_2$ est libre : Soient $a_{i,j} \in \C$ tels que :
    \begin{align*}
        \begin{split}
            a_{1,1} \frac{1}{(X - \alpha_1)} + a_{1,2} \frac{1}{(X - \alpha_1)^2} + \cdots + a_{1,m_1-1} \frac{1}{(X - \alpha_1)^{m_1 - 1}} + \cdots \\
            + a_{n,1} \frac{1}{(X - \alpha_n)} + a_{n,2} \frac{1}{(X - \alpha_n)^2} + \cdots + a_{n,m_n} \frac{1}{(X - \alpha_n)^{m_n}} = 0_{\C(X)}
        \end{split}
    \end{align*}
    En multipliant l'équation par $(X - \alpha_1)^{m_1}$ :
    \begin{align*}
        \begin{split}
            a_{1,1} (X - \alpha_1)^{m_1 - 1} + a_{1,2} (X - \alpha_1)^{m_1 - 2} + \cdots + a_{1,m_1} + \\
            (X - \alpha_1)^{m_1}
            \left( 
            a_{2,1} \frac{1}{(X - \alpha_2)} + \cdots + a_{n1} \frac{1}{(X - \alpha_n)} + \cdots + a_{n,m_n} \frac{1}{(X - \alpha_n)^{m_n}}
            \right)
            = 0_{\C(X)}
         \end{split}
    \end{align*}
    En posant $X = \alpha_1$, on trouve que $a_{1,m_1} = 0$.
    En remplaçant $a_{1,m_1}$ par sa valeur dans l'équation initiale, celle-ci devient : 
    \begin{align*}
        \begin{split}
            a_{1,1} \frac{1}{(X - \alpha_1)} + a_{1,2} \frac{1}{(X - \alpha_1)^2} + \cdots + a_{1,m_1 - 1} \frac{1}{(X - \alpha_1)^{m_1 - 1}} + \cdots \\
            + a_{n,1} \frac{1}{(X - \alpha_n)} + a_{n,2} \frac{1}{(X - \alpha_n)^2} + \cdots + a_{n,m_n} \frac{1}{(X - \alpha_n)^{m_n}} = 0_{\C(X)}
        \end{split}
    \end{align*}
    En multipliant l'équation par $(X - \alpha_1)^{m_1 - 1}$ et en posant $X = \alpha_1$, on trouve que $a_{1,m_1 - 1} = 0$. On procède de la même manière jusqu'à prouver $a_{1,1} = \cdots = a_{1,m_1} = 0$.
    On continue ainsi de suite pour montrer que tous les coefficients sont nuls et donc que $\mathcal{F}_2$ est libre. $\mathcal{F}_2$ est libre et génératrice, elle forme donc une base de $F$.
    \[ \dim(F) = m_1 + \cdots + m_n \]
    Mais aussi : 
    \[ \deg(Q) = \deg \left( \prod_{i=1}^{n} (X - \alpha_i)^{m_i} \right) = m_1 + \cdots + m_n \]
    Ainsi : \[ \dim(F) = \deg(Q) = d \]
    \textbf{Inclusion de $F$ dans $E$ :}
    \\
    Un élément de $F$ est de la forme suivante : 
    \[ \frac{a_{1,1}}{(X - \alpha_1)} + \cdots + \frac{a_{1,m_1}}{(X - \alpha_1)^{m_1}} + \cdots + \frac{a_{n,1}}{(X - \alpha_n)} + \cdots + \frac{a_{n,m_n}}{(X - \alpha_n)^{m_n}} \]
    En mettant toutes les fractions sur le même dénominateur, on obtient :
    \[ \frac{a_{1,1}(X - \alpha_1)^{m_1 - 1} \prod_{i = 2}^{n}(X - \alpha_i)^{m_i} + \cdots + a_{1m_1} \prod_{i = 2}^{n}(X - \alpha_i)^{m_i} + \cdots + a_{n,m_n} \prod_{i = 1}^{n - 1}(X - \alpha_i)^{m_i}}{Q(X)} \]
    Le degré du numérateur est strictement inférieur à celui de $Q$. Donc c'est un élément de $E$. Ainsi :
    \[ F \subset E \]
    Nous avons montré que $\dim(E) = \dim(F)$ et que $F \subset E$, ainsi $E = F$.
    Ce qui revient à dire que toute fraction rationnelle 
    \begin{align*}
        \frac{P(X)}{Q(X)},\ \deg(P) < \deg(Q)
    \end{align*}
    s'exprime sous la forme 
    \begin{align*}
        \sum_{i=1}^n \sum_{j = 1}^{m_i} \frac{a_{i,j}}{(X - \alpha_i)^j}
    \end{align*}
    L'unicité de la décomposition découle du fait que tout élément d'un espace vectoriel s'exprime par une unique combinaison linéaire des vecteurs d'une de ses bases.
\end{proof}