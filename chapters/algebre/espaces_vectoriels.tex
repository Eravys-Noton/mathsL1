\chapter{Espaces vectoriels}\label{chap:espaces_vectoriels}

\section{Définitions}
\begin{definition}[Loi de composition]
	Soient $E$, $F$ deux ensembles et $f$ une application.
	\begin{enumerate}
		\item On dit que $f$ est une loi de composition \emph{interne} si et seulement si : $f : E \times E \to E$.
		\item On dit que $f$ est une loi de composition \emph{externe} si et seulement si : $f : E \times F \to E$.
	\end{enumerate}
\end{definition}

\begin{definition}[Magma]
	On appelle \emph{magma} un ensemble $M$ muni d'une loi de composition \emph{interne} \og $*$ \fg. On note $(M, *)$.
\end{definition}

\begin{definition}[Groupe]
	Soit $G$ un ensemble muni d'une loi de composition interne \og $*$ \fg telle que :
    $\forall (g_1, g_2, g_3) \in G^3$ :
    \begin{multicols}{2}
    	\begin{enumerate}
    		\item $(g_1 * g_2) * g_3 = g_1 * (g_2 * g_3)$
    		\item $\exists 0_G \in G,\ g_1 * 0_G = 0_G * g_1 = g_1$
    		\item $\exists g^{-1} \in G,\ g_1 * g^{-1} = g^{-1} * g_1 = 0_G$
    	\end{enumerate}
    \end{multicols}
    \noindent On dit que $(G, *)$ est un groupe. De plus, si et seulement :
    \[ \forall (g_1, g_2) \in G^2,\ g_1 * g_2 = g_2* g_1 \]
    on dit que $(G, *)$ est un groupe \emph{commutatif} ou \emph{abélien}.
\end{definition}

\begin{definition}[Anneau]
	Soit $A$ un ensemble muni de deux lois de compositions \emph{internes} \og $+$ \fg et \og $\cdot$ \fg sur $A$ telles que : 
	\begin{enumerate}
		\item $(A, +)$ est un groupe \emph{commutatif}.
		\item $\forall (a, b, c) \in A^3,\ a \cdot (b \cdot c) = (a \cdot b) \cdot c$.
		\item \og $\cdot$ \fg possède un \emph{élément neutre}.
		\item $\forall (a, b, c ) \in A^3$ : 
		\begin{enumerate}
			\item $a \cdot (b + c) = a \cdot b + a \cdot c$.
			\item $(b + c) \cdot a = b \cdot a + c \cdot a$.
		\end{enumerate}
	\end{enumerate}
	On dit que $A$ est \emph{intègre} si $\forall (a, b) \in A^2$ :
	\begin{enumerate}
		\item $A$ est commutatif : $a \cdot b = b \cdot a$.
		\item $a \cdot b = 0 \implies a = 0 \lor b = 0$.
	\end{enumerate}
\end{definition}

\begin{definition}[Corps]
    Un corps est un anneau commutatif dans lequel tout élément non nul est inversible.
\end{definition}

\begin{definition}[$\K$-espace vectoriel]
	Soit $\K$ un corps. 
	\\
	Un $\K$-espace vectoriel est un ensemble $E$ composé d'une loi de composition \emph{interne} \og $+$ \fg et une loi de composition \emph{externe} \og $\cdot$ \fg telles que :
	\begin{align*}
		+ : E \times E &\to E \\
		(x, y) &\mapsto x + y
	\end{align*}
	\begin{align*}
		\cdot : \K \times E &\to E \\
		(\lambda, u) &\mapsto \lambda \cdot u
	\end{align*}
	\begin{enumerate}
		\item $(E, +)$ est un groupe commutatif.
		\item $\forall (\lambda_1, \lambda_2) \in \K^2,\ (u, v) \in E^2$
		\begin{multicols}{2}
            \begin{enumerate}
			\item $\lambda_1 \cdot (u + v) = \lambda_1 \cdot u + \lambda_1 \cdot v$
			\item $(\lambda_1 + \lambda_2) \cdot u = \lambda_1 \cdot u + \lambda_2 \cdot u$
			\item $(\lambda_1 \cdot \lambda_2) \cdot u = \lambda_1 \cdot (\lambda_2 \cdot u)$
			\item $1 \cdot u = u$
		\end{enumerate}
        \end{multicols}
	\end{enumerate}
\end{definition}

\begin{definition}[Sous-espace vectoriel]
	Soit $E$ un espace-vectoriel, $F$ est un sous-espace vectoriel de $E$ si :
	\begin{multicols}{2}
	    \begin{enumerate}
    		\item $F \subset E$
    		\item $F \neq \emptyset$
    		\item $\forall (u, v) \in F^2,\ \lambda \in \K,\ u + \lambda v \in F$
    	\end{enumerate}
	\end{multicols}
\end{definition}

\begin{definition}[Somme directe]
	Soient $F_1, F_2 \subset E$. On dit que $F_1$ et $F_2$ sont en \emph{somme directe} ou qu'ils sont \emph{supplémentaires} dans $E$ si et seulement si :
	\begin{multicols}{2}
	    \begin{enumerate}
		\item $F_1 \cap F_2 = \{ 0_E \}$
		\item $F_1 + F_2 = E$
	\end{enumerate}
	\end{multicols}
	On note alors :
	\[ F_1 \oplus F_2 = E \]
\end{definition}

\begin{definition}
	Soient $E$ un espace vectoriel et $\mathcal{F} = (u_1, \ldots, u_n)$ une famille de vecteurs dans $E$.
	\begin{enumerate}
		\item $\mathcal{F}$ est libre :
		\begin{align*}
			\forall \lambda_i \in \K,\ u_i \in E,\ \sum_{i = 1}^n \lambda_i u_i = 0_E \implies \lambda_1 = \cdots = \lambda_n = 0
		\end{align*}
		\item $\mathcal{F}$ est génératrice :
		\begin{align*}
			\forall x \in E,\ \exists \lambda_i \in \K,\ x = \sum_{i = 1}^n \lambda_i u_i 
		\end{align*}
	\end{enumerate}
\end{definition}

\section{Base et dimension}
\begin{definition}[Base]
	Une famille de vecteurs est une base si elle est \emph{libre} et \emph{génératrice}.
\end{definition}

\begin{proposition}
	Soient une famille de vecteurs $\mathcal{F}$ telle que $\mathcal{F} = (u_1, \ldots, u_n)$ et un espace vectoriel $E$.
	\[ 
	\mathcal{F} \text{ est une base} \iff 
	\forall x \in E,\ \exists ! \lambda_i \in \R,\ x = \sum_{i = 1}^n \lambda_i u_i
	\]
\end{proposition}

\begin{proposition}
	Soit $\mathcal{F} = (u_1, \ldots, u_n)$ une famille de vecteurs. $\mathcal{F}$ est une base de $\R^n$ si et seulement si :
	\[ \det(\mathcal{F}) \neq 0 \]
\end{proposition}

\begin{definition}[Dimension d'un espace vectoriel]
	Soit $E$ un espace vectoriel, on appelle dimension de $E$, notée $\dim(E)$, le nombre d'éléments d'une base de $E$. 
\end{definition}


\begin{proposition}
	Soient $E$ un espace vectoriel de dimension $n$ et $\mathcal{F} = (u_1, \ldots, u_n)$ une famille de $n$ vecteurs dans $E$. Alors on a :
	\begin{multicols}{3}
	    \begin{enumerate}
		\item $\mathcal{F}$ est une base.
		\item $\mathcal{F}$ est libre.
		\item $\mathcal{F}$ est génératrice.
	\end{enumerate}
	\end{multicols}
	\noindent Ainsi il suffit de montrer que $\mathcal{F}$ est libre pour montrer les deux autres propriétés.
\end{proposition}

\begin{theorem}[Théorème de la base incomplète]
	Toute famille libre peut être complétée en une base.
\end{theorem}

\begin{theorem}[Théorème de la base extraite]
    De toute famille génératrice, on peut extraire une base.
\end{theorem}

\begin{theorem}
    Chaque espace vectoriel admet une base.
\end{theorem}

\begin{proposition}
	Soient $E$ un espace vectoriel et $F, G\subseteq E$.
	\begin{multicols}{2}
	    \begin{enumerate}
    		\item $\dim(F \oplus G) = \dim(F) + \dim(G)$ 
    		\item $\dim(F + G) \leq \dim(E)$
    	\end{enumerate}
	\end{multicols}
\end{proposition}

\begin{theorem}[Théorème de Grassman]
	Soient $F, G$ des espaces vectoriels.
	\[ \dim(F+G) = \dim(F) + \dim(G) - \dim(F\cap G) \]
\end{theorem}

