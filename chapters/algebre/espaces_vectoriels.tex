\chapter{Espaces vectoriels}\label{chap:espaces_vectoriels}
\def\arraystretch{1}

\section{Définitions}
\begin{definition}[Loi de composition]
	Soient $E$, $F$ deux ensembles et $f$ une application.
	\begin{enumerate}
		\item On dit que $f$ est une loi de composition \textbf{interne} si et seulement si : $f : E \times E \to E$.
		\item On dit que $f$ est une loi de composition \textbf{externe} si et seulement si : $f : E \times F \to E$.
	\end{enumerate}
\end{definition}

\begin{definition}[Magma]
	On appelle \textbf{magma} un ensemble $M$ muni d'une loi de composition \textbf{interne} \og $*$ \fg. 
	\\
	On le note $(M, *)$.
\end{definition}

\begin{definition}[Monoïde \cite{monoide_bibmath}]
	On appelle \textbf{monoïde} un ensemble $M$ muni d'une loi de composition \textbf{interne} \og $*$ \fg \textbf{associative}. C'est-à-dire que pour tous $x, y, z \in M$ :
	\[ x * (y * z) = (x * y) * z \]
\end{definition}

\begin{definition}[Groupe]
	Soit $(G, *)$ un \textbf{monoïde}.
	\\ 
	On dit que $(G, *)$ est un \textbf{groupe} si et seulement si pour tous $g_1, g_2 \in G$ : 
    \begin{enumerate}
    		\item $\exists 0_G \in G,\ g_1 * 0_G = 0_G * g_1 = g_1$
    		\item $\exists g^{-1} \in G,\ g_1 * g^{-1} = g^{-1} * g_1 = 0_G$
    	\end{enumerate}
    \noindent De plus, si et seulement :
    \[ g_1 * g_2 = g_2 * g_1 \]
    on dit que $(G, *)$ est un groupe \textbf{commutatif} ou \textbf{abélien}.
\end{definition}

\begin{definition}[Anneau \cite{bibmath_resume_groupes}]
	Soit $A$ un ensemble muni de deux lois de compositions \textbf{internes} \og $+$ \fg et \og $\cdot$ \fg sur $A$ telles que pour tous $a, b, c \in A$ : 
	\begin{enumerate}
			\item $(A, +)$ est un groupe \textbf{commutatif}.
			\item $a \cdot (b \cdot c) = (a \cdot b) \cdot c$.
			\item $a \cdot (b + c) = a \cdot b + a \cdot c$.
			\item $(b + c) \cdot a = b \cdot a + c \cdot a$.
			\item \og $\cdot$ \fg possède un \textbf{élément neutre}.
		\end{enumerate}
	\noindent On dit que $A$ est \textbf{intègre} si :
	\begin{enumerate}
			\item $A$ est commutatif : $a \cdot b = b \cdot a$.
			\item $a \cdot b = 0 \implies a = 0 \text{ ou } b = 0$.
		\end{enumerate}
\end{definition}

\begin{definition}[Corps \cite{bibmath_resume_groupes}]
    Un corps est un anneau \textbf{commutatif} dans lequel tout élément non nul est inversible.
\end{definition}

\begin{definition}[$\K$-espace vectoriel]
	Soit $\K$ un corps. 
	\\
	Un $\K$-espace vectoriel est un ensemble $E$ composé d'une loi de composition \textbf{interne} \og $+$ \fg et d'une loi de composition \textbf{externe} \og $\cdot$ \fg telles que :
	\begin{center}
		$
		\begin{array}{cc}
			\appli{+}{E \times E}{E}{(x,y)}{x+y}
			&
			\appli{\cdot}{\K \times E}{E}{(\lambda, u)}{\lambda \cdot u}
		\end{array}
		$	
	\end{center}		
	
	\begin{enumerate}
		\item $(E, +)$ est un \textbf{groupe commutatif}.
		\item $\forall \lambda_1, \lambda_2 \in \K,\ u, v \in E :$
		\begin{enumerate}
			\item $\lambda_1 \cdot (u + v) = \lambda_1 \cdot u + \lambda_1 \cdot v$
			\item $(\lambda_1 + \lambda_2) \cdot u = \lambda_1 \cdot u + \lambda_2 \cdot u$
			\item $(\lambda_1 \cdot \lambda_2) \cdot u = \lambda_1 \cdot (\lambda_2 \cdot u)$
			\item $1 \cdot u = u$
		\end{enumerate}
	\end{enumerate}
\end{definition}

\begin{definition}[Sous-espace vectoriel]
	Soit $E$ un $\K$-espace-vectoriel, $F$ est un sous-espace vectoriel de $E$ si :
	\begin{enumerate}
    		\item $F \subseteq E$
    		\item $F \neq \varnothing$
    		\item $\forall u, v \in F,\ \lambda \in \K : u + \lambda v \in F$
    	\end{enumerate}
\end{definition}

\begin{definition}[Somme directe]
	Soient $E$ un $\K$-espace vectoriel et $F_1, F_2 \subseteq E$. On dit que $F_1$ et $F_2$ sont en \textbf{somme directe} ou qu'ils sont \textbf{supplémentaires} dans $E$ si et seulement si :
	\begin{multicols}{2}
	    \begin{enumerate}
		\item $F_1 \cap F_2 = \{ 0_E \}$
		\item $F_1 + F_2 = E$
	\end{enumerate}
	\end{multicols}
	On note alors :
	\[ F_1 \oplus F_2 = E \]
\end{definition}

\begin{proposition}
	Soient $E$ un $\K$-espace vectoriel et $F_1, F_2 \subseteq E$, on a $F_1 + F_2 \subseteq E$. 
\end{proposition}

\begin{proof}
	Tout d'abord $0_E \in F_1$ et $0_E \in F_2$, on a donc $0_E \in F_1 + F_2$.
	\\
	Ensuite, soient $x, y \in F_1 + F_2,\ \lambda \in \K$, posons pour $v_1, w_1 \in F_1$ et $v_2, w_2 \in F_2$ :
	\begin{align*}
		\begin{cases}
			x = v_1 + v_2 \\
			y = w_1 + w_2
		\end{cases}
	\end{align*}
	On a :
	\begin{align*}
		x + \lambda y &= v_1 + v_2 + \lambda (w_1 + w_2) \\
		&= v_1 + \lambda w_1 + v_2 + \lambda w_2
	\end{align*}
	ce qui implique que $x + \lambda y \in F_1 + F_2$.
\end{proof}

\begin{proposition}
	$\Vect(u_1, \ldots, u_k) \subseteq \R^n$ est un sous-espace vectoriel.
\end{proposition}

\begin{proof}
	$0_{\R^n} \in \Vect(u_1, \ldots, u_k)$ pour $\lambda_1 = \cdots = \lambda_k = 0$.
	\\
	Soient $v = \sum_{i=1}^{k} \alpha_i u_i$, $w = \sum_{i=1}^{k} \beta_i u_i$ et $\lambda \in \R$ alors on a :
	\begin{align*}
		v + \lambda w &= \sum_{i=1}^{k} (\alpha_i u_i) + \lambda \sum_{i=1}^{k} (\beta_i u_i) \\
		&= \sum_{i=1}^{k} (\alpha_i + \lambda \beta_i) u_i \in \Vect(u_1, \ldots, u_k)
	\end{align*}
\end{proof}

\begin{proposition}
	Soient $E$ un $\K$-espace vectoriel, $F_1, F_2 \subseteq E$ alors $F_1 \cap F_2 \subseteq E$.
\end{proposition}

\begin{proof}
	Tout d'abord, $0_E \in F_1$, $0_E \in F_2$ alors $0_E \in F_1 \cap F_2$.
	\\
	Ensuite pour tout $u, v \in F_1 \cap F_2,\ \lambda \in \K$ on a :
	\begin{enumerate}
		\item $u + \lambda v \in F_1$ car $F_1 \subseteq E$. 
		\item $u + \lambda v \in F_2$ car $F_2 \subseteq E$.
	\end{enumerate}
	ainsi $u + \lambda v \in F_1 \cap F_2$.
\end{proof}

\begin{definition}
	Soient $n \in \N^*$, $E$ un espace vectoriel et $\mathcal{F} = (u_1, \ldots, u_n)$ une famille de vecteurs de $E$.
	\begin{enumerate}
		\item On dit que $\mathcal{F}$ est \textbf{libre} si et seulement si :
		\begin{align*}
			\forall \lambda_i \in \K : \sum_{i = 1}^n \lambda_i u_i = 0_E \implies \lambda_1 = \cdots = \lambda_n = 0
		\end{align*}
		\item On dit que $\mathcal{F}$ est \textbf{génératrice} si et seulement si :
		\begin{align*}
			\forall x \in E,\ \exists \lambda_i \in \K : x = \sum_{i = 1}^n \lambda_i u_i 
		\end{align*}
	\end{enumerate}
\end{definition}

\begin{remark}
	Si une famille n'est pas libre, on dit qu'elle est liée.
\end{remark}

\section{Base et dimension}
\begin{definition}[Base]
	Une famille de vecteurs est une base si elle est \textbf{libre} et \textbf{génératrice}.
\end{definition}

\begin{proposition}
	Soient un espace vectoriel $E$ et $\mathcal{F} = (u_1, \ldots, u_n)$ une famille de vecteurs de $E$.
	\[ 
	\mathcal{F} \text{ est une base} \iff 
	\forall x \in E,\ \exists ! \lambda_i \in \K^n : x = \sum_{i = 1}^n \lambda_i u_i
	\]
\end{proposition}

\begin{proof}
	L'existence est évidente car $\mathcal{F}$ est génératrice. 
	\\
	Montrons l'unicité : Soient $\lambda_1, \ldots, \lambda_n, \mu_1, \ldots, \mu_n \in \K$.
	On a d'une part :
	\[ \forall x \in E,\ x = \sum_{i=1}^{n} \lambda_i u_i \]
	puis d'autre part :
	\[ \forall x \in E,\ x = \sum_{i=1}^{n} \mu_i u_i \]
	Donc on a :
	\begin{align*}
		&\sum_{i=1}^{n} \lambda_i u_i = \sum_{i=1}^{n} \mu_i u_i \\
		\iff &\sum_{i=1}^{n} (\lambda_i - \mu_i) u_i = 0 
	\end{align*}
	Or $\mathcal{F}$ est libre, donc pour tout $i \in \llbracket 1, n \rrbracket,\ \lambda_i - \mu_i = 0 \implies \lambda_i = \mu_i$.
\end{proof}

\begin{proposition}
	Soit $\mathcal{F} = (u_1, \ldots, u_n)$ une famille de vecteurs de $\R^n$. $\mathcal{F}$ est une base de $\R^n$ si et seulement si :
	\[ \det(\mathcal{F}) \neq 0 \]
\end{proposition}

\begin{proof}
	$\mathcal{F} = (u_1, \ldots, u_n)$ est une base $\iff \forall x \in \R^n,\ \exists ! (\lambda_1, \ldots, \lambda_n) \in \R^n : x = \sum_{i=1}^{n} \lambda_i u_i$.
	\begin{align*}
		x = \sum_{i=1}^{n} \lambda_i u_i &= 
		\begin{pmatrix}
			\lambda_1 u_{1,1} + \cdots + \lambda_n u_{1,n} \\
			\vdots \\
			\lambda_1 u_{n,1} + \cdots + \lambda_n u_{n,n}
		\end{pmatrix}
		\\
		&= 
		\begin{pmatrix}
			u_{1,1} & \cdots & u_{1,n} \\
			\vdots & \ddots & \vdots \\
			u_{n,1} & \cdots & u_{n,n}
		\end{pmatrix}
		\begin{pmatrix}
			\lambda_1 \\
			\vdots \\
			\lambda_n
		\end{pmatrix}
	\end{align*}
	Posons 
	$
	A =
	\begin{pmatrix}
		u_{1,1} & \cdots & u_{1,n} \\
		\vdots & \ddots & \vdots \\
		u_{n,1} & \cdots & u_{n,n}
	\end{pmatrix}
	$
	et 
	$\lambda =
	\begin{pmatrix}
		\lambda_1 \\
		\vdots \\
		\lambda_n
	\end{pmatrix}
	$\\
	On a donc :
	\[ x = A \cdot \lambda \]
	Il existe une solution unique si et seulement s'il existe l'inverse de $A$, c'est-à-dire que $\det(A) \neq 0$.
\end{proof}

\begin{definition}[Dimension d'un espace vectoriel]
	Soit $E$ un espace vectoriel, on appelle dimension de $E$, notée $\dim(E)$, le nombre d'éléments d'une base de $E$. 
\end{definition}

\begin{proposition}
	Soient $E$ un espace vectoriel de dimension $n$ et $\mathcal{F} = (u_1, \ldots, u_n)$ une famille de vecteurs de $E$. Alors on a :
	\begin{enumerate}
		\item $\mathcal{F}$ est une base.
		\item $\mathcal{F}$ est libre.
		\item $\mathcal{F}$ est génératrice.
	\end{enumerate}
	\noindent Ainsi il suffit de montrer que $\mathcal{F}$ est libre pour montrer les deux autres propriétés.
\end{proposition}

\begin{theorem}[Théorème de la base incomplète]
	Toute famille libre peut être complétée en une base.
\end{theorem}

\begin{theorem}[Théorème de la base extraite]
    De toute famille génératrice, on peut extraire une base.
\end{theorem}

\begin{theorem}
    Chaque espace vectoriel admet une base.
\end{theorem}

\begin{corollary}
	Soient $n, N \in \N^*,\ \mathcal{F} = (u_1, \ldots, u_N)$, $E$ un espace vectoriel tel que $\dim(E) = n$.
	\begin{enumerate}
		\item Si $N > n$ alors $\mathcal{F}$ n'est pas libre.
		\item Si $N < n$ alors $\mathcal{F}$ n'est pas génératrice.
	\end{enumerate}
\end{corollary}

\begin{proposition}
	Soient $E$ un espace vectoriel et $F, G\subseteq E$.
	\begin{multicols}{2}
	    \begin{enumerate}
    		\item $\dim(F \oplus G) = \dim(F) + \dim(G)$ .
    		\item $\dim(F + G) \leq \dim(E)$.
    	\end{enumerate}
	\end{multicols}
\end{proposition}

\begin{theorem}[Théorème de Grassman]
	Soient $F$ et $G$ deux espaces vectoriels.
	\[ \dim(F+G) = \dim(F) + \dim(G) - \dim(F\cap G) \]
\end{theorem}

