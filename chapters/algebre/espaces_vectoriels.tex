\chapter{Espaces vectoriels}\label{chap:espaces_vectoriels}
\def\arraystretch{1}

\section{Définitions}
\begin{definition}[Loi de composition]
	Soient $E$, $F$ deux ensembles et $f$ une application.
	\begin{enumerate}
		\item On dit que $f$ est une loi de composition \textbf{interne} si et seulement si : $f : E \times E \to E$.
		\item On dit que $f$ est une loi de composition \textbf{externe} si et seulement si : $f : E \times F \to E$.
	\end{enumerate}
\end{definition}

\begin{definition}[Magma]
	On appelle \textbf{magma} un ensemble $M$ muni d'une loi de composition \textbf{interne} \og $*$ \fg. 
	\\
	On le note $(M, *)$.
\end{definition}

\begin{definition}[Groupe]
	Soit $(G, *)$ un magma.
	\\ 
	On dit que $(G, *)$ est un groupe si et seulement si pour $g_1, g_2, g_3 \in G$ : 
    \begin{enumerate}
    		\item $(g_1 * g_2) * g_3 = g_1 * (g_2 * g_3)$
    		\item $\exists 0_G \in G,\ g_1 * 0_G = 0_G * g_1 = g_1$
    		\item $\exists g^{-1} \in G,\ g_1 * g^{-1} = g^{-1} * g_1 = 0_G$
    	\end{enumerate}
    \noindent De plus, si et seulement :
    \[ g_1 * g_2 = g_2* g_1 \]
    on dit que $(G, *)$ est un groupe \textbf{commutatif} ou \textbf{abélien}.
\end{definition}

\begin{definition}[Anneau \cite{bibmath_resume_groupes}]
	Soit $A$ un ensemble muni de deux lois de compositions \textbf{internes} \og $+$ \fg et \og $\cdot$ \fg sur $A$ telles que pour $a, b, c \in A$ : 
	\begin{enumerate}
			\item $(A, +)$ est un groupe \textbf{commutatif}.
			\item $a \cdot (b \cdot c) = (a \cdot b) \cdot c$.
			\item $a \cdot (b + c) = a \cdot b + a \cdot c$.
			\item $(b + c) \cdot a = b \cdot a + c \cdot a$.
			\item \og $\cdot$ \fg possède un \textbf{élément neutre}.
		\end{enumerate}
	\noindent On dit que $A$ est \textbf{intègre} si :
	\begin{enumerate}
			\item $A$ est commutatif : $a \cdot b = b \cdot a$.
			\item $a \cdot b = 0 \implies a = 0 \lor b = 0$.
		\end{enumerate}
\end{definition}

\begin{definition}[Corps \cite{bibmath_resume_groupes}]
    Un corps est un anneau \textbf{commutatif} dans lequel tout élément non nul est inversible.
\end{definition}

\begin{definition}[$\K$-espace vectoriel]
	Soit $\K$ un corps. 
	\\
	Un $\K$-espace vectoriel est un ensemble $E$ composé d'une loi de composition \textbf{interne} \og $+$ \fg et d'une loi de composition \textbf{externe} \og $\cdot$ \fg telles que :
	\begin{center}
		$
		\begin{array}{cc}
			\appli{+}{E \times E}{E}{(x,y)}{x+y}
			&
			\appli{\cdot}{\K \times E}{E}{(\lambda, u)}{\lambda \cdot u}
		\end{array}
		$	
	\end{center}		
	
	\begin{enumerate}
		\item $(E, +)$ est un groupe commutatif.
		\item $\forall \lambda_1, \lambda_2 \in \K,\ u, v \in E :$
		\begin{enumerate}
			\item $\lambda_1 \cdot (u + v) = \lambda_1 \cdot u + \lambda_1 \cdot v$
			\item $(\lambda_1 + \lambda_2) \cdot u = \lambda_1 \cdot u + \lambda_2 \cdot u$
			\item $(\lambda_1 \cdot \lambda_2) \cdot u = \lambda_1 \cdot (\lambda_2 \cdot u)$
			\item $1 \cdot u = u$
		\end{enumerate}
	\end{enumerate}
\end{definition}

\begin{definition}[Sous-espace vectoriel]
	Soit $E$ un espace-vectoriel, $F$ est un sous-espace vectoriel de $E$ si :
	\begin{enumerate}
    		\item $F \subset E$
    		\item $F \neq \varnothing$
    		\item $\forall u, v \in F,\ \lambda \in \K,\ u + \lambda v \in F$
    	\end{enumerate}
\end{definition}

\begin{definition}[Somme directe]
	Soient $F_1, F_2 \subseteq E$. On dit que $F_1$ et $F_2$ sont en \textbf{somme directe} ou qu'ils sont \textbf{supplémentaires} dans $E$ si et seulement si :
	\begin{multicols}{2}
	    \begin{enumerate}
		\item $F_1 \cap F_2 = \{ 0_E \}$
		\item $F_1 + F_2 = E$
	\end{enumerate}
	\end{multicols}
	On note alors :
	\[ F_1 \oplus F_2 = E \]
\end{definition}

\begin{definition}
	Soient $n \in \N^*$, $E$ un espace vectoriel et $\mathcal{F} = (u_1, \ldots, u_n)$ une famille de vecteurs de $E$.
	\begin{enumerate}
		\item On dit que $\mathcal{F}$ est \textbf{libre} si et seulement si :
		\begin{align*}
			\forall (\lambda_1, \ldots, \lambda_n) \in \K^n,\ \sum_{i = 1}^n \lambda_i u_i = 0_E \implies \lambda_1 = \cdots = \lambda_n = 0
		\end{align*}
		\item On dit que $\mathcal{F}$ est \textbf{génératrice} si et seulement si :
		\begin{align*}
			\forall x \in E,\ \exists (\lambda_1, \ldots, \lambda_n) \in \K^n,\ x = \sum_{i = 1}^n \lambda_i u_i 
		\end{align*}
	\end{enumerate}
\end{definition}

\begin{remark}
	Si une famille n'est pas libre, on dit qu'elle est liée.
\end{remark}

\section{Base et dimension}
\begin{definition}[Base]
	Une famille de vecteurs est une base si elle est \textbf{libre} et \textbf{génératrice}.
\end{definition}

\begin{proposition}
	Soient un espace vectoriel $E$ et $\mathcal{F} = (u_1, \ldots, u_n)$ une famille de vecteurs de $E$.
	\[ 
	\mathcal{F} \text{ est une base} \iff 
	\forall x \in E,\ \exists ! (\lambda_1, \ldots, \lambda_n) \in \K^n,\ x = \sum_{i = 1}^n \lambda_i u_i
	\]
\end{proposition}

\begin{proof}
	L'existence est évidente car $\mathcal{F}$ est génératrice. 
	\\
	Montrons l'unicité : Soient $\lambda_1, \ldots, \lambda_n, \mu_1, \ldots, \mu_n \in \K$.
	On a d'une part :
	\[ \forall x \in E,\ x = \sum_{i=1}^{n} \lambda_i u_i \]
	puis d'autre part :
	\[ \forall x \in E,\ x = \sum_{i=1}^{n} \mu_i u_i \]
	Donc on a :
	\begin{align*}
		&\sum_{i=1}^{n} \lambda_i u_i = \sum_{i=1}^{n} \mu_i u_i \\
		\iff &\sum_{i=1}^{n} (\lambda_i - \mu_i) u_i = 0 
	\end{align*}
	Or $\mathcal{F}$ est libre, donc pour tout $i \in \llbracket 1, n \rrbracket,\ \lambda_i - \mu_i = 0 \implies \lambda_i = \mu_i$.
\end{proof}

\begin{proposition}
	Soit $\mathcal{F} = (u_1, \ldots, u_n)$ une famille de vecteurs de $\R^n$. $\mathcal{F}$ est une base de $\R^n$ si et seulement si :
	\[ \det(\mathcal{F}) \neq 0 \]
\end{proposition}

\begin{definition}[Dimension d'un espace vectoriel]
	Soit $E$ un espace vectoriel, on appelle dimension de $E$, notée $\dim(E)$, le nombre d'éléments d'une base de $E$. 
\end{definition}


\begin{proposition}
	Soient $E$ un espace vectoriel de dimension $n$ et $\mathcal{F} = (u_1, \ldots, u_n)$ une famille de vecteurs de $E$. Alors on a :
	\begin{enumerate}
		\item $\mathcal{F}$ est une base.
		\item $\mathcal{F}$ est libre.
		\item $\mathcal{F}$ est génératrice.
	\end{enumerate}
	\noindent Ainsi il suffit de montrer que $\mathcal{F}$ est libre pour montrer les deux autres propriétés.
\end{proposition}

\begin{theorem}[Théorème de la base incomplète]
	Toute famille libre peut être complétée en une base.
\end{theorem}

\begin{theorem}[Théorème de la base extraite]
    De toute famille génératrice, on peut extraire une base.
\end{theorem}

\begin{theorem}
    Chaque espace vectoriel admet une base.
\end{theorem}

\begin{proposition}
	Soient $E$ un espace vectoriel et $F, G\subset E$.
	\begin{multicols}{2}
	    \begin{enumerate}
    		\item $\dim(F \oplus G) = \dim(F) + \dim(G)$ 
    		\item $\dim(F + G) \leq \dim(E)$
    	\end{enumerate}
	\end{multicols}
\end{proposition}

\begin{theorem}[Théorème de Grassman]
	Soient $F$ et $G$ deux espaces vectoriels.
	\[ \dim(F+G) = \dim(F) + \dim(G) - \dim(F\cap G) \]
\end{theorem}

