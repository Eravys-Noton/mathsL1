\chapter{Calcul Algèbrique}
\def\arraystretch{1}

\par \noindent Dans cette partie, $\K$ désigne soit $\N$, soit $\Z$, soit $\Q$, soit $\R$.

\begin{proposition}
    Soient $a, b, c \in \K$ :
    \begin{enumerate}
    	\item Associativité :
    	\begin{enumerate}
    		\item $a + (b + c) = (a + b) + c$.
    		\item $a \cdot (b \cdot c) = (a \cdot b) \cdot c$.
    	\end{enumerate}
    	\item Commutativité : 
    	\begin{enumerate}
    		\item $a + b = b + a$.
    		\item $a \cdot b = b \cdot a$.
    	\end{enumerate}
    	\item \'Elément neutre :
    	\begin{enumerate}
    		\item $a + 0 = a$.
    		\item $a \cdot 1 = a$.
    	\end{enumerate}
    	\item \'Elément absorbant : $a \cdot 0 = 0$.
    	\item \'Elément symétrique si $\K \neq \N$ : $\exists a' \in \K : a + a' = 0$
    	\item \'Elément inverse si $\K \in \{\Q, \R\}$ : $\exists a' \in \K : a \cdot a' = 1$.
    	\item Distributivité :
    	\begin{enumerate}
    		\item $a \cdot (b + c) = a \cdot b + a \cdot c$.
    		\item $(a + b) \cdot c = a \cdot c + b \cdot c$.
    	\end{enumerate}
    \end{enumerate}
\end{proposition}

\begin{proposition}[Opérations sur les fractions]
    Soient $a, c \in \Z,\ b, d \in \Z^*$.
    \begin{enumerate}
    	\item $\frac{a}{b} + \frac{c}{d} = \frac{ad + bc}{bd}$.
    	\item $\frac{a}{b} \cdot \frac{c}{d} = \frac{a \cdot c}{b \cdot d}$.
    \end{enumerate}
\end{proposition}

\begin{proof}
	Soient $a, a', c, c' \in \N$ et $b, b', d, d' \in \N^*$ tels que :
	\begin{align*}
		\frac{a}{b} &= \frac{a'}{b'} & \frac{c}{d} &= \frac{c'}{d'}
	\end{align*}
	ce qui revient à dire :
	\begin{align*}
		ab' &= a'b & cd' &= c'd
	\end{align*}
	\begin{enumerate}
		\item Montrons que :
		\[  \]
		ce qui revient à montrer que :
		\begin{align*}
			&\frac{ad + bc}{bd} = \frac{a'd' + b'c'}{b'd'} \\
			\iff &(ad + bc)(b'd') = (a'd' + b'c')(bd)
			\\
			\iff &(ad + bc)(b'd') - (a'd' + b'c')(bd) = 0
		\end{align*}
		\begin{align*}
			(ad + bc)(b'd') - (a'd' + b'c')(bd) &= adb'd' + bcb'd' - a'd'bd - b'c'bd \\
			&= (ab' - a'b)(dd') + (cd' - c'd)(bb')
		\end{align*}
		Sachant que $ab' = a'b$ et $cd' = c'd$, on a $ab' - a'b = 0$ et $cd' - c'd = 0$.
		\\
		Ainsi :
		\[ (ab' - a'b)(dd') + (cd' - c'd)(bb') = 0 \]
		\item Montrons que :
		\begin{align*}
			&\frac{ac}{bd} = \frac{a'c'}{b'd'} \\
			\iff &(ac)(b'd') = (a'c')(bd) \\
			\iff &(ac)(b'd') - (a'c')(bd) = 0
		\end{align*}
		\begin{align*}
			(ac)(b'd') - (a'c')(bd) &= acb'd' - a'c'bd + a'bcd' - a'bcd' \\
			&= (ab' - a'b)(cd') + (cd' - c'd)(a'b) 
		\end{align*}
		Sachant que $ab' = a'b$ et $cd' = c'd$, on a $ab' - a'b = 0$ et $cd' - c'd = 0$.
		\\
		Ainsi :
		\[ (ab' - a'b)(cd') + (cd' - c'd)(a'b) = 0 \]
	\end{enumerate}
\end{proof}

\begin{definition}[Somme]
    Soient $a_k \in \R$, $m, n \in \N$.
    \[ \sum_{k = m}^n a_k = a_m + a_{m + 1} + \cdots + a_n \]
\end{definition}

\begin{proposition}[Linéarité de la somme]
    $\forall m, n \in \N^2,\ a_k, b_k, \lambda \in \R,\ m \leq k \leq n$.
    \[ \sum_{k = m}^{n} (a_k + \lambda b_k) = \sum_{k = m}^n a_k + \lambda \sum_{k = m}^n b_k \]
\end{proposition}

\begin{proof}
	Nous pouvons le vérifier en développant les sommes.
\end{proof}

\begin{proposition}[Somme téléscopique]
    $\forall m, n \in \N,\ a_k \in \R,\ m \leq k \leq n$.
    \[ \sum_{k = m}^n (a_k - a_{k - 1}) = a_n - a_{m - 1} \]
\end{proposition}

\begin{proof}
	Nous pouvons le vérifier en développant la somme.
\end{proof}

\begin{proposition}
	$\forall n \in \N$.
	\begin{align*}
		\sum_{k = 0}^{n} k &= \frac{n(n+1)}{2} & 
		\sum_{k = 0}^{n} k^2 &= \frac{n(n+1)(2n+1)}{6} &
		\sum_{k = 0}^{n} k^3 &= \frac{n^2(n+1)^2}{4}
	\end{align*}
\end{proposition}

\begin{proof}
	On procède par récurrence simple.
\end{proof}

\begin{definition}
	Pour tout $n \in \N$.
	\[ n! = n \cdot (n - 1) \cdot \ldots \cdot 2 \cdot 1 \]
\end{definition}

\begin{proposition}
	$\forall n, k \in \N,\ n \geq k$.
	\[ \binom{n}{p} = \frac{n!}{p!(n - p)!} \]
\end{proposition}

\begin{proposition}
	Soient $n, k \in \N$ tels que $n \geq k$.
	\[
	\binom{n}{k - 1} + \binom{n}{k} = \binom{n+1}{k}
	\]
\end{proposition}

\begin{proof}
	D'une part :
	\[ \binom{n}{k - 1} = \frac{n!}{(k-1)!(n - (k - 1))!} \]
	D'autre part :
	\[ \binom{n}{k} = \frac{n!}{k!(n - k)!} \]
	On a ensuite : 
	\begin{align*}
		\binom{n}{k - 1} + \binom{n}{k} &= 
		\frac{n!}{(k-1)!(n - (k - 1))!} + \frac{n!}{k!(n - k)!} \\
		&= \frac{n!}{(k-1)!(n - k + 1))!} + \frac{n!}{k!(n - k)!} \\
		&= \frac{n!k}{k!(n-k+1)!} + \frac{n!(n-k+1)}{k!(n-k+1)!} \\
		&= \frac{n!k + n!(n - k + 1)}{k!(n - k + 1)!} \\
		&= \frac{n!(k + n - k + 1)}{k!(n - k + 1)!} \\
		&= \frac{n!(n+1)}{(n - k + 1)!} \\
		&= \frac{(n+1)!}{k!(n + 1 - k)!} \\
		&= \binom{n+1}{k}
	\end{align*}
\end{proof}
	
\begin{proposition}[Binôme de Newton]
	$\forall a, b \in \K,\ n \in \N$.
	\begin{align*}
		(a + b)^n = \sum_{k = 0}^{n} \binom{n}{k} a^kb^{n-k}
	\end{align*}
\end{proposition}

\begin{proof}
	Procédons par récurrence sur $n \in \N$ pour montrer $P(n) : \forall a, b \in \K,\ n \in \N : (a + b)^n = \sum_{k = 0}^{n} \binom{n}{k} a^kb^{n-k}$.
	\begin{enumerate}
		\item \textbf{Initialisation} : Pour $n = 0$.
		\\
		D'une part :
		\[ (a + b)^0 = 1 \]
		D'autre part :
		\[ \sum_{k = 0}^{0} \binom{0}{k} a^k b^{-k} = 1 \]
		$P(0)$ est vraie.
		\item \textbf{Hérédité} : Supposons que $P(n)$ est vraie pour un $n \geq 1$ fixé.
		\\
		\begin{align*}
			(a + b)^n &= \sum_{k=0}^{n} \binom{n}{k} a^k b^{n-k} \\
			(a + b)^{n+1} &= (a+b) \sum_{k=0}^{n} \binom{n}{k} a^k b^{n-k} \\
			&= \sum_{k = 0}^{n} \binom{n}{k} a^{k+1} b^{n-k} + \sum_{k=0}^{n} \binom{n}{k} a^k b^{n - k + 1}
		\end{align*}
		Procédons à un changement de variable $j = k = 1$.
		\begin{align*}
			(a + b)^{n+1} &= \sum_{j = 1}^{n + 1} \binom{n}{j - 1} a^j b^{n - j + 1} + \sum_{k = 0}^{n} \binom{n}{k} a^k b^{n-k+1} \\
			&= \sum_{k = 1}^{n + 1} \binom{n}{k - 1} a^k b^{n - k + 1} + \sum_{k = 0}^{n} \binom{n}{k} a^k b^{n - k + 1} \\
			&= \binom{n}{n} a^{n+1} + \binom{n}{0} b^{n+1} + \sum_{k = 1}^{n} \binom{n}{k - 1} a^k b^{n - k + 1} + \sum_{k = 1}^{n} \binom{n}{k} a^k b^{n - k + 1} \\
			&= a^{n+1} + b^{n+1} + \sum_{k = 1}^{n} \left[ \binom{n}{k - 1} + \binom{n}{k} \right] a^k b^{n - k + 1} \\
			&= a^{n+1} + b^{n+1} + \sum_{k=1}^{n} \binom{n+1}{k} a^k b^{n - k + 1} \\
			&= \sum_{k=0}^{n+1} \binom{n + 1}{k} a^k b^{n + 1 - k}
		\end{align*}
		Ce qui montre que $P(n+1)$ est vraie. 
		\\
		Par principe de récurrence, la propriété $P(n)$ est vraie pour tout $n \in \N$.
	\end{enumerate}
\end{proof}

\begin{definition}[Produit]
    $\forall m, n \in \N,\ m \leq n,\ a_k \in \R,\ m \leq k \leq n$
    \[ \prod_{k = m}^{n} = a_m \cdot a_{m+1} \cdot \ldots \cdot a_n \]
\end{definition}