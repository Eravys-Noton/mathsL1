\chapter{Calcul Algèbrique}
\def\arraystretch{1}

\par \noindent Dans cette partie, $\K$ désigne soit $\N$, soit $\Z$, soit $\Q$, soit $\R$.

\begin{axiom}[Loi de composition \og $+$ \fg]
    $\forall a, b, c \in \K$.
    \begin{enumerate}
    	\item Associativité :
    	\[ a + (b + c) = (a + b) + c \]
    	\item Commutativité : 
    	\[ a + b = b + a \]
    	\item \'Elément neutre :
    	\[ a + 0 = a \]
    	\item Symétrie si $\K \neq \N$ :
    	\[ \exists a' \in \K,\ a + a' = 0 \]
    \end{enumerate}
\end{axiom}

\begin{axiom}[Loi de composition \og $\cdot$ \fg]
    $\forall a, b, c \in \K$. 
    \begin{enumerate}
    	\item Associativité :
    	\[ a \cdot (b \cdot c) = (a \cdot b) \cdot c \]
    	\item Commutativité : 
    	\[ a \cdot b = b \cdot a \]
    	\item \'Elément neutre :
    	\[ a \cdot 1 = a \]
    	\item \'Elément absorbant :
    	\[ a \cdot 0 = 0 \]
    	\item Distributivité :
    	\[ a \cdot (b + c) = a \cdot b + a \cdot c \]
    	\[ (a + b) \cdot c = a \cdot c + b \cdot c \]
    \end{enumerate}
\end{axiom}

\begin{proposition}[Opérations sur les fractions]
    $\forall a, c \in \Z,\ b, d \in \Z^*$.
    \begin{align*}
    	\frac{a}{b} + \frac{c}{d} &= \frac{ad + bc}{bd} & \frac{a}{b} \cdot \frac{c}{d} &= \frac{a \cdot c}{b \cdot d}
    \end{align*}
\end{proposition}

\begin{proof}
	Soient $a, a', c, c' \in \N$ et $b, b', d, d' \in \N^*$ tels que :
	\begin{align*}
		\frac{a}{b} &= \frac{a'}{b'} & \frac{c}{d} &= \frac{c'}{d'}
	\end{align*}
	ce qui revient à dire :
	\begin{align*}
		ab' &= a'b & cd' &= c'd
	\end{align*}
	\begin{enumerate}
		\item Montrons que :
		\[  \]
		ce qui revient à montrer que :
		\begin{align*}
			&\frac{ad + bc}{bd} = \frac{a'd' + b'c'}{b'd'} \\
			\iff &(ad + bc)(b'd') = (a'd' + b'c')(bd)
			\\
			\iff &(ad + bc)(b'd') - (a'd' + b'c')(bd) = 0
		\end{align*}
		\begin{align*}
			(ad + bc)(b'd') - (a'd' + b'c')(bd) &= adb'd' + bcb'd' - a'd'bd - b'c'bd \\
			&= (ab' - a'b)(dd') + (cd' - c'd)(bb')
		\end{align*}
		Sachant que $ab' = a'b$ et $cd' = c'd$, on a $ab' - a'b = 0$ et $cd' - c'd = 0$.
		\\
		Ainsi :
		\[ (ab' - a'b)(dd') + (cd' - c'd)(bb') = 0 \]
		\item Montrons que :
		\begin{align*}
			&\frac{ac}{bd} = \frac{a'c'}{b'd'} \\
			\iff &(ac)(b'd') = (a'c')(bd) \\
			\iff &(ac)(b'd') - (a'c')(bd) = 0
		\end{align*}
		\begin{align*}
			(ac)(b'd') - (a'c')(bd) &= acb'd' - a'c'bd + a'bcd' - a'bcd' \\
			&= (ab' - a'b)(cd') + (cd' - c'd)(a'b) 
		\end{align*}
		Sachant que $ab' = a'b$ et $cd' = c'd$, on a $ab' - a'b = 0$ et $cd' - c'd = 0$.
		\\
		Ainsi :
		\[ (ab' - a'b)(cd') + (cd' - c'd)(a'b) = 0 \]
	\end{enumerate}
\end{proof}

\begin{definition}[Somme]
    $\forall m, n \in \N,\ a_k \in \R,\ m \leq k \leq n$.
    \[ \sum_{k = m}^n a_k = a_m + a_{m + 1} + \cdots + a_n \]
\end{definition}

\begin{proposition}[Linéarité de la somme]
    $\forall m, n \in \N^2,\ a_k, b_k, \lambda \in \R,\ m \leq k \leq n$.
    \[ \sum_{k = m}^{n} (a_k + \lambda b_k) = \sum_{k = m}^n a_k + \lambda \sum_{k = m}^n b_k \]
\end{proposition}

\begin{proof}
	Nous pouvons le vérifier en développant les sommes.
\end{proof}

\begin{proposition}[Somme téléscopique]
    $\forall m, n \in \N,\ a_k \in \R,\ m \leq k \leq n$.
    \[ \sum_{k = m}^n (a_k - a_{k - 1}) = a_n - a_{m - 1} \]
\end{proposition}

\begin{proof}
	Nous pouvons le vérifier en développant la somme.
\end{proof}

\begin{proposition}
	$\forall n, p \in \N,\ n \geq p$.
	\[ \binom{n}{p} = \frac{n!}{p!(n - p)!} \]
	$\forall n \in \N$.
	\begin{align*}
		\sum_{k = 0}^{n} k &= \frac{n(n+1)}{2} & 
		\sum_{k = 0}^{n} k^2 &= \frac{n(n+1)(2n+1)}{6} &
		\sum_{k = 0}^{n} k^3 &= \frac{n^2(n+1)^2}{4}
	\end{align*}
	$\forall a, b \in \K,\ n \in \N$.
	\begin{align*}
		(a + b)^n = \sum_{k = 0}^{n} \binom{n}{k} a^kb^{n-k}
	\end{align*}
\end{proposition}
\begin{proof}
    Pour les démonstrations, on procède par interprétation combinatoire et par récurrence.
\end{proof}

\begin{definition}[Produit]
    $\forall m, n \in \N,\ m \leq n,\ a_k \in \R,\ m \leq k \leq n$
    \[ \prod_{k = m}^{n} = a_1 \cdot a_2 \cdot \ldots \cdot a_n \]
\end{definition}