\chapter{Nombres complexes}
\def\arraystretch{1}
On définit l'ensemble des nombres complexes, noté $\C$, comme une extension de l'ensemble des nombres réels.
Cette extension introduit un nouvel élément, noté $i$, appelé \textbf{nombre imaginaire} et défini comme $i^2=-1$.

\section{Vision algébrique des nombres complexes}
\begin{definition}[Forme algébrique d'un nombre complexe]
    Soient $a, b \in \R$ et $z \in \C$, on appelle \textbf{forme algébrique} de $z$ l'expression $z = a + ib$.
    \\
    $a$ est appelé \og \textbf{partie réelle} \fg, notée $\Re(z)$ et $b$ est appelé \og \textbf{partie imaginaire} \fg, notée $\Im(z)$.
\end{definition}

\begin{proposition}
	Pour tous $z_1, z_2, z_3 \in \C$ :
	\begin{enumerate}
		\item Associativité : 
		\begin{enumerate}
			\item $(z_1 + z_2) + z_3 = z_1 + (z_2 + z_3)$.
			\item $(z_1 \cdot z_2) \cdot z_3 = z_1 \cdot (z_1 \cdot z_3)$.
		\end{enumerate}
		\item Commutativité :
		\begin{enumerate}
			\item $z_1 + z_2 = z_2 + z_1$.
			\item $z_1 \cdot z_2 = z_2 \cdot z_1$.
		\end{enumerate}
		\item \'Elément neutre :
		\begin{enumerate}
			\item $0 + i0 = 0$.
			\item $z_1 + 0 = z_1$.
			\item $z_1 \cdot 1 = z_1$
		\end{enumerate}
		\item \'Elément absorbant : $z_1 \cdot 0 = 0$.
		\item \'Elément symétrique : $\exists z' \in \C : z_1 + z' = 0$.
		\item \'Elément inverse : $\exists z' \in \C :  z_1 \cdot z' = 1$.
		\item Distributivité :
		\begin{enumerate}
			\item $(z_1 + z_2) \cdot z_3 = z_1 \cdot z_3 + z_2 \cdot z_3$.
			\item $z_1 \cdot (z_2 + z_3) = z_1 \cdot z_2 + z_1 \cdot z_3$.
		\end{enumerate}
	\end{enumerate}
\end{proposition}

\begin{definition}[Module d'un nombre complexe]
    Soit $z = a + ib$ avec $a, b \in \R$. 
    \\
    On définit $\abs{z}$ tel que :
    \[ \abs{z} = \sqrt{a^2 + b^2} \]
    qu'on appelle \textbf{module} de $z$.
\end{definition}

\begin{definition}[Conjugué d'un nombre complexe]
    Soit $z = a + ib$ avec $a, b \in \R$. On appelle conjugué de $z$ qu'on note $\overline{z}$ tel que :
    \[ \overline{z} = a - ib \]
\end{definition}

\begin{proposition}
    Pour tous $z_1, z_2 \in \C$ :
    \begin{enumerate}
            \item $\abs{z_1 + z_2} \leq \abs{z_1} + \abs{z_2}$.
            \item $\abs{z_1 - z_2} \geq \abs{z_1} - \abs{z_2}$.
            \item $\abs{z_1 \cdot z_2} = \abs{z_1} \cdot \abs{z_2}$.
            \item Si $z_1 \neq 0$ : $\abs{\frac{1}{z_1}} = \frac{1}{\abs{z_1}}$.
            \item Si $z_2 \neq 0$ : $\abs{\frac{z_1}{z_2}} = \frac{\abs{z_1}}{\abs{z_2}}$.
            \item $\abs{z}^2 = z \cdot \overline{z}$.
            \item $\abs{z} \geq 0$.
            \item $\abs{z} = 0 \iff z = 0$.
            \item $\abs{z} = \abs{\overline{z}} = \abs{-z} = \abs{-\overline{z}}$.
        \end{enumerate}
\end{proposition}

\section{Vision géométrique des nombres complexes}
Il est possible de représenter les nombres complexes sur un plan complexe avec l'axe des ordonnées représentant la partie imaginaire et l'axe des abscisses la partie réelle.

\begin{definition}
	Soit $z \in \C$, on appelle l'\textbf{argument} de $z$, noté $\arg(z)$, l'angle entre l'axe de la partie réelle et la droite issue de l'origine passant par $z$. 
\end{definition}

\begin{figure}[!h]
	\centering
	\begin{tikzpicture}
		\coordinate (O) at (0, 0);
		\coordinate (Z) at (4, 3);
		\coordinate (X) at (5, 0);
		\coordinate (Y) at (0, 5);
		\draw[step=1cm, gray, very thin] (-2, -2) grid (6,6);
		\draw[thick, ->] (O) -- (X) node[anchor=south east] {$\Re(z)$};
		\draw[thick, ->] (O) -- (Y) node[anchor=north west] {$\Im(z)$};
		\draw[->] (O) -- (Z) node[circle, fill, inner sep=0pt, scale=0.5, label=above:$z$]{$z$};
		\draw (1.9, 1.6) node[anchor=south]{$\abs{z}$};
		\path[gray]
		pic["$\arg(z)$" shift={(30pt, 5pt)}, draw, -, angle radius=1cm] {angle = X--O--Z};
	\end{tikzpicture}
	\caption{Vision géométrique des nombres complexes}
\end{figure}

\begin{proposition}
	Pour tous $z_1, z_2 \in \C$ et $n \in \N$ :
	\begin{enumerate}
		\item $\arg(z_1 \cdot z_2) = \arg(z_1) + \arg(z_2)$.
		\item $\arg(z_1^n) = n \arg(z_1)$.
		\item Si $z_1 \neq 0$ : $\arg \left(\frac{1}{z_1} \right) = - \arg(z_1)$.
		\item Si $z_2 \neq 0$ : $\arg \left( \frac{z_1}{z_2} \right) = \arg(z_1) - \arg(z_2)$.
	\end{enumerate}
\end{proposition}

\begin{definition}
	Soient $z \in \C,\ r = \abs{z}$ et $\theta = \arg(z)$, il est possible d'exprimer $z$ dans sa forme trigonométrique :
	\[ z = r \left( \cos(\theta) + i \sin(\theta) \right) \]
\end{definition}

\begin{proposition}
	Pour tous $z_1 = r_1 \left( \cos(\theta_1) + i \sin(\theta_1) \right)$ et $z_2 = r_2 \left( \cos(\theta_2) + i \sin(\theta_2) \right)$ deux nombres complexes tels que :
	\begin{enumerate}
			\item $r_1 = \abs{z_1}$.
			\item $r_2 = \abs{z_2}$.
			\item $\theta_1 = \arg(z_1)$.
			\item $\theta_2 = \arg(z_2)$.
		\end{enumerate}
	\[ z_1 z_2 = r_1 r_2 \left( \cos(\theta_1+\theta_2) + i \sin(\theta_1+ \theta_2) \right) \]
\end{proposition}

\begin{proof}
	\begin{align*}
		z_1 z_2 &= r_1(\cos(\theta_1) + i \sin(\theta_1)) r_2(\cos(\theta_2) + i \sin(\theta_2)) \\
		&= (r_1 \cos(\theta_1) + i r_1 \sin(\theta_1))(r_2 \cos(\theta_2) + i r_2 \sin(\theta_2)) \\
		&= r_1 r_2 \cos(\theta_1) \cos(\theta_2) + i r_1 r_2 \cos(\theta_1) \sin(\theta_2) + i r_1 r_2 \sin(\theta_1) \cos(\theta_2) - r_1 \sin(\theta_1) \sin(\theta_2) \\
		&= r_1 r_2 (\cos(\theta_1) \cos(\theta_2) - \sin(\theta_1) \sin(\theta_2)) + i (r_1 r_2(\cos(\theta_1) \sin(\theta_2) + \sin(\theta_1)\cos(\theta_2))) \\
		&= r_1 r_2 (\cos(\theta_1 + \theta_2) + i \sin(\theta_1 + \theta_2))
	\end{align*}
\end{proof}

\begin{definition}
	Soient $z \in \C,\ r = \abs{z}$ et $\theta = \arg(z)$ tels que :
	\[ z = r (\cos(\theta) + i \sin(\theta)) \]
	On peut écrire $z$ sous une forme utilisant l'exponentielle :
	\[ z = re^{i \theta} \]
\end{definition}

\begin{proposition}[Formule de Moivre]
	Pour tous $\theta \in \R$ et $n \in \N$ :
	\[ (\cos(\theta) + i \sin(\theta))^n = \cos(n \theta) + i \sin(n \theta) \]
\end{proposition}

\begin{proposition}[Identité d'Euler]
	\[ e^{i\pi} + 1 = 0 \]
\end{proposition}

\begin{proposition}[Formules d'Euler]
	Pour tout $\theta \in \R$ :
	\begin{enumerate}
			\item $\cos(\theta) = \frac{e^{i\theta} + e^{-i\theta}}{2}$.
			\item $\sin(\theta) = \frac{e^{i\theta} - e^{-i\theta}}{2i}$.
		\end{enumerate}
\end{proposition}

\begin{definition}
	Soient $z \in \C$ et $n \in \N$. On appelle \textbf{racine $n$-ième} de $z$ tout $\omega \in \C$ tel que :
	\[ \omega^n = z \] 
\end{definition}

\begin{proposition}
	Soient $z \in \C^*$, $\theta = \arg(z)$ et $\rho = \abs{z}$ tels que :
	\[ z = \rho e^{i\theta} \]
	$z$ admet $n$ racines $n$-ièmes, pour $0 \leq k \leq n - 1$ :
	\[ \omega_k = \rho^{\frac{1}{n}} e^{i \left( \frac{\theta}{n} + \frac{2 k \pi}{n} \right)} \]
\end{proposition}

\begin{proof}
	Soit $z = \rho e^{i \theta}$. \\
	Les racines $n$-ièmes de $z$ sont les nombres $\omega = re^{i \theta'}$ pour $r = \abs{\omega}$ et $\theta' = \arg(\omega)$ tels que $\omega^n = z$.
	\begin{align*}
		(r e^{i \theta'})^n &= \rho e^{i \theta} \\
		r^n (e^{i \theta'})^n &= \rho e^{i \theta} \\
		r^n e^{i n \theta'} &= \rho e^{i \theta}
	\end{align*} 
	Par identification nous avons pour $0 \leq k \leq n - 1$:
	\begin{align*}
		\begin{cases}
			r^n = \rho \\
			n \theta' = \theta + 2 k \pi
		\end{cases}
		\iff 
		\begin{cases}
			r = \rho^{\frac{1}{n}} \\
			\theta' = \frac{\theta}{n} + \frac{2 k \pi}{n}
		\end{cases}
	\end{align*}
	On a finalement : 
	\[ \omega_k = \rho^{\frac{1}{n}} e^{i \left( \frac{\theta}{n} + \frac{2 k \pi}{n} \right) } \]
\end{proof}

\section{Géométrie des nombres complexes}
\begin{proposition}
	Soit :
	\begin{center}
		$
		\appli{f}{\C}{\C}{z}{f(z)}
		$
	\end{center}
	\begin{enumerate}
		\item Soit $a \in \C,\ f(z) = z + a$ : translation d'affixe $a$.
		\item Soit $a \in \R^*,\ f(z) = az$ : homothétie de rapport $a$.
		\item Soient $a \in \C$ et $\theta \in \R,\ f(z) = (z-a)e^{i\theta} + a$ : rotation d'angle $\theta$ et de centre $a$.
		\item Soit $\theta \in \R,\ f(z) = \overline{z} e^{2i\theta}$ : réflexion par rapport à la droite formant un angle $\theta$ avec l'axe des réels.
	\end{enumerate}
\end{proposition}

\begin{proposition}
	\begin{enumerate}
		\item L'axe des réels : $\overline{z} = z$.
		\item Un axe formant une angle $\theta$ avec l'axe des réels : $\overline{e^{-i \theta}z} = e^{-i \theta} z$.
		\item L'asymptote verticale de partie réelle $a$ : $z + \overline{z} = 2a$.
	\end{enumerate}
\end{proposition}