\chapter{Nombres complexes}

On définit l'ensemble des nombres complexes, noté $\C$, comme une extension de l'ensemble des nombres réels.
Cette extension introduit un nouvel élément, noté $i$, appelé \textbf{nombre imaginaire} et défini comme $i^2=-1$.

\subsection{Vision algébrique des nombres complexes}
\begin{definition}[Forme algébrique d'un nombre complexe]
    Soient $a, b \in \R$ et $z \in \C$, on appelle \textbf{forme algébrique} de $z$ l'expression $z = a + ib$.
    \\
    $a$ est appelé \og \textbf{partie réelle} \fg, notée $\Re(z)$ et $b$ est appelé \og \textbf{partie imaginaire} \fg, notée $\Im(z)$.
\end{definition}

\begin{definition}[Module d'un nombre complexe]
    Soit $z = a + ib$ avec $(a, b) \in \R^2$. 
    \\
    On définit $\abs{z}$ tel que :
    \[ \abs{z} = \sqrt{a^2 + b^2} \]
    qu'on appelle \textbf{module} de $z$.
\end{definition}

\begin{definition}[Conjugué d'un nombre complexe]
    Soit $z = a + ib$ avec $(a, b) \in \R^2$. On appelle conjugué de $z$ qu'on note $\overline{z}$ tel que :
    \[ \overline{z} = a - ib \]
\end{definition}

\begin{proposition}
    Soient $(z_1, z_2) \in \C^2$.
    \begin{multicols}{3}
        \begin{enumerate}
            \item $\abs{z_1 + z_2} \leq \abs{z_1} + \abs{z_2}$.
            \item $\abs{z_1 - z_2} \geq \abs{z_1} - \abs{z_2}$.
            \item $\abs{z_1 \cdot z_2} = \abs{z_1} \cdot \abs{z_2}$.
            \item Si $z_2 \neq 0$ : $\abs{\frac{1}{z_2}} = \frac{1}{\abs{z_2}}$.
            \item $\abs{z}^2 = z \cdot \overline{z}$.
            \item $\abs{z} \geq 0$.
            \item $\abs{z} = 0 \iff z = 0$.
            \item $\abs{z} = \abs{\overline{z}}$.
        \end{enumerate}
    \end{multicols}
\end{proposition}

\section{Vision géométrique des nombres complexes}
Il est possible de représenter les nombres complexes sur un plan complexe avec l'axe des ordonnées représentant la partie imaginaire et l'axe des abscisses la partie réelle.

\begin{definition}
	Soit $z \in \C$, on appelle l'argument de $z$, noté $\arg(z)$, l'angle entre l'axe de la partie réelle et la droite issue de l'origine passant par $z$. 
\end{definition}

\begin{proposition}
	Soient $(z, z_1, z_2) \in \C^3$ et $n \in \N$.
	\begin{enumerate}
		\item $\arg(z_1 \cdot z_2) = \arg(z_1) + \arg(z_2)$.
		\item $\arg(z^n) = n \arg(z)$.
		\item Si $z_2 \neq 0$ : $\arg(\frac{1}{z_2}) = - \arg(z_2)$.
	\end{enumerate}
\end{proposition}

\begin{figure}[!ht]
	\centering
	\begin{tikzpicture}[line cap=round,line join=round,>=triangle 45,x=1cm,y=1cm]
		\begin{axis}[
			x=1.85cm,y=1.85cm,
			axis lines=middle,
			xmin=-1,
			xmax=5,
			ymin=-1,
			ymax=3,
			xtick={-1,0,...,5},
			ytick={-3,-2,...,6},]
			\clip(-1.4725109228142914,-3.6020142531794552) rectangle (5.3902234350649305,6.061762219150244);
			\draw [line width=0pt,color=qqwuqq,fill=qqwuqq,fill opacity=1] (0,0) -- (0:1.0092256408645914) arc (0:26.56505117707799:1.0092256408645914) -- cycle;
			\draw [line width=2pt] (0,0)-- (4,2);
			\draw [line width=2pt,dash dot] (0,2)-- (4,2);
			\draw [line width=2pt,dash dot] (4,2)-- (4,0);
			\draw [color=qqwuqq](0.96981512807802,0.4020581817220694) node[anchor=north west] {$\arg(z_1)$};
			\begin{scriptsize}
				\draw [fill=qqqqff] (4,2) circle (2.5pt);
				\draw[color=qqqqff] (4.1085068711669,2.2457835389154512) node {$z_{1}$};
				\draw[color=black] (2.0,1.3) node {|z|};
				\draw [fill=xdxdff] (0,2) circle (2.5pt);
				\draw[dash dot, color=xdxdff] (0.34494358814519,2.2457835389154512) node {$Im(z_1)$};
				\draw [fill=xdxdff] (4,0) circle (2.5pt);
				\draw[dash dot, color=xdxdff] (4.3982973705234,0.24429008544130865) node {$Re(z_1)$};
			\end{scriptsize}
		\end{axis}
	\end{tikzpicture}
	\caption{Vision géométrique des nombres complexes}
\end{figure}

\needspace{5cm}

\begin{definition}
	Soient $z \in \C,\ r = \abs{z}$ et $\theta = \arg(z)$, il est possible d'exprimer $z$ dans sa forme trigonométrique :
	\[ z = r \left( \cos(\theta) + i \sin(\theta) \right) \]
\end{definition}

\begin{proposition}
	Soient $z_1 = r_1 \left( \cos(\theta_1) + i \sin(\theta_1) \right)$ et $z_2 = r_2 \left( \cos(\theta_2) + i \sin(\theta_2) \right)$ deux nombres complexes tels que :
	\begin{multicols}{4}
		\begin{enumerate}
			\item $r_1 = \abs{z_1}$.
			\item $r_2 = \abs{z_2}$.
			\item $\theta_1 = \arg(z_1)$.
			\item $\theta_2 = \arg(z_2)$.
		\end{enumerate}
	\end{multicols}
	\[ z_1 z_2 = r_1 r_2 \left( \cos(\theta_1+\theta_2) + i \sin(\theta_1+ \theta_2) \right) \]
\end{proposition}

\begin{proof}
	En utilisant les formules d'additions de $\cos$ et $\sin$.
\end{proof}

\begin{definition}
	Soient $z \in \C,\ r = \abs{z}$ et $\theta = \arg(z)$ tels que :
	\[ z = r (\cos(\theta) + i \sin(\theta)) \]
	On peut écrire $z$ sous une forme utilisant l'exponentielle :
	\[ z = re^{i \theta} \]
\end{definition}

\begin{proposition}[Formule de Moivre]
	Soient $\theta \in \R$ et $n \in \N$.
	\[ (\cos(\theta) + i \sin(\theta))^n = \cos(n \theta) + i \sin(n \theta) \]
\end{proposition}

\begin{proposition}[Identité d'Euler]
	\[ e^{i\pi} + 1 = 0 \]
\end{proposition}

\begin{proposition}[Formules d'Euler]
	Soit $\theta \in \R$.
	\begin{multicols}{2}
		\begin{enumerate}
			\item $\cos(\theta) = \frac{e^{i\theta} + e^{-i\theta}}{2}$.
			\item $\sin(\theta) = \frac{e^{i\theta} - e^{-i\theta}}{2i}$.
		\end{enumerate}
	\end{multicols}
\end{proposition}

\begin{definition}
	Soient $z \in \C$ et $n \in \N$. On appelle \textbf{racine $n$-ième} de $z$ tout $\omega \in \C$ tel que :
	\[ \omega^n = z \] 
\end{definition}

\begin{proposition}
	Soient $z \in \C^*$, $\theta \in \R$ et $\rho = \abs{z}$ tels que :
	\[ z = \rho e^{i\theta} \]
	$z$ admet $n$ racines $n$-ièmes de la forme :
	\[ \omega_k = \rho^{\frac{1}{n}} e^{i \left( \frac{\theta}{n} + \frac{2 k \pi}{n} \right)},\ k \in \llbracket 0, n - 1 \rrbracket \]
\end{proposition}

\begin{proof}
	En utilisant la forme exponentielle.
\end{proof}

\section{Géométrie des nombres complexes}
\begin{proposition}
	Soit :
	\begin{align*}
		f : \C &\to \C \\ 
		z &\mapsto f(z)
	\end{align*}
	\begin{enumerate}
		\item Soit $a \in \C,\ f(z) = z + a$ : translation d'affixe $a$.
		\item Soit $a \in \R^*,\ f(z) = az$ : homothétie de rapport $a$.
		\item Soient $a \in \C$ et $\theta \in \R,\ f(z) = (z-a)e^{i\theta} + a$ : rotation d'angle $\theta$ et de centre $a$.
		\item Soit $\theta \in \R,\ f(z) = \overline{z} e^{2i\theta}$ : réflexion par rapport à la droite formant un angle $\theta$ avec l'axe des réels.
	\end{enumerate}
\end{proposition}

\begin{proposition}
	\leavevmode
	\begin{enumerate}
		\item L'axe des réels : $\overline{z} = z$.
		\item Un axe formant une angle $\theta$ avec l'axe des réels : $\overline{e^{-i \theta}z} = e^{-i \theta} z$.
		\item L'asymptote verticale de partie réelle $a$ : $z + \overline{z} = 2a$.
	\end{enumerate}
\end{proposition}