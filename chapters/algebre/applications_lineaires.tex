\chapter{Applications linéaires}
\def\arraystretch{1}

\section{Définitions}
\begin{definition}[Application linéaire]
    Soient $E$ et $F$ deux $\K$-espaces vectoriels et $f : E \to F$.
    On dit que $f$ est une \textbf{application linéaire} si et seulement si : 
    \[ \forall x_1, x_2 \in E,\ \lambda \in \K : f(x_1 + \lambda x_2) = f(x_1) + \lambda f(x_2) \]
\end{definition}

\begin{definition}
    Soient $E$ et $F$ deux $\K$-espaces vectoriels et $f : E \to F$ une application linéaire.
    \\
    On définit le \textbf{noyau} de $f$, noté $\ker(f)$ tel que :
    \[ \ker(f) = \{ x \in E : f(x) = 0_F \} \] 
    On définit l'\textbf{image} de $f$, notée $\Im(f)$ telle que :
    \[ \Im(f) = \{ y \in F,\ \exists x \in E : y = f(x) \} \]
\end{definition}

\begin{theorem}[\cite{applications_lineaires_bibmath}]
	Soient $E$ et $F$ deux $\K$-espaces vectoriels. \\
	$f \in \mathcal{L}(E, F)$ est injective si et seulement si $\ker(f) = \{ 0_E \}$.
\end{theorem}

\begin{proof}
	\leavevmode
	\begin{enumerate}
		\item \boxed{\implies} : Supposons $f$ injective.
		\\
		Si $x \in \ker(f)$ alors $f(x) = 0_F$. On sait que $f(0_E) = 0_F$, or $f$ est injective donc $0_E = x$ et :
		\[ \ker(f) = \{ 0_E \} \]
		\item \boxed{\impliedby} : Supposons que $\ker(f) = \{ 0_E \}$.
		\\
		Soient $x, y \in E$ tels que $f(x) = f(y)$.
		On a ensuite :
		\[ 0_F = f(x) - f(y) \]
		puis car $f$ est linéaire :
		\[ 0_F = f(x - y) \]
		ce qui veut dire que $x - y \in \ker(f)$ or $\ker(f) = \{ 0_E \}$ donc $0_E = x - y$, c'est-à-dire $x = y$.
		\\
		On a bien montré que $f$ est injective.
	\end{enumerate}
\end{proof}

\begin{theorem}
	Soient $E, F$ des espaces vectoriels et $f : E \to F$ une application linéaire.
	\begin{enumerate}
		\item $\ker(f) \subseteq E$.
		\item $\Im(f) \subseteq F$.
	\end{enumerate}
\end{theorem}

\begin{proof}
	\leavevmode 
	\begin{enumerate}
		\item $\forall x_1, x_2 \in \ker(f),\ \lambda \in \K$.
		\begin{align*}
			f(x_1 + \lambda x_2) &= f(x_1) + \lambda f(x_2) \\ 
								 &= 0_F + \lambda 0_F \\
								 &= 0_F
		\end{align*}
		\item $\forall y_1, y_2 \in \Im(f),\ \lambda \in \K$.
		\[ y_1 \in \Im(f) \iff \exists x_1 : f(x_1) = y_1 \]
		\[ y_2 \in \Im(f) \iff \exists x_2 : f(x_2) = y_2 \]
		\begin{align*}
			y_1 + \lambda y_2 &= f(x_1) + \lambda f(x_2) \\
			                  &= f(x_1 + \lambda x_2)
		\end{align*}
		Ainsi $y_1 + \lambda y_2 \in \Im(f)$.
	\end{enumerate}
\end{proof}

\begin{definition}
    \par \noindent Soient $E$ et $F$ deux $\K$-espaces vectoriels et $f$ une application linéaire de $E$ vers $F$.
    \begin{enumerate}
        \item On dit que $f$ est un \textbf{morphisme} de $E$ vers $F$, on note $f \in \mathcal{L}(E, F)$.
        \item Si $E = F$, on dit que $f$ est un \textbf{endomorphisme} de $E$, on note $f \in \mathcal{L}(E)$ ou $\End(E)$.
        \item Si $f \in \mathcal{L}(E, F)$ est une \textbf{bijection}, alors $f$ est un \textbf{isomorphisme}. On le note $f : E \overset{\sim}{\to} F$.
        \item Si $f \in \mathcal{L}(E)$ est un \textbf{isomorphisme}, on dit que $f$ est un \textbf{automorphisme} de $E$ et on le note $\Aut(E)$.
        \\
        $E$ et $F$ sont appelés \textbf{isomorphes} s'il existe un \textbf{isomorphisme} de l'un vers l'autre, on écrit parfois $E \cong F$.
    \end{enumerate}
\end{definition}

Soit $\mathcal{E} = \{ f \in \mathcal{C}^n(I) : f^{(n)} + a_{n - 1} f^{(n - 1)} + \cdots + a_1 f' + a_0 f = 0 \},\ n \in \N,\ a_i \in \mathcal{C}^0(I),\ I$ un intervalle ouvert.

\begin{proposition}
    Soit $n \in \N$.
    \begin{multicols}{2}
        \begin{enumerate}
            \item $\mathcal{E} \subseteq \mathcal{C}^n(I)$.
            \item $\dim(\varepsilon) = n$.
        \end{enumerate}
    \end{multicols}
\end{proposition}

\begin{proof}
	Montrons 1. \\
	On définit :
	\begin{center}
		$\appli{\Psi}{\mathcal{C}^n(I)}{\mathcal{C}^0(I)}{f}{f^{(n)} + a_{n-1}f^{(n-1)} + \cdots + a_1 f' + a_0 f}$
	\end{center}
	Alors on voit que $\mathcal{E} = \ker(\Psi)$.
\end{proof}

Soient $E = \{ (u_n)_{n \in \N},\ u_n \in \R,\ \forall n \in \N \}$, $N \in \N$ et $a = (a_0, \ldots, a_{N-1}) \in \R^N$.
\\
On définit :
\[ F = \{ (u_n)_{n \in \N} : u_{n + N} + a_{N-1} u_{n + N - 1} + \cdots + a_n u_{n+1} + a_0 u_n = 0,\ \forall n \in \N \} \]

\begin{theorem}
	\begin{multicols}{2}
		\begin{enumerate}
			\item $F \subseteq E$.
			\item $\dim(F) = N$.
		\end{enumerate}
	\end{multicols}
\end{theorem}

\begin{definition}[Rang d'un morphisme]
    Soit $f \in \mathcal{L}(E, F)$. On appelle $\rg(f) = \dim(\Im(f))$ le rang de $f$.
\end{definition}

\begin{theorem}[Théorème du rang]
    Soient $E, F$ des espaces vectoriels de dimensions finies ou infinies et $f \in \mathcal{L}(E, F)$, alors :
    \[ \dim(E) = \dim(\ker(f)) + \rg(f) \]
\end{theorem}

\begin{corollary}
    Soit $f \in \mathcal{L}(E)$.
    \begin{align*}
        \begin{cases}
            \ker(f) + \Im(f) = \ker(f) \oplus \Im(f) \\
            \ker(f) \oplus \Im(f) = E
        \end{cases}
        \iff 
        \ker(f) \cap \Im(f) = \{0_E\}
    \end{align*}
\end{corollary}

\begin{proposition}
    Soit $f \in \mathcal{L}(E, F)$ et $(b_1, \ldots, b_n)$ une base de $E$. Si l'on connait $f(b_i) \in F$, $1 \leq i \leq n$, on connait toute l'application $f$.
\end{proposition}

\begin{proof}
	$\forall x \in E,\ \exists ! (\lambda_1, \ldots, \lambda_n) \in \R^n$ tel que $x = \sum_{i=1}^{n} \lambda_i b_i$.
	\[ f(x) = f \left( \sum_{i=1}^{n} \lambda_i b_i \right) = \sum_{i=1}^{n} \lambda_i f(b_i) \]
\end{proof}

\begin{corollary}
    Soit $(b_1, \ldots, b_n)$ une base de $E$. Alors $\varphi \in \mathcal{L}(E, \R^n)$ est un isomorphisme donné par :
    \[ \varphi(b_i) = e_i,\ 1 \leq i \leq n \]
    \[ \varphi \left( \sum_{i = 1}^{n} x_i b_i \right) = \sum_{i = 1}^{n} x_i e_i = 
    \begin{pmatrix}
        x_1 \\
        \vdots \\
        x_n
    \end{pmatrix}
    \in \R^n
     \]
\end{corollary}

\begin{lemma} Soient $\K$ un corps et $E$ un $\K$-espace-vectoriel de dimension $n \in \N$.
    \[ \dim(E) = n \implies E \cong \K^n \]
\end{lemma}

\begin{lemma}
    Soient $\mathcal{B} = (b_1, \ldots, b_n)$ une base de $E$, $\lambda_1, \ldots, \lambda_n \in \K$ et $\varphi \in \mathcal{L}(E, \K^n)$  alors :
    \begin{center}
    	$
    	\appli{\varphi}{E}{\K^n}{\sum_{i = 1}^{n} \lambda_i b_i}{
    	\begin{pmatrix}
    		\lambda_1 \\
    		\vdots \\
    		\lambda_n
    	\end{pmatrix}
    	}
    	$
    \end{center}
    est une bijection, alors c'est un isomorphisme.
\end{lemma}

\begin{proposition}
    Soient $E, F$ des espaces vectoriels $f \in \mathcal{L}(E, F)$ un morphisme et $A$ sa matrice associée. 
    \[ y = f(x) \iff y = A \cdot x \]
\end{proposition}

\section{Projecteurs et symétries}
\begin{definition}
    Soit $f \in \mathcal{L}(E)$. Notons $f^2 = f \circ f$.
    \begin{enumerate}
        \item On dit que $f$ \textbf{est idempotente/une projection} si et seulement si : $f^2 = f$.
        \item On dit que $f$ \textbf{est involutive/une symétrie linéaire} si et seulement si : $f^2 = id_E$.
    \end{enumerate}
\end{definition}

\begin{proposition} 
    Soit $p \in \mathcal{L}(E)$.
    \begin{enumerate}
        \item $id_E - p \text{ est une projection} \iff p \text{ est une projection}$
        \item $2p - id_E \text{ est une symétrie} \iff p \text{ est une projection}$
    \end{enumerate}
\end{proposition}

\begin{proof}
	\leavevmode
    \begin{enumerate}
        \item Posons $f(x) = id_E(x) - p(x)$. Montrons que $f(f(x)) = id_E(x) - p(x)$ si et seulement si $p(p(x))$.
        \begin{align*}
            f(f(x)) &= id_E(id_E(x) - p(x)) - p(id_E(x) - p(x)) \\ 
            &= id_E(id_E(x)) - id_E(p(x)) - p(id_E(x)) + p(p(x)) \\ 
            &= id_E(x) - p(x) - p(x) + p(p(x)) \\ 
            &= id_E(x) - 2p(x) + p(p(x)) \\ 
            &= id_E(x) - [2p(x) - p(p(x))]
        \end{align*}
        \begin{align*}
            id_E(x) - [2p(x) - p(p(x))] = id_E(x) - p(x) &\iff p(p(x)) = p(x) \\ 
            &\iff 2p(x) - p(p(x)) = p(x) \\ 
            &\iff 2p(x) = p(x) + p(p(x)) \\
            &\iff p(x) = p(p(x))
        \end{align*}
        \item Posons $s(x) = 2p(x) - id_E(x)$. 
        \\
        Montrons que $s(s(x)) = id_E(x)$ si et seulement si $p(p(x)) = p(x)$.
        \begin{align*}
            s(s(x)) &= s(2p(x) - id_E(x)) \\
            &= 2p(2p(x) - id_E(x)) - id_E(2p(x) - id_E(x)) \\
            &= 4p(p(x)) - 2p(id_E(x)) - 2id_E(p(x)) + id_E(id_E(x))
        \end{align*}
        \begin{align*}
            4p(p(x)) - 4p(x) + id_E(x) = id_E(x) &\iff 4p(p(x)) - 4p(x) = 0 \\
            &\iff 4p(p(x)) = 4p(x) \\
            &\iff p(p(x)) = p(x)
        \end{align*}
    \end{enumerate}
\end{proof}

\begin{definition}
    Soient $E = F \oplus G$, $u \in F$, $v \in G$, alors l'application 
    \begin{center}
    	$
    	\appli{p_F}{E}{E}{u+v}{u}
    	$
    \end{center}
    est appelée un \textbf{projecteur} sur $F$ \textbf{parallélement} à $G$.
\end{definition}

\begin{proposition}
    Soit $p_F$ définie comme dans la définition précédente.
    \begin{enumerate}
        \item $p_F$ est une projection.
        \item Soit $p$ une projection, $p$ est un projecteur sur $\Im(p)$ parallélement à son noyau $\ker(p)$.
    \end{enumerate}
\end{proposition}

\section{Rotations.}
\begin{definition}
	Soit $n \in \N^*$.
    \[ \operatorname{GL}(\R, n) = \{ A \in \mathcal{M}_n(\R) : \det(A) \neq 0 \}. \]
\end{definition}

\begin{definition}
	Soit $n \in \N^*$.
    \[ \operatorname{SL}(\R, n) = \{ A \in \mathcal{M}_n(\R) : \det(A) = 1 \}. \]
\end{definition}

\begin{definition}
	Soit $n \in \N^*$.
    \[ \operatorname{O}(n) = \{ R \in \mathcal{M}_n(\R) : \forall x, y \in \R^n : \langle Rx|Ry \rangle = \langle x|y \rangle \}. \]
\end{definition}

\begin{definition}
    Soit $n \in \N^*$.
    \[ \operatorname{SO}(n) = \{ R \in \operatorname{O}(n) : \det(R) = 1 \}. \]
\end{definition}

\begin{proposition}
	Soit $R \in \mathcal{M}_n(\K)$.
    \begin{enumerate}
        \item $R \in \operatorname{O}(n) \iff R^T \cdot R = I_n$.
        \item $R \in \operatorname{O}(n) \implies \det(R) \in \{ \pm 1 \}$.
    \end{enumerate}
\end{proposition}

\begin{corollary}
	Soit $n \in \N^*$.
    \[ \operatorname{O}(n) = \{ R \in \mathcal{M}_n(\K) : R^T \cdot R = I_n \}. \]
\end{corollary}

\begin{lemma}
	Soient $x, y \in \R^n$ et $A \in \mathcal{M}_n(\R)$.
	\[ \langle y \mid A \cdot x \rangle = \langle A^T \cdot y \mid x \rangle. \]
\end{lemma}

\begin{proof}
	Soient $a_{1,1}, \ldots, a_{n,n} \in \R$ les coefficients de $A$.
	\begin{align*}
		\langle y \mid A \cdot x \rangle &= \sum_{i = 1}^{n} y_i (A \cdot x)_i \\
		&= \sum_{i = 1}^{n} y_i \left( \sum_{j=1}^{n} a_{i,j} x_{j,i} \right) \\
		&= \sum_{j = 1}^{n} \left( \sum_{i = 1}^{n} a_{j,i} y_i \right) \cdot x_j \\
		&= \langle A^T \cdot y \mid x \rangle
	\end{align*}
\end{proof}

\section{Changements de bases et matrices associées aux applications linéaires.}

\begin{definition}[Matrice d'une application linéaire]
    Soient $n, p \in \N^*$, $E$ et $F$ deux $\K$-espaces vectoriels, $\mathcal{B} = (b_1, \ldots, b_p)$ une base de $E$, $\mathcal{B}' = (b_1', \ldots, b_n')$ une base de $F$ et $f \in \mathcal{L}(E, F)$.
    \\
    On peut définir une matrice $\Mat_{\mathcal{B},\mathcal{B}'} (f) \in \mathcal{M}_{n,p}(\K)$ telle que :
    \begin{align*}
        \hspace{0.4cm}
        \begin{matrix}
            f(b_1) & \cdots & f(b_p) 
        \end{matrix}
    \end{align*}
    \begin{align*}
        \begin{matrix}
            b_1' \\
            \vdots \\ 
            b_n'
        \end{matrix}
        \begin{pmatrix}
            a_{1,1} & \cdots & a_{1,p} \\
            \vdots & \ddots & \vdots \\ 
            a_{n,1} & \cdots & a_{n,p}
        \end{pmatrix}
    \end{align*}
    avec les coefficients $a_{1,1}, \ldots, a_{n,p} \in \K$ tels que : 
    \[ f(b_j) = \sum_{i = 1}^n a_{i,j} b_i' \]
\end{definition}

\begin{example}
	Soit $\Psi$ une application linéaire telle que :
	\begin{center}
		$\appli{\Psi}{\R_2[X]}{\R_3[X]}{P}{P + (X + X^2)P' + (-2 + 3X - X^3)P''}$
	\end{center}
	On prend $\mathcal{B} = (1, X, X^2)$ comme base de $\R_2[X]$ et $\mathcal{B}' = (1, X, X^2, X^3)$ comme base de $\R_3[X]$.
	\\
	On calcule :
	\[ \Psi(1) = 1 \cdot 1 + (X + X^2) \cdot 0 + (-2 + 3X - X^3) \cdot 0 = 1 \cdot 1 + 0 \cdot X + 0 \cdot X^2 + 0 \cdot X^3 \]
	\[ \Psi(X) = X + (X + X^2) \cdot 1 + (-2 + 3X - X^3) \cdot 0 = 0 \cdot 1 + 2 \cdot X + 1 \cdot X^2 + 0 \cdot X^3 \]
	\[ \Psi(X^2) = X^2 + (X + X^2) \cdot 2X + (-2 + 3X - X^3) \cdot 2 = -4 \cdot 1 + 6 \cdot X + 2 \cdot X^2 + 1 \cdot X^3 \]
	\begin{align*}
		&\begin{matrix}
			\Psi(1) & \Psi(X) & \Psi(X^2)
		\end{matrix}
		\\
		\begin{matrix}
			1 \\
			X \\
			X^2 \\
			X^3
		\end{matrix}
		&
		\begin{pmatrix}
			1 & 0 & -4 \\
			0 & 2 & 6 \\
			0 & 1 & 2 \\
			0 & 0 & 1
		\end{pmatrix}
		= \Mat_{\mathcal{B}, \mathcal{B'}}(\Psi)
	\end{align*}
\end{example}

\begin{definition}[Matrice de passage]
    Soient $n \in \N^*$, $E$ un $\K$-espace vectoriel de dimension $n$, $\mathcal{B} = (b_1, \ldots, b_n)$ et $\mathcal{B}' = (b_1', \ldots, b_n')$ deux bases de $E$. On appelle \textbf{matrice de passage} de $\mathcal{B}$ à $\mathcal{B}'$ la matrice carrée de taille $n$ dont la $j-$ième colonne est formée des coordonnées de $b_j'$ dans la base $\mathcal{B}$. Nous la noterons $P_{\mathcal{B}, \mathcal{B}'}$.
    \begin{align*}
    	\hspace{0.45cm}
    	\begin{matrix}
    		b_1' & \cdots & b_n' 
    	\end{matrix}
    \end{align*}
    \begin{align*}
    	\begin{matrix}
    		b_1 \\
    		\vdots \\ 
    		b_n
    	\end{matrix}
    	\begin{pmatrix}
    		a_{1,1} & \cdots & a_{1,n} \\
    		\vdots & \ddots & \vdots \\ 
    		a_{n,1} & \cdots & a_{n,n}
    	\end{pmatrix}
    \end{align*}
    Avec les coefficients $a_{1,1}, \ldots, a_{n,n} \in \K$ tels que :
    \[ b_j' = \sum_{i=1}^{n} a_{i,j} b_i  \]
\end{definition}

\begin{example}
	Soient $\mathcal{E} = ((1, 0), (0, 1))$ et $\mathcal{B} = ((1, 2), (3, -1))$ deux bases de $\R^2$.
	On a :
	\[
	\begin{pmatrix}
		1 \\
		2
	\end{pmatrix}
	= 
	1 \cdot
	\begin{pmatrix}
		1 \\
		0
	\end{pmatrix}
	+ 2 \cdot
	\begin{pmatrix}
		0 \\
		1
	\end{pmatrix}
	\]
	
	\[
	\begin{pmatrix}
		3 \\
		-1
	\end{pmatrix}
	=
	3 \cdot
	\begin{pmatrix}
		1 \\
		0
	\end{pmatrix}
	- 1 \cdot 
	\begin{pmatrix}
		0 \\
		1
	\end{pmatrix}
	\]
	Ainsi on a :
	\begin{align*}
		&\begin{matrix}
			(1, 2) & (3, -1)
		\end{matrix}
		\\
		\begin{matrix}
			(1, 0) \\
			(0, 1)
		\end{matrix}
		&\begin{pmatrix}
			1 & 3 \\
			2 & -1
		\end{pmatrix} = P_{\mathcal{E},\mathcal{B}}
	\end{align*}
\end{example}

\begin{definition}
    Soient $A, B \in \mathcal{M}_{m,n}(\K)$.
    \begin{enumerate}
        \item On dit que $A$ et $B$ sont \textbf{équivalentes} si et seulement si : 
        \[ \exists P \in \operatorname{GL}(n),\ Q \in \operatorname{GL}(m),\ B = Q^{-1} A P \]
        On note $A \sim B$.
        \item On dit que $A$ et $B$ sont \textbf{semblables} si et seulement si :
        \[ \exists P \in \operatorname{GL}(n),\ A = P B P^{-1} \]
        On note $A \simeq B$.
    \end{enumerate}
\end{definition}

\begin{lemma}
	$\forall A, B \in \mathcal{M}_n(\K),\ A \simeq B : \tr(A) = \tr(B)$.
\end{lemma}