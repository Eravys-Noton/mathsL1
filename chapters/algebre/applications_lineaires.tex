\chapter{Applications linéaires}
\def\arraystretch{1}

\section{Définitions}
\begin{definition}[Application linéaire]
    Soient $E$ et $F$ deux $\K$-espaces vectoriels et $f : E \to F$.
    On dit que $f$ est une \textbf{application linéaire} si et seulement si $\forall (x_1, x_2) \in E^2,\ \lambda \in \K$ :
    \[ f(x_1 + \lambda x_2) = f(x_1) + \lambda f(x_2) \]
\end{definition}

\begin{definition}
    Soient $E$ et $F$ deux $\K$-espaces vectoriels et $f : E \to F$ une application linéaire.
    \\
    On définit le \textbf{noyau} de $f$, noté $\ker(f)$ tel que :
    \[ \ker(f) \coloneqq \{ \forall x \in E : f(x) = 0_E \} \] 
    On définit l'\textbf{image} de $f$, notée $\Im(f)$ telle que :
    \[ \Im(f) \coloneqq \{ \forall y \in F : \exists x \in E,\ y = f(x) \} \]
\end{definition}

\begin{definition}
    \par \noindent Soient $E$ et $F$ deux $\K$-espaces vectoriels et $f$ une application linéaire de $E$ vers $F$.
    \begin{enumerate}
        \item On dit que $f$ est un \textbf{morphisme} de $E$ vers $F$, on note $f \in \mathcal{L}(E, F)$.
        \item Si $E = F$, on dit que $f$ est un \textbf{endomorphisme} de $E$, on note $f \in \mathcal{L}(E)$ ou $\End(E)$.
        \item Si $f \in \mathcal{L}(E, F)$ est une \textbf{bijection}, alors $f$ est un \textbf{isomorphisme}. On le note $f : E \overset{\sim}{\to} F$.
        \item Si $f \in \mathcal{L}(E)$ est un \textbf{isomorphisme}, on dit que $f$ est un \textbf{automorphisme} de $E$ et on le note $\Aut(E)$.
        \\
        $E$ et $F$ sont appelés \textbf{isomorphes} s'il existe un \textbf{isomorphisme} de l'un vers l'autre, on écrit parfois $E \cong F$.
    \end{enumerate}
\end{definition}

Soit $\varepsilon \coloneqq \{ f \in \mathcal{C}^N(I) : f^{(n)} + a_{n - 1} f^{(n - 1)} + \cdots + a_1 f' + a_0 f = 0 \},\ n, N \in \N^*,\ a_i \in \mathcal{C}^0(I),\ I$ un intervalle ouvert.

\begin{proposition}
    Soit $n \in \N$.
    \begin{multicols}{2}
        \begin{enumerate}
            \item $\varepsilon \subset \mathcal{C}^n(I)$
            \item $\dim(E) = n$
        \end{enumerate}
    \end{multicols}
\end{proposition}

\begin{definition}[Rang d'un morphisme]
    Soit $f \in \mathcal{L}(E, F)$. On appelle $\rg(f) \coloneqq \dim(\Im(f))$ le rang de $f$.
\end{definition}

\begin{theorem}[Théorème du rang]
    Soient $E, F$ des espaces vectoriels de dimensions finies ou infinies et $f \in \mathcal{L}(E, F)$, alors :
    \[ \dim(E) = \dim(\ker(f)) + \rg(f) \]
\end{theorem}

\begin{corollary}
    Soit $f \in \mathcal{L}(E)$.
    \begin{align*}
        \begin{cases}
            \ker(f) + \Im(f) = \ker(f) \oplus \Im(f) \\
            \ker(f) \oplus \Im(f) = E
        \end{cases}
        \iff 
        \ker(f) \cap \Im(f) = \{0_E\}
    \end{align*}
\end{corollary}

\begin{proposition}
    Soit $f \in \mathcal{L}(E, F)$ et $(b_1, \ldots, b_n)$ une base de $E$. Si l'on connait $f(b_i) \in F$, $\forall i = 1, \ldots, n$, on connait toute l'application $f$.
\end{proposition}

\begin{corollary}
    Soit $(b_1, \ldots, b_n)$ une base de $E$. Alors $\varphi \in \mathcal{L}(E, \R^n)$ est un isomorphisme donné par :
    \[ \varphi(b_i) = e_i,\ \forall i \in \llbracket 1, n \rrbracket \]
    \[ \varphi \left( \sum_{i = 1}^{n} x_i b_i \right) = \sum_{i = 1}^{n} x_i e_i \equiv 
    \begin{pmatrix}
        x_1 \\
        \vdots \\
        x_n
    \end{pmatrix}
    \in \R^n
     \]
\end{corollary}

\begin{definition}
    Soient $f \in \mathcal{L}(E, F)$ et $\mathcal{B} \coloneqq (b_1, \ldots, b_n)$ une base de $E$ et $\mathcal{C} \coloneqq (c_1, \ldots, c_n)$ une base de $F$.
    Alors on peut définir une matrice $A \in \mathcal{M}_{m,n}(\mathcal{K}),\ \forall i \in \llbracket 1, n \rrbracket$. 
    \[ f(b_i) = \sum_{j = 1}^{n} A_{ji} c_j \]
\end{definition}

\begin{lemma} Soient $\K$ un corps et $E$ un espace-vectoriel de dimension $n \in \N$.
    \[ \dim(E) = n \implies E \cong \K^n \]
\end{lemma}

\begin{lemma}
    Soient $\mathcal{B} = (b_1, \ldots, b_n)$ une base de $E$, $(\lambda_1, \ldots, \lambda_n) \in \K^n$ et $\varphi \in \mathcal{L}(E, \K^n)$  alors :
    \begin{center}
    	$
    	\appli{\varphi}{E}{\K^n}{\sum_{i = 1}^{n} \lambda_i b_i}{(\lambda_i)_{1 \leq i \leq n}}
    	$
    \end{center}
    est une bijection, alors c'est un isomorphisme.
\end{lemma}

\begin{proposition}
    Soient $E, F$ des espaces vectoriels $f \in \mathcal{L}(E, F)$ un morphisme et $A$ sa matrice associée. 
    \[ y = f(x) \iff y = A \cdot x \]
\end{proposition}

\section{Projecteurs et symétries}
\begin{definition}
    Soit $f \in \mathcal{L}(E)$. Notons $f^2 \equiv f \circ f$.
    \begin{enumerate}
        \item On dit que $f$ \textbf{est idempotente/une projection} si et seulement si : $f^2 = f$.
        \item On dit que $f$ \textbf{est involutive/une symétrie linéaire} si et seulement si : $f^2 = id_E$.
    \end{enumerate}
\end{definition}

\begin{proposition} 
    Soit $p \in \mathcal{L}(E)$.
    \begin{enumerate}
        \item $id_E - p \text{ est une projection} \iff p \text{ est une projection}$
        \item $2p - id_E \text{ est une symétrie} \iff p \text{ est une projection}$
    \end{enumerate}
\end{proposition}

\begin{proof}
    \begin{enumerate}
        \item Posons $f(x) = id_E(x) - p(x)$. Montrons que $f(f(x)) = id_E(x) - p(x)$ si et seulement si $p(p(x))$.
        \begin{align*}
            f(f(x)) &= id_E(id_E(x) - p(x)) - p(id_E(x) - p(x)) \\ 
            &= id_E(id_E(x)) - id_E(p(x)) - p(id_E(x)) + p(p(x)) \\ 
            &= id_E(x) - p(x) - p(x) + p(p(x)) \\ 
            &= id_E(x) - 2p(x) + p(p(x)) \\ 
            &= id_E(x) - [2p(x) - p(p(x))]
        \end{align*}
        \begin{align*}
            id_E(x) - [2p(x) - p(p(x))] = id_E(x) - p(x) &\iff p(p(x)) = p(x) \\ 
            &\iff 2p(x) - p(p(x)) = p(x) \\ 
            &\iff 2p(x) = p(x) + p(p(x)) \\
            &\iff p(x) = p(p(x))
        \end{align*}
        \item Posons $s(x) = 2p(x) - id_E(x)$. 
        \\
        Montrons que $s(s(x)) = id_E(x)$ si et seulement si $p(p(x)) = p(x)$.
        \begin{align*}
            s(s(x)) &= s(2p(x) - id_E(x)) \\
            &= 2p(2p(x) - id_E(x)) - id_E(2p(x) - id_E(x)) \\
            &= 4p(p(x)) - 2p(id_E(x)) - 2id_E(p(x)) + id_E(id_E(x))
        \end{align*}
        \begin{align*}
            4p(p(x)) - 4p(x) + id_E(x) = id_E(x) &\iff 4p(p(x)) - 4p(x) = 0 \\
            &\iff 4p(p(x)) = 4p(x) \\
            &\iff p(p(x)) = p(x)
        \end{align*}
    \end{enumerate}
\end{proof}

\begin{definition}
    Soient $E = F \oplus G$, $u \in F$, $v \in G$, alors l'application 
    \begin{center}
    	$
    	\appli{p_F}{E}{E}{u+v}{u}
    	$
    \end{center}
    est appelée un \textbf{projecteur} sur $F$ \textbf{parallélement} à $G$.
\end{definition}

\begin{proposition}
    Soit $p_F$ définie comme dans la définition précédente.
    \begin{enumerate}
        \item $p_F$ est une projection.
        \item Soit $p$ une projection, $p$ est un projecteur sur $\Im(p)$ parallélement à son noyau $\ker(p)$.
    \end{enumerate}
\end{proposition}

\section{Rotations.}
\begin{definition}
	Soit $n \in \N^*$.
    \[ \operatorname{GL}(\R, n) \coloneqq \{ A \in \mathcal{M}_n(\R) : \det(A) \neq 0 \} \]
\end{definition}

\begin{definition}
	Soit $n \in \N^*$.
    \[ \operatorname{SL}(\R, n) \coloneqq \{ A \in \mathcal{M}_n(\R) : \det(A) = 1 \} \]
\end{definition}

\begin{definition}
	Soit $n \in \N^*$.
    \[ \operatorname{O}(n) \coloneqq \{ R \in \mathcal{M}_n(\R) : \forall (x, y) \in (\R^n)^2,\ \langle Rx|Ry \rangle = \langle x|y \rangle \} \]
\end{definition}

\begin{definition}
    Soit $n \in \N^*$.
    \[ \operatorname{SO}(n) \coloneqq \{ R \in \operatorname{O}(n) : \det(R) = 1 \} \]
\end{definition}

\begin{proposition}
	Soit $R \in \mathcal{M}_n(\K)$.
    \begin{enumerate}
        \item $R \in \operatorname{O}(n) \iff R^T \cdot R = I_n$.
        \item $R \in \operatorname{O}(n) \implies \det(R) \in \{ \pm 1 \}$.
    \end{enumerate}
\end{proposition}

\begin{corollary}
	Soit $n \in \N^*$.
    \[ \operatorname{O}(n) = \{ R \in \mathcal{M}_n(\K) : R^T \cdot R = I_n \} \]
\end{corollary}

\section{Changements de bases et matrices associées aux applications linéaires.}

\begin{definition}
    Soient $(n,p) \in (\N^*)^2$, $E$ et $F$ deux $\K$-espaces vectoriels, $\mathcal{B} = (b_1, \ldots, b_p) \in E^p$ une base de $E$, $\mathcal{B}' = (b_1', \ldots, b_n') \in F^n$ une base de $F$ et $f \in \mathcal{L}(E, F)$.
    \\
    On peut définir une matrice $\operatorname{Mat}_{\mathcal{B}\mathcal{B}'} (f) \in \mathcal{M}_{n,p}(\K)$ telle que :
    \begin{align*}
        \hspace{0.4cm}
        \begin{matrix}
            f(b_1) & \cdots & f(b_p) 
        \end{matrix}
    \end{align*}
    \vspace{-0.8cm}
    \begin{align*}
        \begin{matrix}
            b_1' \\
            \vdots \\ 
            b_n'
        \end{matrix}
        \begin{pmatrix}
            a_{11} & \cdots & a_{1p} \\
            \vdots & \ddots & \vdots \\ 
            a_{n1} & \cdots & a_{np}
        \end{pmatrix}
    \end{align*}
    avec les coefficients $(a_{11}, \ldots, a_{np}) \in \K^{np}$ tels que : 
    \[ f(b_j) = \sum_{i = 0}^n a_{ij} b_i' \]
\end{definition}

\begin{definition}[Matrice de passage]
    Soient $n \in \N^*$, $E$ et $F$ deux $\K$-espaces vectoriels de dimension $n$, $\mathcal{B} = (b_1, \ldots, b_n) \in E^n$ et $\mathcal{B}' = (b_1', \ldots, b_n') \in E^n$ deux bases de $E$. On appelle \textbf{matrice de passage} de $\mathcal{B}$ à $\mathcal{B}'$ la matrice carrée de taille $n$ dont la $j-$ième colonne est formée des coordonnées de $b_j'$ dans la base $\mathcal{B}$.
\end{definition}

\par \noindent Concrètement, écrivons : 
\begin{align*}
    \hspace{0.45cm}
    \begin{matrix}
        b_1' & \cdots & b_n' 
    \end{matrix}
\end{align*}
\vspace{-1.1cm}
\begin{align*}
    \begin{matrix}
        b_1 \\
        \vdots \\ 
        b_n
    \end{matrix}
    \begin{pmatrix}
        a_{11} & \cdots & a_{1n} \\
        \vdots & \ddots & \vdots \\ 
        a_{n1} & \cdots & a_{nn}
    \end{pmatrix}
\end{align*}
Avec les coefficients $(a_{11}, \ldots, a_{nn}) \in \K^{nn}$ tels que :
\[ b_j' = \sum_{i=1}^{n} a_{ij} b_i  \]

\begin{definition}
    Soient $A, B \in \mathcal{M}_{m,n}(\K)$.
    \begin{enumerate}
        \item On dit que $A$ et $B$ sont \textbf{équivalentes} si et seulement si : 
        \[ \exists P \in \operatorname{GL}(n),\ Q \in \operatorname{GL}(m),\ B = Q^{-1} A P \]
        \item On dit que $A$ et $B$ sont \textbf{similaires} si et seulement si :
        \[ \exists P \in \operatorname{GL}(n),\ B = P^{-1} A P \]
    \end{enumerate}
\end{definition}

