\chapter{Systèmes linéaires et matrices}
\def\arraystretch{1}

\section{Définitions et opérations élémentaires}
\begin{definition}[Matrice]
	Soient $m,n \in \N^*$ et $a_{1,1}, \ldots, a_{m,n} \in \K$. 
	\\
    Une \textbf{matrice} est un tableau de données appartenant à $\mathcal{M}_{m,n} (\K)$ :
    \begin{align*}
        \begin{pmatrix}
            a_{1,1} & a_{1,2} & \cdots & a_{1,n} \\
            a_{2,1} & a_{2,2} & \cdots & a_{2,n} \\
            \vdots & \vdots & \ddots & \vdots \\
            a_{m,1} & a_{m,2} & \cdots & a_{m,n}
        \end{pmatrix}
        = 
        (a_{i,j})_{\substack{1 \leq i \leq m \\ 1 \leq j \leq n}}
    \end{align*}
\end{definition}

\begin{definition}[Système linéaire]
    Soient $m,\ n \in \N^*$, $x_1, \ldots, x_n \in \K$ $n$ inconnues, $a_{1, 1}, \ldots, a_{m,n} \in \K$ et $b_1, \ldots, b_n \in \K$.
    \\
    Un \textbf{système linéaire} est décrit par :
    \begin{align*}
        \begin{cases}
            a_{1,1} \cdot x_1 + a_{1,2} \cdot x_2 + \cdots + a_{1,n} \cdot x_n = b_1 \\
            a_{2,1} \cdot x_1 + a_{2,2} \cdot x_2 + \cdots + a_{2,n} \cdot x_n = b_2 \\
            \vdots \\
            a_{m,1} \cdot x_1 + a_{m,2} \cdot x_2 + \cdots + a_{m,n} \cdot x_n = b_m
        \end{cases}
    \end{align*}        
    Sa matrice associée est : 
    \begin{align*}
        \left(
        \begin{matrix}    
            a_{1,1} & a_{1,2} & \cdots & a_{1,n} \\
            a_{2,1} & a_{2,2} & \cdots & a_{2,n} \\
            \vdots & \vdots & \ddots & \vdots \\
            a_{m,1} & a_{m,2} & \cdots & a_{m,n}
        \end{matrix}
        \
        \middle|
        \
        \begin{matrix}
            b_1 \\
            b_2 \\
            \vdots \\
            b_m
        \end{matrix}
        \right)
    \end{align*}
\end{definition}

\begin{definition}[Opérations élémentaires]
	Soient $i, j$ tels que $i \neq j$ des numéros de ligne et $\lambda \in \K^*$.
	\begin{enumerate}
		\item $L_i \leftarrow L_i + \lambda L_j$ 
		\item $L_i \leftrightarrow L_j$
		\item $L_i \leftarrow \lambda \cdot L_i$
	\end{enumerate}
	Opérations analogues sur les colonnes.
\end{definition}

\begin{definition}
	\begin{enumerate}
		\item Une matrice est \textbf{échelonnée} en lignes si et seulement si le nombre de zéros commençant une ligne croît strictement ligne par ligne jusqu'à ce qu'il ne reste plus que des zéros.
		\item Une matrice est dite dite \textbf{échelonnée réduite} en lignes si les pivots valent 1 et si les autres coefficients dans les colonnes des pivots sont nuls.
	\end{enumerate}
\end{definition}

\par Décrivons maintenant l'algorithme du pivot de Gauss utilisé pour résoudre des systèmes. On expliquera la méthode sur les lignes mais les principes sont analogues pour les colonnes.
\par \noindent La méthode consiste à appliquer les opérations élémentaires sur le système ou la matrice afin d'en faire un système ou une matrice échelonnée réduite.

\begin{definition}[Rang d'une matrice]
	Le rang d'une matrice $A$ est son nombre de lignes non nulles après échelonnage. Il est noté $\rg(A)$.
\end{definition}

\begin{theorem}
    Soient $\mathcal{S}$ un système linéaire de $m$ lignes et $n$ inconnues, $A \in \mathcal{M}_{m,n}(\K)$ et $B \in \mathcal{M}_{m,1}(\K)$ telles que $A|B$ forme la matrice associée à $\mathcal{S}$. $\mathcal{S}$ est solvable si et seulement si :
    \[ \rg(A) = \rg(A|B) \]
\end{theorem}

\section{Opérations sur les matrices}
\begin{definition}[Opérations sur les matrices]
	Soient $m, n, p, q \in \N^*$, $A \in \mathcal{M}_{m, n}(\K)$, $B \in \mathcal{M}_{p, q}(\K)$, $\lambda  \in \K$ tels que :
	\begin{align*}
		A &=
		\begin{pmatrix}
			a_{11} & \cdots & a_{1n} \\
			\vdots & \ddots & \vdots \\
			a_{m1} & \cdots & a_{mn}
		\end{pmatrix}
		&
		B &= 
		\begin{pmatrix}
			b_{11} & \cdots & b_{1q} \\
			\vdots & \ddots & \vdots \\
			b_{p1} & \cdots & b_{pq}
		\end{pmatrix}
	\end{align*}
	\begin{enumerate}
		\item Si $m = p$ et $n = q$ alors on peut définir l'addition entre $A$ et $ B$ et la multiplication par $\lambda$.
		\begin{align*}
			A + \lambda B &= 
			\begin{pmatrix}
				(a_{1,1} + \lambda b_{1,1}) & \cdots & (a_{1,n} + \lambda b_{1,q}) \\
				\vdots & \ddots & \vdots \\
				(a_{m,1} + \lambda b_{p,1}) & \cdots & (a_{m,n} + \lambda b_{p,q})
			\end{pmatrix}
		\end{align*}
		\item Si $n = p$ alors on peut définir la multiplication entre $A$ et $B$.\\
		Soit $C = AB$. Chaque coefficient $c_{i,j}$ est défini par :
		\begin{align*}
			c_{i,j} = \sum_{k=1}^{n} a_{i,k} b_{k,j}
		\end{align*}
        Autrement dit, le coefficient $c_{i,j}$ pour $C = AB$ est donné par :
        \begin{align*}
            &\begin{pmatrix}
                b_{1,1} & \cdots & \color{red}b_{1,j} & \cdots & b_{1,q} \\
                b_{2,1} & \cdots & \color{green}b_{2,j} & \cdots & b_{2,q} \\
                \vdots & \cdots & \vdots & \cdots & \vdots \\
                b_{p,1} & \cdots & \color{blue}b_{pj} & \cdots & b_{p,q}
            \end{pmatrix}
            \\
            \begin{pmatrix}
                a_{1,1} & a_{1,2} & \cdots & a_{1,n} \\ 
                \vdots & \vdots & \vdots & \vdots \\
                \color{red}a_{i,1} & \color{green}a_{i,2} & \cdots & \color{blue}a_{i,n} \\ 
                \vdots & \vdots & \vdots & \vdots \\
                a_{m,1} & a_{m,2} & \cdots & a_{m,n}
            \end{pmatrix}
            &\begin{pmatrix}
                c_{1,1} & c_{1,2} & \cdots & \cdots & c_{1,q} \\
                \vdots & \vdots & \vdots & \vdots & \vdots \\
                \cdots & \cdots & \boxed{c_{i,j}} & \cdots & c_{i,q} \\
                \vdots & \vdots & \vdots & \vdots & \vdots \\
                c_{m,1} & \cdots & \cdots & \cdots & c_{m,q}
            \end{pmatrix}
        \end{align*}
        \[ \boxed{c_{i,j}} = \color{red}a_{i,1} \cdot b_{1,j} \color{black}+ \color{green} a_{i,2} \cdot b_{2,j} \color{black}+ \color{blue}\cdots + a_{i,n} \cdot b_{p,j} \]
		Attention, la multiplication n'est pas commutative $(AB \neq BA)$.
	\end{enumerate}
\end{definition}

\begin{proposition}
	Soient $m, n, k, l \in \N^*$.
	\begin{enumerate}
		\item Soient $A, B, C \in \mathcal{M}_{m,n} (\K)$.
		\[A + B = B + A\]
		\[A + (B + C) = (A + B) + C\]
		\item Soient $A \in \mathcal{M}_{m, k}(\K),\ B \in \mathcal{M}_{k, l}(\K),\ C \in \mathcal{M}_{l, n}(\K)$.
		\[A \cdot (B \cdot C) = (A \cdot B) \cdot C\]
		\item Soient $A \in \mathcal{M}_{m, k}(\K),\ B, C \in \mathcal{M}_{k, n}(\K),\ \lambda \in \K$.
		\[A \cdot (B + \lambda C) = A \cdot B + \lambda \cdot A \cdot C\]
		\[(A + \lambda B) \cdot C = A \cdot C + \lambda \cdot B \cdot C\]
	\end{enumerate}
\end{proposition}

\begin{definition}[Matrice nulle]
	La matrice nulle est la matrice dont tous les coefficients sont $0$. On la note $0_{m,n}$, $m$ étant le nombre de lignes, $n$ le nombre de colonnes. 
\end{definition}

\begin{definition}[Symbole de Kronecker]
	On définit le symbole de Kronecker ainsi pour $i, j \in \N^*$ :
	\[ \delta_{i,j} = 
	\begin{cases}
		1 \text{ si } i = j \\
		0 \text{ si } i \neq j
	\end{cases}
	\]
\end{definition}

\begin{definition}[Matrice identité]
	La matrice identité est la matrice dont tous les coefficients sont $0$ à l'exception de ceux de la diagonale principale à $1$. On la note $I_n$, $n$ étant le nombre de lignes.
	\begin{align*}
		I_n = 
		\begin{pmatrix}
			1 & 0 & \cdots & 0 \\
			0 & 1 & \cdots & 0 \\
			\vdots & \ddots & \ddots & \vdots \\
			0 & \cdots & 0 & 1
		\end{pmatrix}
	\end{align*}
	On peut également dire que :
	\[ I_n = (\delta_{i,j})_{\substack{1 \leq i \leq n \\ 1 \leq j \leq n}} \]
\end{definition}

\begin{remark}
    La matrice identité est parfois notée : $\mathds{1}_n$.
\end{remark}

\begin{lemma}
	Soient $m,n \in \N^*$.
	\begin{multicols}{2}
	    \begin{enumerate}
    		\item $\forall A \in \mathcal{M}_{m, n}(\K),\ A + 0_{m,n} = A$
    		\item $\forall A \in \mathcal{M}_{n}(\K),\ A \cdot I_n = I_n \cdot A = A$
    	\end{enumerate}
	\end{multicols}
\end{lemma}

\begin{definition}[Transposée]
	Soient $m, n \in \N^*$, $A \in \mathcal{M}_{m, n}(\K)$ de coefficients $a_{1,1}, \ldots, a_{m,n} \in \K$.
	\\
	La transposée de $A$ est notée $A^T$. $A^T \in \mathcal{M}_{n,m}(\K)$ et pour $a^T_{1,1}, \ldots, a^T_{n,m} \in \K$ ses coefficients :
	\[ (a^T_{j,i})_{\substack{1 \leq j \leq n \\ 1 \leq i \leq m}} = (a_{i,j})_{\substack{1 \leq i \leq m \\ 1 \leq j \leq n}} \]
\end{definition}

\begin{example}
	Soit $A$ une matrice telle que :
	\[ 
	A = 
	\begin{pmatrix}
	1 & 2 & 3 \\
	4 & 5 & 6 \\
	7 & 8 & 9
	\end{pmatrix}
	\]
	\[
	A^T =
	\begin{pmatrix}
	1 & 4 & 7 \\
	2 & 5 & 8 \\
	3 & 6 & 9
	\end{pmatrix}
	\]
\end{example}

\begin{definition}
	Soient $n \in \N^*$ et $M \in \mathcal{M}_n(\K)$.
	    \begin{enumerate}
    		\item $M$ est symétrique $\iff M^T = M$
    		\item $M$ est anti-symétrique $\iff M^T = -M$
    	\end{enumerate}
\end{definition}

Pour calculer le déterminant d'une matrice carrée noté $\det(A)$ ou s'il n'y a pas d'ambiguïté $\abs{A}$, il existe plusieurs méthodes.
	\\
    Soit $A$ une matrice carrée de taille $n \times n$ de coefficients $a_{1,1},\ldots,a_{n,n} \in \K$. 
    \begin{enumerate}
        \item $n = 2$. 
        \[ 
        A = 
        \begin{pmatrix}
            a_{1,1} & a_{1,2} \\
            a_{2,1} & a_{2,2}
        \end{pmatrix}
        \]
        \[ \det(A) = a_{1,1} a_{2,2} - a_{1,2} a_{2,1} \]
        \item $n = 3$
        \[
        A = 
        \begin{pmatrix}
            a_{1,1} & a_{1,2} & a_{1,3} \\
            a_{2,1} & a_{2,2} & a_{2,3} \\
            a_{3,1} & a_{3,2} & a_{3,3}
        \end{pmatrix}
        \]
        \[ \det(A) = (a_{1,1} a_{2,2} a_{3,3} + a_{1,2} a_{2,3} a_{3,1} + a_{1,3} a_{2,1} a_{3,2} ) - (a_{1,3} a_{2,2} a_{3,1} + a_{1,2} a_{2,1} a_{3,3} + a_{1,1} a_{2,3} a_{3,2}) \]
        \item $n \geq 3$ \cite{wikipedia_determinant}.
        On peut calculer le déterminant d'une matrice en calculant à l'aide des déterminants des matrices de taille $n-1$. Pour se faire on développe en lignes ou en colonnes.
        Notons $A_{i,j}$ ma matrice obtenue en enlevant à $A$ sa $i$-ème ligne et sa $j$-ème colonne.
        \begin{align*}
        	A_{i,j} = 
        	\begin{pmatrix}
        		a_{1,1} & \cdots & a_{1, j-1} & a_{1, j+1} & \cdots & a_{1,n} \\
        		\vdots & & \vdots & \vdots & & \vdots \\
        		a_{i-1, 1} & \ldots & a_{i-1, j-1} & a_{i-1, j+1} & \cdots & a_{i-1,n} \\
        		a_{i+1, 1} & \cdots & a_{i+1, j-1} & a_{i+1, j+1} & \cdots & a_{i+1, n} \\
        		\vdots & & \vdots & \vdots & & \vdots \\
        		a_{n,1} & \cdots & a_{n,j-1} & a_{n,j+1} & \cdots & a_{n,n}
        	\end{pmatrix}
        \end{align*}
        On peut alors développer le calcul du déterminant  suivant une ligne ou une colonne : 
        \begin{enumerate}
        	\item Suivant la ligne $i$ : 
        	\[ \det(A) = \sum_{j=1}^n a_{i,j} (-1)^{i+j} \det(A_{i,j}) \]
        	\item Suivant la colonne $j$ : 
        	\[ \det(A) = \sum_{i=1}^n a_{i,j} (-1)^{i+j} \det(A_{i,j}) \]
        \end{enumerate}
    \end{enumerate}

\begin{example}
	Calculons le déterminant de la matrice 
	\begin{align*}
		A = 
		\begin{pmatrix}
		2 & 1 & -2 \\
		1 & 1 & -2 \\
		1 & 2 & -1
		\end{pmatrix}
	\end{align*}
	en développant suivant la première ligne.
	\begin{align*}
		\det(A) &= 2 (-1)^{1+1} 
		\begin{vmatrix}
			1 & -2 \\
			2 & -1	
		\end{vmatrix}			
		+ 1 (-1)^{1 + 2}
		\begin{vmatrix}
			1 & -2 \\
			1 & -1
		\end{vmatrix}
		+ (-2) (-1)^{1 + 3}
		\begin{vmatrix}
			1 & 1 \\
			1 & 2
		\end{vmatrix}
		\\
		&=
		2[1 \cdot (-1) - (-2) \cdot 2] -1[1 \cdot (-1) - (-2) \cdot 1] -2 [1 \cdot 2 - 1 \cdot 1]
		\\
		&= 6 - 1 - 2 \\
		&= 3
	\end{align*}
\end{example}

\begin{proposition}
	Soient $n \in \N^*$, $A, B \in \mathcal{M}_n(\K),\ \lambda \in \K$.
	\begin{multicols}{2}
	    \begin{enumerate}
		\item $\det(A \cdot B) = \det(A) \cdot \det(B)$
		\item $\det(\lambda A) = \lambda^n \det(A)$
		\item $\det(A^T) = \det(A)$
		\item $\det(A^{-1}) = \frac{1}{\det(A)}$ si $\det(A) \neq 0$
	\end{enumerate}
	\end{multicols}
\end{proposition}

\begin{definition}[Matrice inversible]
	Soient $n \in \N^*$ et $A \in \mathcal{M}_n(\K)$.
	\\
	$A$ est inversible si et seulement si existe une unique matrice $B = A^{-1}$ telle que 
	\[AB = BA = I_n.\]
\end{definition}

\begin{proposition}
	Une matrice est inversible si et seulement si son déterminant est non nul.
\end{proposition}

\begin{definition}[Comatrice]
	La comatrice $\com(A)$ d'une matrice $A \in \mathcal{M}_n(\K)$ est une matrice est définie ainsi :
	\[ (\com(A))_{i,j} = (-1)^{i+j} \det(A_{i,j}) \]
\end{definition}

Lorsque nous devons calculer l'inverse d'une matrice, plusieurs cas sont possibles.
    Si on a une matrice
    $A =
    \begin{pmatrix}
        a & b \\
        c & d
    \end{pmatrix}
    $.
    \[
    A^{-1} = \frac{1}{\det(A)} 
    \begin{pmatrix}
        d & -b \\
        -c & a
    \end{pmatrix}
    \]
    Une méthode générale pour calculer l'inverse d'une matrice $A$ est d'utiliser l'algorithme de Gauss-Jordan.
    On part de la matrice 
    \[ A|I_n \]
    et on applique les opérations élémentaires pour avoir une matrice de la forme
    \[I_n|B\]
    $B = A^{-1}$ est l'inverse de $A$.
    \\
    Une autre méthode générale est d'utiliser la comatrice de $A$.
    \[ A^{-1} = \frac{1}{\det(A)} \left( \com(A) \right)^T \]

\begin{definition}[Trace d'une matrice]
	Soient $A \in \mathcal{M}_n(\K)$ et $a_{1,1}, \ldots, a_{n,n}$ les coefficients de A. 
	\\
	La trace de $A$, notée $\tr(A)$ est définie par l'application suivante.
	\begin{center}
		$
		\appli{\tr}{\mathcal{M}_n(\K)}{\K}{A}{\sum_{i=1}^n a_{i,i}}
		$
	\end{center}
\end{definition}

\begin{lemma}
	\begin{multicols}{2}
	    \begin{enumerate}
    		\item $\tr(A^T) = \tr(A)$
    		\item $\tr(A B) = \tr(B A)$
    	\end{enumerate}
	\end{multicols}
\end{lemma}

\begin{proof}
	\leavevmode
    \begin{enumerate}
        \item Par application directe de la définition de la trace et de la transposée.
        \item Les coefficients $a_{1,1}, \ldots, a_{n,n}$ et $b_{1,1}, \ldots, b_{n,n}$ sont respectivement les coefficients des matrices $A$ et $B$. Les coefficients $(ab)_{1,1}, \ldots, (ab)_{n,n}$ et $(ba)_{1,1}, \ldots, (ba)_{n,n}$ sont respectivement ceux des matrices $AB$ et $BA$.
        \begin{align*}
        	\tr(AB) &= \sum_{i=1}^n (ab)_{i,i} \\
        			&= \sum_{i=1}^n \left( \sum_{k=1}^n a_{i,k} b_{k,i} \right) \\
        			&= \sum_{i=1}^n (a_{i,1} b_{1,i} + \cdots + a_{i,n} b_{n,i}) \\
        			&= (a_{1,1} b_{1,1} + \cdots + a_{1,n} b_{n,1}) + \cdots + (a_{n,1} b_{1,n} + \cdots + a_{n,n} b_{n,n}) \\
        			&= (b_{1,1}a_{1,1} + \cdots + b_{1,n}a_{n,1}) + \cdots + (b_{n,1} a_{1,n} + \cdots + b_{n,n} a_{n,n}) \\
        			&= \sum_{i=1}^n \left( b_{i,1} a_{1,i} + \cdots + b_{i,n} a_{n,i} \right) \\
        			&= \sum_{i=1}^n \left( \sum_{k=1}^n b_{i,k} a_{k,i} \right) \\
        			&= \sum_{i=1}^n (ba)_{i,i} \\
        			&= \tr(BA)
        \end{align*}
    \end{enumerate}
\end{proof}

\begin{proposition}
    Pour passer d'une forme cartésienne à une forme paramétrique, on applique le pivot de Gauss sur les lignes.
    Pour passer d'une forme paramétrique à une forme cartésienne, on utilise le déterminant.
\end{proposition}

\begin{definition}[Produit scalaire dans $\R^n$] 
	Soient $\vec{x} = \begin{pmatrix} x_1 \\ \vdots \\ x_n \end{pmatrix}$ et $\vec{y} = \begin{pmatrix} y_1 \\ \vdots \\ y_n \end{pmatrix}$ deux vecteurs de $\R^n$.
	\\
	On définit le produit scalaire ainsi :
	\begin{center}
		$
		\appli{\langle \cdot | \cdot \rangle}{\R^n \times \R^n}{\R}{(\vec{x}, \vec{y})}{\sum_{i = 1}^n x_i y_i}
		$
	\end{center}
\end{definition}

\begin{definition}[Produit vectoriel dans $\R^3$]
	Soient $\vec{x} = \begin{pmatrix} x_1 \\ x_2 \\ x_3 \end{pmatrix}$ et $\vec{y} = \begin{pmatrix} y_1 \\ y_2 \\ y_3 \end{pmatrix}$ deux vecteurs de $\R^3$.
	\\
	On définit le produit vectoriel ainsi :
	\begin{center}
		$
		\appli{\wedge}{\R^3 \times \R^3}{\R^3}{(\vec{x}, \vec{y})}{
		\begin{pmatrix}
			x_2 y_3 - x_3 y_2 \\
			x_1 y_3 - x_3 y_1 \\
			x_1 y_2 - x_2 y_1
		\end{pmatrix}
		}
		$
	\end{center}
\end{definition}

\section{Géométrie en basse dimension}
\begin{figure}[!h]
	\centering
	\begin{tikzpicture}
		\draw[->] (-3, 0) -- (3, 0) node[anchor=west]{$x$};
		\draw[->] (0, -3) -- (0, 3) node[anchor=south]{$y$};
	\end{tikzpicture}
	\caption{Une droite dans $\R^2$}
\end{figure}