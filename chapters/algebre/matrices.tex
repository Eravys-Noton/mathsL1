\chapter{Systèmes linéaires et matrices}
\def\arraystretch{1}

\section{Définitions et opérations élémentaires}
\begin{definition}[Matrice]
	Soient $(m,n) \in (\N^*)^2$ et $(a_{11}, \ldots, a_{mn}) \in \K^{mn}$. \\
    Une \emph{matrice} est un tableau de données appartenant à $\mathcal{M}_{m,n} (\K)$ :
    \begin{align*}
        \begin{pmatrix}
            a_{11} & a_{12} & \cdots & a_{1n} \\
            a_{21} & a_{22} & \cdots & a_{2n} \\
            \vdots & \vdots & \ddots & \vdots \\
            a_{m1} & a_{m2} & \cdots & a_{mn}
        \end{pmatrix}
    \end{align*}
\end{definition}

\begin{definition}[Système linéaire]
    Soient $(m,n) \in (\N^*)^2$, $(x_1, \ldots, x_n) \in \K^n$, $(a_{11}, \ldots, a_{mn}) \in \K^{mn}$ et $(b_1, \ldots, b_m) \in \K^m$.
    Un \emph{système linéaire} est décrit par :
    \begin{align*}
        \begin{cases}
            a_{11} \cdot x_1 + a_{12} \cdot x_2 + \cdots + a_{1n} \cdot x_n = b_1 \\
            a_{21} \cdot x_1 + a_{22} \cdot x_2 + \cdots + a_{2n} \cdot x_n = b_2 \\
            \vdots \\
            a_{m1} \cdot x_1 + a_{m2} \cdot x_2 + \cdots + a_{mn} \cdot x_n = b_m
        \end{cases}
    \end{align*}        
    Sa matrice associée est : 
    \begin{align*}
        \left(
        \begin{matrix}    
            a_{11} & a_{12} & \cdots & a_{1n} \\
            a_{21} & a_{22} & \cdots & a_{2n} \\
            \vdots & \vdots & \ddots & \vdots \\
            a_{m1} & a_{m2} & \cdots & a_{mn}
        \end{matrix}
        \
        \middle|
        \
        \begin{matrix}
            b_1 \\
            b_2 \\
            \vdots \\
            b_m
        \end{matrix}
        \right)
    \end{align*}
\end{definition}

\begin{definition}[Opérations élémentaires]
	Soient $i, j$ tels que $i \neq j$ des numéros de ligne et $\lambda \in \K^*$.
	\begin{multicols}{2}
        \begin{enumerate}
		\item $L_i \leftarrow L_i + \lambda L_j$ 
		\item $L_i \leftrightarrow L_j$
		\item $L_i \leftarrow \lambda \cdot L_i$
	\end{enumerate}
    \end{multicols}
	Opérations analogues sur les colonnes.
\end{definition}

\begin{proposition}[Pivot de Gauss]
    L'algorithme sera décrit ici pour les lignes, l'énoncé pour les colonnes est analogue sauf qu'on applique les opérations sur les colonnes. \\
	Soit un système ou une matrice. On applique les opérations élémentaires pour obtenir un système ou une matrice équivalente de la forme :
	\begin{align*}
		\begin{pmatrix}
			a_{11} & a_{12} & \cdots & a_{1n} \\
			0 & a_{22} & \cdots & a_{2n} \\
			\vdots & \vdots & \ddots & \vdots \\
			0 & 0 & \ldots & a_{mn}
		\end{pmatrix}
	\end{align*}
	On appelle cela échelonner.
    \\
    Ensuite le but est d'appliquer les opérations élémentaires pour \og remonter \fg dans l'algorithme et d'obtenir une matrice ou un système de cette forme :
    \begin{align*}
        \begin{pmatrix}
            a_{11} & 0 & \cdots & 0 \\
            0 & a_{22} & \cdots & 0 \\ 
            \vdots & \vdots & \ddots & \vdots \\ 
            0 & 0 & \cdots & a_{mn}
        \end{pmatrix}
    \end{align*}
\end{proposition}

\begin{definition}[Rang d'une matrice]
	Le rang d'une matrice $A$ est son nombre de lignes non nulles après échelonnage. Il est noté $\rg{A}$.
\end{definition}

\begin{theorem}
    Soient $\mathcal{S}$ un système linéaire de $m$ lignes et $n$ inconnues, $A \in \mathcal{M}_{m,n}(\K)$ et $B \in \mathcal{M}_{m,1}(\K)$ telles que $A|B$ forme la matrice associée à $\mathcal{S}$. $\mathcal{S}$ est solvable si et seulement si :
    \[ \rg(A) = \rg(A|B) \]
\end{theorem}

\section{Opérations sur les matrices}
\begin{definition}[Opérations sur les matrices]
	Soient $(m,n,p,q) \in (\N^*)^4$, $A \in \mathcal{M}_{m, n}(\K)$, $B \in \mathcal{M}_{p, q}(\K)$, $\lambda \in \K$, $(a_{11}, \ldots, a_{mn}) \in \K^{mn}$ et $(b_{11}, \ldots, b_{pq}) \in \K^{pq}$ tels que :
	\begin{align*}
		A &=
		\begin{pmatrix}
			a_{11} & \cdots & a_{1n} \\
			\vdots & \ddots & \vdots \\
			a_{m1} & \cdots & a_{mn}
		\end{pmatrix}
		&
		B &= 
		\begin{pmatrix}
			b_{11} & \cdots & b_{1q} \\
			\vdots & \ddots & \vdots \\
			b_{p1} & \cdots & b_{pq}
		\end{pmatrix}
	\end{align*}
	\begin{enumerate}
		\item Si $m = p$ et $n = q$ alors on peut définir l'addition entre $A$ et $ B$ et la multiplication par $\lambda$.
		\begin{align*}
			A + \lambda B &= 
			\begin{pmatrix}
				(a_{11} + \lambda b_{11}) & \cdots & (a_{1n} + \lambda b_{1q}) \\
				\vdots & \ddots & \vdots \\
				(a_{m1} + \lambda b_{p1}) & \cdots & (a_{mn} + \lambda b_{pq})
			\end{pmatrix}
		\end{align*}
		\item Si $n = p$ alors on peut définir la multiplication entre $A$ et $B$.\\
		Soit $C = AB$. Chaque coefficient $C_{ij}$ est défini par :
		\begin{align*}
			C_{ij} = \sum_{k=1}^{n} a_{ik} b_{kj}
		\end{align*}
        Autrement dit, soient $A, B$ deux matrices telles que :
        \begin{align*}
            A &= 
            \begin{pmatrix}
                a_{11} & a_{12} & \cdots & a_{1n} \\ 
                \vdots & \vdots & \vdots & \vdots \\
                a_{i1} & a_{i2} & \cdots & a_{in} \\ 
                \vdots & \vdots & \vdots & \vdots \\
                a_{m1} & a_{m2} & \cdots & a_{mn}
            \end{pmatrix}
            &
            B &= 
            \begin{pmatrix}
                b_{11} & \cdots & b_{1j} & \cdots & b_{1q} \\
                b_{21} & \cdots & b_{2j} & \cdots & b_{2q} \\
                \vdots & \cdots & \vdots & \cdots & \vdots \\
                b_{p1} & \cdots & b_{pj} & \cdots & b_{pq}
            \end{pmatrix}
        \end{align*}
        Alors le coefficient $(C)_{ij}$ pour $C = AB$ est donné par :
        \begin{align*}
            &\begin{pmatrix}
                b_{11} & \cdots & \color{red}b_{1j} & \cdots & b_{1q} \\
                b_{21} & \cdots & \color{green}b_{2j} & \cdots & b_{2q} \\
                \vdots & \cdots & \vdots & \cdots & \vdots \\
                b_{p1} & \cdots & \color{blue}b_{pj} & \cdots & b_{pq}
            \end{pmatrix}
            \\
            \begin{pmatrix}
                a_{11} & a_{12} & \cdots & a_{1n} \\ 
                \vdots & \vdots & \vdots & \vdots \\
                \color{red}a_{i1} & \color{green}a_{i2} & \cdots & \color{blue}a_{in} \\ 
                \vdots & \vdots & \vdots & \vdots \\
                a_{m1} & a_{m2} & \cdots & a_{mn}
            \end{pmatrix}
            &\begin{pmatrix}
                c_{11} & c_{12} & \cdots & \cdots & c_{1q} \\
                \vdots & \vdots & \vdots & \vdots & \vdots \\
                \cdots & \cdots & \boxed{c_{ij}} & \cdots & c_{iq} \\
                \vdots & \vdots & \vdots & \vdots & \vdots \\
                c_{m1} & \cdots & \cdots & \cdots & c_{mq}
            \end{pmatrix}
        \end{align*}
        \[ \boxed{c_{ij}} = \color{red}a_{i1} \cdot b_{1j} \color{black}+ \color{green} a_{i2} \cdot b_{2j} \color{black}+ \color{blue}\cdots + a_{in} \cdot b_{pj} \]
		Attention, la multiplication n'est pas commutative $(AB \neq BA)$.
	\end{enumerate}
\end{definition}

\begin{proposition}
	Soit $(m, n, k, l) \in (\N^*)^4$.
	\begin{enumerate}
		\item Soit $(A, B, C) \in (\mathcal{M}_{m,n} (\K))^3$.
		\[A + B = B + A\]
		\[A + (B + C) = (A + B) + C\]
		\item Soient $A \in \mathcal{M}_{m, k}(\K),\ B \in \mathcal{M}_{k, l}(\K),\ C \in \mathcal{M}_{l, n}(\K)$.
		\[A \cdot (B \cdot C) = (A \cdot B) \cdot C\]
		\item Soient $A \in \mathcal{M}_{m, k}(\K),\ (B, C) \in (\mathcal{M}_{k, n}(\K))^2,\ \lambda \in \K$.
		\[A \cdot (B + \lambda C) = A \cdot B + \lambda \cdot A \cdot C\]
		\[(A + \lambda B) \cdot C = A \cdot C + \lambda \cdot B \cdot C\]
	\end{enumerate}
\end{proposition}

\begin{definition}[Matrice nulle]
	La matrice nulle est la matrice dont tous les coefficients sont $0$. On la note $0_{m,n}$, $m$ étant le nombre de lignes, $n$ le nombre de colonnes. 
\end{definition}

\begin{definition}[Matrice identité]
	La matrice identité est la matrice dont tous les coefficients sont $0$ à l'exception de ceux de la diagonale principale à $1$. On la note $I_n$, $n$ étant le nombre de lignes.
	\begin{align*}
		I_n = 
		\begin{pmatrix}
			1 & 0 & \cdots & 0 \\
			0 & 1 & \cdots & 0 \\
			\vdots & \ddots & \ddots & \vdots \\
			0 & \cdots & 0 & 1
		\end{pmatrix}
	\end{align*}
\end{definition}

\begin{remark}
    La matrice identité est parfois notée : $\mathds{1}_n$.
\end{remark}

\begin{lemma}
	Soit $(m,n) \in (\N^*)^2$.
	\begin{multicols}{2}
	    \begin{enumerate}
    		\item $\forall A \in \mathcal{M}_{m, n}(\K),\ A + 0_{m,n} = A$
    		\item $\forall A \in \mathcal{M}_{n}(\K),\ A \cdot I_n = I_n \cdot A = A$
    	\end{enumerate}
	\end{multicols}
\end{lemma}

\begin{definition}[Transposée]
	Soient $(m,n) \in (\N^*)^2$, $A \in \mathcal{M}_{m, n}(\K)$.
	\\
	On définit $A^T \in \mathcal{M}_{n, m}(\K)$ la transposée de $A$ par :
	\[(A^T)_{ij} = A_{ji}\]
    Autrement dit, les colonnes de $A$ deviennent ses lignes et réciproquement.
\end{definition}

\begin{definition}
	Soient $n \in \N^*$ et $M \in \mathcal{M}_n(\K)$.
	\begin{multicols}{2}
	    \begin{enumerate}
    		\item $M$ est symétrique $\iff M^T = M$
    		\item $M$ est anti-symétrique $\iff M^T = -M$
    	\end{enumerate}
	\end{multicols}
\end{definition}

\begin{proposition}[Calculer le déterminant d'une matrice carrée]
    Pour calculer le déterminant d'une matrice carrée, il existe plusieurs méthodes.
    Soit $A$ une matrice carrée de taille $n \times n$. 
    \begin{itemize}
        \item $n = 2$. 
        \[ 
        A = 
        \begin{pmatrix}
            a & b \\
            c & d
        \end{pmatrix}
        \quad 
        \det(A) = ad - bc 
        \]
        \item $n = 3$
        \[
        A = 
        \begin{pmatrix}
            a & b & c \\
            d & e & f \\
            g & h & i
        \end{pmatrix}
        \quad 
        \det(A) = (aei + bfg + cdh) - (ceg + bdi + afh)
        \]
        \item $n \geq 3$.
        (Meilleure rédaction prévue)
        \begin{align*}
            \det(A) = \sum_{j = 1}^{n} A_{ij} = (-1)^{i+j} \cdot \det(A^{ij})
        \end{align*}
    \end{itemize}
\end{proposition}

\begin{proposition}
	Soient $n \in \N^*$, $(A, B) \in (\mathcal{M}_n(\K))^2,\ \lambda \in \K$.
	\begin{multicols}{2}
	    \begin{enumerate}
		\item $\det(A \cdot B) = \det(A) \cdot \det(B)$
		\item $\det(\lambda A) = \lambda^n \det(A)$
		\item $\det(A^T) = \det(A)$
		\item $\det(A^{-1}) = \frac{1}{\det(A)}$ si $\det(A) \neq 0$
	\end{enumerate}
	\end{multicols}
\end{proposition}

\begin{definition}[Matrice inversible]
	Soient $n \in \N^*$ et $A \in \mathcal{M}_n(\K)$.
	\\
	$A$ est inversible si et seulement si existe une unique matrice $B \equiv A^{-1}$ telle que 
	\[AB = BA = I_n.\]
\end{definition}

\begin{proposition}
	Une matrice est inversible si et seulement si son déterminant est non nul.
\end{proposition}

\begin{definition}[Comatrice]
	La comatrice $\com(A)$ d'une matrice $A \in \mathcal{M}_n(\K)$.
	$$
	[\com(A)]_{ij} = (-1)^{i + j} \det(A^{(ij)}) 
	$$
\end{definition}

\begin{proposition}[Calculer l'inverse d'une matrice]
    Lorsque nous devons calculer l'inverse d'une matrice, plusieurs cas sont possibles.
    Si on a une matrice
    $A =
    \begin{pmatrix}
        a & b \\
        c & d
    \end{pmatrix}
    $.
    \[
    A^{-1} = \frac{1}{\det(A)} 
    \begin{pmatrix}
        d & -b \\
        -c & a
    \end{pmatrix}
    \]
    Une méthode générale pour calculer l'inverse d'une matrice $A$ est d'utiliser l'algorithme de Gauss-Jordan.
    On part de la matrice 
    \[ A|I_n \]
    et on applique les opérations élémentaires pour avoir une matrice de la forme
    \[I_n|B\]
    $B \equiv A^{-1}$ est l'inverse de $A$.
    \\
    Une autre méthode générale est d'utiliser la comatrice de $A$.
    \[ A^{-1} = \frac{1}{\det(A)} \left( \com(A) \right)^T \]
\end{proposition}

\begin{definition}[Trace d'une matrice]
	Soit $A \in \mathcal{M}_n(\K)$, la trace de $A$, notée $\tr(A)$ est définie par l'application suivante.
	\begin{align*}
		\tr : \mathcal{M}_n(\K) &\to \K \\
		A &\mapsto \sum_{i = 1}^{n} A_{ii}
	\end{align*}
    Autrement dit, c'est la somme des coefficients de la diagonale principale (celle allant du premier au dernier coefficient).
\end{definition}

\begin{lemma}
	\begin{multicols}{2}
	    \begin{enumerate}
    		\item $\tr(A^T) = \tr(A)$
    		\item $\tr(A \cdot B) = \tr(B \cdot A)$
    	\end{enumerate}
	\end{multicols}
\end{lemma}

\begin{proof}\leavevmode
    \begin{enumerate}
        \item Par application directe de la définition de la trace et de la transposée.
        \item On utilise la définition du produit matriciel.
        \begin{align*}
            \tr(AB) = \tr(BA) \iff \tr(AB) - \tr(BA) = 0
        \end{align*}
        \begin{align*}
            \tr(AB) - \tr(BA) &=\sum_{i=1}^n (AB)_{ii} - \sum_{i=1}^n (BA)_{ii} \\ 
            &= \sum_{i=1}^n [(AB)_{ii} - (BA)_{ii}] \\ 
            &= \sum_{k=1}^n \left[ \sum_{i=1}^n \left( A_{ik} B_{ki} - B_{ik}A_{ki} \right) \right] \\ 
            = 
            \begin{split}
                &[(\cancel{a_{11} b_{11}} - \cancel{b_{11} a_{11}}) +  (\cancel{a_{12} b_{21}} - \cancel{b_{12} a_{21}}) + \cdots + (\cancel{a_{1n} b_{n1}} - \cancel{b_{1n} a_{n1}})] \\
                + &[(\cancel{a_{21} b_{12}} - \cancel{b_{21}a_{12}}) + (\cancel{a_{22} b_{22}} - \cancel{b_{22} a_{22}}) + \cdots + (\cancel{a_{2n}b_{n2}} - \cancel{b_{2n}a_{n2}})] \\
                + &\cdots + \\ 
                + &[(\cancel{a_{n1} b_{1n}} - \cancel{b_{n1} a_{1n}}) + (\cancel{a_{n2} b_{2n}} - \cancel{b_{n2} a_{2n}}) + \cdots + (\cancel{a_{nn} b_{nn}} - \cancel{b_{nn} a_{nn}})]
            \end{split}
            \\
            &= 0
        \end{align*}
    \end{enumerate}
\end{proof}

\begin{proposition}
    Pour passer d'une forme cartésienne à une forme paramétrique, on applique le pivot de Gauss sur les lignes.
    Pour passer d'une forme paramétrique à une forme cartésienne, on utilise le déterminant.
\end{proposition}
