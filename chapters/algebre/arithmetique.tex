\chapter{Arithmétique}
\def\arraystretch{1}

\begin{definition}
    Soient $a \in \Z$ et $b \in \Z^*$.
    \[ a \text{ est un multiple de } b \iff b \text{ est un diviseur de } a \iff b \mid a \iff \exists q \in \Z, a = bq \]
\end{definition}

\section{Divisibilité}

\begin{theorem}[Division euclidienne]
    Soient $a \in \Z$ et $b \in \Z^*$.
    \[ \exists ! (q, r) \in \Z^2,\ 0 \leq r < \abs{b} : a = bq + r  \]
    \begin{enumerate}
    	\item $a$ est appelé le \textbf{dividende}.
        \item $b$ est appelé le \textbf{diviseur}.
        \item $q$ est appelé le \textbf{quotient}.
        \item $r$ est appelé le \textbf{reste}.
    \end{enumerate}
\end{theorem}

\begin{proof}\cite{livre_prepa}
	\begin{enumerate}
		\item \textbf{Existence :} Supposons $a \in \N$ et considérons $M = \{n \in \N : nb \leq a\}$ l'ensemble des multiples de $b$ inférieurs à $a$. $M$ est une partie de $\N$. Nous avons deux propriétés : 
		\begin{enumerate}
			\item $M$ est non vide car 0 est un multiple de $b$ inférieur à $a$.
			\item $M$ est majoré par $a$ d'après sa définition.
		\end{enumerate}
		Ainsi $M$ admet un plus grand élément que l'on note $q$, vérifiant :
		\begin{enumerate}
			\item $qb \leq a$ car $q \in M$ 
			\item $(q + 1)b > a$ car $q + 1 > q$ sachant que $q$ est le plus grand élément de $M$, $q + 1 \notin M$.
		\end{enumerate}
		Posons : $r = a - bq$. Sachant que $a \geq bq,\ r \geq 0$. On a $r < b$ car $b = (q + 1)b - qb > a - bq = r$. Supposons que $a \in \Z$. 
		\begin{enumerate}
			\item Si $a$ est positif, on se ramène au cas précédent.
			\item Dans le cas où $a < 0,\ -a \geq 0$, ainsi il existe $(q', r') \in \Z^2$ tel que :
		\[ -a = bq' + r',\ 0 \leq r' < \abs{b} \]
		\[a = b(-q') - r' \]
			\begin{enumerate}
				\item Si $r' = 0$, on pose $q = -q'$ et $r = 0$ et on obtient le couple recherché.
				\item Si $r' \neq 0$, $r' \in \llbracket 1, b-1 \rrbracket$ et $a = b(-q' -1) + (b - r')$, on pose $q = -q' - 1$ et $r = b - r'$ et on obtient le couple recherché. 
			\end{enumerate}
		\end{enumerate}
		\item \textbf{Unicité :} Soient $q,\ q',\ r,\ r' \in \Z$. \\
	On a d'une part : $a = bq + r$ et d'autre part : $a = bq' + r'$.
	On sait que $0 \leq r < b$ et $0 \leq r' < b$ donc :
	\[ b \abs{q' - q} = \abs{r' - r} < b \]
	ce qui n'est possible que si $\abs{q' - q} = 0$ ce qui impliquerait $q = q'$. Ceci entraîne donc $r = r'$.
	\end{enumerate}
\end{proof}

\begin{example}
	Donnons la division euclidienne de $5$ par $2$.
	\[ 5 = 2 \cdot 2 + 1 \]
\end{example}

\section{PGCD et PPCM}

\begin{definition}
	Soient $a, b \in \Z^*$. 
	\begin{enumerate}
		\item L'ensemble des diviseurs de $\N^*$ commun à $a$ et $b$ admet un plus grand élément. C'est le \textbf{plus grand commun diviseur} des entiers $a$ et $b$. On le note $\pgcd(a, b)$.
		\item L'ensemble des diviseurs de $\N^*$ commun à $a$ et $b$ admet un plus petit élément. C'est le \textbf{plus petit commun multiple} des entiers $a$ et $b$. On le note $\ppcm(a, b)$.
	\end{enumerate}
\end{definition}

\begin{theorem}
	Soient $a, b \in \Z^*$ et $d \in \Z$.
	\begin{enumerate}
		\item $a \mid d \text{ et } b \mid d \implies \ppcm(a, b) \mid d$.
		\item $d \mid a \text{ et } d \mid b \implies d \mid \pgcd(a, b)$.
	\end{enumerate}
\end{theorem}

\begin{proof}
	\leavevmode
	\begin{enumerate}
		\item Posons $\ell = \ppcm(a,b)$. 
		\[ \exists ! (q,r) \in \Z^2,\ d = q \ell + R,\ 0 \leq r < \ell \]
		\[ r = d - q \ell,\ d \text{ et } \ell \text{ sont multiples de } a \text{ et } r \text{ est aussi un multiple de } a \text{ et } b \]
		Par la minimalité de $\ell,\ r = 0 \implies m = q \ell$.
		\item Posons $m = \pgcd(a, b)$. Montrons que :
		\[ \pgcd(m, d) = m \]
		Soit $\ell = \ppcm(m,d)$, $\ell \geq m$, $a$ et $b$ sont multiples de $m$ et $d$. D'après 1. :
		\[ \ell \mid a \text{ et } \ell \mid b,\ \ell \leq m \]
		Sachant que $\ell \geq m$ et $\ell \leq m$, $\ell = m$.
	\end{enumerate}
\end{proof}

\begin{definition}
	Soient $a, b \in \Z^*$. 
	\\	
	On dit que $a$ et $b$ sont \textbf{premiers entre eux} si et seulement si $\pgcd(a, b) = 1$.
\end{definition}

\section{Algorithme d'Euclide}
\begin{proposition}[Algorithme d'Euclide]
	Soient $a, b \in \Z^*$ tels que $\abs{a} > \abs{b}$.
	\[ \exists ! (q, r) \in \Z^2,\ 0 \leq r < \abs{b} : a = bq + r,\  \]
	\[ \pgcd(a, b) = \pgcd(b, a) = \pgcd(b, a - qb) = \pgcd(b, r) \]
	\begin{enumerate}
		\item Si $r = 0$ alors $a = qb$ et donc $\pgcd(a, b) = b$.
		\item Si $r \neq 0$ alors :
		\[ \exists ! (q_1, r_1) \in \Z^2,\ 0 \leq r_1 < r : b = q_1 r + r_1 \]
		Ensuite : 
		\begin{enumerate}
			\item Si $r_1 = 0$ alors $b = q_1 r$ et donc $\pgcd(a, b) = r$.
			\item Si $r_1 \neq 0$ alors :
			\[ \exists ! (q_2, r_2) \in \Z^2,\ 0 \leq r_2 < r_1 : r = q_2 r_1 + r_2 \]
		\end{enumerate}
	\end{enumerate}
	On procède de cette manière jusqu'à obtenir un reste nul, le $\pgcd$ de $a$ et $b$ est le dernier reste non nul.
\end{proposition}

\begin{example}
	Déterminons $\pgcd(216, 126)$.
	\begin{align*}
		216 &= 126 \cdot 1 + 90 \\
		126 &= 90 \cdot 1 + 36 \\
		90  &= 36 \cdot 2 + 18 \\
		36  &= 18 \cdot 2 + 0
	\end{align*}
	\[ \pgcd(216, 126) = 18 \]
\end{example}

\begin{theorem}[Identité de Bézout]
	Soient $a, b \in \Z$.
	\[ \exists (u, v) \in \Z^2,\ au + bv = \pgcd(a, b) \]
\end{theorem}

\begin{example}
	On cherche $u, v \in \Z$ tels que :
	\[ 216u + 126v = 18 \]
	On utilise l'algorithme d'Euclide :
	\begin{align*}
		216 &= 126 \cdot 1 + 90 \\
		126 &= 90 \cdot 1 + 36 \\
		90  &= 36 \cdot 2 + 18 \\
		36  &= 18 \cdot 2 + 0
	\end{align*}
	\begin{align*}
		90 - 36 \cdot 2 &= 18 \\
		90 - (126 - 90) \cdot 2 &= 18 \\
		90 \cdot 3 - 126 \cdot 2 &= 18 \\
		(216 - 126) \cdot 3 - 126 \cdot 2 &= 18 \\
		216 \cdot 3 + 126 \cdot (-5) &= 18
	\end{align*}
\end{example}

\begin{corollary}
	Soient $a, b \in \Z^*$ et $d \in \Z$.
	\[ \exists (u, v) \in \Z^2,\ au + bv = d \iff \pgcd(a,b) \mid d \]
\end{corollary}

\begin{lemma}[Lemme de Gauss]
	Soient $a, b \in \Z^*$ tels que $\pgcd(a, b) = 1$.
	\[ \forall c \in \Z,\ a \mid bc \implies a \mid c \]
\end{lemma}

\begin{proof}
	\begin{align*}
		\pgcd(a, b) = 1 &\implies \exists(u, v) \in \Z^2,\ au + bv = 1 \\
		&\implies a(cu) + b(cv) = c \\ 
		&\implies \pgcd(a, bc) \mid c
	\end{align*}
\end{proof}

\section{Nombres premiers}
\begin{definition}[Nombre premier]
	Soit $p \in \N^*$. 
	\\
	On dit que $p$ est \textbf{premier} si et seulement s'il admet \textbf{exactement} deux diviseurs : $1$ et $p$.
\end{definition}

\begin{theorem}[Théorème d'Euclide]
	Il existe une infinité de nombres premiers.
\end{theorem}

\begin{proof}
	Supposons qu'il existe $k$ nombres premiers $p_1, \ldots, p_k$.
	\[ N = p_1 \cdots p_k + 1 \implies p_i \nmid N \]
\end{proof}

\begin{lemma}
	Soit $n \in \N$ tel que $n \geq 2$.
	\\
	Si $p$ est le plus petit diviseur de $n$ tel que $p > 2$ alors $p$ est premier.
\end{lemma}

\begin{theorem}[Décomposition en facteurs premiers]
	Soit $n \in \N^*$ tel que $n \geq 2$. 
	\\
	Il existe une unique écriture de $n$ sous la forme de :
	\[ p_1^{\alpha_1} \cdots p_k^{\alpha_k} \]
	\begin{enumerate}
		\item Pour $1 \leq i \leq k$, les $p_i$ sont premiers.
		\item $\alpha_i \in \N^*$.
		\item $p_1 < p_2 < \cdots < p_k$.
	\end{enumerate}
\end{theorem}

\begin{example}
	Décomposons $40$ en facteurs premiers.
	\begin{align*}
		40 = 20 \cdot 2 \\
		20 = 10 \cdot 2 \\
		10 = 5 \cdot 2
	\end{align*}
	Ainsi $40 = 2^3 \cdot 5$.
\end{example}

\begin{proposition}
	Soient $a, b \in \Z$ et $i, k \in \N$. Pour déterminer $\pgcd(a, b)$ on peut utiliser leurs décompositions en facteurs premiers.
	\[ a = p_1^{\alpha_1} \cdots p_k^{\alpha_k} \]
	\[ b = n_1^{\beta_1} \cdots n_i^{\beta_i} \]
	$\pgcd(a, b)$ correspond au produit des facteurs premiers communs.
\end{proposition}

\section{Congruences}
\begin{definition}
	Soient $a, b \in \Z$ et $n \in \N$ tel que $n \geq 2$.
	\\
	On dit que $a$ et $b$ sont \textbf{congrus modulo} $n$ s'il existe un $k$ tel que :
	\[ a - b = k n \]
	On écrit généralement $a \equiv b \pmod n$ ou $a \equiv b \ [n]$.
\end{definition}

\begin{proposition}
	Soient $a, b, c, d \in \Z$ et $k, n \in \N$ tel que $n \geq 2$.
	\begin{enumerate}
		\item $a \equiv a \ [n]$. 
		\item $a \equiv b \ [n] \iff b \equiv a \ [n]$. 
		\item $a \equiv b \ [n] \text{ et } b \equiv c  \ [n] \implies a \equiv c \ [n]$.
		\item $a \equiv b \ [n] \text{ et } c \equiv d     \ [n] \implies a + c \equiv b + d \ [n] \text{ et } ac \equiv bd \ [n]$.
		\item $a \equiv b \ [n] \implies a^k \equiv b^k \ [n]$.
	\end{enumerate}
\end{proposition}

\begin{proof}
	Immédiate en utilisant la définition de la congruence.
\end{proof}

\begin{theorem}
	Soient $m_1, m_2 \in \N$ tel que :
	\begin{enumerate}
		\item $m_1 > 1$.
		\item $\pgcd(m_1, m_2) = 1$.
	\end{enumerate}
	Soient $a_1, a_2 \in \Z$ tel que pour $x \in \Z$ :
	\begin{enumerate}
		\item $x \equiv a_1 \ [m_1]$.
		\item $x \equiv a_2 \ [m_2]$.
	\end{enumerate}
	Notons $\mathcal{S}$ l'ensemble des solutions :
	$\exists k_0,\ k \in \Z$ tel que :
	\begin{enumerate}
		\item $\mathcal{S} = k_0 + k m_1 m_2$.
		\item $0 \leq k_0 < m_1 m_2$.
	\end{enumerate}
\end{theorem}

