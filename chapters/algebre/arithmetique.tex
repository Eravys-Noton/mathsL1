\chapter{Arithmétique}
\begin{definition}
    Soient $a \in \Z, b \in \Z^*$.
    \[ a \text{ est un multiple de } b \iff b \text{ est un diviseur de } a \iff b \mid a \iff \exists q \in \Z, a = bq \]
\end{definition}

\begin{theorem}[Division euclidienne]
    Soient $a \in \Z, b \in \Z^*$.
    \[ \exists ! (q, r) \in \Z^2, a = bq + r, (0 \leq r < \abs{b}) \]
\end{theorem}

\begin{nomenclature}
    Pour $a, b, c, d$ définis comme dans le théorème précédent.
    \begin{multicols}{2}
        \begin{itemize}
        \item $a$ est appelé le \emph{dividende}
        \item $b$ est appelé le \emph{diviseur}
        \item $q$ est appelé le \emph{quotient}
        \item $r$ est appelé le \emph{reste}
    \end{itemize}
    \end{multicols}
\end{nomenclature}
