\chapter{Continuité et limites de fonctions}
\def\arraystretch{1}

\begin{definition}[Limite d'une fonction]
    \begin{enumerate}
        \item En un point $a \in \R$, $\ell \in \R$ :
        \begin{enumerate}
            \item $ \lim_{x \to a} f(x) = \ell \iff 
	\forall \varepsilon > 0,\ \exists \delta > 0,\ \abs{x - a} \leq \delta \implies \abs{f(x) - \ell} \leq \varepsilon $.
            \item $ \lim_{x \to a} f(x) = +\infty \iff 
	\forall A \in \R,\ \exists \delta > 0,\ \abs{x - a} \leq \delta \implies f(x) \geq A $.
            \item $ \lim_{x \to a} f(x) = -\infty \iff 
	\forall A \in \R,\ \exists \delta > 0,\ \abs{x - a} \leq \delta \implies f(x) \leq A $.
        \end{enumerate}
        \item En l'infini, $\ell \in \R$ : 
        \begin{enumerate}
            \item $ \lim_{x \to +\infty} f(x) = \ell \iff 
	\forall \varepsilon > 0,\ \exists A \in \R,\ x \geq A \implies \abs{f(x) - \ell} \leq \varepsilon $.
            \item $ \lim_{x \to -\infty} f(x) = \ell \iff 
	\forall \varepsilon > 0,\ \exists A \in \R,\ x \leq A \implies \abs{f(x) - \ell} \leq \varepsilon $.
            \item $\lim_{x \to +\infty} f(x) = +\infty \iff 
	\forall A \in \R,\ \exists B \in \R,\ x \geq B \implies f(x) \geq A $.
            \item $ \lim_{x \to +\infty} f(x) = -\infty \iff 
	\forall A \in \R,\ \exists B \in \R,\ x \geq B \implies f(x) \leq A $.
            \item $ \lim_{x \to -\infty} f(x) = +\infty \iff 
	\forall A \in \R,\ \exists B \in \R,\ x \leq B \implies f(x) \geq A $.
            \item $\lim_{x \to -\infty} f(x) = -\infty \iff 
	\forall A \in \R,\ \exists B \in \R,\ x \leq B \implies f(x) \leq A$.
        \end{enumerate}
    \end{enumerate}
\end{definition}

\begin{theorem}
	Soient $I \subseteq \R,\ f : I \to \R,\ a \in I \cup \{-\infty, +\infty\}$ et $\ell \in \R \cup \{-\infty, +\infty\}$.
	\[ \lim_{x \to a} f(x) = \ell \iff \forall (u_n)_{n \in \N} : \lim_{n \to +\infty} u_n = a \implies \lim_{n \to +\infty} f(u_n) = \ell \]
\end{theorem}

\begin{definition}[Limite à gauche et à droite]
    Soient $I \subseteq \R,\ a \in I$ et $f : I \to \R $.
    \begin{enumerate}
        \item $ \lim_{\substack{x \to a \\ x < a}} f(x) = \lim_{x \to a^-} f(x) = \ell \iff \forall \varepsilon > 0,\ \exists \delta > 0,\ a - \delta < x < a \implies \abs{f(x) - \ell} \leq \varepsilon $.
        \item $\lim_{\substack{x \to a \\ x > a}} f(x) = \lim_{x \to a^+} f(x) = \ell \iff \forall \varepsilon > 0,\ \exists \delta > 0,\ a < x < a + \delta \implies \abs{f(x) - \ell} \leq \varepsilon$.
    \end{enumerate}
\end{definition}

\begin{definition}[Continuité]
	Soient $I \subseteq \R,\ a \in I$ et $f : I \to \R$.
        \\ 
        On dit que $f$ est \textbf{continue} si et seulement si :
        \[\lim_{x \to a} f(x) = f(a).\] 
	On dit que $f$ est continue sur $I$ si elle est continue en tout point de $I$.\\
	On peut également définir la continuité à gauche et à droite.
\end{definition}

\begin{remark}
	Les théorèmes d'opérations avec les limites, de comparaison et des gendarmes sont analogues à ceux vus  dans le chapitre sur les suites réelles.
\end{remark}

\begin{theorem}[Composition de limites]
	Soient $I, J \subseteq \R$, $f : I \to J,\ g : J \to \R$ et $a \in I$ ou éventuellement $a \in \{ \pm \infty \}$ tels que :
	\begin{enumerate}
		\item $\lim_{x \to a} f(x) = z \in I$.
		\item $\lim_{y \to z} g(y) = \ell$ existe.
	\end{enumerate}
	\[ \lim_{x \to a} g(f(x)) = \ell \]
\end{theorem}

\begin{example}
	Calculons $\lim_{x \to +\infty} \sin \left( \frac{1}{x} \right)$.
	\begin{enumerate}
		\item D'une part : $\lim_{x \to +\infty} \frac{1}{x} = 0$.
		\item D'autre part : $\lim_{y \to 0} \sin(y) = 0$.
	\end{enumerate}
	Donc par composition de limites : 
	\[ \lim_{x \to +\infty} \sin \left( \frac{1}{x} \right) = 0 \]
\end{example}

\begin{theorem}[Théorème des valeurs intermédiaires]
	Soient $a, b \in \R$ tels que $a < b$ et $f : [a, b] \to \R$ une fonction continue.
	\[ \forall y \in [f(a), f(b)],\ \exists c \in [a, b] : f(c) = y \]
\end{theorem}

\begin{proof}
	On utilise la borne supérieure.
	\\
	Soit $E = \{ x \in I \mid f(x) \leq y \}$. $a \in E$ donc $E \neq \varnothing$. On sait que $E \subseteq I$ donc $E$ est majoré.
	\\
	Posons $c = \sup(E)$.
	\\
	Puisque $c = \sup(E)$, il existe une suite $(c_n)_{n \in \N}$ d'éléments de $E$ telle que $\lim_{n \to +\infty} c_n = c$. 
	\\
	Comme $f$ est continue, on a :
	\[ \lim_{n \to +\infty} f(c_n) = f(c) \] 
	Puisque $c_n \in E,\ f(c_n) \leq y$. En passant à la limite, on a :
	\[ f(c) \leq y \]
	Montrons maintenant que $f(c) \geq y$.
	\begin{itemize}
		\item Si $c = b$, on a bien :
		\[ f(c) = f(b)  \geq y \]
		\item Si $c < b$, pour $n$ assez grand :
		\[ c < c + \frac{1}{n} \leq b \]
		Sachant que $c = \sup(E)$, $c + \frac{1}{n} \notin E$, on a donc :
		\[ f \left( c + \frac{1}{n} \right) > y \]
		On a $\lim_{n \to +\infty} c + \frac{1}{n} = c$ et $f$ étant continue :
		\[ \lim_{n \to +\infty} f \left( c + \frac{1}{n} \right) = f(c) \]
		Sachant que $f \left( c + \frac{1}{n} \right) > y$, en passant à la limite :
		\[ f(c) \geq y \]
	\end{itemize}
\end{proof}

\begin{theorem}
	Soient $a, b \in \R$ tels que $a < b$ et $f :\ ]a, b[ \ \to \R$. 
	Si $f$ est croissante.
	\begin{enumerate}
		\item $f$ admet une limite en $b$, qui est finie si et seulement si $f$ est \textbf{majorée}.
		\item $f$ admet une limite en $a$, qui est finie si et seulement si $f$ est \textbf{minorée}.
	\end{enumerate}
	Si $f$ est décroissante.
	\begin{enumerate}
		\item $f$ admet une limite en $b$, qui est finie si et seulement si $f$ est \textbf{minorée}.
		\item $f$ admet une limite en $a$, qui est finie si et seulement si $f$ est \textbf{majorée}.
	\end{enumerate}
	Soient $x_0 \in \ ]a, b[$, $f$ a une limite à gauche et à droite en $x_0$ et :
	\[ \lim_{x \to x_0^-} f(x) \leq f(x_0) \leq \lim_{x \to x_0^+} f(x) \]
\end{theorem}

\begin{theorem}
	Soient $a, b \in \R$ tels que $a < b$ et $f : [a, b] \to \R$ une fonction continue.
	\begin{enumerate}
	    \item $ f \text{ strictement croissante} \implies f : [a, b] \to [f(a), f(b)] \text{ est une bijection} $.
            \item $ f \text{ strictement décroissante} \implies f : [a, b] \to [f(b), f(a)] \text{ est une bijection} $.
	\end{enumerate}
\end{theorem}

\begin{theorem}
	Soient $I \subseteq \R$ et $f : I \to \R$ une injection continue.
	\\
	Alors $f$ est strictement monotone, donc bijective. Si on pose $J = f(I)$, $f^{-1} : J \to I$ est continue.
\end{theorem}

\begin{definition}[Segment]
	Un segment est un intervalle fermé borné.
\end{definition}

\begin{theorem}
	Soient $a, b \in \R$ tels que $a < b$ et $f : [a, b] \to \R$ une fonction continue.
	\\
	Alors $f$ est bornée sur $[a, b]$ et elle atteint ses bornes.
	\[ \exists m, M \in \R,\ \forall x \in [a, b] : m \leq f(x) \leq M \text{ et } \exists x_0, x_1 \in [a, b] : f(x_0) = m \text{ et } f(x_1) = M \]
\end{theorem}

\begin{definition}[Prolongement par continuité]
	Soient $I \subseteq \R$, $x_0 \in I$ et $f : I\backslash\{x_0\} \to \R$.
	\\
	On suppose que pour $\ell \in \R$, $\lim_{x \to x_0} f(x) = \ell$ existe. Alors la fonction :
	\begin{center}
		$
		\appli{\overset{\sim}{f}}{I}{\R}{x}{
		\begin{cases}
			f(x) \text{ si } x \neq x_0 \\
			\ell \text{ sinon}
		\end{cases}		
		}
		$
	\end{center}
\end{definition}