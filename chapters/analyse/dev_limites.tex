\chapter{Développements limités et formules de Taylor}
\def\arraystretch{1}
\label{chap:dev_limites}

\par Il arrive parfois des situations où l'on se retrouve avec des formes indéterminées de type \og $\frac{0}{0}$ \fg ou \og $\frac{\infty}{\infty}$ \fg lorsque nous essayons de calculer les limites, le théorème suivant permet de lever l'indétermination assez simplement.
Plus tard dans ce chapitre, nous pourrons également utiliser les développements limités pour lever les indéterminations.
\section{Règle de l'Hôpital}
\begin{theorem}[Règle de l'Hôpital \cite{regle_hopital_bibmath}]
    Soient $a, b \in \R,\ \ell \in \R \cup \{-\infty, +\infty\}$ tels que $a < b$ et $f, g : \ ]a, b[ \ \to \R$ deux fonctions dérivables telles que $g'$ ne s'annule pas. 
    \begin{enumerate}
        \item Si $\lim_{x \to a} f(x) = \lim_{x \to a} g(x) = 0$ et si $\lim_{x \to a} \frac{f'(x)}{g'(x)} = \ell$, alors $\lim_{x \to a} \frac{f(x)}{g(x)} = \ell$.
        \item Si $\lim_{x \to a} g(x) = \pm \infty$ et si $\lim_{x \to a} \frac{f'(x)}{g'(x)} = \ell$, alors $\lim_{x \to a} \frac{f(x)}{g(x)} = \ell$.
    \end{enumerate}
\end{theorem}

\begin{example}
    $\lim_{x \to 0} \frac{\sin(2x)}{x}$, on a ici une forme indéterminée \og $\frac{0}{0}$ \fg.
    On remarque que $\sin(2x)$ et $x$ sont dérivables en 0 et $x' = 1 \neq 0$, on utilise donc la règle de l'Hôpital pour lever l'indétermination.
    \begin{align*}
        \lim_{x \to 0} \frac{\sin(2x)}{x} &= \lim_{x \to 0} \frac{\sin'(2x)}{x'} \\
        &= \lim_{x \to 0} \frac{2 \cos(x)}{1} \\
        &= 2 \lim_{x \to 0} \cos(x) \\
        &= 2
    \end{align*}
\end{example}

\begin{definition} 
    Soit $a \in \R$.
    On dit que $x$ est au voisinage d'un point $a$ si et seulement si :
    \[ \exists \varepsilon > 0, \text{ tel que } x \in \ ]a - \varepsilon, a + \varepsilon[ \]
\end{definition}

\section{Relations de négligeabilité, domination, d'équivalence}
\begin{definition}
    Soient $I \subseteq \R$, $f, g : I \to \R$, $a \in I \cup \{-\infty, +\infty\}$ et $\varepsilon$ telle que $\lim_{x \to a} \varepsilon(x) = 0$.
    \begin{enumerate}
        \item On dit que $f$ est \textbf{dominée} par $g$ au voisinage de $a$ s'il existe un $B \in \R_+$ tel que $\abs{f(x)} \leq B \abs{g(x)}$ au voisinage de $a$. 
        On écrit alors $f \underset{a}{=} O(g)$ ou $f = O_a(g)$.
        \item On dit que $f$ est \textbf{négligeable} devant $g$ au voisinage de $a$ si $f(x) = g(x) \cdot \varepsilon(x)$.
        On écrit alors $f \underset{a}{=} o(g)$ ou $f = o_a(g)$.
        \item On dit que $f$ est \textbf{équivalente} à $g$ au voisinage de $a$ si $f(x) = g(x) (1 + \varepsilon(x))$.
        On écrit alors $f \underset{a}{\sim} g$.
    \end{enumerate}
\end{definition}

\begin{proposition}
	Pour toutes fonctions $f, g$ telles que $g$ ne s'annule pas :
    \begin{enumerate}
        \item Si $\lim_{x \to a} \frac{f(x)}{g(x)} = 0$ alors $f$ est \textbf{négligeable} devant $g$.
        \item Si $\lim_{x \to a} \frac{f(x)}{g(x)} = 1$ alors $f$ est \textbf{équivalente} à $g$.
        \item Si $\lim_{x \to a} \frac{f(x)}{g(x)}$ est bornée, alors $f$ est \textbf{dominée} par $g$.
    \end{enumerate}   
\end{proposition}

\begin{proposition}
	\begin{enumerate}
        \item $o(1) + o(1) = o(1)$.
        \item $\forall \lambda \in \R : \lambda \cdot o(1) = o(1)$.
        \item $\forall n \in \N^* : (o(1))^n = o(1)$.
        \item $\forall \alpha > 0 : (o(1))^{\alpha} = o(1)$.
        \item $\forall \alpha \in \R : (1 + o(1))^{\alpha} = 1 + o(1)$.
        \item $O(1) + O(1) = O(1)$.
        \item $\forall \lambda \in \R : \lambda \cdot O(1) = O(1)$.
        \item $\forall n \in \N^* : (O(1))^n = O(1)$.
        \item $o(1) \cdot O(1) = o(1)$.
    \end{enumerate}
\end{proposition}

\begin{proposition}
	\begin{enumerate}
        \item $\forall x \in \R_+^* : \ln(x) \underset{+\infty}{\sim} o(x)$.
        \item $\forall \alpha, \beta, x \in \R_+^* : (\ln(x))^{\beta} = o(x^{\alpha})$.
        \item $\forall \alpha, \beta \in \R_+^*,\ x \in \R : x^{\beta} = o(e^{\alpha x})$.
    \end{enumerate}
\end{proposition}

\begin{proposition}
    Soit $f$ une fonction polynomiale. 
    \begin{enumerate}
        \item Un équivalent de $f$ en l'infini est un son monôme de \textbf{plus haut degré}.
        \item Un équivalent de $f$ en 0 est son monôme de \textbf{plus bas degré}.
    \end{enumerate}
\end{proposition}

\section{Développements limités}

\begin{definition}[Polynôme de Taylor]
    Soit $f \in \mathcal{D}^n(I, \R)$ alors son polynôme de Taylor en $x_0$ est :
    \[ T_{n, x_0}^f(x) = \sum_{k = 0}^{n} \frac{f^{(k)}(x_0)}{k!} (x - x_0)^k \]
\end{definition}

\begin{theorem}[Formule de Taylor-Young]
    Soient $I \subseteq \R,\ f \in \mathcal{C}^{n}(I, \R),\ x_0 \in I$.
    \begin{align*}
        f(x) = T_{n, x_0}^f(x) + o((x - x_0)^n) 
    \end{align*}
\end{theorem}

\begin{theorem}[Formue de Taylor-Lagrange]
    Soient $I \subseteq \R,\ f \in \mathcal{C}^{n+1}(I, \R),\ x_0 \in I$. 
    \begin{align*}
        \exists c \in 
        \begin{cases}
            ]x_0, x[ \text{ si } x > x_0 \\
            ]x, x_0[ \text{ si } x < x_0
        \end{cases},\
        f(x) = T_{n, x_0}^f(x) + 
        \frac{f^{(n+1)}(c)}{(n+1)!} (x - x_0)^{n+1}
    \end{align*}
\end{theorem}

\begin{theorem}[Formule de Taylor avec reste intégral]
    Soient $I \subseteq \R,\ f \in \mathcal{C}^{n+1}(I, \R),\ x_0 \in I$.
    \begin{align*}
        f(x) = T_{n, x_0}^f(x) + \int_{x_0}^{x} \frac{f^{(n+1)}(t)}{n!} (x - t)^n \diffd t
    \end{align*}
\end{theorem}

\begin{corollary}[Inégalité de Taylor-Lagrange]
    Si $I \subseteq \R,\ \forall x \in I,\ \abs{f^{(n+1)}(x)} \leq M$ alors
    \[ \abs{\int_{x_0}^{x} \frac{f^{(n+1)}(t)}{n!} (x - t)^n \ \diffd t} \leq M \frac{\abs{x - x_0}^{n+1}}{(n+1)!} \]
\end{corollary}

\begin{definition}[Développement limité]
    Un polynôme $P_n(x)$ de degré $n$ satisfaisant :
    \[ f(x) = P_n(x) + o((x - x_0)^n) \]
    est un développement limité d'ordre $n$ de la fonction $f$.
\end{definition}

\begin{remark}
    Il est courant d'abréger développement limité par DL s'il n'y a pas d'ambiguïté.
\end{remark}

\begin{proposition}
    Si une fonction admet un développement limité, alors il est unique.
\end{proposition}

\subsection{Opérations sur les développements limités}
\begin{proposition}[\cite{exo7_analyse1}]
    Soient $c_0, \ldots, c_n \in \R$, $d_0, \ldots, d_n \in \R$ et $f,\ g$ deux fonctions admettant des développements limités en 0 telles que :
    \[ f(x) = c_0 + c_1x + \cdots + c_n x^n + o(x^n) \]
    \[ g(x) = d_0 + d_1x + \cdots + d_n x^n + o(x^n) \]
    \begin{enumerate}
        \item Addition : $f + g$ admet un développement limité en 0 à l'ordre $n$.
        \[ f(x) + g(x) = (c_0 + d_0) + (c_1 + d_1)x + \cdots + (c_n + d_n) x^n + o(x^n) \]
        \item Multiplication : $f \cdot g$ admet un développement limité en 0 à l'ordre $n$.
        \[ (c_0 + c_1 x + \cdots + c_n x^n) \cdot (d_0 + d_1 x + \cdots + d_n x^n) \]
        où l'on conserve les monômes de degré inférieur ou égal à $n$.
        \item Composition : Si $g(0) = 0$ alors la fonction $f \circ g$ admet un développement limité en 0 à l'ordre $n$.\\
        Posons $C(x) = c_0 + c_1x + \cdots + c_n x^n$ et $D(x) = d_0 + d_1x + \cdots + d_n x^n$.\\
        Sa partie polynomiale est le polynôme tronqué à l'ordre $n$ (on conserve les monômes de degré inférieur ou égal à $n$) de la composition $C(D(x))$. 
    \end{enumerate}
\end{proposition}
