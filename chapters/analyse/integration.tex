\chapter{Intégration}
\def\arraystretch{1}

\begin{definition}[Fonction en escalier]
	Soient $a, b \in \R$ tels que $a < b$, $n \in \N$ et $f : [a, b] \to \R$.
	\\
	$f$ est une fonction en escalier s'il existe une division de $[a, b]$ :
	\[ a = x_0 < \cdots < x_n = b \]
	telle que pour $0 \leq i \leq n,\ x_i \in \R,\  f_{| ]x, x_{i+1}[}$ est constante.
	Autrement dit c'est une fonction constante par morceaux.
\end{definition}

\begin{example}
	La fonction partie entière est une fonction en escalier.
\end{example}

\begin{definition}[Intégrale d'une fonction en escalier]
	Soient $f : [a, b] \to \R$ une fonction en escalier et $a_i$ les morceaux de la \textbf{subdivision}.
	\\
	Alors on définit :
	\[ \int_{a}^{b} f(x) \ \diffd x = \sum_{i = 1}^{n} f(a_i) (a_{i+1} - a_i) \]
\end{definition}

\begin{definition}
	Soit $f : [a, b] \to \R$. 
	\begin{enumerate}
		\item $f$ est \textbf{Riemann-intégrable} si pour tout $\varepsilon > 0$, il existe $e, E$ des fonctions en escalier telles que :
		\begin{enumerate}
			\item $e \leq f \leq E$.
			\item $\int_{a}^{b} (E(x) - e(x)) \ \diffd x < \varepsilon$.
		\end{enumerate}
		\item Soit $f$ \textbf{Riemann-intégrable} sur $[a, b]$.
		\[ \int_{a}^{b} f(x) \ \diffd x = \underset{e \leq f}{\sup} \int_{a}^{b} e(x) \ \diffd x = \underset{E \geq f}{\inf} \int_{a}^{b} E(x) \ \diffd x \]
	\end{enumerate}
\end{definition}

\begin{lemma}
	Soient $f, g : [a, b] \to \R$ deux fonctions Riemann-intégrables telles que $f(x) \leq g(x)$, alors :
	\begin{enumerate}
		\item $\int_{a}^{b} f(x) \ \diffd x \leq \int_{a}^{b} g(x) \ \diffd x$.
		\item $\abs{\int_{a}^{b} f(x) \ \diffd x} \leq \int_{a}^{b} \abs{f(x)} \ \diffd x$.
	\end{enumerate}
\end{lemma}

\begin{theorem}[Théorème fondamental de l'analyse]
	Soient $f \in \mathcal{C}^0([a, b])$ et $x \in [a, b]$.
	\begin{align*}
		F(x) = \int_a^x f(t) \ \diffd t
	\end{align*}
	alors $F \in \mathcal{C}^1(]a, b[)$ et pour tout $x \in \ ]a, b[,\ F'(x) = f(x)$.    
\end{theorem}

\begin{corollary}
    Si $F \in \mathcal{C}^1(]a, b[)$ tel que pour tout $x \in \ ]a, b[,\ F'(x) = f(x)$ alors
	\begin{align*}
		\int_a^b f(x) \ \diffd x = [F(x)]_a^b = F(b) - F(a) 
	\end{align*}
\end{corollary}

\begin{proposition}
	Soient $\lambda, a, b \in \R$ et $f, g \in \mathcal{C}^0([a, b])$.
	\begin{align*}
		\int_a^b f(x) + \lambda g(x) \ \diffd x = \int_a^b f(x) \ \diffd x + \lambda \int_a^b g(x) \ \diffd x 
	\end{align*}
\end{proposition}

\begin{proof}
    \begin{align*}
        \int_{a}^{b} f(x) + \lambda g(x) \ \diffd x &= [F(x) + \lambda G(x)]_a^b \\
                                                    &= (F(b) + \lambda G(b)) - (F(a) + \lambda G(a)) \\
                                                    &= F(b) - F(a) + \lambda (G(b) - G(a)) \\
                                                    &= \int_{a}^{b} f(x) \ \diffd x + \lambda \int_{a}^{b} g(x) \ \diffd x
    \end{align*}
\end{proof}

\begin{proposition}
	Soient $a, b \in \R$, $c \in \ ]a, b[$ et $f \in \mathcal{C}^0([a, b])$.
	\begin{align*}
		\int_a^b f(x) \ \diffd x = \int_a^c f(x) \ \diffd x + \int_c^b f(x) \ \diffd x
	\end{align*}
\end{proposition}

\begin{theorem}[Théorème de la moyenne]
    Soit $f \in \mathcal{C}^0([a, b])$. $\exists c \in \ ]a, b[$  tel que :
    \[ \frac{\int_{a}^{b} f(x) \ \diffd x}{b - a} = f(c) \]
\end{theorem}

\begin{theorem}[Intégration par parties]
	Soient $u, v \in \mathcal{C}^1([a, b])$ alors
	\begin{align*}
		\int_a^b u'(x) v(x) \ \diffd x = [u(x)v(x)]_a^b - \int_a^b u(x) v'(x) \ \diffd x
	\end{align*}
  \end{theorem}

\begin{proof}
    \begin{align*}
        (uv)'(x) &= u'(x) v(x) + u(x)v'(x) \\
        \iff u'(x)v(x) &= (uv)'(x) - u(x)v'(x)
    \end{align*}
    Par croissance et linéarité de l'intégrale :
    \begin{align*}
        \int_a^b u'(x)v(x) \ \diffd x &= \int_a^b (uv)'(x) \ \diffd x - \int_a^b u(x)v'(x) \ \diffd x \\
        \int_a^b u'(x)v(x) \ \diffd x &=  [u(x)v(x)]_a^b - \int_a^b u(x)v'(x) \ \diffd x 
    \end{align*}
\end{proof}

\begin{example}
	Soit $I = \int_{0}^{1} x e^x \ \diffd x$.
	\\
	Procédons par intégration par parties, posons :
	\[
	\begin{cases}
		u'(x) = e^x,\ v(x) = x \\
		u(x) = e^x,\ v'(x) = 1
	\end{cases}
	\]
	On a ensuite :
	\begin{align*}
		I &= [xe^x]_0^1 - \int_{0}^{1} e^x \ \diffd x \\
		  &= e - [e^x]_0^1 \\
		  &= e - (e - 1) \\
		  &= 1
	\end{align*}
\end{example}

\begin{theorem}[Intégration par changement de variable]
	Soient $I \subseteq \R,\ f \in \mathcal{C}^0(I)$ et $\varphi \in \mathcal{C}^1([a, b])$ tel que $\varphi ([a, b]) \subseteq I$ alors 
	\begin{align*}
		\int_a^b f(\varphi(x)) \cdot \varphi'(x) \ \diffd x = \int_{\varphi(a)}^{\varphi(b)} f(x) \ \diffd x
	\end{align*}
\end{theorem}

\begin{proof}
    \begin{align*}
        (f \circ \varphi)'(x) &= f' \circ \varphi(x) \cdot \varphi'(x) \\
        \int_a^b (f \circ \varphi)'(x) \ \diffd x &= \int_a^b f' \circ \varphi(x)  \cdot \varphi'(x) \ \diffd x \\
         &= [F(\varphi(x))]_a^b \\
         &= F(\varphi(b)) - F(\varphi(a)) \\
         &= [F(x)]_{\varphi(a)}^{\varphi(b)} \\
         &= \int_{\varphi(a)}^{\varphi(b)} f(x) \ \diffd x 
    \end{align*}
\end{proof}

\begin{example}
	Soit $I = \int_{1}^{e} \frac{(\ln(x))^n}{x} \ \diffd x$ pour $n \in \N$.
	\\
	Procédons par changement de variable, posons :
	\[
	\begin{cases}
		u = \ln(x) \\
		\diffd u = \frac{1}{x} \ \diffd x
	\end{cases}
	\]
	\begin{align*}
		I &= \int_{\ln(1)}^{\ln(e)} u^n \ \diffd u \\
		  &= \int_{0}^{1} u^n \ \diffd u \\
		  &= \left[ \frac{u^{n+1}}{n+1} \right]_0^1 \\
		  &= \frac{1}{n + 1}
	\end{align*}
\end{example}

Lorsque nous sommes confrontés à une intégrale de fonctions trigonométriques, on peut se ramener à une intégrale de fraction rationnelle en posant le changement de variable suivant :
\[ u = \tan(\frac{x}{2}) \]
\begin{multicols}{3}
    \begin{enumerate}
        \item $\cos(x) = \frac{1 - u^2}{1 + u^2}$
        \item $\sin(x) = \frac{2u}{1 + u^2}$
        \item $\diffd x = \frac{2}{1 + u^2} \diffd u$
    \end{enumerate}
\end{multicols}
\begin{proof}
	\leavevmode
    \begin{enumerate}
        \item \[ \cos(x) = \cos^2\left( \frac{x}{2} \right) - \sin^2\left( \frac{x}{2} \right) \]
            \begin{align*}
                \cos^2\left( \frac{x}{2} \right) - \sin^2\left( \frac{x}{2} \right) &= \left( \cos^2\left( \frac{x}{2} \right) - \sin^2\left( \frac{x}{2} \right) \right) \cdot \frac{1 + \tan^2 \left( \frac{x}{2} \right)}{1 + \tan^2 \left( \frac{x}{2} \right)}
            \end{align*}
            Posons $A(x) = \left( \cos^2\left( \frac{x}{2} \right) - \sin^2\left( \frac{x}{2} \right) \right) \left( 1 + \tan^2\left( \frac{x}{2} \right) \right)$.
            \begin{align*}
                A(x) &= \cos^2\left( \frac{x}{2} \right) - \sin^2\left( \frac{x}{2} \right) + \left[ \cos^2\left( \frac{x}{2} \right) - \sin^2\left( \frac{x}{2} \right) \right] \frac{\sin^2 \left( \frac{x}{2} \right)}{\cos^2 \left( \frac{x}{2} \right)} \\
                &= \cos^2 \left( \frac{x}{2} \right) - \frac{\sin^4 \left( \frac{x}{2} \right)}{\cos^2 \left( \frac{x}{2} \right)} \\
                &= \frac{\cos^4 \left( \frac{x}{2} \right) - \sin^4 \left( \frac{x}{2} \right)}{\cos^2 \left( \frac{x}{2} \right)}\\
                &= \frac{\left[ \cos^2\left( \frac{x}{2} \right) - \sin^2 \left( \frac{x}{2} \right) \right] \left[ \cos^2\left( \frac{x}{2} \right) + \sin^2 \left( \frac{x}{2} \right) \right]}{\cos^2 \left( \frac{x}{2} \right)}
            \end{align*}
             On sait que $\forall x \in \R, \cos^2(x) + \sin^2(x) = 1$.
             Ainsi : 
             \begin{align*}
                 A(x) = 1 - \frac{\sin^2\left( \frac{x}{2} \right)}{\cos^2\left( \frac{x}{2} \right)} = 1 - \tan^2 \left( \frac{x}{2} \right)
             \end{align*}
             Ainsi en posant $u = \tan(\frac{x}{2})$, on retrouve bien :
             \begin{align*}
                 \cos^2\left( \frac{x}{2} \right) - \sin^2\left( \frac{x}{2} \right) = \cos(x) = \frac{1 - u^2}{1 + u^2}
             \end{align*}
     \item \[ \sin(x) =  2 \sin(\frac{x}{2}) \cos(\frac{x}{2}) \]
        \begin{align*}
            2\sin(\frac{x}{2})\cos(\frac{x}{2}) &= 2\sin(\frac{x}{2})\cos(\frac{x}{2}) \cdot \frac{1 + \tan^2 \left( \frac{x}{2} \right)}{1 + \tan^2 \left( \frac{x}{2} \right)} \\
            &= \frac{2\sin(\frac{x}{2})\cos(\frac{x}{2}) \cdot \left( 1 + \tan^2 \left( \frac{x}{2} \right) \right)}{1 + \tan^2 \left( \frac{x}{2} \right)}
        \end{align*}
        Posons $A(x) = 2\sin(\frac{x}{2})\cos(\frac{x}{2}) \cdot \left( 1 + \tan^2 \left( \frac{x}{2} \right) \right)$.
        \begin{align*}
            A(x) &= 2\sin(\frac{x}{2})\cos(\frac{x}{2}) + 2\sin(\frac{x}{2})\cos(\frac{x}{2}) \tan^2 \left( \frac{x}{2} \right) \\
            &= 2\sin(\frac{x}{2})\cos(\frac{x}{2}) + 2\sin(\frac{x}{2})\cos(\frac{x}{2}) \frac{\sin^2 \left( \frac{x}{2} \right)}{\cos^2 \left( \frac{x}{2} \right)} \\
            &= 2\sin(\frac{x}{2})\cos(\frac{x}{2}) + \frac{2 \sin^3 \left( \frac{x}{2} \right)}{\cos(\frac{x}{2})} \\ 
            &= 2 \left( \sin(\frac{x}{2}) \left[ \cos(\frac{x}{2}) + \frac{\sin^2 \left( \frac{x}{2} \right)}{\cos(\frac{x}{2})} \right] \right) \\
            &= 2 \left( \sin(\frac{x}{2}) \left[ \frac{\cos^2 \left(\frac{x}{2}\right) + \sin^2 \left( \frac{x}{2} \right)}{\cos(\frac{x}{2})} \right] \right)
        \end{align*}
        On sait que $\forall x \in \R, \cos^2(x) + \sin^2(x) = 1$. Ainsi :
        \begin{align*}
            A(x) &= 2 \frac{\sin(\frac{x}{2})}{\cos(\frac{x}{2})} \\
            &= 2 \tan(\frac{x}{2})
        \end{align*}
        On a donc :
        \begin{align*}
            2\sin(\frac{x}{2})\cos(\frac{x}{2}) = \sin(x) = \frac{2 \tan(\frac{x}{2})}{1 + \tan^2 \left( \frac{x}{2} \right)}
        \end{align*}
        En posant $u = \tan(\frac{x}{2})$ on retrouve :
        \[ \sin(x) = \frac{2u}{1 + u^2} \]
        \item 
        \begin{align*}
            u = \tan(\frac{x}{2}) &\iff \arctan(u) = \frac{x}{2} \\
            &\iff x = 2 \arctan(u) \\
            &\iff \diffd x = \frac{2}{1 + u^2} \diffd u
        \end{align*}
    \end{enumerate}
\end{proof}



