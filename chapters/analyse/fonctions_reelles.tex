\chapter{Fonctions réelles}
\def\arraystretch{1}

\section{Définitions}
\begin{definition}[Fonction]
	Une fonction $f$ est la donnée de :
	\begin{enumerate}
		\item Un ensemble de départ $E$.
		\item Un ensemble d'arrivée $F$.
		\item Une flèche : $ f : E \to F $ à tout élément $x \in E$ associe un élément $f(x) \in F$.
	\end{enumerate}		
	On appelle \textbf{images} les éléments de $F$ et \textbf{antécédents} les éléments de $E$.
\end{definition}

\begin{definition}[Graphe d'une fonction]
	Soient $E$ et $F$ deux ensembles et $f : E \to F$ une fonction.
	\\
	Le graphe de $f$ est défini comme :
	\[ \operatorname{Gr}(f) = \operatorname{Gph}(f) = \{ (x, f(x)),\ x \in E \} \subseteq E \times F \] 
\end{definition}

\begin{definition}
	Soient $E$ et $F$ deux ensembles et $f : E \to F$ une fonction. On dit que :
    \begin{enumerate}
        \item $f$ est \textbf{injective} si et seulement si : 
        \[ \forall x_1, x_2 \in E : f(x_1) = f(x_2) \implies x_1 = x_2 \]
        \item $f$ est \textbf{surjective} si et seulement si : 
        \[ \forall y \in F,\ \exists x \in E : f(x) = y \]
        \item $f$ est \textbf{bijective} si et seulement si :
        \[ \forall y \in E,\ \exists ! x \in E : f(x) = y \]
    \end{enumerate}
\end{definition}

\begin{definition}[Bijection réciproque]
	Lorsque $f : E \to F$ est une bijection. On peut définir $f^{-1} : F \to E$ la bijection réciproque de $f$ qui associe à tout élément de $F$ son unique antécédent dans $E$.
\end{definition}

\begin{proposition}
	Soient $E$ et $F$ deux ensembles. Pour tous $x \in E$ et $y \in F$.
    \begin{multicols}{2}
        \begin{enumerate}
            \item $f^{-1} (f(x)) = x$
            \item $f(f^{-1}(y)) = y$
        \end{enumerate}
    \end{multicols}
\end{proposition}

\begin{definition}
	Soient $I \subseteq \R$, $f : I \to \R$ et $T \in \R_+$. On dit que :
    \begin{enumerate}
        \item $f$ est \textbf{paire} si et seulement si : 
        \[ \forall x \in I : f(x) = f(-x) \]
        \item $f$ est \textbf{impaire} si et seulement si  :
        \[ \forall x \in I : -f(x) = f(-x) \]
        \item $f$ est \textbf{$T$-périodique} si et seulement si :
        \[ \forall x \in \R,\ n \in \Z : f(x + nT) = f(x) \]
    \end{enumerate}
\end{definition}

\begin{definition}
	Soient $I \subseteq \R,\ f : I \to \R$. On dit que :
	\begin{enumerate}
	    \item $f$ est \textbf{majorée} ou qu'elle admet un \textbf{majorant} si et seulement si : 
	    \[ \exists M \in \R,\ \forall x \in I : f(x) \leq M \]
        \item $f$ est \textbf{minorée} ou qu'elle admet un \textbf{minorant} si et seulement si :
        \[ \exists m \in \R,\ \forall x \in I : f(x) \geq m \]
        \item $f$ est \textbf{bornée} si et seulement si elle est \textbf{majorée} et \textbf{minorée}.
	\end{enumerate}
\end{definition}

\begin{definition}
	Soient $I \subseteq \R$, $a \in \R$ et $f : I \to \R$. On dit que :
    \begin{enumerate}
        \item $f$ est \textbf{croissante} si et seulement si :
        \[ \forall x, y \in I : x \leq y \implies f(x) \leq f(y) \]
        \item $f$ est \textbf{décroissante} si et seulement si : 
        \[ \forall x, y \in I : x \leq y \implies f(x) \geq f(y) \]
        \item $f$ est \textbf{monotone} si et seulement si elle est \textbf{croissante} ou \textbf{décroissante}.
        \item $f$ est \textbf{strictement croissante} si et seulement si : 
        \[ \forall x, y \in I : x < y \implies f(x) < f(y) \]
        \item $f$ est \textbf{strictement décroissante} si et seulement si :
        \[ \forall x, y \in I : x < y \implies f(x) > f(y) \]
        \item $f$ est \textbf{strictement monotone} si et seulement si elle est \textbf{strictement croissante} ou \textbf{strictement décroissante}.
        \item $f$ est \textbf{constante} si et seulement si :
        \[ \forall x \in I : f(x) = a \]
    \end{enumerate}
\end{definition}

\section{Opérations sur les fonctions}
\begin{definition}[Opérations sur les fonctions]
	Soient $f, g$ deux fonctions et $A \subseteq \R$.
	\begin{center}
		$
		\begin{array}{cc}
			\appli{f+g}{A}{\R}{x}{f(x)+g(x)}
			&
			\appli{fg}{A}{\R}{x}{f(x)g(x)}
		\end{array}
		$
	\end{center}
	Si $g$ ne s'annule pas :
	\begin{center}
		$
		\appli{\frac{f}{g}}{A}{\R}{x}{\frac{f(x)}{g(x)}}
		$
	\end{center}
\end{definition}

\begin{definition}[Composition de fonctions]
	Soient $E, F, G, H \subseteq \R,\ F \subseteq G,\ f : E \to F,\ g : G \to H$.
	\begin{center}
		$
		\appli{g \circ f}{E}{H}{x}{g(f(x))}
		$
	\end{center}
\end{definition}

\begin{definition}[Fonction identité]
    Soient $E, F \subseteq \R$.
	\begin{center}
		$
		\appli{id_E}{E}{F}{x}{x}
		$
	\end{center}
\end{definition}

\begin{proposition}
	Soient $E$ et $F$ deux ensembles et $f : E \to F$ une bijection.
    \begin{multicols}{2}
        \begin{enumerate}
            \item $f^{-1} \circ f = id_E$.
            \item $f \circ f^{-1} = id_F$.
        \end{enumerate}
    \end{multicols}
\end{proposition}

\begin{definition}[Image directe]
    Soient $E$ et $F$ deux ensembles $f : E \to F$ et $A \subseteq E$.
	\[ f(A) = \{ f(x),\ x \in A \} : f(A) \subseteq F \]
\end{definition}

\begin{definition}[Image réciproque]
	Soient $E$ et $F$ deux ensembles $f : E \to F$ et $B \subseteq F$.
	\[ f^{-1}(B) = \{ \forall x \in E,\ f(x) \in B \} : f^{-1} \subseteq E \]
\end{definition}

\begin{proposition}
    Soient $E$ et $F$ deux ensembles $f : E \to F,\ A_1, A_2 \subseteq E$ et $B_1, B_2 \subseteq F$.
    \begin{enumerate}
        \item $f(A_1 \cup A_2) = f(A_1) \cup f(A_2)$.
        \item $f^{-1} (B_1 \cup B_2) = f^{-1} (B_1) \cup f^{-1} (B_2)$.
        \item $f^{-1} (B_1 \cap B_2) = f^{-1} (B_1) \cap f^{-1} (B_2)$.
    \end{enumerate}
\end{proposition} 

