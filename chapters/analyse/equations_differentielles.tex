\chapter{Equations différentielles linéaires}
\def\arraystretch{1}

\noindent Pour résoudre une équation différentielle, nous allons suivre ces 3 étapes :
\begin{enumerate}
	\item Déterminer les solutions de l'équation homogène associée.
	\item Déterminer une solution particulière de l'équation différentielle.
	\item Combiner les solutions précédentes pour obtenir la solution générale.
\end{enumerate}

\section{\'Equations différentielles d'ordre 1}

\begin{definition}[\'Equation différentielle d'ordre 1]
	Soient $y \in \mathcal{C}^1(\R)$ et $a, b \in \mathcal{C}^0(\R)$. \\
	Une équation différentielle d'ordre 1 est une équation de la forme
	\[ y' + a(x)y = b(x) \]
\end{definition}

\begin{theorem}
	Soient $y \in \mathcal{C}^1(\R)$ et $a \in \mathcal{C}^0(\R)$, les solutions de :
	\[ y' + a(x)y = 0 \]
	sont de la forme $C \in \R,\ A'(x) = a(x)$:
	\[ y = C \cdot \exp(-A(x)) \]
\end{theorem}

\begin{proof}
    Il faut montrer l'inclusion dans les deux sens. \\
    \boxed{\supseteq} :
	Définissons
	\[ y = C e^{-A(x)} \]
	on aurait donc
	\[ y' = -C a(x)e^{-A(x)} \]
	puis
	\[ y' + a(x)y(x) = -C a(x) e^{-A(x)} + a(x)Ce^{-A(x)} = 0 \]
    \boxed{\subseteq} : \cite{bibmath_resolution_eq_diff} :
    Supposons $y$ solution de $y' + a(x) y = 0$.
    Alors il existerait un $C \in \R$ tel que $y(x) = Ce^{-A(x)}$.
    Posons $f(x) = y(x)e^{A(x)}$.
    \begin{align*}
        f'(x) &= y'(x)e^{A(x)} + y(x)e^{A(x)}A'(x) \\
        &= -a(x)y(x)e^{A(x)} + a(x)y(x)e^{A(x)} \\
        &= 0
    \end{align*}
    Cela implique que $f(x) = C,\ C \in \R$. Donc :
    \begin{align*}
        C = y(x)e^{A(x)} \iff y(x) = Ce^{-A(x)}
    \end{align*}
\end{proof}

\begin{example}
    $(E_1) : y' - 2xy = 0$.
    En appliquant le théorème précédent on obtient les solutions :
    \begin{align*}
        y_0 = Ce^{x^2}
    \end{align*}
\end{example}

Pour trouver une solution particulière d'une équation différentielle d'ordre 1. Nous utilisons $y_{h}(x)$, sauf qu'ici, $C$ n'est plus une constante mais une fonction. Cette méthode est appelée \textbf{variation de la constante}.
    \begin{align*}
        \begin{cases}
            y_{p} = C(x)e^{-A(x)} \\
            y_{p}' = C'(x)e^{-A(x)} - C(x)a(x)e^{-A(x)}
        \end{cases}
    \end{align*}
    On obtient alors en remplaçant dans l'équation générale :
    \begin{align*}
        &y_{p}' + a(x) y_{p} = b(x) \\
        \iff &C'(x)e^{-A(x)} - C(x)a(x) e^{-A(x)} + a(x)C(x)e^{-A(x)} = b(x) \\
        \iff &C'(x) e^{-A(x)} = b(x) \\
        \iff &C(x) = \int b(x) e^{A(x)}
    \end{align*}

\begin{example}
    $(E_1) : y' - 2xy = \exp(x^2 - x)$.
    On utilise donc la solution homogène trouvée précédemment avec la variation de la constante.
    \begin{align*}
        \begin{cases}
            y_p = C(x) e^{x^2} \\ 
            y_p' = C'(x) e^{x^2} + C(x)2x e^{x^2}
        \end{cases}
    \end{align*}
    On remplace dans l'équation $(E_1)$.
    \begin{align*}
        y_p' - 2x y_p &= \exp(x^2 - x) \\
        C'(x) e^{x^2} + C(x)2x e^{x^2} - 2x C(x)e^{x^2} &= \exp(x^2 - x) \\
        C'(x)e^{x^2} &= \exp(x^2 - x) \\ 
        C'(x) &= \frac{\exp(x^2 - x)}{e^{x^2}} \\
        C'(x) &= \exp(x^2 - x - x^2) \\
        C'(x) &= \exp(-x) \\
        C(x) &= -\exp(-x)
    \end{align*}
    Ainsi une solution particulière de $(E_1)$ :
    \[ y_p = -\exp(-x) \exp(x^2) = -\exp(x^2 - x) \]
\end{example}

\section{\'Equations différentielles d'ordre 2}

\begin{definition}[\'Equations différentielle d'ordre 2]
	Soient $y \in \mathcal{C}^2(\R)$, $b \in \mathcal{C}^0(\R)$ et $p, q \in \R$. \\
	Une équation différentielle d'ordre 2 est une équation de la forme
	\[ y'' + py' + q = b(x) \]
\end{definition}

\begin{theorem}
  Soient $y \in \mathcal{C}^2(\R),\ p, q \in \R$ et $(E)$ l'équation suivante :
  \[ (E) : y'' + py' + qy = 0, \]
  On s'intéresse d'abord à cette équation associée :
  \[ \ \lambda^{2} + p\lambda + q = 0 \]
  Pour $\Delta = p^2 - 4q$. 
  \begin{enumerate}
	\item Cas 1 :
		  $\Delta > 0$
		  \[ y = C_{1} e^{\lambda_{1} x} + C_{2} e^{\lambda_{2} x},\ \lambda_{i} C_{i} \in \R \]
        \[ \lambda_{1/2} = \frac{-p \pm \sqrt{\Delta}}{2} \]
	\item Cas 2:
		  $\Delta = 0$
		  \[ y = (C_{1} + C_{2}x) e^{\lambda x},\ C_{i}, \lambda \in \R \]
        \[ \lambda = \frac{-p}{2} \]
	\item Cas 3 :
		  $\Delta < 0$
		  $,\ \lambda_{1} = a + ib,\ a, b \in \R,\ \lambda_{2} = a - ib = \overline{\lambda_{1}}$
		  \[ y = e^{ax}(C_1 \cos(\abs{b}x) + C_2 \sin(\abs{b}x)), C_i \in \R \]
        \[ \lambda_{1/2} = \frac{-p \pm i \sqrt{-\Delta}}{2} \]
  \end{enumerate}
\end{theorem}

\par La variation de la constante est difficilement applicable sur les équations différentielles de second ordre, nous devons trouver d'autres méthodes.
Toutes les équations de second ordre qu'on étudiera dans ce chapitre auront pour second membre une composé de fonctions polynomiales, exponentielles et trigonométriques. Ainsi, nous pouvons utiliser ces propriétés.
\\
\par \noindent Voici une méthode pour trouver une solution particulière d'une équation de type :
    \[ y'' + py' + qy = b(x) \]
    pour $p, q \in \R,\ b \in \mathcal{C}^0(\R),\ y \in \mathcal{C}^2(\R)$. \\
    Soient $\alpha, \beta \in \R$, $m \in \N$, $P, P_1, P_2, Q_1, Q_2 \in \R[X]$. 
    \begin{itemize}
        \item \cite{exo7_analyse1} Si $b(x) = e^{\alpha x}(P_1(x) \cos(\beta x) + P_2(x) \sin(\beta x) )$
        \[y_{p} = x^{m} e^{\alpha x}(Q_1(x) \cos(\beta x) + Q_2(x) \sin(\beta x)) ,\ \deg(Q_1), \deg(Q_2) \leq \max\left( \deg(P_1), \deg(P_2) \right) \]
        \[ m = \begin{cases}
            0 \text{ si } \alpha + i\beta \text{ n'est pas racine de l'équation caractéristique} \\
            1 \text{ sinon}
        \end{cases} \]
        \item Si $b(x) = P(x) e^{\alpha x}$ 
        \[ y_p = x^m Q(x)e^{\alpha x},\ \deg(Q) \leq \deg(P) \]
        avec $m$ l'ordre de multiplicité (voir \autoref{def:ordre_mult}) de la racine $\alpha$ par rapport à l'équation caractéristique associée.
\end{itemize}

\begin{remark}
    Il existe des propriétés analogues pour les équations différentielles du premier ordre, parfois cela est plus rapide qu'avec la variation de la constante.
\end{remark}

\begin{example}
    $(E) : y'' - 2y' + 3y = 9x^2 e^{2x} + 4e^x$.
    Tout d'abord, résolvons l'équation homogène associée :
    \[ (E_0) : y'' - 2y' + 3y = 0 \]
    \[ \lambda^2 - 2\lambda + 3 = 0 \]
    \[ \Delta = (-2)^2 - 4 \cdot 3 = -8 < 0 \]
    \begin{align*}
        \lambda_1 &= \frac{2 - i\sqrt{8}}{2} & \lambda_2 &= \frac{2 + i\sqrt{8}}{2} \\
        &= 1 - i\sqrt{2} & &= 1 + i\sqrt{2}
    \end{align*}
    Ainsi les solutions de $(E_0)$ sont :
    \[ y_0 = e^x \left(C_1 \cos(\sqrt{2}x) + C_2 i \sin(\sqrt{2}x) \right) \]
    Trouvons une solution particulière de l'équation :
    \[ (E_1) : y'' - 2y' + 3y = 9x^2 e^{2x} \]
    On remarque que le second membre (le \og $b(x)$ \fg) est de la forme $P(x) e^{\alpha x}$ avec $\deg(P) = 2$ et $\alpha \notin \{1 \pm i\sqrt{2}\}$, ainsi la solution particulière est de la forme $y_1 = (ax^2 + bx + c)e^{2x},\ a, b, c \in \R$.
    On a donc :
    \begin{align*}
        \begin{cases}
            y_1 = (ax^2 + bx + c)e^{2x} \\
            y_1' = (2ax + b)e^{2x} + 2(ax^2 + bx + c)e^{2x} = (2ax + b)e^{2x} + 2y_1 \\
            y_1'' = 2a e^{2x} + 2(2ax + b)e^{2x} + 2y_1'
        \end{cases}
    \end{align*}
    En remplaçant dans $(E_1)$ :
    \begin{align*}
        2ae^{2x} + 2(2ax + b)e^{2x} + 2y_1' - 2[(2ax + b)e^{2x} + 2y_1] + 3y_1 &= 9x^2 e^{2x} \\
        2ae^{2x} + (4ax + 2b)e^{2x} + 2y_1' - (4ax + 2b)e^{2x} - 4y_1 + 3y_1 &= 9x^2 e^{2x} \\
        2ae^{2x}+ 2y_1' - y_1 &= 9x^2e^{2x} \\
        2ae^{2x} + 2[(2ax + b)e^{2x} + 2y_1] - y_1 &= 9x^2e^{2x} \\
        2ae^{2x} + (4ax + 2b)e^{2x} + 4 y_1 - y_1 &= 9x^2 e^{2x} \\
        2ae^{2x} + (4ax + 2b)e^{2x} + 3y_1 &= 9x^2e^{2x} \\
        2ae^{2x} + (4ax + 2b)e^{2x} + 3[(ax^2 + bx + c)e^{2x}] &= 9x^2 e^{2x} \\ 
        2ae^{2x} + (4ax + 2b)e^{2x} + (3ax^2 + 3bx + 3c)e^{2x} &= 9x^2 e^{2x} \\
        (2a + 4ax + 2b + 3ax^2 + 3bx + 3c)e^{2x} &= 9x^2 e^{2x} \\
        [3ax^2 + (4a + 3b)x + (2a + 2b + 3c)]e^{2x} &= 9x^2 e^{2x}
    \end{align*}
    On procède par identification :
    \begin{align*}
        \systeme{
            3a = 9,
            4a + 3b = 0,
            2a + 2b + 3c = 0
        }
        \iff 
        \systeme{
            a = 3,
            4a + 3b = 0,
            2a + 2b + 3c = 0
        }
        \iff 
        \systeme{
            a = 3,
            3b = -12,
            2a + 2b + 3c = 0
        }
        \\ 
        \iff 
        \systeme{
            a = 3,
            b = -4,
            2a + 2b + 3c = 0
        }
        \iff 
        \systeme{
            a = 3,
            b = -4,
            3c = 2
        }
        \iff 
        \systeme{
            a = 3,
            b = -4,
            c = \frac{2}{3}
        }
    \end{align*}
    \[ y_1 = \left(3x^2 - 4x + \frac{2}{3} \right)e^{2x} \]
    Maintenant trouvons une solution particulière de l'équation :
    \[ (E_2) : y'' - 2y' + 3y = 4e^x \]
    Ici on a encore une forme $P(x)e^{\alpha x}$ avec $P(x) = 1, \deg(P) = 0$ et $\alpha \notin \{ 1 \pm i \sqrt{2} \}$ ainsi $y_2 = k e^{x}, k \in \R$
    \begin{align*}
        \begin{cases}
            y_2 = k e^{x} \\
            y_2' = k e^{x} \\
            y_2'' = k e^{x}
        \end{cases}
    \end{align*}
    \begin{align*}
        k e^{x} -2k e^{x} + 3k e^{x} &= 4e^x \\ 
        2k e^x &= 4e^x \\
        k e^x = 2e^x
    \end{align*}
    \[ y_2 = 2e^x \]
    Solution générale :
    \begin{align*}
        y &= y_0 + y_1 + y_2 \\
        y &= e^x \left(C_1 \cos(\sqrt{2}x) + C_2 i \sin(\sqrt{2}x) \right) + \left(3x^2 - 4x + \frac{2}{3} \right)e^{2x} + 2e^x \\
        y &= \left(C_1 \cos(\sqrt{2}x) + C_2 i \sin(\sqrt{2}x) + 2 \right)e^x + \left(3x^2 - 4x + \frac{2}{3} \right)e^{2x}
    \end{align*} 
\end{example}