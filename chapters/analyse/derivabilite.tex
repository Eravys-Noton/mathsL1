\chapter{Dérivabilité et accroissements finis}

\par \noindent Dans ce chapitre, en l'absence de précisions supplémentaires, $I$ désigne un intervalle de $\R$.
\section{Dérivabilité, théorèmes de Rolle et des accroissements finis}
\begin{definition}
	Soit $f : I \to \R$, $f$ est dérivable en $a \in I$ si et seulement s'il existe $\ell \in \R$ tel que :
	\begin{align*}
		  \lim_{x \to a} \frac{f(x) - f(a)}{x - a} &= \ell
	\end{align*}
    Une autre manière de définir la dérivabilité :
    \begin{align*}
        \lim_{h \to 0} \frac{f(a + h) - f(a)}{h} &= \ell
    \end{align*}
	On note $f'(a) \coloneqq \ell$ la dérivée de $f$ en $a$.
	Ainsi une fonction dérivable est une fonction dérivable en tout point de $I$.
	On peut également vérifier la limite à gauche et à droite de $a$.
\end{definition}

\begin{proposition}
	Soient $f : I \to \R$ et $a \in I$, si $f$ est \textbf{dérivable} en $a$ alors elle est \textbf{continue} en $a$.
\end{proposition}

\begin{definition}[Maximum, minimum]
    Soient $f : I \to \R$ et $a \in I$.
    \begin{enumerate}
        \item On dit que $a$ est un \textbf{maximum} si et seulement si : $\forall x \in I,\ f(a) \geq f(x)$.
        \item On dit que $a$ est un \textbf{minimum} si et seulement si : $\forall x \in I,\ f(a) \leq f(x)$.
    \end{enumerate}
    On appelle extremum un point qui est soit un maximum soit un minimum.
    \begin{enumerate}
        \item On dit que $a$ est un \textbf{maximum local} si et seulement si : $\exists \varepsilon > 0,\ a \text{ est un maximum de } f_{|]a - \varepsilon, a + \varepsilon[}$.
        \item On dit que $a$ est un \textbf{minimum local} si et seulement si : $\exists \varepsilon > 0,\ a \text{ est un minimum de } f_{|]a - \varepsilon, a + \varepsilon[}$.
    \end{enumerate}
\end{definition}

\begin{theorem}
	Soient $f : I \to \R$ une fonction dérivable et $a \in \overset{\circ}{I}$, $\overset{\circ}{I}$ désigne l'intérieur de $I$. 
	\[ a \text{ est un extremum local} \implies f'(a) = 0 \]
\end{theorem}

\begin{theorem}[Théorème de Rolle]
    Soient $(a, b) \in \R^2$ tel que $a < b$ et $f : [a, b] \to \R$ une fonction telle que :
    \begin{enumerate}
            \item $f$ est continue sur $[a, b]$.
            \item $f$ est dérivable sur $]a, b[$.
            \item $f(a) = f(b)$.
        \end{enumerate}
    \par \noindent Il existe un $c \in ]a, b[$ tel que : \[ f'(c) = 0 \]
\end{theorem}

\begin{theorem}[Théorème des accroissements finis]
    Soient $(a, b) \in \R^2$ tel que $a < b$ et $f : [a, b] \to \R$, telle que : 
    \begin{enumerate}
            \item $f$ est continue sur $[a, b]$.
            \item $f$ est dérivable sur $]a, b[$.
        \end{enumerate}
    \par \noindent Il existe un $c \in ]a, b[$ tel que :
	\[ f'(c) = \frac{f(b) - f(a)}{b - a} \]
\end{theorem}

\begin{corollary}[Inégalité des accroissements finis]
	Soit $M$ tel que $\abs{f'} \leq M$ sur $]a, b[$.
	\[ \abs{\frac{f(b) - f(a)}{b - a}} \leq M \]
\end{corollary}

\begin{proposition}
    Soient $f : [a, b] \to \R$ une fonction continue sur $[a, b]$ et dérivable sur $]a, b[$.
    \begin{enumerate}
            \item $f \text{ croissante} \iff f' \geq 0$.
            \item $f \text{ décroissante} \iff f' \leq 0$
        \end{enumerate}
\end{proposition}

\begin{definition}
	Soit $f : I \to \R$ et $n \in \N$.
	\begin{enumerate}
		\item $f \in \mathcal{D}^n(I, \R)$ signifie que $f$ est $n$ fois dérivable sur $I$.
		\item $f \in \mathcal{C}^n(I, \R)$ signifie que $f$ est $n$ fois dérivable et que sa dérivée $n$-ième est continue.
		\item $f \in \mathcal{C}^{\infty}(I, \R) = \mathcal{D}^{\infty}(I, \R)$ signifie que $f \in \mathcal{C}^n(I, \R),\ \forall n \in \N$. On dit que les fonctions $\mathcal{C}^{\infty}$ sont des \textbf{fonctions lisses}.
	\end{enumerate}
\end{definition}

\begin{proposition}
	$\forall (f, g) \in (\mathcal{C}^n(I, \R))^2 \implies f + g, f \cdot g, f \circ g \in \mathcal{C}^n(I,\R)$
\end{proposition}

\section{Convexité}
\begin{definition}
    Soit $f : I \to \R$.
    \begin{enumerate}
        \item On dit que $f$ est \textbf{convexe} si et seulement si : 
        \[ \forall (x, y) \in I^2,\ \lambda \in [0, 1],\ f(\lambda x + (1 - \lambda)y) \leq \lambda f(x) + (1 - \lambda) f(y) \]
        \item On dit que $f$ est \textbf{concave} si et seulement si : 
        \[ \forall (x, y) \in I^2,\ \lambda \in [0, 1],\ f(\lambda x + (1 - \lambda)y) \geq \lambda f(x) + (1 - \lambda) f(y) \]
    \end{enumerate}
\end{definition}

\par Géométriquement, $f$ est convexe signifie que son graphe passe sous les cordes de $f$ et que les tangentes passent sous le graphe. $f$ est concave signifie que son graphe au-dessus des cordes de $f$ et que les tangentes passent par-dessus le graphe.

\begin{theorem}
	Soit $f \in \mathcal{D}^2(I, \R)$.
        \begin{enumerate}
                \item $f \text{ convexe} \iff f'' \geq 0$.
                \item $f \text{ concave} \iff f'' \leq 0$.
            \end{enumerate}
\end{theorem}

\begin{proposition}
	Soient $f : I \to \R$ et $a \in I$, la tangente de $f$ en $a$ est :
	\[ \mathcal{T}_a(x) = f(a) + f'(a)(x - a) \]
\end{proposition}

\begin{proposition}
	Soient $(a, b) \in \R^2$ tel que $a < b,\ f : [a, b] \to \R$.
	La corde $c$ reliant les points $(a, f(a))$ et $(b, f(b))$ est définie par l'équation suivante :
	\[ c = \frac{f(b) - f(a)}{b - a} (x - a) + f(a) \]
\end{proposition}

\begin{proposition}
	Soient $f \in \mathcal{D}^2(I, \R),\ a \in I$.
	\begin{align*}
		\begin{cases}
			f'(a) = 0 \\
			f''(a) < 0
		\end{cases}
		\implies a \text{ est un maximum local, si } f''(a) > 0,\ a \text{  est un minimum local}
	\end{align*}
\end{proposition}

\begin{definition}[Suite récurrente]
	Soient $f : I \to \R$ et $u_0 \in I$.
	Si $\forall n \in \N,\ u_n \in I$ alors on peut définir :
	\[ u_{n + 1} = f(u_n) \]
\end{definition}

\begin{lemma}
	S'il existe un $\ell \in I$ tel que $\lim_{n \to +\infty} u_n = \ell$ et si $f$ est continue en $\ell$ alors 
	\[ f(\ell) = \ell \]
\end{lemma}

\begin{definition}
	Soit $f : I \to \R$. On dit que $f$ est stable sur $I$ si et seulement si : 
	\[ f(I) \subset I \]
\end{definition}

\begin{proposition}
	Si $f$ est \textbf{croissante} sur $I$ alors 
	$
	(u_n)_{n \in \N} = 
	\begin{cases}
		u_0 \in I \\
		u_{n + 1} = f(u_n)
	\end{cases}
	$
	est monotone.
	\[ u_1 \geq u_0 \iff (u_n) \text{ croissante} \]
	\[ u_1 \leq u_0 \iff (u_n) \text{ décroissante} \]
	Si $f$ est \textbf{décroissante} sur $I$, alors les suites extraites $(v_n)_{n \in \N} \coloneqq u_{2n}$ et $(w_n)_{n \in \N} \coloneqq u_{2n + 1}$ sont monotones, l'une est \textbf{croissante}, l'autre est \textbf{décroissante}.
\end{proposition}

\begin{definition}[Point fixe]
	Soient $f : \mathcal{D}_f \to \R$ et $x \in \mathcal{D}_f,\ f(x) = x$. On dit que $x$ est un \textbf{point fixe} de $f$.
\end{definition}

\begin{definition}[Coefficient de convergence]
    Soient $(u_n)_{n \in \N}$ tel que $\lim_{n \to +\infty} u_n = \ell \in \R$, $\varepsilon_n \coloneqq \abs{u_n - \ell}$ et supposons que $\lim_{n \to +\infty} \frac{\varepsilon_{n + 1}}{\varepsilon_n} = K \in \R_+$. 
    On appelle $K$ coefficient de la convergence.
    \begin{itemize}
        \item Si $K = 1$ la convergence est \textbf{lente}.
        \item Si $K = 0$ la convergence est \textbf{rapide}.
        \item Si $0 < K < 1$ la convergence est \textbf{géométrique}.
    \end{itemize}
\end{definition}

\begin{definition}[Fonction contractante]
	Soit $f : I \to \R$.
    On dit que $f$ est \textbf{contractante} si et seulement si :
    \[ \exists k \in ]0, 1[,\ \forall (x, y) \in I^2 : \abs{f(x) - f(y)} \leq k \abs{x - y} \]
\end{definition}

\begin{theorem}[Théorème du point fixe]
	Soient $I$ un intervalle fermé et $f : I \to I$ une fonction contractante et continue et $(u_n)_{n \in \N}$ sa suite récurrente associée.
	\begin{enumerate}
        \item Il existe un unique point fixe $\ell \in I$.
        \item $\lim_{n \to +\infty} u_n = \ell$.
        \item La convergence de $(u_n)_{n \in \N}$ est géométrique.
    \end{enumerate}
\end{theorem}

\begin{theorem}[Suites récurrentes linéaires d'ordre 2]
	Soit $u_{n+2} = a u_{n+1} + b u_n$ pour $a, b$ des réels.
	On pose l'équation suivante pour $r \in \R$ :
	\[ r^2 - ar - b = 0 \qquad \Delta = (-a)^2 + 4b \]
	Pour $(\lambda, \mu, \alpha) \in \R^3$ :
	\begin{itemize}
        \item $\Delta > 0 \implies u_n = \lambda r_1^n + \mu r_2^n$ avec $r_1, r_2$ tels que :
        \begin{align*}
            r_1 &= \frac{a - \sqrt{\Delta}}{2} & r_2 &= \frac{a + \sqrt{\Delta}}{2}
        \end{align*}
        \item $\Delta = 0 \implies u_n = \lambda r_0^n + \mu n r_0^n$ avec $r_0$ tel que :
        \begin{align*}
            r_0 = \frac{a}{2}
        \end{align*}
        \item $\Delta < 0 \implies u_n = \lambda r^n \cos(n \alpha) + \mu r^n \sin(n \alpha)$ avec les racines de la forme $re^{i\alpha}$ et $re^{-i\alpha}$.
    \end{itemize}    
    On trouve $\lambda, \mu$ grâce aux conditions sur les deux premiers termes de la suite.
\end{theorem}
