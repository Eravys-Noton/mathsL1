\chapter{Fonctions usuelles}


\begin{definition}[Fonction polynomiale]
	Soient $n \in \N,\ a_0, \ldots, a_n \in \R$.
	\begin{align*}
		f : \R &\to \R \\
            x &\mapsto \sum_{i = 0}^{n} a_i x^i
	\end{align*}
\end{definition}

\begin{definition}[Fonction partie entière]
	\[ \forall x \in \R,\ \exists ! E(x) \in \Z,\ E(x) \leq x < E(x + 1) \]
	\begin{align*}
		E : \R &\to \Z \\
        x &\mapsto E(x)
	\end{align*}
\end{definition}

\begin{definition}[Fonction puissance]
	Soit $a \in \R$.
	\begin{align*}
		f : \R_+^* &\to \R \\
        x &\mapsto x^a
	\end{align*}
\end{definition}

\begin{proposition}
	$\forall (a, b, x) \in \R^3$.
    \begin{multicols}{4}
        \begin{enumerate}
            \item $1^a = 1$.
            \item $x^a \cdot x^b = x^{a + b}$.
            \item $(xy)^a = x^a y^a$.
            \item $(x^a)^b = x^{ab}$.
        \end{enumerate}
    \end{multicols}
\end{proposition}

\section{Fonctions trigonométriques}
\begin{definition}[Fonctions trigonométriques]
	\begin{align*}
		\cos : \R &\to [-1, 1] \\
		        x &\mapsto \cos(x)
	\end{align*}
	\begin{align*}
		\sin : \R &\to [-1, 1] \\
		        x &\mapsto \sin(x)
	\end{align*}
	Pour tout $k \in \Z$ :
	\begin{align*}
		\tan : \R \backslash \left\{ \frac{k \pi}{2} \right\} &\to \R \\
		                                   					x &\mapsto \frac{\sin(x)}{\cos(x)}
	\end{align*}
\end{definition}

\begin{proposition}
	$\forall x \in \R,\ k \in \Z$.
	\begin{enumerate}
		\item $\cos(-x) = \cos(x)$.
		\item $\sin(-x) = -\sin(x)$.
		\item $\tan(-x) = -\tan(x)$.
		\item $\cos(x + 2k\pi) = \cos(x)$.
		\item $\sin(x + 2k\pi) = \sin(x)$.
		\item $\tan(x + k\pi) = \tan(x)$.
		\item $\cos$ est bijective sur $[0, \pi]$.
		\item $\sin$ est bijective sur $[-\frac{\pi}{2}, \frac{\pi}{2}]$.
		\item $\tan$ est bijective sur $]-\frac{\pi}{2}, \frac{\pi}{2}[$.
	\end{enumerate}
\end{proposition}

\begin{definition}
	On définit $\arccos,\ \arcsin$ et $\arctan$ comme étant les bijections réciproques des fonctions $\cos,\ \sin$ et $\tan$.
\end{definition}

\begin{proposition}
    \begin{enumerate}
    	\item $\forall x \in [-\frac{\pi}{2}, \frac{\pi}{2}],\ \arcsin(\sin(x)) = x$.
        \item $\forall x \in ]-\frac{\pi}{2}, \frac{\pi}{2}[,\ \arctan(\tan(x)) = x$.
        \item $\forall x \in [0, \pi],\ \arccos(\cos(x)) = x$.
        \item $\forall x \in [-1, 1]$ :
        \begin{multicols}{2}
            \begin{enumerate}
                \item $\sin(\arcsin(x)) = x$.
                \item $\cos(\arccos(x)) = x$.
            \end{enumerate}
        \end{multicols}
        \item $\forall x \in \R,\ \tan(\arctan(x)) = x$.
    \end{enumerate}
\end{proposition}

\begin{proposition}
	$\forall (a, b) \in \R^2$.
    \begin{multicols}{2}
        \begin{enumerate}
            \item $\sin(a + b) = \sin(a) \cos(b) + \sin(b) \cos(a)$.
            \item $\cos(a + b) = \cos(a) \cos(b) - \sin(a) \sin(b)$.
        \end{enumerate}
    \end{multicols}
\end{proposition}

\begin{proof}
    Nous pouvons procéder avec des produits scalaires, mais nous allons utiliser les nombres complexes ici.
    \\
    D'une part :
    \[e^{i (a + b)} = \cos(a + b) + i\sin(a + b)\]
    D'autre part :
    \begin{align*}
        e^{i (a + b)} &= e^{ia} \cdot e^{ib} \\
        &= [\cos(a) + i \sin(a)] \cdot [\cos(b) + i \sin(b)] \\
        &= \cos(a) \cos(b) + i\sin(b)\cos(a) + i\sin(a)\cos(b) - \sin(a)\sin(b) \\
        &= \cos(a)\cos(b) - \sin(a)\sin(b) + i [\sin(b)\cos(a) + \sin(a) \cos(b)].
    \end{align*}
    Par identification de la partie réelle et de la partie imaginaire :
    \[ \sin(a + b) = \sin(a) \cos(b) + \sin(b) \cos(a) \]
	\[ \cos(a + b) = \cos(a) \cos(b) - \sin(a) \sin(b) \]
\end{proof}

\begin{proposition}
    $\forall x \in \R,\ \cos^2(x) + \sin^2(x) = 1$.
\end{proposition}

\begin{proof}
    C'est une application du théorème de Pythagore sachant que le rayon du cercle trigonométrique est égal à 1.
\end{proof}

\begin{proposition}
	
    \begin{enumerate}
        \item $\lim_{x \to +\infty} \arctan(x) = \frac{\pi}{2}$.
        \item $\lim_{x \to -\infty} \arctan(x) = -\frac{\pi}{2}$.
    \end{enumerate}
\end{proposition}

\section{Exponentielle et logarithme}

\begin{definition}[Fonction exponentielle]
	\begin{align*}
		\exp : \R &\to \R_+^* \\
        x &\mapsto \exp(x) \equiv e^x
	\end{align*}
\end{definition}

\begin{proposition}
	La fonction exponentielle est \textbf{bijective} et \textbf{strictement croissante} et $\exp(0) = 1$.
    \begin{multicols}{2}
        \begin{enumerate}
            \item $\lim_{x \to -\infty} e^x = 0$.
            \item $\lim_{x \to +\infty} e^x = +\infty$.
        \end{enumerate}
    \end{multicols}
    \noindent $\forall (x, y) \in \R^2$.
    \begin{multicols}{2}
        \begin{enumerate}
            \item $\exp(x + y) = \exp(x) \cdot \exp(y)$.
            \item $\exp(-x) = \frac{1}{\exp(x)}$.
            \item $\exp(x - y) = \frac{\exp(x)}{\exp(y)}$.
        \end{enumerate}
    \end{multicols}
\end{proposition}

\begin{definition}[Logarithme néperien]
	\begin{align*}
		\ln : \R_+^* &\to \R \\
        x &\mapsto \ln(x)
	\end{align*}
\end{definition}

\begin{proposition}
	$\forall (x, y) \in \R_+^* \times \R$.
    \begin{multicols}{2}
        \begin{enumerate}
            \item $\exp(\ln(x)) = x$.
            \item $\ln(\exp(y)) = y$.
        \end{enumerate}
    \end{multicols}
\end{proposition}

\begin{proposition}
	La fonction logarithme néperien est \textbf{bijective} et \textbf{strictement croissante} et $\ln(1) = 0$.
    \begin{multicols}{2}
        \begin{enumerate}
            \item $\lim_{x \to 0} \ln(x) = -\infty$.
            \item $\lim_{x \to +\infty} \ln(x) = +\infty$.
        \end{enumerate}
    \end{multicols}
    \noindent $\forall (x, y) \in \R^2$.
    \begin{multicols}{2}
        \begin{enumerate}
            \item $\ln(xy) = \ln(x) + \ln(y)$.
            \item $\ln(\frac{1}{x}) = -\ln(x)$.
            \item $\ln(\frac{x}{y}) = \ln(x) - \ln(y)$.
            \item $\ln(x^n) = n\ln(x)$.
        \end{enumerate}
    \end{multicols}
\end{proposition}

\section{Fonctions hyperboliques}

\begin{definition}[Fonctions hyperboliques]
	\begin{align*}
		\cosh : \R &\to \R \\
		         x &\mapsto \frac{e^x + e^{-x}}{2}
	\end{align*}
	\begin{align*}
		\sinh : \R &\to \R \\ 
			     x &\mapsto \frac{e^x - e^{-x}}{2}
	\end{align*}
	\begin{align*}
		\tanh : \R &\to \R \\ 
				 x &\mapsto \frac{\sinh(x)}{\cosh(x)} = \frac{e^x - e^{-x}}{e^x + e^{-x}}
	\end{align*}
\end{definition}

\begin{proposition}
	$\forall (x,k) \in \R \times \Z$.
	\begin{enumerate}
		\item $\cosh(-x) = \cosh(x)$.
		\item $\sinh(-x) = \sinh(x)$.
		\item $\tanh(-x) = \tanh(x)$.
	\end{enumerate}
\end{proposition}

\begin{proposition}
	$\forall (x, y) \in \R^2$.
    \begin{enumerate}
        \item $\cosh(x + y) = \cosh(x) \cosh(y) + \sinh(x) \sinh(y)$.
        \item $\sinh(x + y) = \cosh(x) \sinh(y) + \sinh(x) \cosh(y)$.
    \end{enumerate}
\end{proposition}

\begin{proof}
    Calcul direct avec les définitions de $\cosh$ et de $\sinh$.
\end{proof}
