\chapter{Nombres réels}\label{chap:nb_reels}
\def\arraystretch{1}

\begin{definition}[Nombre réel \cite{wikipedia_nb_reel}]
    Un nombre réel est un nombre qui peut être représenté par une partie entière et une liste finie ou infinie de décimales.
\end{definition}

\begin{proposition}[Addition et multiplication sur $\R$]
    On peut définir sur $\R$ une addition (notée \og + \fg) et une multiplication (notée \og $\times$ \fg ou \og $\cdot$ \fg) qui prolonge l'addition et la multiplication de $\N$ et vérifie les règles suivantes pour $a, b, c \in \R$ :
    \begin{enumerate}
        \item Commutativité : 
        \begin{enumerate}
        	\item $a + b = b + a$.
        	\item $a \cdot b = b \cdot a$.
        \end{enumerate}
        \item Associativité : 
        \begin{enumerate}
        	\item $a + (b + c) = (a + b) + c$.
        	\item $a \cdot (b \cdot c) = (a \cdot b) \cdot c$.
        \end{enumerate}
        \item Distributivité : 
		\begin{enumerate}
			\item $a \cdot (b + c) = a \cdot b + a \cdot b$.
			\item $(a + b) \cdot c = a \cdot c + b \cdot c$.
		\end{enumerate}
        \item \'Element neutre :
        \begin{enumerate}
        	\item $a + 0 = a$.
        	\item $a \cdot 1 = a$.
        \end{enumerate}
        \item \'Elément absorbant : $a \cdot 0 = 0$.
    \end{enumerate}
\end{proposition}

\begin{proposition}[Relation d'ordre sur $\R$]
    On peut définir une relation d'ordre sur $\R$, notée \og $\leq$ \fg, qui prolonge l'ordre de $\N$ et vérifie les règles suivantes pour $a, b, c \in \R$ :
    \begin{enumerate}
        \item Réflexivité : $a \leq a$.
        \item Antisymétrie : $a \leq b \land b \leq a \implies a = b$.
        \item Transitivité : $a \leq b \land b \leq c \implies a \leq c$.
        \item Ordre total : $a \leq b \lor b \leq a$.
        \item Compatibilité avec l'addition : $a + c \leq b + c$.
        \item Compatibilité avec la multiplication par un réel positif : $a \leq b \land c \geq 0 \implies a \cdot c \leq b \cdot c$.
    \end{enumerate}
\end{proposition}

\begin{definition}[Valeur absolue]
    Soit $x \in \R$, on définit la valeur absolue ainsi :
    \begin{align*}
        \abs{x} =
        \begin{cases}
            x &\text{ si } x \geq 0 \\
            -x &\text{ si } x < 0
        \end{cases}
    \end{align*}
\end{definition}

\begin{proposition}
   	Soient $a, b \in \R$.
    \begin{enumerate}
            \item $\abs{a + b} \leq \abs{a} + \abs{b}$
            \item $\abs{a \cdot b} = \abs{a} \cdot \abs{b}$
            \item $\abs{a - b} \geq \abs{a} - \abs{b}$
            \item $\abs{a} = \sqrt{a^2}$
        \end{enumerate}
\end{proposition}

\begin{definition}[Intervalle]
    Un intervalle $I$ est une partie de $\R$ tel que :
    \begin{align*}
        \forall x, y \in I \implies \forall z \in I \text{ et } z \in \R,\ x \leq z \leq y
    \end{align*}
\end{definition}

\begin{definition}
    Soient $A \subseteq \R$ et $m \in \R$.
    \begin{enumerate}
        \item On dit que $m$ est un \textbf{majorant} de $A$ si et seulement si : $A \iff \forall x \in A,\ x \leq m$.
        \item On dit que $m$ est un \textbf{minorant} de $A$ si et seulement si : $\forall x \in A,\ x \geq m$.
    \end{enumerate}
    On dit que $A$ est \textbf{majorée} si elle admet un \textbf{majorant}, \textbf{minorée} si elle admet un \textbf{minorant} et \textbf{bornée} si elle est \textbf{majorée} et \textbf{minorée}.
\end{definition}

\begin{theorem}
    Soit $A \subseteq \R,\ A \neq \varnothing$. 
    \begin{enumerate}
    	\item Si $A$ est \textbf{majorée}, elle admet un \textbf{plus petit majorant} appelé la \textbf{borne supérieure} de $A$, notée : $\sup(A)$.
    	\item Si $A$ est \textbf{minorée}, elle admet un \textbf{plus grand minorant} appelé la \textbf{borne inférieure} de $A$, notée : $\inf(A)$.
    \end{enumerate}
\end{theorem}

\begin{proposition}
    Soient $A \subseteq \R,\ A \neq \varnothing$, $M$ un majorant de $A$ et $m$ un minorant de $A$.  
    \begin{enumerate}
        \item $M = \sup(A) \iff \forall \varepsilon > 0,\ ]M - \varepsilon, M] \cap A \neq \varnothing$.
        \item $m = \inf(A) \iff \forall \varepsilon > 0,\ [m, m + \varepsilon[ \cap A \neq \varnothing $
    \end{enumerate}
\end{proposition}